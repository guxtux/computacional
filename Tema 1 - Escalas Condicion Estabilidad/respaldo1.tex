\section{Instrucciones de entrada/salida}
\begin{frame}[fragile]
\frametitle{{Instrucciones de entrada y salida}
Entrada de datos:
\begin{itemize}
\item \verb|raw_input("entrada")|: lee una l\'{i}nea de entrada que es convertida a \texttt{string}.
\item \verb|eval(string)| : convierte \texttt{string} en un valor num\'{e}rico.
\end{itemize}
\fontsize{12}{12}\selectfont
\begin{minipage}{5.5cm}
\begin{exampleblock}{}<1->
	\verb|a = raw_input("Ingrese a: ")| \\
	\pause
	\texcolor{blue}{\texttt{Ingrese a: 2}}
\end{exampleblock}
\begin{exampleblock}{}<2->
	\verb|print a| \\
	\pause
	\textcolor{blue}{2}
\end{exampleblock}
\begin{exampleblock}{}<3->
	\verb|a| \\
	\pause
	\textcolor{blue}{'a'}
\end{exampleblock}
\begin{exampleblock}{}<4->
	\verb|type(a)| \\
	\pause
	\textcolor{blue}{\texttt{<type 'str'>}}
\end{exampleblock}
\begin{exampleblock}{}<5->
	\verb|b = eval(a)|
\end{exampleblock}
\end{minipage}
\hspace{0.5cm}
\begin{minipage}{5.5cm}
\begin{exampleblock}{}<6->
	\verb|print b, type(b)| \\
	\pause
	\textcolor{blue}{\texttt{2 <type 'int'>}}
\end{exampleblock}
\begin{exampleblock}{}<7->
	\verb|s=eval(raw_input("Ingrese s :"))| \\
	\pause
	\textcolor{blue}{\texttt{Ingrese s: 2*3}}
\end{exampleblock}
\begin{exampleblock}{}<8->
	\verb|print s, type(s)| \\
	\pause
	\textcolor{blue}{\texttt{6 <type 'int'>}}
\end{exampleblock}
\begin{exampleblock}{}<9->
	\verb|m=eval(raw_input("Ingrese m :"))| \\
	\pause
	\textcolor{blue}{\texttt{Ingrese s: hola}}
\end{exampleblock}
\end{minipage}
\end{frame}
\end{document}