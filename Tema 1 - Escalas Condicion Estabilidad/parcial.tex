\section{Errores numéricos}
\begin{frame}[fragile]
\frametitle{Errores numéricos}
Los errores causados por problemas numéricos han producido catástrofes que han derivado en cuantiosas pérdidas humanas y económicas.
\pause
\begin{itemize}[<+->]
\item Sonda espacial Mars.
\item Misil Patriot.
\item Explosión del Arianne 5.
\end{itemize}
\end{frame}
\begin{frame}[fragile]
\frametitle{Sonda Mars Climate Observer}
\begin{itemize}[<+->]
\item El 23 de septiembre de 1999 la sonda espacial Mars Climate Observer se pierde en la órbita marciana.
\item La falla se debió al ''omitir'' la conversión de unidades inglesas (millas) en unidades métricas en uno de los módulos del sistema de navegación.
\item La sonda entró a la atmósfera marciana 49 segundos antes de lo previsto con una velocidad superior a la prevista, con lo que se desintegró.
\end{itemize}
\end{frame}
\begin{frame}[fragile]
\frametitle{Fallo en misil Patriot}
\begin{itemize}[<+->]
\item El 25 de febrero de 1991 durante la Guerra del Golfo.
\item Un misil Patriot falla al interceptar un misil Scud, y cae en un campamento americano, matando a 28 soldados.
\item El contador de tiempo monitorizaba la posición cada $1/10$ segundos (pregunta: ¿la representación de $1/10$ en binario es periódica?)
\item El misil utilizaba registros de 24 bits.
\item El error que se generaba era de $0.000000095$ segundos.
\item Autonomía del misil Patriot: 100 horas.
\item Error total: $0.34$ segundos.
\item Velocidad del misil: 1676 m/s
\item Error de posición: $\sim500$ metros.
\end{itemize}
\end{frame}
\begin{frame}[fragile]
\frametitle{Explosión del Arianne 5}
\begin{itemize}[<+->]
\item El 4 de junio de 1996, un cohete Arianne 5 explotó 40 segundos después del lanzamiento.
\item Las pérdidas económicas se calcularon en 500 millones de euros (actualizado)
\item El sistema de control de la posición horizontal en el que el lanzamiento actuaba sobre los motores para evitar que se inclinara.
\item La posición se obtenía como un número real de 64 bits, pero se truncaba a un tipo de dato entero con signo.
\item Cuando la velocidad horizontal superó esta cantidad (\emph{overflow}), el sistema produjo un error de ejecución.
\item Curiosamente, una vez superados los primeros segundos, este sistema dejaba de ser importante.
\end{itemize}
\end{frame}
