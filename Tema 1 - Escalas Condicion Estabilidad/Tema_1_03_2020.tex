\documentclass[12pt]{beamer}
\usepackage[utf8]{inputenc}
\usepackage[spanish]{babel}
\usepackage{color}
\usepackage{hyperref}
\usepackage{amsmath}
\usepackage{amsthm}
\usepackage{multicol}
\usepackage{graphicx}
\usepackage{tikz}
\usepackage[autostyle,spanish=mexican]{csquotes}
%\usepackage[sfdefault]{roboto}  %% Option 'sfdefault' only if the base font of the document is to be sans serif
\renewcommand{\arraystretch}{1.5}
\renewcommand{\rmdefault}{cmr}% cmr = Computer Modern Roman
\usefonttheme[onlymath]{serif}

\newcommand{\python}{\texttt{python}}
\newcommand{\textoazul}[1]{\textcolor{blue}{#1}}
\newcommand{\azulfuerte}[1]{\textcolor{blue}{\textbf{#1}}}
\newcounter{saveenumi}
\newcommand{\seti}{\setcounter{saveenumi}{\value{enumi}}}
\newcommand{\conti}{\setcounter{enumi}{\value{saveenumi}}}

\linespread{1.5}
\beamertemplatenavigationsymbolsempty
\usefonttheme{professionalfonts}
\usefonttheme{serif}
\DeclareGraphicsExtensions{.pdf,.png,.jpg}
\renewcommand {\arraystretch}{1.25}
\mode<presentation>
{
  \usetheme{Warsaw}
  \setbeamertemplate{headline}{}
  %\useoutertheme{infolines}
  \useoutertheme{default}
  \setbeamercovered{invisible}
  % or whatever (possibly just delete it)
  \setbeamertemplate{section in toc}[sections numbered]
  \setbeamertemplate{subsection in toc}[subsections numbered]
  \setbeamertemplate{subsection in toc}{\leavevmode\leftskip=3.2em\rlap{\hskip-2em\inserttocsectionnumber.\inserttocsubsectionnumber}\inserttocsubsection\par}
  \setbeamercolor{section in toc}{fg=blue}
  \setbeamercolor{subsection in toc}{fg=blue}
  \setbeamercolor{frametitle}{fg=yellow}

  \setbeamertemplate{footline} 
{
  \leavevmode%
  \hbox{%
  \begin{beamercolorbox}[wd=.333333\paperwidth,ht=2.25ex,dp=1ex,center]{author in head/foot}%
    \usebeamerfont{author in head/foot}\insertsection
  \end{beamercolorbox}%
  \begin{beamercolorbox}[wd=.333333\paperwidth,ht=2.25ex,dp=1ex,center]{title in head/foot}%
    \usebeamerfont{title in head/foot}\textcolor{yellow}{\insertsubsection}
  \end{beamercolorbox}%
  \begin{beamercolorbox}[wd=.333333\paperwidth,ht=2.25ex,dp=1ex,right]{date in head/foot}%
    \usebeamerfont{date in head/foot}\insertshortdate{}\hspace*{2em}
    \insertframenumber{} / \inserttotalframenumber\hspace*{2ex} 
  \end{beamercolorbox}}%
  \vskip0pt%
}
}
\makeatother

\makeatletter
\patchcmd{\beamer@sectionintoc}
  {\vfill}
  {\vskip\itemsep}
  {}
  {}
\makeatother
\title{\large{Tema 1 - Escalas, condición y estabilidad}}
\subtitle{Curso de Física Computacional}
\author[]{M. en C. Gustavo Contreras Mayén}
\date{\today}
\institute{Facultad de Ciencias - UNAM}
\titlegraphic{\includegraphics[width=2cm]{Imagenes/escudo-facultad-ciencias}\hspace*{4.75cm}~%
   \includegraphics[width=2cm]{Imagenes/escudo-unam}
}
\begin{document}
\maketitle
\section*{Contenido}
\frame[allowframebreaks]{\tableofcontents[currentsection, hideallsubsections]}
\fontsize{14}{14}\selectfont
\spanishdecimal{.}
\section{Sumando series}
\frame{\tableofcontents[currentsection, hideothersubsections]}
\subsection{La serie para $\sin x$}
\begin{frame}
\frametitle{Sumando una serie conocida}
Un problema numérico clásico es la suma de una serie infinita que evalúa una función.
\\
\bigskip
\pause
Consideremos la serie infinita para la función $\sin x$:
\begin{align*}
\sin x =  x - \dfrac{x^{3}}{3!} + \dfrac{x^{5}}{5!} - \dfrac{x^{7}}{7!} + \ldots
\end{align*}
\end{frame}
\begin{frame}
\frametitle{Nuestra tarea}
Nuestra tarea será utilizar esta serie para calcular valores de $\sin x$ para $x < 2 \pi$ y $x > 2 \pi$, con un error absoluto en cada caso, que sea menor que $1$ parte en $10^{8}$.
\end{frame}
\begin{frame}
\frametitle{Primera propuesta}
Si bien una serie infinita es \emph{exacta} en un sentido matemático, no es en si, un buen algoritmo porque debemos dejar de sumar en algún momento.
\\
\bigskip
\pause
Un algoritmo podría ser la suma finita:
\begin{align}
\sin x \simeq \sum_{n = 1}^{N} \dfrac{(-1)^{n-1} \, x^{2 n - 1}}{(2 n - 1)!}
\label{suma_seno_algoritmo}
\end{align}
\end{frame}
\begin{frame}
\frametitle{Pregunta necesaria}
Debemos de plantearnos la siguiente pregunta: \textbf{¿En qué momento debemos de detener la suma?}
\\
\bigskip
NO deberíamos de considerar: \emph{\enquote{Cuando la solución corresponda con una tabla o con el valor de una función de la librería matemática}.} 
\end{frame}
\begin{frame}
\frametitle{El método numérico}
No olvidemos que el algoritmo (\ref{suma_seno_algoritmo}) nos pide que debemos de calcular $(-1)^{n-1} \, x^{2 n - 1}$ para luego dividir entre $(2 n - 1)!$
\\
\bigskip
\pause
No es una buena idea para calcular los valores:
\begin{itemize}[<+->]
\item Tanto $(-1)^{n-1} \, x^{2 n - 1}$ como $(2 n - 1)!$ pueden ser valores muy grandes y generar \emph{overflows}, mientras que el cociente no.
\item Las potencias y factoriales consumen demasiado tiempo de cómputo.
\end{itemize}
\end{frame}
\begin{frame}
\frametitle{Segunda propuesta}
Una mejor aproximación es utilizar una multiplicación para relacionar el siguiente término de la serie, a partir del anterior:
\begin{align}
\fontsize{12}{12}\selectfont
\begin{aligned}
&\dfrac{(-1)^{n-1} \, x^{2 n - 1}}{(2 n - 1)!} = \dfrac{x^{2}}{(2 n - 1)(2 n - 2)} \, \dfrac{(1-)^{n-2} \, x^{2 n - 3}}{(2 n - 3!)} \\[0.5em]
&\Rightarrow n-\mbox{término} = \dfrac{- x^{2}}{(2 n - 1)(2 n - 2)} \times (n-1)-\mbox{término}
\end{aligned}
\label{eq:termino_suma}
\end{align}
\end{frame}
\begin{frame}
\frametitle{Observación importante}
Si bien queremos garantizar una precisión definitiva para la función $\sin x$, eso no es tan fácil de hacer.
\\
\bigskip
\pause
Lo que es fácil de hacer es \emph{suponer que el error en la suma es aproximadamente el último término sumado}, esto supone que no hay error de redondeo.
\end{frame}
\begin{frame}
\frametitle{Valor de tolerancia}
Para obtener un error absoluto de $1$ parte en $10^{8}$, debemos de detener el cálculo cuando se cumpla la condición:
\begin{align}
\abs{\dfrac{n-\mbox{término}}{\mbox{suma}}} < 10^{-8}
\label{eq:tolerancia}
\end{align}
\end{frame}
\begin{frame}
\frametitle{Valor de tolerancia}
\begin{align*}
\abs{\dfrac{n-\mbox{término}}{\mbox{suma}}} < 10^{-8}
\end{align*}
donde el \enquote{término} es el último en la serie (\ref{suma_seno_algoritmo}) y la \enquote{suma} es la suma acumulada para todos los términos.
\end{frame}
\begin{frame}
\frametitle{Sobre la tolerancia}
En general, podemos determinar cualquier nivel de tolerancia.
\\
\bigskip
Aunque si está demasiado cerca o es más pequeño que el épsilon de la máquina, es posible que el cálculo no se pueda alcanzar.
\end{frame}
\begin{frame}[fragile]
\frametitle{Pseudocódigo}
Proponemos el siguiente pseudocódigo con el que intentaremos resolver el problema:
\begin{lstlisting}
term = x
suma = x
tolerancia = 10e+-08

mientras abs(term/suma) > tolerancia
    term = -term * x * x /((2 * n - 1)(2* n - 2))
    suma = suma + term
\end{lstlisting}
\end{frame}
\subsection*{Implementación}
\begin{frame}[fragile]
\frametitle{Implementación}
\setbeamercolor{item projected}{bg=blue!70!black,fg=yellow}
\setbeamertemplate{enumerate items}[circle]
\begin{enumerate}[<+->]
\item Escribe un programa en \python{} para los calcule los siguientes valores de $x$:
\begin{align*}
x &= -2 \, \pi, -1.5 \, \pi, - \pi, -0.5 \, \pi, -0.25 \, \pi, \\
& 0.25 \, \pi , 0.5 \, \pi, \, \pi, 1.5 \, \pi, 2 \, \pi
\end{align*}
\seti
\end{enumerate}
\end{frame}
\begin{frame}[fragile]
\frametitle{Implementación}
\setbeamercolor{item projected}{bg=blue!70!black,fg=yellow}
\setbeamertemplate{enumerate items}[circle]
\begin{enumerate}[<+->]
\conti   
\item Presenta los resultados en una tabla de la siguiente forma:
\begin{table}
\centering
\begin{tabular}{c | c | c}
$x$ & suma & $\dfrac{\abs{\sin(x) - \mbox{suma}}}{\sin x}$ 
\end{tabular}
\end{table}
donde imax es el número de iteraciones realizadas, y $\sin (x)$ es el valor obtenido de la función seno incluida en el paquete \textoazul{math}.
\seti
\end{enumerate}
\end{frame}
\begin{frame}[fragile]
\frametitle{Tabla de resultados}
\begin{table}
\centering
\begin{tabular}{c | c | c}
$x$ & suma & $\dfrac{\abs{\sin(x) - \mbox{suma}}}{\sin x}$ \\ \hline
$-2 \, \pi$ & & \\ \hline
$-1.5 \, \pi$ & & \\ \hline
\vdots & & \\ \hline
$1.5 \, \pi$ & & \\ \hline
$2 \, \pi$ & & \\ \hline
\end{tabular}
\end{table}
\end{frame}
\begin{frame}[fragile]
\frametitle{Implementación}
\setbeamercolor{item projected}{bg=blue!70!black,fg=yellow}
\setbeamertemplate{enumerate items}[circle]
\begin{enumerate}[<+->]
\conti   
\item Modifica el código tal que sume la serie de \enquote{buena forma} (sin factoriales), para calcular la suma de \enquote{mala forma} (con factoriales explícitos).
\end{enumerate}
\end{frame}
\begin{frame}[allowframebreaks, fragile]
\frametitle{Evaluando valores individuales}
\begin{lstlisting}[style=codigopython]
import math

def error_relativo(exacto, aproximado):
    return math.fabs(math.sin(exacto)- aproximado)/math.sin(exacto)*100

x = float(input('Teclea el valor a evaluar: '))

n = 10
suma = x
term =  x

print('x \t exacta \t   suma \t  error')
for i in range(2, n):
    term = (-term *x*x)/((2*i-1)*(2*i-2))
    suma  = suma + term
    print('{0:} \t {1:1.10f} \t {2:1.10f} \t {3:1.5e}'.format(x, math.sin(x), suma, error_relativo(x, suma)))
\end{lstlisting}
\end{frame}
\begin{frame}[fragile]
\frametitle{Resultado para el valor}
\begin{table}
\fontsize{12}{12}\selectfont
\begin{tabular}{c c c c}
x & exacta & suma & error \\ \hline
$-2.0$ & $-0.9092974268$ & $-0.6666666667$  $-2.66833e+01$ \\ \hline
$-2.0$ & $-0.9092974268$ & $-0.9333333333$  $-2.64335e+00$ \\ \hline
$-2.0$ & $-0.9092974268$ & $-0.9079365079$  $-1.49667e-01$ \\ \hline
$-2.0$ & $-0.9092974268$ & $-0.9093474427$  $-5.50049e-03$ \\ \hline
$-2.0$ & $-0.9092974268$ & $-0.9092961360$  $-1.41963e-04$ \\ \hline
$-2.0$ & $-0.9092974268$ & $-0.9092974515$  $-2.71572e-06$ \\ \hline
$-2.0$ & $-0.9092974268$ & $-0.9092974265$  $-4.00566e-08$ \\ \hline
$-2.0$ & $-0.9092974268$ & $-0.9092974268$  $-4.69511e-10$ \\ \hline
\end{tabular}
\end{table}
\end{frame}
\begin{frame}
\frametitle{Mejora para calcular más valores}
Introducir los valores uno por uno resulta una tarea tediosa, por lo que habrá que mejorar el código para que en un sólo paso, se devuelvan los valores pedidos y completar la tabla.
\\
\bigskip
\textcolor{red}{¿Cómo le hacemos?}
\end{frame}
\begin{frame}[fragile]
\frametitle{Tabla con resultados}
\begin{table}
\fontsize{12}{12}\selectfont
\begin{tabular}{c c c c}
x & exacta & suma & error \\ \hline
$-2$  & $-0.9092974268$ &  $-0.9092974268$ &  $-4.69511e-10$ \\ \hline
$-1.5$ & $-0.9974949866$ &  $-0.9974949866$ &  $-1.82534e-12$ \\ \hline
$-1$ & $-0.8414709848$ &  $-0.8414709848$ &  $-0.00000e+00$ \\ \hline
$-0.5$ & $-0.4794255386$ &  $-0.4794255386$ &  $-0.00000e+00$ \\ \hline
$0.5$ & $0.4794255386 $ &  $0.4794255386$ &  $0.00000e+00$ \\ \hline
$1$ & $0.8414709848$ &  $0.8414709848$ &  $0.00000e+00$ \\ \hline
$1.5$ & $0.9974949866$ &  $0.9974949866$ &  $1.82534e-12$ \\ \hline
$2$ & $0.9092974268$ &  $0.9092974268$ &  $4.69511e-10$ \\ \hline
\end{tabular}
\end{table}
\end{frame}
\begin{frame}[allowframebreaks, fragile]
\frametitle{Código para la mejora}
\begin{lstlisting}[style=codigopython]
x = [-2, -1.5, -1, -0.5, 0.5, 1, 1.5, 2]

n = 10

print('x \t exacta \t   suma \t  error')
for j in x:
    suma = j
    term =  j
    for i in range(2, n):
        term = (-term *j*j)/((2*i-1)*(2*i-2))
        suma  = suma + term
    print('{0:} \t {1:1.10f} \t {2:1.10f} \t {3:1.5e}'.format(j, math.sin(j), suma, error_relativo(j, suma)))
\end{lstlisting}
\end{frame}
\end{document}