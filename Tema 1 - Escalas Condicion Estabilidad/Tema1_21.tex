\documentclass[12pt]{beamer}
\usepackage[utf8]{inputenc}
%\usepackage[latin1]{inputenc}
\usepackage[spanish]{babel}
%\usetheme{Warsaw}
%\usepackage{euler}
\usepackage{amsmath}
\usepackage{amsthm}
\usepackage{multicol}
\usepackage{graphicx}
\usepackage{tikz}
\usepackage{color}
\usepackage{listings}
\lstset{ %
language=Python,                % choose the language of the code
basicstyle=\footnotesize,       % the size of the fonts that are used for the code
numbers=left,                   % where to put the line-numbers
numberstyle=\footnotesize,      % the size of the fonts that are used for the line-numbers
stepnumber=1,                   % the step between two line-numbers. If it is 1 each line will be numbered
numbersep=5pt,                  % how far the line-numbers are from the code
backgroundcolor=\color{white},  % choose the background color. You must add \usepackage{color}
showspaces=false,               % show spaces adding particular underscores
showstringspaces=false,         % underline spaces within strings
showtabs=false,                 % show tabs within strings adding particular underscores
frame=single,   		% adds a frame around the code
tabsize=2,  		% sets default tabsize to 2 spaces
captionpos=b,   		% sets the caption-position to bottom
breaklines=true,    	% sets automatic line breaking
breakatwhitespace=false,    % sets if automatic breaks should only happen at whitespace
escapeinside={\%}{)}          % if you want to add a comment within your code
}
%\usepackage{epstopdf}
\DeclareGraphicsExtensions{.pdf,.png,.jpg}
\renewcommand {\arraystretch}{1.5}
\mode<presentation>
{
  \usetheme{Warsaw}
  \setbeamercovered{transparent}
  % or whatever (possibly just delete it)
}
\title{Tema 1 - Propagaci\'{o}n de errores}
\subtitle{Curso de F\'{i}sica Computacional}
\author{M. en C. Gustavo Contreras May\'{e}n}
\date{}
%\email{curso.fisica.comp@gmail.com}
%\ptsize{10}
\begin{document}
\maketitle
\fontsize{14}{14}\selectfont
\spanishdecimal{.}
\begin{frame}{Contenido}
\tableofcontents[pausesections]
\end{frame}
\section{Modelos para el desastre}
\begin{frame}
\frametitle{Modelos para el desastre}
Un c\'{a}lculo que utiliza n\'{u}meros que se almacenan de manera aproximada en la computadora, puede devolver una soluci\'{o}n aproximada.
\\
\bigskip
Para demostrar este hecho, consideremos que hay un valor de incertidumbre, sea $x_{c}$ el valor representado en la computadora del valor exacto $x$, tal que:
\[x_{c} \simeq  x(1+\epsilon_{x}) \]
\end{frame}
\begin{frame}
Donde $\epsilon_{x}$ es el error relativo de $x_{c}$, el cual se espera que sea similar en magnitud al \'{e}psilon de la m\'{a}quina $\epsilon_{m}$.
\\
\bigskip
Si usamos \'{e}sta notación para una diferencia entre dos valores, tenemos que:
\begin{eqnarray*}
a=b-c & \rightarrow a_{c} \simeq b_{c} - c_{c} \simeq b(1+\epsilon_{b}) - c(1+\epsilon_{c}) \\
{} & \rightarrow \dfrac{a_{c}}{a} \simeq 1 + \epsilon_{b} \dfrac{b}{a} - \dfrac{c}{a} \epsilon_{c}
\end{eqnarray*} 
\end{frame}
\begin{frame}
Vemos que el error resultante en $a$ es un promedio de los errores en $b$ y $c$, y no hay seguridad en que los dos t\'{e}rminos se cancelen.
\end{frame}
\begin{frame}
El error en $a_{c}$ se incrementa cuando se restan dos valores cercanos $(b \simeq c)$, dado que cuando se restan las cifras m\'{a}s significativas de ambos n\'{u}meros, el error de las cifras menos significativas toma la forma de:
\[ \dfrac{a_{c}}{a}=1+\epsilon_{a} \simeq 1 + \dfrac{b}{a}(\epsilon_{b} - \epsilon_{c}) \simeq 1 + \dfrac{b}{a} max(\vert \epsilon_{b} \vert, \vert \epsilon_{c} \vert) \]
\end{frame}
\subsection{Ejercicio 1}
\begin{frame}
\frametitle{Ejercicio 1}
Aprendimos en la secundaria a resolver la ecuaci\'{o}n homog\'{e}nea de segundo grado:
\[ ax^{2} + bx + c = 0\]
que tiene una soluci\'{o}n anal\'{i}tica que se puede escribir como
\\
\medskip
\begin{minipage}{5cm}
\[ x_{1,2} = \dfrac{-b \pm \sqrt{b^{2} - 4ac}}{2a}\]
\end{minipage}
\hspace{0.5cm}
\begin{minipage}{5cm}
\[ x_{1,2} = \dfrac{-2c}{b \pm \sqrt{b^{2}-4ac}}\]
\end{minipage}
\end{frame}
\begin{frame}
Revisando la expresi\'{o}n anterior vemos que la cancelación de la diferencia (y por tanto, un incremento en el error) aumenta cuando $b^{2} >> 4ac$ debido a que la ra\'{i}z cuadrada y el siguiente t\'{e}rmino est\'{a}n muy pr\'{o}ximas a cancelarse.
\end{frame}
\begin{frame}
\frametitle{Manos al c\'{o}digo}
\begin{enumerate}
\item Escribe un programa que calcule las cuatro soluciones para valores arbitrarios de $a$, $b$ y $c$.
\item Revisa c\'{o}mo los errores obtenidos en los c\'{a}lculos, aumentan conforme hay una cancelaci\'{o}n de la diferencia de t\'{e}rminos y su relaci\'{o}n con la precisi\'{o}n de la m\'{a}quina. Prueba con los siguientes valores $a=1$, $b=1$, $c=10^{n}, n=1,2,3,\ldots$
\item C\'{o}mo mejorar\'{i}as el programa para obtener la mayor precisi\'{o}n en tu respuesta?
\end{enumerate}
\end{frame}
\end{document}