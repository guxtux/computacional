\documentclass[11pt]{article}
\usepackage[utf8]{inputenc}
%\usepackage[latin1]{inputenc}
\usepackage[spanish]{babel}
\decimalpoint
\usepackage{anysize}
\usepackage{graphicx} 
\usepackage{amsmath}
\usepackage{float}
\usepackage{tikz}
\usepackage{color}
\marginsize{1cm}{2cm}{2cm}{2cm}  
\title{Examen 1: Errores, condición y estabilidad. \\ Curso de Física Computacional}
\author{M. en C. Gustavo Contreras Mayén}
\date{ }
\begin{document}
\maketitle
\fontsize{14}{14}\selectfont
\textbf{Indicaciones: } Para cada uno de los problemas, deber\'{a}s de anotar tu c\'{o}digo en Python, adem\'{a}s de incluir gr\'{a}ficas en un archivo jpg, si es que lo menciona la pregunta.
\\
\\
En caso de que tengas alguna complicaci\'{o}n para resolver el problema, comenta dentro del mismo c\'{o}digo para que sepamos en d\'{o}nde se te presenta la dificultad.
\begin{enumerate}
\item \textbf{(2 puntos) } Los dos ejercicios en clase que pidieron trabajar.  
\item \textbf{(1.5 puntos) }Un problema cl\'{a}sico en c\'{o}mputo cient\'{i}fico, es la suma de una serie para evaluar una funci\'{o}n. Sea la serie de potencias para la funci\'{o}n exponencial:
\[e^{-x} = 1 - x + \dfrac{x^{2}}{2!} + \dfrac{x^{3}}{3!} +\cdots \hspace{1.5cm} (x^{2} < \infty)  \]
Utiliza la serie anterior para calcular el valor de $e^{-x}$ para $x=0.1,1,10, 100, 1000$ con un error absoluto para cada caso, menor a $10^{-8}$.
\item \textbf{(1.5 puntos) }Usando el m\'{e}todo de Horner, env\'{i}a la tabla de valores tanto de los puntos de evaluaci\'{o}n, como los de funci\'{o}n evaluada a un archivo de datos, el polinomio es:
\[p(x)= 2x^{4} - 20x^{3} + 70x^{2}+ 100x+48 \]
para valores de $x$ en el intervalo $[-4,-1]$, con saltos de $x$ de valor $\Delta x = 0.5$.
\\
\\
Grafica los puntos obtenidos y el polinomio $p(x)$, interpreta los resultados obtenidos.
\item \textbf{(2.5 puntos) } El valor de $\pi$ se puede calcular aproximando el \'{a}rea de un c\'{i}rculo unitario como el l\'{i}mite de una sucesi\'{o}n $p_{1}, p_{2}, \ldots$ descrita a continuaci\'{o}n:
\\
Se divide un c\'{i}rculo unitario en $2^{n}$ sectores (en el ejemplo, $n=3$). Se aproxima el \'{a}rea del sector por el \'{a}rea del tri\'{a}ngulo is\'{o}celes. El \'{a}ngulo $\theta_{n}$ es $2 \pi / 2^{n}$. El \'{a}rea del tri\'{a}ngulo es $1/2 \sin \theta_{n}$.
\\
\begin{figure}[H]
\centering
\begin{tikzpicture}
\draw (3,3) circle (2);
\draw (1,3) -- (5,3);
\draw (3,1) -- (3,5);
\draw (1.6,4.39) -- (4.39,1.6);
\draw (4.39,4.39) -- (1.6,1.6);
\draw (4.39,4.39) -- (5,3);
\draw (5,3) -- (4.39,1.6);
\draw (5,3) -- (4.39,1.6);
\draw (4.39,1.6) -- (3,1);
\draw (3,1) -- (1.6,1.6);
\draw (1.6,1.6) -- (1,3);
\draw (1,3) -- (1.6,4.39);
\draw (1.6,4.39) -- (3,5);
\draw (3,5) -- (4.39,4.39);
\draw (3.3,3) arc (0:16:1);
\draw [font=\small] (3.7,3.2) node {$\theta_{n}$};
\draw [font=\small] (3.7,4) node {1};
\draw [font=\small] (4.2,2.8) node {1};
\end{tikzpicture}
\caption{Divisi\'{o}n en $n$ sectores.}
\end{figure}
La en\'{e}sima aproximaci\'{o}n a $\pi$ es: $p_{n}= 2^{n-1} \sin \theta_{n}$. Demuestra que
\[\sin \theta_{n} = \dfrac{\sin \theta_{n-1}}{\left( 2 \left[ 1+ (1-\sin^{2}\theta_{n-1})^{\frac{1}{2}} \right] \right)^{\frac{1}{2}}} \]
Usa esta relaci\'{o}n de recurrencia para generar las sucesiones $\sin \theta_{n}$ y $p_{n}$ en el rango $3 \leq n \leq 20$ iniciando con $\sin \theta_{2}=1$. Compara tus resultados con el valor de $4.0 \arctan(1.0)$
\item \textbf{(2.5 puntos) } La sucesi\'{o}n de Fibonacci $1,1,2,3,5,8,13,\ldots$ est\'{a} definida por la relaci\'{o}n de recurrencia lineal
\begin{equation*}
\begin{cases}
\lambda_{1} = 1 \hspace{0.5cm} \lambda_{2}= 1 \\
\lambda_{n} = \lambda_{n-1} + \lambda_{n-2} \hspace{0.5cm} (n \geq 3)
\end{cases}
\end{equation*}
Una f\'{o}rmula para obtener el n-\'{e}simo n\'{u}mero de Fibonacci es
\[ \lambda_{n} = \dfrac{1}{\sqrt{5}} \left\lbrace \left[ \dfrac{1}{2} (1 + \sqrt{5}) \right]^{n} - \left[ \dfrac{1}{2} (1 - \sqrt{5}) \right]^{n} \right\rbrace \]
Calcula $\lambda_{n}$ en $3\leq n \leq 50$ usando tanto la relaci\'{o}n de recurrencia como la f\'{o}rmula. Discute los resultados obtenidos.
\end{enumerate}
\end{document}