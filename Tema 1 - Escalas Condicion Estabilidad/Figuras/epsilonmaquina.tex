\documentclass{standalone}
\usepackage[utf8]{inputenc}
\usepackage{tikz}
\usepackage{mathrsfs}
\usetikzlibrary{positioning, shapes.geometric, arrows}
\tikzstyle{startstop} = [draw, rectangle, rounded corners, minimum width=3cm, minimum height=1cm,text centered, draw=black]
\tikzstyle{bola} = [draw, circle , minimum size = 10, draw=black, text centered]
\tikzstyle{elipse} = [draw, ellipse, minimum height = 10]
\usepackage[T1]{fontenc}
\renewcommand*\familydefault{\ttdefault} %% Only if the base font of the document is to be typewriter style
\begin{document}
\begin{tikzpicture}
\draw (0,3) -- (13,3);
\draw (0,3) -- (0,2.7);

\foreach \x in {0, 0.3, 0.6, 0.9, 1.2, 5, 5.3, 5.6, 5.9, 11.8, 12.1, 12.4, 12.7, 13}
	\foreach \y in {3.2, 2.8}
		\draw (\x, 2.8) -- (\x, 3.2); 

\node[] at (0,2.4) {0};
\node[] at (5,2.4) {1};
\node[] at (13,2.4) {$2.17 \times 10^{38}$};
\draw[->] (0.3,4.5) -- (0.3,3.4);
\draw[->] (5.3,0) -- (5.3,2.6);
\draw[->] (13,0) -- (13,2);
\node[text width=4cm , align=center] at (2.5,5) {Valor de punto flotante positivo más pequeño};
\node[text width=2cm , align=center] at (5.3,-1) {Epsilón de la máquina};
\node[text width=3cm, align=center] at (12,-1) {Máximo valor de punto flotante};
\end{tikzpicture}
\end{document}