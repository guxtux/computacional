\documentclass[11pt]{article}
\usepackage[utf8]{inputenc}
%\usepackage[latin1]{inputenc}
\usepackage[spanish]{babel}
\usepackage{anysize}
\usepackage{graphicx} 
\usepackage{amsmath}
\usepackage{tikz}
\usepackage{float}
\renewcommand{\baselinestretch}{1.2}
\spanishdecimal{.} 
\marginsize{1.5cm}{1.5cm}{0cm}{2cm}  
\title{Examen Reposición 1: Errores, condición y estabilidad \\ Curso de Física Computacional}
\author{M. en C. Gustavo Contreras Mayén}
\date{ }
\begin{document}
\maketitle
\fontsize{14}{14}\selectfont
\begin{enumerate}
\item La serie de Maclaurin para la función $\arctan$ converge en $-1 \leq x \leq 1$ y está dada por
\[ \arctan x = \lim_{n \rightarrow \infty} P_{n} (x) = \lim_{n \rightarrow \infty} \sum_{i=1}^{n} (-1)^{i+1} \dfrac{x^{2i-1}}{2i-1}	\]
\begin{enumerate}
\item Usa el hecho de que $\tan (\frac{\pi}{4})=1$ para determinar el número de términos $n$ de la serie que debemos de sumar para garantizar que $\vert 4 P_{n}(1) - \pi \vert < 10^{-3}$
\item En un experimento simulado en la computadora se requiere que el valor aproximado de $\pi$ esté dentro de $10^{-10}$. ¿Cuántos términos de la serie debemos de sumar para obtener este grado de precisión?
\end{enumerate}
\item Identifica los números de punto flotante correspondientes a las siguientes cadenas de bits (debes de resolverlo mediante un código, para simplificar la tarea, considera una conversión de binario a decimal simple, es decir, no uses el estándar IEEE)
\begin{enumerate}
\item \fbox{0 \hspace{0.1cm} 00000000 \hspace{0.1cm} 00000000000000000000000}
\item \fbox{1 \hspace{0.1cm} 00000000 \hspace{0.1cm} 00000000000000000000000}
\item \fbox{0 \hspace{0.1cm} 11111111 \hspace{0.1cm} 00000000000000000000000}
\item \fbox{1 \hspace{0.1cm} 11111111 \hspace{0.1cm} 00000000000000000000000}
\item \fbox{0 \hspace{0.1cm} 00000001 \hspace{0.1cm} 00000000000000000000000}
\item \fbox{0 \hspace{0.1cm} 10000001 \hspace{0.1cm} 01100000000000000000000}
\item \fbox{0 \hspace{0.1cm} 01111111 \hspace{0.1cm} 00000000000000000000000}
\item \fbox{0 \hspace{0.1cm} 01111011 \hspace{0.1cm} 10011001100110011001100}
\end{enumerate}
\item Da la representación en binario con precisión simple de los siguientes números decimales
\begin{enumerate}
\item -9876.54321
\item 0.2343375
\item -285.75
\item $10^{2}$
\item $+0.0$ y $-0.0$
\end{enumerate}
\end{enumerate}
\end{document}








