\documentclass[12pt]{beamer}
\newenvironment{ConCodigo}[1]
  {\begin{frame}[fragile,environment=ConCodigo]{#1}}
  {\end{frame}}
\graphicspath{{Imagenes/}{../Imagenes/}}
\usepackage[utf8]{inputenc}
\usepackage[spanish]{babel}
\usepackage{hyperref}
\usepackage{etex}
\reserveinserts{28}
\usepackage{amsmath}
\usepackage{amsthm}
\usepackage{mathtools}
\usepackage{multicol}
\usepackage{multirow}
\usepackage{tabulary}
\usepackage{booktabs}
\usepackage{nccmath}
\usepackage{biblatex}
\usepackage{epstopdf}
\usepackage{graphicx}
%\usepackage{enumitem,xcolor}
\usepackage{siunitx}
\sisetup{scientific-notation=true}
%\usepackage{fontspec}
\usepackage{lmodern}
\usepackage{float}
\usepackage[format=hang, font=footnotesize, labelformat=parens]{caption}
\usepackage[autostyle,spanish=mexican]{csquotes}
\usepackage{standalone}
\usepackage{blkarray}
\usepackage{algorithm}
\usepackage{algorithmic}
\usepackage{tikz}
\usepackage[siunitx]{circuitikz}
\usetikzlibrary{arrows,patterns,shapes}
\usetikzlibrary{decorations.markings}
\usetikzlibrary{arrows}
\usepackage{color}
\usepackage{xcolor}
%\usepackage{beton}
%\usepackage{euler}
%\usepackage[T1]{fontenc}
\usepackage[sfdefault]{roboto}  %% Option 'sfdefault' only if the base font of the document is to be sans serif
\usepackage[T1]{fontenc}
\renewcommand*\familydefault{\sfdefault}
\DeclareGraphicsExtensions{.pdf,.png,.jpg}
\usepackage{hyperref}
\renewcommand {\arraystretch}{1.5}
\newcommand{\python}{\texttt{python}}
\usefonttheme[onlymath]{serif}
\setbeamertemplate{navigation symbols}{}
\usetikzlibrary{patterns}
\usetikzlibrary{decorations.markings}
\tikzstyle{every picture}+=[remember picture,baseline]
%\tikzstyle{every node}+=[inner sep=0pt,anchor=base,
%minimum width=2.2cm,align=center,text depth=.15ex,outer sep=1.5pt]
%\tikzstyle{every path}+=[thick, rounded corners]
\setbeamertemplate{caption}[numbered]
\newcommand{\ptm}{\fontfamily{ptm}\selectfont}
%Se usa la plantilla Warsaw modificada con spruce
\mode<presentation>
{
  \usetheme{Warsaw}
  \setbeamertemplate{headline}{}
  \useoutertheme{default}
  \usecolortheme{seahorse}
  \setbeamercovered{invisible}
}
% \AtBeginSection[]
% {
% \begin{frame}<beamer>{Contenido}
% \normalfont\mdseries
% \tableofcontents[currentsection]
% \end{frame}
%}

\usepackage{listings}
\lstset{ %
language=Python,                % choose the language of the code
basicstyle=\small,       % the size of the fonts that are used for the code
numbers=left,                   % where to put the line-numbers
numberstyle=\footnotesize,      % the size of the fonts that are used for the line-numbers
stepnumber=1,                   % the step between two line-numbers. If it is 1 each line will be numbered
numbersep=5pt,                  % how far the line-numbers are from the code
backgroundcolor=\color{white},  % choose the background color. You must add \usepackage{color}
showspaces=false,               % show spaces adding particular underscores
showstringspaces=false,         % underline spaces within strings
showtabs=false,                 % show tabs within strings adding particular underscores
frame=single,   		% adds a frame around the code
tabsize=4,  		% sets default tabsize to 2 spaces
captionpos=b,   		% sets the caption-position to bottom
breaklines=true,    	% sets automatic line breaking
breakatwhitespace=false,    % sets if automatic breaks should only happen at whitespace
escapeinside={\#}{)}          % if you want to add a comment within your code
}

\makeatletter
\setbeamertemplate{footline}
%\setbeamercolor{title in head/foot}{fg=Green}
{
  \leavevmode%
  \hbox{%
  \begin{beamercolorbox}[wd=.333333\paperwidth,ht=2.25ex,dp=1ex,center]{author in head/foot}%
    \usebeamerfont{author in head/foot} \textcolor{white}{\insertsection}
  \end{beamercolorbox}}%
  \begin{beamercolorbox}[wd=.333333\paperwidth,ht=2.25ex,dp=1ex,center]{title in head/foot}%
    \usebeamerfont{title in head/foot} \textcolor{white}\insertsubsection
  \end{beamercolorbox}%
  \begin{beamercolorbox}[wd=.333333\paperwidth,ht=2.25ex,dp=1ex,right]{date in head/foot}%
    \usebeamerfont{date in head/foot} \textcolor{white}\insertshortdate{}\hspace*{2em}
    \textcolor{white}\insertframenumber{} / \textcolor{white}\inserttotalframenumber\hspace*{2ex} 
  \end{beamercolorbox}}%
  \vskip0pt%
\makeatother
\normalfont
\usepackage{ccfonts}% http://ctan.org/pkg/{ccfonts}
\usepackage[T1]{fontenc}% http://ctan.or/pkg/fontenc
\renewcommand{\rmdefault}{cmr}% cmr = Computer Modern Roman
\linespread{1.3}
\title{Algebra matricial}
\subtitle{Curso de Física Computacional}
\author[]{M. en C. Gustavo Contreras Mayén}
\newcounter{saveenumi}
\newcommand{\seti}{\setcounter{saveenumi}{\value{enumi}}}
\newcommand{\conti}{\setcounter{enumi}{\value{saveenumi}}}
\resetcounteronoverlays{saveenumi}
\begin{document}
\newcommand{\localtextbulletone}{\textcolor{gray}{\raisebox{.45ex}{\rule{.6ex}{.6ex}}}}
\maketitle
\fontsize{14}{14}\selectfont
\spanishdecimal{.}
\begin{frame}{Contenido}
\tableofcontents[pausesections]
\end{frame}
\section{Algebra matricial con python}
\subsection{Introducción}
\begin{frame}
\frametitle{Introducción}
Nuestro punto de partida es el siguiente:
\\
\bigskip
Encontrar $\lambda$ para el cual existe una solución no trivial de
\[ \mathbf{A \; x} = \lambda \; \mathbf{x} \]
\end{frame}
\begin{frame}
La \emph{forma estándar} de un problema matricial de valores propios es
\begin{equation}
\mathbf{A \; x} = \lambda \mathbf{x}
\label{eq:ecuacion_09_01}
\end{equation}
donde $\mathbf{A}$ es una matriz dada de tamaño $n \times n$. El problema que debemos resolver, es calcular un escalar $\lambda$ y un vector $\mathbf{x}$.
\end{frame}
\begin{frame}
Re-escribimos la ecuación (\ref{eq:ecuacion_09_01}) de la forma
\begin{equation}
( \mathbf{A} - \lambda \; \mathbf{I} ) \mathbf{x} = \mathbf{0}
\label{eq:ecuacion_09_02}
\end{equation}
Se hace evidente que se trata de un sistema de $n$ ecuaciones homogéneas.
\end{frame}
\begin{frame}
Una solución obvia es la trivial $x = 0$. Una solución no trivial sólo puede existir si el determinante de la matriz de coeficientes es nulo, es decir
\begin{equation}
\vert \mathbf{A} - \lambda \; \mathbf{I} \vert = \mathbf{0}
\label{eq:ecuacion_09_03}
\end{equation}
\end{frame}
\begin{frame}
La expansión del determinante nos lleva a la ecuación polinomial, también conocida como la \emph{ecuación característica}
\[ a_{0} + a_{1} \; \lambda + a_{2} \; \lambda^{2} + \ldots + a_{n} \; \lambda^{n} = 0 \]
Que tiene las raíces $\lambda_{i} \; i = 1, 2, \ldots,  n$, llamados valores propios (autovalores, \emph{eigenvalores}) de la matriz $\mathbf{A}$.
\\
\medskip
Las soluciones $x_{i}$ de $(\mathbf{A}- \lambda_{i} \; \mathbf{I}) \mathbf{x} = \mathbf{0}$ son conocidas como vectores propios \emph{eigenvectores}.
\end{frame}
\begin{frame}
\frametitle{Ejemplo}
Considermos la matriz
\begin{equation}
\mathbf{A} = \begin{vmatrix}
1 & -1 & 0 \\
-1 & 2 & -1 \\
0 & -1 & 1
\end{vmatrix}
\label{eq:ecuacion_09_a}
\end{equation}
\end{frame}
\begin{frame}
La ecuación característica es
\begin{equation}
\vert \mathbf{A} - \lambda \; \mathbf{I} =  \begin{vmatrix}
1 - \lambda & -1 & 0 \\
-1 & 2 - \lambda & -1 \\
0 & -1 & 1 - \lambda
\end{vmatrix} 
= -3 \lambda +  4 \lambda^{2} - \lambda^{3} = 0
\label{eq:ecuacion_09_b}
\end{equation}
\pause
Las raíces de esta ecuación son $\lambda_{1} = 0$, $\lambda_{2} = 1$, $\lambda_{3} = 3$.
\end{frame}
\begin{frame}
Para calcular el vector propio correspondiente a $\lambda_{3}$, sustituimos $\lambda = \lambda_{3}$ en la ecuación (\ref{eq:ecuacion_09_02}), para obtener
\begin{equation}
\begin{bmatrix}
-2 & -1 & 0 \\
-1 & -1 & -1 \\
0 & -1 & -2
\end{bmatrix}
\begin{bmatrix}
x_{1} \\
x_{2} \\
x_{3} 
\end{bmatrix} = 
\begin{bmatrix}
0 \\
0 \\
0
\end{bmatrix}
\label{eq:ecuacion_09_c}
\end{equation}
\end{frame}
\begin{frame}
Sabemos que el determinante de la matriz de coeficientes es cero, de modo que las ecuaciones no son linealmente independientes. 
\\
\bigskip
Por lo tanto, podemos asignar un valor arbitrario a cualquier componente de $x$ y usar dos de las ecuaciones para calcular los otros dos componentes.
\end{frame}
\begin{frame}
Escogiendo $x_{1} = 1$, la primera ecuación de la ecuación (\ref{eq:ecuacion_09_c}) nos devuelve $x_{2} = -2$, y de la tercera ecuación se obtiene $x_{3} = 1$. 
\\
\bigskip
Así, el vector propio asociado con $\lambda_{3}$ es
\[ \mathbf{x}_{3}  = \begin{bmatrix}
1 \\
-2 \\
1
\end{bmatrix} \]
\end{frame}
\begin{frame}
Obtenemos los otros dos vectores propios de la misma forma
\[ \mathbf{x}_{2} = \begin{bmatrix}
1 \\
0 \\
-1
\end{bmatrix} \hspace{1cm}
\mathbf{x}_{1} =  \begin{bmatrix}
1 \\
1 \\
1
\end{bmatrix} \]
\end{frame}
\begin{frame}
A veces es conveniente mostrar los vectores propios como columnas de una matriz $\mathbf{X}$. Para nuestro primer ejemplo, esta matriz es
\[ \mathbf{X} = \left[ \mathbf{x}_{1} \hspace{0.3cm} \mathbf{x}_{2} \hspace{0.3cm} \mathbf{x}_{3} \right] = 
\begin{bmatrix}
1 & 1 & 1 \\
1 & 0 & -2 \\
1 & -1 & 1
\end{bmatrix} \]
\end{frame}
\begin{frame}
De este ejemplo se desprende claramente que la magnitud de un vector propio es indeterminada; sólo su dirección se puede calcular a partir de la ecuación (\ref{eq:ecuacion_09_02}).
\\
\bigskip
Es costumbre normalizar los vectores propios asignando una magnitud unitaria a cada vector. 
\pause
\\
\medskip
Así, los vectores propios normalizados en nuestro ejemplo son
\[ \mathbf{X} = \begin{bmatrix}
1 / \sqrt{3} & 1 / \sqrt{2} & 1 / \sqrt{6} \\
1 / \sqrt{3} & 0 & -2 / \sqrt{6} \\
1 / \sqrt{3} & -1 / \sqrt{2} & 1 / \sqrt{6}
\end{bmatrix} \]
\end{frame}
\begin{frame}
\frametitle{Supondremos que los vectores propios están normalizados}
Mencionaremos algunas propiedades útiles de los valores propios y de los vectores propios(dados sin prueba):
\setbeamercolor{item projected}{bg=green!70!black,fg=white}
\setbeamertemplate{enumerate items}[circle]
\begin{enumerate}[<+->]
\item Todos los valores propios de una matriz simétrica son reales.
\item Todos los valores propios de una matriz simétrica definida positiva son reales y positivos.
\seti
\end{enumerate}
\end{frame}
\begin{frame}
\setbeamercolor{item projected}{bg=green!70!black,fg=white}
\setbeamertemplate{enumerate items}[circle]
\begin{enumerate}[<+->]
\conti
\item Los vectores propios de una matriz simétrica son ortonormales; Es decir, $\mathbf{X}^{T} \; \mathbf{X} = \mathbf{I}$.
\item Si los valores propios de $\mathbf{A}$ son $\lambda_{i}$, entonces los valores propios de $\mathbf{A}^{-1}$ son $\lambda_{i}^{-1}$
\end{enumerate}
\end{frame}
\begin{frame}
Los problemas de valores propios que se originan a partir de problemas físicos a menudo terminan con una matriz $\mathbf{A}$ simétrica.
\\
\bigskip
Esto es de cierta manera afortunado, porque los problemas simétricos de los valores propios son más fáciles de resolver que sus homólogos no simétricos (que pueden tener valores propios complejos).
\end{frame}
\begin{frame}
Tomaremos en nuestra discusión a los valores propios y vectores propios de matrices simétricas.
\\
\bigskip
Las fuentes comunes de problemas de los valores propios son los análisis de las vibraciones y la estabilidad.
\end{frame}
\begin{frame}
Estos problemas a menudo tienen las siguientes características:
\setbeamercolor{item projected}{bg=green!70!black,fg=white}
\setbeamertemplate{enumerate items}[circle]
\begin{enumerate}[<+->]
\item Las matrices son grandes y escasas (por ejemplo, tienen una estructura con bandas).
\item Necesitamos conocer sólo los valores propios; si se requieren vectores propios, sólo unos cuantos de ellos son de interés.
\end{enumerate}
\end{frame}
\begin{frame}
Un algortimo útil de valores propios, debe ser capaz de utilizar estas características para minimizar los cálculos.
\\
\bigskip
En particular, debe ser lo suficientemente flexible como para calcular sólo lo que necesitamos y no más.
\end{frame}
\section{Métodos estacionarios}
\begin{frame}
A continuación se presenta una breve descripción de cada uno de los métodos a revisar.
\end{frame}
\subsection{Método de Jacobi}
\begin{frame}
\frametitle{Método de Jacobi}
El método de Jacobi está basado en resolver cualquier variable local con respecto a otras variables; cada iteración del método corresponde a la solución de cada variable a la vez. 
\\
\bigskip
El método es fácil de entender e implementar, pero la convergencia es lenta.
\end{frame}
\subsection{Método de Gauss-Siedel}
\begin{frame}
\frametitle{Método de {Gauss-Siedel}}
El método de Gauss-Siedel es parecido al método de Jacobi, excepto en que utiliza valores actualizados tan pronto como los tenga disponibles. 
\\
\bigskip
En general, si el método de Jacobi converge, el método de Gauss-Siedel convergerá más rápido que el método de Jacobi.
\end{frame}
\subsection{Método de Sobrerelajación (SOR)}
\begin{frame}
\frametitle{Método de Sobrerelajación (SOR)}
El método de sobrerelajación (\textit{Succesive Over Relaxation}) puede considerarse como un derivado del método de Gauss-Siedel, ya que se introduce un parámetro de extrapolación $\omega$. 
\\
\bigskip
Para valores óptimos de $\omega$, el método SOR puede converger más rápido que el método de Gauss-Siedel en un orden de magnitud.
\end{frame}
\subsection{Método de Sobrerelajación Simétrico (SSOR)}
\begin{frame}
\frametitle{Método de Sobrerelajación Simétrico (SSOR)}
El método de sobrerelajación simétrico (\textit{Symmetric Successive Over Relaxation}) no tiene una mayor ventaja que el método SOR como método iterativo, pero es bastante útil como precondicionador para métodos no estacionarios.
\end{frame}
\end{document}