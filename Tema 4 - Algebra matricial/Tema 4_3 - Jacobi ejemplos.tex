\documentclass[12pt]{beamer}
\newenvironment{ConCodigo}[1]
  {\begin{frame}[fragile,environment=ConCodigo]{#1}}
  {\end{frame}}
\graphicspath{{Imagenes/}{../Imagenes/}}
\usepackage[utf8]{inputenc}
\usepackage[spanish]{babel}
\usepackage{hyperref}
\usepackage{etex}
%\reserveinserts{28}
\usepackage{amsmath}
\usepackage{amsthm}
\usepackage{mathtools}
\usepackage{multicol}
\usepackage{multirow}
\usepackage{tabulary}
\usepackage{booktabs}
\usepackage{nccmath}
\usepackage{physics}
\usepackage{biblatex}
\usepackage[outdir=./]{epstopdf}
%\epstopdfsetup{outdir=./}
\usepackage{graphicx}
%\usepackage{enumitem,xcolor}
\usepackage{siunitx}
%\sisetup{scientific-notation=true}
%\usepackage{fontspec}
\usepackage{lmodern}
\usepackage{float}
\usepackage[format=hang, font=footnotesize, labelformat=parens]{caption}
\usepackage[autostyle,spanish=mexican]{csquotes}
\usepackage{standalone}
\usepackage{blkarray}
\usepackage{algorithm}
\usepackage{algorithmic}
\usepackage{tikz}
\usepackage[siunitx, RPvoltages]{circuitikz}
\usetikzlibrary{arrows,patterns,shapes}
\usetikzlibrary{decorations.markings}
\usetikzlibrary{arrows}
\usepackage{color}
\usepackage{xcolor}
%\usepackage{beton}
%\usepackage{euler}
%\usepackage[T1]{fontenc}
\usepackage[sfdefault]{roboto}  %% Option 'sfdefault' only if the base font of the document is to be sans serif
\usepackage[T1]{fontenc}
\renewcommand*\familydefault{\sfdefault}
\DeclareGraphicsExtensions{.pdf,.png,.jpg}
\usepackage{hyperref}
\renewcommand {\arraystretch}{1.5}
\newcommand{\python}{\texttt{python}}
\usefonttheme[onlymath]{serif}
\setbeamertemplate{navigation symbols}{}
\usetikzlibrary{patterns}
\usetikzlibrary{decorations.markings}
\tikzstyle{every picture}+=[remember picture,baseline]
%\tikzstyle{every node}+=[inner sep=0pt,anchor=base,
%minimum width=2.2cm,align=center,text depth=.15ex,outer sep=1.5pt]
%\tikzstyle{every path}+=[thick, rounded corners]
\setbeamertemplate{caption}[numbered]
\newcommand{\ptm}{\fontfamily{ptm}\selectfont}
%Se usa la plantilla Warsaw modificada con spruce
\mode<presentation>
{
  \usetheme{Warsaw}
  \setbeamertemplate{headline}{}
  \useoutertheme{default}
  \usecolortheme{albatross}
  \setbeamercovered{invisible}
}
% \AtBeginSection[]
% {
% \begin{frame}<beamer>{Contenido}
% \normalfont\mdseries
% \tableofcontents[currentsection]
% \end{frame}
% }

\include{pre_codigo}
\makeatletter
\setbeamertemplate{footline}
%\setbeamercolor{title in head/foot}{fg=Green}
{
  \leavevmode%
  \hbox{%
  \begin{beamercolorbox}[wd=.333333\paperwidth,ht=2.25ex,dp=1ex,center]{author in head/foot}%
    \usebeamerfont{author in head/foot} \textcolor{white}{\insertsection}
  \end{beamercolorbox}}%
  \begin{beamercolorbox}[wd=.333333\paperwidth,ht=2.25ex,dp=1ex,center]{title in head/foot}%
    \usebeamerfont{title in head/foot} \textcolor{white}\insertsubsection
  \end{beamercolorbox}%
  \begin{beamercolorbox}[wd=.333333\paperwidth,ht=2.25ex,dp=1ex,right]{date in head/foot}%
    \usebeamerfont{date in head/foot} \textcolor{white}\insertshortdate{}\hspace*{2em}
    \textcolor{white}\insertframenumber{} / \textcolor{white}\inserttotalframenumber\hspace*{2ex} 
  \end{beamercolorbox}}%
  \vskip0pt%
\makeatother
\normalfont
\usepackage{ccfonts}% http://ctan.org/pkg/{ccfonts}
\usepackage[T1]{fontenc}% http://ctan.or/pkg/fontenc
\renewcommand{\rmdefault}{cmr}% cmr = Computer Modern Roman
\linespread{1.3}
\title{Algebra matricial - El método de Jacobi 2}
\subtitle{Curso de Física Computacional}
\author[]{M. en C. Gustavo Contreras Mayén}
\newcounter{saveenumi}
\newcommand{\seti}{\setcounter{saveenumi}{\value{enumi}}}
\newcommand{\conti}{\setcounter{enumi}{\value{saveenumi}}}
\resetcounteronoverlays{saveenumi}

\begin{document}
\newcommand{\localtextbulletone}{\textcolor{gray}{\raisebox{.45ex}{\rule{.6ex}{.6ex}}}}
\maketitle
\fontsize{14}{14}\selectfont
\spanishdecimal{.}
\begin{frame}{Contenido}
\tableofcontents[pausesections]
\end{frame}
\section{Método de Jacobi}
\subsection{Ejemplo con el método de Jacobi}
\begin{frame}
\frametitle{Ejemplo con el método de Jacobi}
\fontsize{12}{12}\selectfont
Consideremos los esfuerzos aplicados al siguiente cubo
\\
\begin{minipage}{5cm}
\begin{figure}
	\centering
	\includegraphics[scale=0.45]{cubotensor.eps}
\end{figure} \hspace{0.5cm}
\end{minipage}
\begin{minipage}{5cm}
\[ \mathbf{S} = \begin{bmatrix}
80 & 30 & 0 \\
30 & 40 & 0 \\
0 & 0 & 60
\end{bmatrix} \mbox{ MPa} \]
\end{minipage}
\\
\medskip
En la matriz de esfuerzos, cada renglón consta de los tres componentes de tensión que actúan sobre un plano de coordenadas.
\end{frame}
\begin{frame}
Se puede demostrar que los valores propios de $S$ son las esfuerzos (tensiones) principales y los vectores propios son normales a los planos principales.
\pause
\setbeamercolor{item projected}{bg=red!70!black,fg=white}
\setbeamertemplate{enumerate items}[circle]
\begin{enumerate}[<+->]
\item Determina las tensiones principales diagonalizando $S$ con una rotación de Jacobi.
\item Calcula los vectores propios.
\end{enumerate}
\end{frame}
\begin{frame}
\frametitle{Solución inciso 1)}
Para eliminar $S_{12}$ debemos de aplicar la rotación en el plano $1-2$. Con $k = 1$ y $\ell = 2$, resulta
\[ \phi = - \dfrac{s_{11} - S_{22}}{2 \; S_{12}}  = - \dfrac{80 - 40}{2(30)} = - \dfrac{2}{3}\]
\pause
por lo que ahora podemos calcular $t$
\[ t = \dfrac{sgn(\phi)}{\vert \phi \vert + \sqrt{ \phi^{2} + 1}} = \dfrac{-1}{2/3 + \sqrt{(2/3)^{2} + 1}}  = -0.53518\]
\end{frame}
\begin{frame}
De acuerdo a las ecuaciones a las que llegamos en donde se determina el cambio de $\mathbf{S}$ debido a la rotación, tenemos que
\fontsize{12}{12}\selectfont
\begin{align*}
S_{11}^{*} &= S_{11} - t \; S_{12} = 80 - (-0.53518)(30) = 96.055 \mbox{ MPa} \\
S_{22}^{*} &= S_{22} - t \; S_{12} = 40 + (-0.53518)(30) = 23.945 \mbox{ MPa} \\
S_{12}^{*} &= S_{21}^{*}
\end{align*}
\end{frame}
\begin{frame}
Por lo que la matriz de esfuerzos diagonalizada es
\[ \mathbf{S}^{*} = \begin{bmatrix}
96.055 & 0 & 0 \\
0 & 23.945 & 0 \\
0 &  0 & 60
\end{bmatrix} \]
donde los elementos de la diagonal son los esfuerzos principales.
\end{frame}
\begin{frame}
\frametitle{Solución inciso 2)}
Para calcular los valores propios, obtenemos primero los valores de $c$, $s$ y $\tau$:
\begin{align*}
c &= \dfrac{1}{\sqrt{1 + t^{2}}} = \dfrac{1}{\sqrt{1 + (-0.53518)^{2}}} = 0.88168 \\
s &= tc = (-0.5319(0.88168) = -0.47186 \\
\tau &= \dfrac{s}{1 + c} = \dfrac{-0.47186}{1 + 0.88168} = -0.25077
\end{align*}
\end{frame}
\begin{frame}
Para obtener los cambios en la matriz de transformación $\mathbf{P}$, recordemos que $\mathbf{P}$ se inicializa como una matriz identidad, la primera ecuación que obtenemos es
\[ \begin{split}
P_{11}^{*} &=  P_{11} - s(P_{12} - \tau \; P_{11}) \\
&= 1 - (-0.47186)(0 + (-0.25077)(1)) = 0.88167 \\
P_{21}^{*} &= P_{21} - s(P_{22} + \tau \; P_{21} \\
&= 0 - (-0.47186)[1 + (0.25077)(0)] = 0.47186
\end{split} \]
\end{frame}
\begin{frame}
De manera similar la segunda ecuación resulta se
\[ P_{12}^{*} = -04786 \hspace{1.5cm} P_{22}^{*} = 0.88167 \]
La tercera columna y renglón de $\mathbf{P}$ no son afectadas por la transformación.
\end{frame}
\begin{frame}
\frametitle{La matriz resultate es}
Entonces tenemos el resultado
\[ \mathbf{P}^{*} = \begin{bmatrix}
0.88167 & -0.47186 & 0 \\
0.47186 & 0.88167 & 0 \\
0 & 0 & 1
\end{bmatrix} \]
Las columnas de $\mathbf{P}^{*}$ son los vectores propios de $\mathbf{S}$.
\end{frame}
\section{Problemas de valores propios generalizados}
\subsection{Conceptos generales}
\begin{frame}
\frametitle{{Problemas de valores propios generalizados}}
En ciertos problemas de la física, derivados por ejemplo del formalismo de la mecánica cuántica, aparecen los denominados \emph{problemas de valores propios generalizados}.
\\
\bigskip
Que no se definen por una sola matriz, sino por dos matrices cuadradas $\mathbf{A}$ y $\mathbf{B}$
\begin{equation}
\mathbf{A} \; \mathbf{x} =  \lambda \; \mathbf{B} \; \mathbf{c}
\label{eq:ecuacion_08_30}
\end{equation}
\end{frame}
\begin{frame}
En la mayoría de los casos, tales problemas son representaciones matriciales de ecuaciones de operadores con respecto a bases no ortogonales en un espacio de dimensión finita, como $\mathbb{R}_{n}$.
\end{frame}
\begin{frame}
Suponiendo que la matriz $\mathbf{B}$ es no singular, una posibilidad directa de transformar la ecuación generalizada (\ref{eq:ecuacion_08_30}) en una forma estándar
\[ \mathbf{A} \; \mathbf{x} = \lambda \; \mathbf{x} \]
consiste en multiplicar a la izquierda la ecuación por la inversa $\mathbf{B}^{-1}$
\[ ( \mathbf{B}^{-1} \cdot \mathbf{A}) \; \mathbf{x} = \lambda \; \mathbf{x} \]
\end{frame}
\begin{frame}
Desafortundamente, incluso en el caso particular de las matrices simétricas $\mathbf{A}$ y $\mathbf{B}$, el producto $\mathbf{B \cdot A^{-1}}$ ya no conserva la simetría y, por consiguiente, el problema del valor propio estándar resultante ya no puede resolverse mediante métodos específicos para matrices simétricas (como el método de Jacobi)
\end{frame}
\begin{frame}
Si las matrices $\mathbf{A}$ y $\mathbf{b}$ son simétricas y, además, $\mathbf{B}$ es definida positiva, la transformación del problema generalizado (\ref{eq:ecuacion_08_30}) en un problema estándar puede obtenerse realizando primero la factorización de Cholesky de la matriz $\mathbf{B}$.
\begin{equation}
\mathbf{B} =  \mathbf{L} \cdot \mathbf{L}^{T}
\label{eq:ecuacion_08_31}
\end{equation}
donde $\mathbf{L}$ es una matriz triangular inferior y $\mathbf{L}^{T}$ es su transpuesta.
\end{frame}
\begin{frame}
Multiplicando por la izquierda por $\mathbf{L}^{-1}$ y re-emplazando por la factorización ec. (\ref{eq:ecuacion_08_31}), tenemos que
\[ \mathbf{L}^{-1} \cdot \mathbf{a} \cdot \mathbf{x} = \lambda \; \mathbf{L}^{T} \cdot \mathbf{x} \]
\pause
Además, al incorporar la descomposición trivial de la matriz unitaria
\[ \mathbf{E} = \left( \mathbf{L} \cdot \mathbf{L}^{T} \right)^{T} = \left( \mathbf{L}^{-1} \right)^{T} \cdot \mathbf{L}^{T} \]
\end{frame}
\begin{frame}
En el producto $\mathbf{A} \cdot \mathbf{x}$, el problema de valores propios toma la forma
\[ \mathbf{L}^{-1} \cdot \mathbf{A} \cdot \left( \mathbf{L}^{-1} \right) \cdot \left( \mathbf{L}^{T} \cdot \mathbf{x} \right) = \lambda \left( \mathbf{L}^{T} \cdot \mathbf{x} \right) \]
\pause
Haciendo las anotaciones evidentes
\begin{equation}
\mathbf{\widetilde{x}} = \mathbf{L}^{T} \cdot \mathbf{x}
\label{eq:ecuacion_08_32}
\end{equation}
y
\begin{equation}
\mathbf{\widetilde{A}} = \mathbf{L}^{-1} \cdot \mathbf{A} \cdot \left( \mathbf{L}^{-1} \right)^{T}
\label{eq:ecuacion_08_33}
\end{equation}
\end{frame}
\begin{frame}
Se obtiene finalmente una matriz simétrica $\mathbf{\widetilde{A}}$ para el problema de valores propios estándar:
\begin{equation}
\mathbf{\widetilde{A}} \; \mathbf{x} = \lambda \; \mathbf{\widetilde{x}}
\label{eq:ecuacion_08_34}
\end{equation}
\pause
Los valores propios de $\mathbf{\widetilde{A}}$ coinciden con los valores propios de la matriz original $\mathbf{A}$, mientras que los valores propios de $\mathbf{A}$ deben de recalcularse a partir de $\mathbf{\widetilde{A}}$ por la transformación lineal
\begin{equation}
\mathbf{x} = \left( \mathbf{L}^{-1} \right)^{T} \; \mathbf{x}
\label{eq:ecuacion_08_35}
\end{equation}
\end{frame}
\begin{frame}
Desde el punto de vista de una implementación práctica, la solución del problema de valores propios generalizado (\ref{eq:ecuacion_08_30}) se puede llevar a cabo con las siguientes etapas:
\end{frame}
\begin{frame}
\fontsize{12}{12}\selectfont
\setbeamercolor{item projected}{bg=red!70!black,fg=white}
\setbeamertemplate{enumerate items}[circle]
\begin{enumerate}[<+->]
\item Realizar la descomposición de Cholesky $\mathbf{B} = \mathbf{L} \cdot \mathbf{L}^{T}$.
\item Calcular la inversa $\mathbf{L}^{-1} = [\ell_{ij}]_{nn}$.
\item Generar la matriz transformada $\mathbf{\widetilde{A}} = \mathbf{L}^{-1} \cdot \mathbf{A} \cdot \left( \mathbf{L}^{-1} \right)^{T}$ de acuerdo con
\[ \widetilde{a}_{ij} = \sum_{k = 1}^{n} \sum_{m = 1}^{j} \ell_{ik}^{\prime} \; a_{km} \; \ell_{jm}^{\prime} \]
\item Resolver el problema del valor propio estándar $\mathbf{\widetilde{A}} \cdot \mathbf{\widetilde{x}} = \lambda \mathbf{\widetilde{x}}$ mediante un método adecuado para matrices simétricas (tales como el método de Jacobi).
\item Recalcular los valores propios $\mathbf{x} = (\mathbf{L}^{-1})^{T} \cdot \mathbf{\widetilde{x}}$ del problema inicial de valores propios, a partir de 
\begin{equation}
x_{ij} = \sum_{k = i}^{n} \ell_{ki}^{\prime} \; \widetilde{x}_{kj}
\label{eq:ecuacion_08_37}
\end{equation}
\end{enumerate}
\end{frame}
\subsection{Problema 1}
\begin{frame}
\frametitle{Problema 1}
Utiliza la el método de Jacobi para probar que una matriz simétrica invertible aleatoria $\mathbf{A}$ de orden $n = 100$ cumple la propiedad general de que los valores propios de la matriz inversa $\mathbf{A}^{-1}$ son las inversas $1 / \lambda_{i}$ de los valores propios $\lambda_{i}$ de la matriz original, mientras que los vectores propios coinciden.
\end{frame}
\begin{frame}
Invertir la matriz $\mathbf{A}$ utilizando la rutina \texttt{MatInv} del módulo \texttt{linsys.py}, y empleando la función \texttt{EigSort}, clasificar los autovectores de $\mathbf{A}$ y $\mathbf{A}^{-1}$ por los valores propios correspondientes antes de compararlos.
\end{frame}
\begin{frame}
\frametitle{Trabajemos el código}
Presentamos una propuesta de solución, construimos inicialmente los elementos que vamos a ocupar:
\setbeamercolor{item projected}{bg=red!70!black,fg=white}
\setbeamertemplate{enumerate items}[circle]
\begin{enumerate}[<+->]
\item $n$ es el orden de la matriz.
\item $a$ es la matriz de coeficientes.
\item $a1$ es la inversa de la matriz $a$.
\item $x$ son los vectores propios de $a$.
\item $x1$ son los vectores propios de $a1$.
\item $d$ son los valores propios de  $a$ y $a1$.
\end{enumerate}
\end{frame}
\begin{frame}[fragile]
\begin{lstlisting}[basicstyle=\linespread{0.9}\ttfamily\normalsize, columns=fullflexible]
n = 100

a  = [[0]*(n+1) for i in range(n+1)]

a1 = [[0]*(n+1) for i in range(n+1)]

x  = [[0]*(n+1) for i in range(n+1)]

x1 = [[0]*(n+1) for i in range(n+1)]

d = [0]*(n+1); d1 = [0]*(n+1)  
\end{lstlisting}
\end{frame}
\begin{frame}
\frametitle{¿Cómo generar números aleatorios}
Podemos mencionar propiamente que no hay algoritmos que generen números puramente aleatorios y que sean deterministas, por lo que usando solamente software es imposible obtener números verdaderamente aleatorios.
\end{frame}
\begin{frame}
Para generar números aleatorios en una computadora se necesita muestrear algún proceso físico real: por ejemplo, en el sitio \url{www.random.org} utilizan ruido atmosférico, y en \url{www.fourmilab.ch/hotbits/} desintegración radiactiva.
\end{frame}
\begin{frame}
En computación realmente se dispone de números \emph{pseudoaleatorios}, que son secuencias determinadas a partir de unos ciertos datos iniciales que se parecen bastante a una secuencia aleatoria.
\\
\bigskip
\texttt{NumPy} utiliza un algoritmo llamado \enquote{Mersenne twister}, creado por dos matemáticos japoneses y que utilizan otros programas como \texttt{MATLAB}.
\end{frame}
\begin{frame}
En la librería estándar de \python viene ya incluido el módulo \texttt{random}, con funciones pensadas para trabajar con valores escalares y listas.
\end{frame}
\begin{frame}
A veces nos puede interesar, por ejemplo para pruebas, utilizar siempre una misma secuencia pseudoaleatoria.
\\
\bigskip
Utilizando la función \texttt{random.seed()} se establecen las condiciones iniciales del generador; si no se pasa algún argumento, \texttt{NumPy} escoge la hora del sistema como semilla, de tal forma que cada programa utilice secuencias diferentes.
\end{frame}
\begin{frame}[fragile]
\frametitle{Ejecuta el código con \texttt{Jupyter QtConsole}}
\begin{lstlisting}[basicstyle=\linespread{0.9}\ttfamily\normalsize, columns=fullflexible]
>>>from numpy import random

>>>random.seed(0)

>>>random.rand()
>>> 0.5488135039273248

>>>random.rand(3)
>>>array([ 0.71518937,  0.60276338,  0.54488318])
#devuelve un arreglo de tres numeros aleatorios
\end{lstlisting}
\end{frame}
\begin{frame}[fragile]
\frametitle{Ejecuta el código con \texttt{Jupyter QtConsole}}
\begin{lstlisting}[basicstyle=\linespread{0.9}\ttfamily\normalsize, columns=fullflexible]
>>>random.seed(3)
#cambiamos la semilla del generador

>>>random.rand()
>>>0.5507979025745755

>>>random.rand(3)
>>>array([ 0.70814782,  0.29090474,  0.51082761])
\end{lstlisting}
\end{frame}
\begin{frame}[fragile]
\frametitle{Ejecuta el código con \texttt{Jupyter QtConsole}}
\begin{lstlisting}[basicstyle=\linespread{0.9}\ttfamily\normalsize, columns=fullflexible]
>>>random.seed(0)
#cambiamos la semilla con el valor inicial

>>>random.rand()
>>>0.5488135039273248
#Se obtiene el mismo numero aleatorio

>>>random.rand(3)
>>>array([ 0.71518937,  0.60276338,  0.54488318])
#Se obtiene el mismo arreglo de tres numeros aleatorios
\end{lstlisting}
\end{frame}
\begin{frame}
\frametitle{Regresamos con el código}
Una vez que se revisó el proceso para la generación de números aleatorios, ahora retomamos el código en el punto donde se genera una matriz de $100 \times 100$ cuyos elementos son números aleatorios.
\end{frame}
\begin{frame}[fragile]
Se genera la matriz de $100 \times 100$ de números aleatorios:
\begin{lstlisting}[basicstyle=\linespread{0.9}\ttfamily\normalsize, columns=fullflexible]
random.seed()

for i in range(1,n+1):
   for j in range(1,i+1):
      a[i][j] = a[j][i] = random.random()
      a1[i][j] = a1[j][i] = a[i][j]
\end{lstlisting}
\end{frame}
\begin{frame}[fragile]
\begin{lstlisting}[basicstyle=\linespread{0.9}\ttfamily\normalsize, columns=fullflexible]
# resuelve el problema de valores propios para a
Jacobi(a,x,d,n)

# ordena los valores y vectores propios
EigSort(x,d,n,1)

# calcula la inversa a1 = a^(-1)
MatInv(a1,n)

# resuelve el problema de valores propios para a1
Jacobi(a1,x1,d1,n)

# inverte  los valores propios antes de ordenarlos
for i in range(1,n+1): d1[i] = 1./d1[i]

# ordena los valores propios de a1
EigSort(x1,d1,n,1)
\end{lstlisting}
\end{frame}
\begin{frame}[fragile]
\begin{lstlisting}[basicstyle=\linespread{0.9}\ttfamily\normalsize, columns=fullflexible]
# se hace un bucle sobre los vectores propios
# se revisa el signo y se ajusta para que lo mantenga

for j in range(1,n+1):
   if (x[1][j] * x1[1][j] < 0):
      for i in range(1,n+1): x1[i][j] = -x1[i][j] 
\end{lstlisting}
\end{frame}
\begin{frame}[fragile]
\begin{lstlisting}[basicstyle=\linespread{0.9}\ttfamily\normalsize, columns=fullflexible]
# se calcula la diferencia de valores propios en d1
VecDiff(d,d1,d1,n)

#se calcula la norma de los valores propios
norm = VecNorm(d1,n)

print("Diferencia de la norma de los valores propios = {0:8.2e}".format(norm))
\end{lstlisting}
\end{frame}
\begin{frame}[fragile]
\begin{lstlisting}[basicstyle=\linespread{0.9}\ttfamily\normalsize, columns=fullflexible]
# se calcula la diferencia de vectores propios en x1

MatDiff(x,x1,x1,n,n)

#calcula la norma de los vectores propios
norm = MatNorm(x1,n,n)

print("Diferencia de la norma de los vectores propios = {0:8.2e}".format(norm))
\end{lstlisting}
\end{frame}
\begin{frame}
\frametitle{Resultado}
Al ejecutar el código, se obtiene el siguiente resultado:
\\
\bigskip
\texttt{Diferencia de la norma de los valores propios = 7.00e-12
\\
\medskip
Diferencia de la norma de los vectores propios = 9.51e-13}
\end{frame}
\begin{frame}
\frametitle{Conclusión}
Con lo que se demuestra que se cumplen la propiedades generales:
\setbeamercolor{item projected}{bg=red!70!black,fg=white}
\setbeamertemplate{enumerate items}[circle]
\begin{enumerate}[<+->]
\item Los valores propios de la matriz inversa $\mathbf{A}^{-1}$ son las inversas $1 / \lambda_{i}$ de los valores propios $\lambda_{i}$ de la matriz original.
\item Los valores de los vectores propios de las dos matrices coinciden.
\end{enumerate}
\end{frame}
%\begin{frame}
%\frametitle{Esquema de operación de los módulos}.
%Para la solución del ejercicio, fue necesario ocupar diferentes módulos que resolvieron tareas específicas del problema.
%\\
%\bigskip
%Recuerda que es práctico contar con una modularidad dentro con \ptyhon.
%\end{frame}
%\begin{frame}
%\frametitle{Esquema de operación de los módulos}.
%\begin{figure}
%	\centering
%	\includestandalone{}
%\end{figure}
%\end{frame}
\subsection{Problema 2}
\begin{frame}
\frametitle{Problema 2}
Considera una cuerda elástica sin masa, cargada con $n$ masas idénticas $m$ que están distribuidas de manera equidistante  bajo una tensión constante $T$.
\\
\bigskip
Suponemos que hay pequeños desplazamientos transversales de la cuerda.
\end{frame}
\begin{frame}
\frametitle{A resolver}
\begin{enumerate}
\item Determina las frecuencias y desplazamientos para sus modos normales de vibración. 
\item Comprueba que las frecuencias propias adimensionales son múltiplos de $\pi$.
\end{enumerate}
Considera, en particular, $n = 100$ y $dm / T = 1$.
\end{frame}
\begin{frame}
\frametitle{Solución}
Hagamos $u_{i} = u_{i}(t)$ con $i = 1, 2, \ldots, n$ los desplazamientos transversales instantáneos de los puntos.
\\
\bigskip
La fuerza ejercida por la cuerda en el punto de masa $i$, puede aproximarse por
\[ F_{i} = T \; \dfrac{u_{i-1} - 2 \; u_{i} + u_{i+1}}{d} \]
\end{frame}
\begin{frame}
La respectiva ecuación de movimiento es
\[ m \ddot{u}_{i}  = \left( \dfrac{T}{d} \right) (u_{i-1} - 2 \; u_{i} + u_{i+1}), \hspace{1cm} i = 2, \ldots, n-1  \]
\pause
Este conjunto de ecuaciones acopladas debe complementarse con condiciones de frontera, que en el caso más simple de una cuerda fija en ambos extremos:
\[ u_{1} = u_{n} = 0 \]
\end{frame}
\begin{frame}
Los modos normales de la cuerda se caracterizan por la vibración en fase de todos los puntos con una frecuencia angular común $\omega$.
\\
\bigskip
En consecuencia, la expresión para el movimiento para cada punto tendrá la forma:
\[ u_{i} = x_{i} \; \exp(j \; \omega \; t) \]
\end{frame}
\begin{frame}
\[ u_{i} = x_{i} \; \exp(j \; \omega \; t) \]
Donde $x_{i}$ es el desplazamiento transversal correspondiente (amplitud) y $j$ es la unidad imaginaria.
\\
\bigskip
Las ecuaciones de movimiento discretizadas junto con las condiciones de frontera finalmente llevan al siguiente problema de valor propio con matriz tridiagonal:
\end{frame}
\begin{frame}
\frametitle{Matriz tridiagonal}
\[ \begin{bmatrix}
1 & 0  &    &    & & & 0 \\
0 & 2  & -1 &    & & &   \\
  & -1 & 2  & -1 & & &   \\
  &  & \ddots & \ddots & \ddots & &   \\
  &  & -1  & 2 & -1 & &   \\
  &  &  &  & -1 & 2 & 0   \\
0 &  &  &  & &  0 & 1   \\
  
\end{bmatrix}
\begin{bmatrix}
x_{1} \\
x_{2} \\
x_{3} \\
\vdots \\
x_{n-2} \\
x_{n-1} \\
x_{n} \\
\end{bmatrix} = \lambda \;
\begin{bmatrix}
x_{1} \\
x_{2} \\
x_{3} \\
\vdots \\
x_{n-2} \\
x_{n-1} \\
x_{n}
\end{bmatrix} \]
\end{frame}
\begin{frame}
Donde el valor propio $\lambda$ relaciona la frecuencia angular $\omega$ por
\[ \lambda = \left( \dfrac{d m}{T} \right) \; \omega^{2}  \]
\end{frame}
\begin{frame}[fragile]
Necesitaremos las librerías para resolver el problema, se introduce una librería que genera la interfaz gráfica mediante una ventana.
\begin{lstlisting}[basicstyle=\linespread{0.9}\ttfamily\normalsize, columns=fullflexible]
from eigsys import Jacobi, EigSort
from graphlib import GraphInit, Plot, MainLoop
from math import sqrt
\end{lstlisting}
\end{frame}
\begin{frame}
\frametitle{Interfaces gráficas de usuario con \texttt{Tk}}
La librería \texttt{tkinter} nos ayudará con la creación de GUI (Interfaces gráficas de usuario) en \python.
\\
\bigskip
Ya se encuentra instalada por defecto por lo que no debemos preocuparnos por descargarla e instalarla, es una librería multiplataforma lo que permitirá ejecutar nuestros scripts en diversos sistemas como Windows, Linux o Mac.
\end{frame}
\begin{frame}[fragile]
\begin{lstlisting}[basicstyle=\linespread{0.9}\ttfamily\normalsize, columns=fullflexible]
# se define el numero de puntos
n = 100

# matriz de coeficientes
a = [[0]*(n+1) for i in range(n+1)]

# desplazamientos = vectores propios
x = [[0]*(n+1) for i in range(n+1)]

# frecuecias^2 -- valores propios
d = [0]*(n+1)

# puntos para graficar
xp = [0]*(n+1) # mesh points for plotting

# valores de la funcion para graficar
yp = [0]*(n+1)
\end{lstlisting}
\end{frame}
\begin{frame}[fragile]
Se define la matriz de coeficientes
\\
\medskip
\begin{lstlisting}[basicstyle=\linespread{0.9}\ttfamily\normalsize, columns=fullflexible]
for i in range(1,n+1): a[i][i] = 2.0

for i in range(2,n-1): a[i+1][i] = -1.0; a[i][i+1] = -1.0
\end{lstlisting}
\end{frame}
\begin{frame}[fragile]
Se utiliza la función \texttt{Jacobi} para resolver la matriz y luego se ordenan los valores propios.
\\
\medskip
\begin{lstlisting}[basicstyle=\linespread{0.9}\ttfamily\normalsize, columns=fullflexible]
Jacobi(a, x, d, n)

EigSort(x, d, n, 1)
\end{lstlisting}
\end{frame}
\begin{frame}
\frametitle{Construcción de la gráfica}
Con las funciones contenidas en el módulo \texttt{graphlib}, se construye la ventana en donde se colocarán las cuatro gráficas, una por cada modo normal de oscilación de la cuerda.
\end{frame}
\begin{frame}[fragile]
\begin{lstlisting}[basicstyle=\linespread{0.9}\ttfamily\normalsize, columns=fullflexible]
# se establece el tamano de la ventana grafica
GraphInit(1200,600)

# se define el espacio entre los puntos
h = 1.0/(n-1)

for i in range(1,n+1): xp[i] = (i-1) * h

# se grafica el primer modo de oscilacion
mode = 1

# se calcula la frecuencia
omega = sqrt(d[mode])*(n-1)
\end{lstlisting}
\end{frame}
\begin{frame}[fragile]
\begin{lstlisting}[basicstyle=\linespread{0.9}\ttfamily\normalsize, columns=fullflexible]
# se calculan los desplazamientos
for i in range(1,n+1): yp[i] = x[i][mode]

title = "omega({0:d}) = {1:6.2f}".format(mode,omega)
Plot(xp,yp,n,"blue",3,0.10,0.45,0.60,0.90,"x","y",title)
\end{lstlisting}
\end{frame}
\begin{frame}
\frametitle{Solución gráfica}
\begin{figure}
	\centering
	\includegraphics[scale=0.25]{cuerdaOscilante.eps}
\end{figure}
\end{frame}
\begin{frame}
El uso de librerías o módulos para crear interfases requiere considerar que hay tareas que deben de incluirse, como por ejemplo guardar la ventana en un archivo gráfico.
\\
\bigskip
Con \texttt{matplotlib} tenemos la oportunidad de contar con un menú de opciones, con \texttt{tkinter}, se puede implementar, pero hay que revisar la respectiva documentación.
\end{frame}
\end{document}