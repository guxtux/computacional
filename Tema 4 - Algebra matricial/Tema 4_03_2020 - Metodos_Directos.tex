\documentclass[12pt]{beamer}
\newenvironment{ConCodigo}[1]
  {\begin{frame}[fragile,environment=ConCodigo]{#1}}
  {\end{frame}}
\graphicspath{{Imagenes/}{../Imagenes/}}
\usepackage[utf8]{inputenc}
\usepackage[spanish]{babel}
\usepackage{hyperref}
\usepackage{etex}
%\reserveinserts{28}
\usepackage{amsmath}
\usepackage{amsthm}
\usepackage{mathtools}
\usepackage{multicol}
\usepackage{multirow}
\usepackage{tabulary}
\usepackage{booktabs}
\usepackage{nccmath}
\usepackage{physics}
\usepackage{biblatex}
\usepackage[outdir=./]{epstopdf}
%\epstopdfsetup{outdir=./}
\usepackage{graphicx}
%\usepackage{enumitem,xcolor}
\usepackage{siunitx}
%\sisetup{scientific-notation=true}
%\usepackage{fontspec}
\usepackage{lmodern}
\usepackage{float}
\usepackage[format=hang, font=footnotesize, labelformat=parens]{caption}
\usepackage[autostyle,spanish=mexican]{csquotes}
\usepackage{standalone}
\usepackage{blkarray}
\usepackage{algorithm}
\usepackage{algorithmic}
\usepackage{tikz}
\usepackage[siunitx, RPvoltages]{circuitikz}
\usetikzlibrary{arrows,patterns,shapes}
\usetikzlibrary{decorations.markings}
\usetikzlibrary{arrows}
\usepackage{color}
\usepackage{xcolor}
%\usepackage{beton}
%\usepackage{euler}
%\usepackage[T1]{fontenc}
\usepackage[sfdefault]{roboto}  %% Option 'sfdefault' only if the base font of the document is to be sans serif
\usepackage[T1]{fontenc}
\renewcommand*\familydefault{\sfdefault}
\DeclareGraphicsExtensions{.pdf,.png,.jpg}
\usepackage{hyperref}
\renewcommand {\arraystretch}{1.5}
\newcommand{\python}{\texttt{python}}
\usefonttheme[onlymath]{serif}
\setbeamertemplate{navigation symbols}{}
\usetikzlibrary{patterns}
\usetikzlibrary{decorations.markings}
\tikzstyle{every picture}+=[remember picture,baseline]
%\tikzstyle{every node}+=[inner sep=0pt,anchor=base,
%minimum width=2.2cm,align=center,text depth=.15ex,outer sep=1.5pt]
%\tikzstyle{every path}+=[thick, rounded corners]
\setbeamertemplate{caption}[numbered]
\newcommand{\ptm}{\fontfamily{ptm}\selectfont}
%Se usa la plantilla Warsaw modificada con spruce
\mode<presentation>
{
  \usetheme{Warsaw}
  \setbeamertemplate{headline}{}
  \useoutertheme{default}
  \usecolortheme{albatross}
  \setbeamercovered{invisible}
}
% \AtBeginSection[]
% {
% \begin{frame}<beamer>{Contenido}
% \normalfont\mdseries
% \tableofcontents[currentsection]
% \end{frame}
% }

%Se usa la plantilla Warsaw modificada con spruce
\mode<presentation>
{
  \usetheme{Warsaw}
  \setbeamertemplate{headline}{}
  %\useoutertheme{infolines}
  \usecolortheme{crane}
  \setbeamercovered{invisible}
  

\setbeamertemplate{section in toc}[sections numbered]
\setbeamertemplate{subsection in toc}[subsections numbered]
\setbeamertemplate{subsection in toc}{\leavevmode\leftskip=3.2em\rlap{\hskip-2em\inserttocsectionnumber.\inserttocsubsectionnumber}\inserttocsubsection\par}
\setbeamercolor{section in toc}{fg=blue}
\setbeamercolor{subsection in toc}{fg=blue}
\setbeamerfont{subsection in toc}{size=\small}



\setbeamertemplate{navigation symbols}{}
\setbeamertemplate{caption}[numbered]
% \setbeamercolor{frametitle}{fg=yellow,bg=blue!70!white}
%\setbeamercolor{section in head}{bg=green,fg=white}
%\setbeamercolor{subsection in head/foot}{bg=gray!30,fg=black}
%\setbeamercolor{author in head/foot}{fg=yellow}
%\setbeamercolor{date in head/foot}{fg=blue}

%\mode<presentation>
%{
%  \usetheme{Warsaw}
%  \setbeamertemplate{headline}{}
%  %\useoutertheme{infolines}
%  \useoutertheme{default}
%  \setbeamercovered{invisible}
%  % or whatever (possibly just delete it)
%}
}

\input{../Preambulos/pre_codigo}
\makeatletter
%\setbeamercolor{section in foot}{bg=gray!30, fg=black!90!orange}
%\setbeamercolor{subsection in foot}{bg=blue!30!yellow, fg=red}
\setbeamertemplate{footline}
{
  \leavevmode%
  \hbox{%
  \begin{beamercolorbox}[wd=.333333\paperwidth,ht=2.25ex,dp=1ex,center]{author in foot}%
    \usebeamerfont{author in head/foot} \textcolor{red}{\insertsection}
  \end{beamercolorbox}}%
  \begin{beamercolorbox}[wd=.333333\paperwidth,ht=2.25ex,dp=1ex,center]{title in foot}%
    \usebeamerfont{title in head/foot}\insertsubsection
  \end{beamercolorbox}%
  \begin{beamercolorbox}[wd=.333333\paperwidth,ht=2.25ex,dp=1ex,right]{date in head/foot}%
    \usebeamerfont{date in head/foot} \insertshortdate{} \hspace*{2em}
    \insertframenumber{} / \inserttotalframenumber \hspace*{2ex} 
  \end{beamercolorbox}}%
  \vskip0pt%
\makeatother	
\setbeamercolor{block title example}{use=structure, fg=white}
\normalfont
\usepackage{ccfonts}% http://ctan.org/pkg/{ccfonts}
\usepackage[T1]{fontenc}% http://ctan.or/pkg/fontenc
\renewcommand{\rmdefault}{cmr}% cmr = Computer Modern Roman
\linespread{1.3}
\title{Métodos directos para matrices}
\subtitle{Curso de Física Computacional}
\author{M. en C. Gustavo Contreras Mayén}
\date{}
\institute{Facultad de Ciencias - UNAM}
\titlegraphic{\includegraphics[width=1.75cm]{Imagenes/escudo-facultad-ciencias}\hspace*{4.75cm}~%
   \includegraphics[width=1.75cm]{Imagenes/escudo-unam}
}
\begin{document}
\maketitle
\fontsize{14}{14}\selectfont
\spanishdecimal{.}
\section*{Contenido}
\frame[allowframebreaks]{\tableofcontents[currentsection, hideallsubsections]}
\section{Método de Gauss}
\frame{\tableofcontents[currentsection, hideothersubsections]}
\subsection{Definición}
\begin{frame}
\frametitle{El método de Gauss}
El \funcionazul{método de eliminación de Gauss} es el método más conocido para resolver sistemas de ecuaciones simultáneas.
\\
\bigskip
Consiste en dos etapas:
\setbeamercolor{item projected}{bg=blue!70!black,fg=yellow}
\setbeamertemplate{enumerate items}[circle]
\begin{enumerate}[<+->]
\item Fase de eliminación.
\item Fase de sustitución.
\end{enumerate}
\end{frame}
\begin{frame}
\frametitle{El método de Gauss}
La función de la fase de eliminación es transformar las ecuaciones a la forma $U \, x = c$.
\\
\bigskip
Las ecuaciones obtenidas son resueltas por sustitución hacia atrás.
\end{frame}
\begin{frame}
\frametitle{El método de Gauss}
Con el fin de ilustrar el procedimiento, vamos a resolver el siguiente sistema:
\begin{align}
4 \, x_{1} - 2 \, x_{2} + x_{3} &= 11 \label{eq:ecuacion_a} \\
-2 \, x_{1} + 4 \, x_{2} - 2 \, x_{3} &= -16 \label{eq:ecuacion_b} \\
x_{1} - 2 \, x_{2} + 4 \, x_{3} &= 17 \label{eq:ecuacion_c}
\end{align}
\end{frame}
\subsection{Fase de eliminación}
\begin{frame}
\frametitle{Fase de eliminación}
La fase de eliminación \emph{utiliza solamente una de las operaciones elementales}: multiplicar una ecuación (por ejemplo, ecuación $j$) por una constante $\lambda$ y restarla de otra ecuación (ecuación $i$).
\end{frame}
\begin{frame}
\frametitle{Fase de eliminación}
La representación simbólica de esta operación es:
\begin{align*}
Eq.(i) \leftarrow Eq.(i) - \lambda \times Eq.(j) 
\end{align*}
La ecuación que se resta, a saber, la $Eq.(j)$, se llama \emph{ecuación pivote}.
\end{frame}
\begin{frame}
\frametitle{Inicia la fase de eliminación}
Comenzamos la eliminación eligiendo la ec. (\ref{eq:ecuacion_a}) como ecuación pivote y eligiendo los multiplicadores $\lambda$ a fin de eliminar $x_{1}$ en las ecs. (\ref{eq:ecuacion_b}) y (\ref{eq:ecuacion_c}):
\end{frame}
\begin{frame}
\frametitle{Inicia la fase de eliminación}
Las operaciones son entonces:
\begin{align*}
Eq.(\ref{eq:ecuacion_b}) \leftarrow& Eq.(\ref{eq:ecuacion_b}) - (-0.5) \times Eq.(\ref{eq:ecuacion_a}) \\
Eq.(\ref{eq:ecuacion_c}) \leftarrow& Eq.(\ref{eq:ecuacion_c}) - 0.25 \times Eq.(\ref{eq:ecuacion_a})
\end{align*}
\end{frame}
\begin{frame}
\frametitle{Inicia la fase de eliminación}
Después de esta primera transformación, el sistema de ecuaciones queda:
\begin{align}
4 \ , x_{1} - 2 \, x_{2} + x_{3} &= 11 \label{eq:ecuacion_d} \\
3 \, x_{2} - 1.5 \, x_{3} &= -10.5 \label{eq:ecuacion_e} \\
-1.5 \, x_{2} + 3.75 \, x_{3} &= 14.25 \label{eq:ecuacion_f}
\end{align}
Esto completa el primer paso.
\end{frame}
\begin{frame}
\frametitle{Continua la fase de eliminación}
Ahora elegimos ec. (\ref{eq:ecuacion_e}) como ecuación pivote y eliminamos $x_{2}$ de la ec. (\ref{eq:ecuacion_f}):
\begin{align*}
Eq.(\ref{eq:ecuacion_f}) \leftarrow Eq.(\ref{eq:ecuacion_f}) - (-0.5) \times Eq.(\ref{eq:ecuacion_e})
\end{align*}
que nos devuelve:
\end{frame}
\begin{frame}
\frametitle{Continua la fase de eliminación}
Que nos devuelve:
\begin{align}
4 \, x_{1} - 2 \, x_{2} + x_{3} &= 11 \label{eq:ecuacion_g} \\
3 \, x_{2} - 1.5 \, x_{3} &= -10.5 \label{eq:ecuacion_h} \\
3 \, x_{3} &= 9 \label{eq:ecuacion_i}
\end{align}
Tenemos que \emph{la fase de eliminación está completa}.
\end{frame}
\begin{frame}
\frametitle{Fase de eliminación completa}
Las ecuaciones iniciales del sistema fueron reemplazadas por un conjunto de ecuaciones equivalentes que pueden resolverse fácilmente por sustitución hacia atrás.
\end{frame}
\begin{frame}
\frametitle{Uso de la matriz aumentada}
Como se ha señalado antes, la matriz de coeficientes aumentada es un instrumento más conveniente para realizar los cálculos.
\\
\bigskip
\pause
Así, las ecuaciones originales se escribirían matricialmente como
\begin{align*}
\begin{bmatrix}[ccc|c]
4 & -2 & 1 & 11 \\
-2 & 4 & -2 & -16 \\
1 & -2 & 4 & 17
\end{bmatrix}
\end{align*}
\end{frame}
\begin{frame}
\frametitle{Matrices equivalentes}
Las ecuaciones equivalentes obtenidas por el primer paso de la eliminación de Gauss, quedarían como:
\begin{align*}
\begin{bmatrix}[ccc|c]
4 & -2 & 1 & 11.00 \\
0 & 3 & -1.5 & -10.50 \\
0 & -1.5 & 3.75 & 14.25
\end{bmatrix}
\end{align*}
\end{frame}
\begin{frame}
\frametitle{Matrices equivalentes}
Las ecuaciones equivalentes para el segundo paso de la eliminación de Gauss, quedarían como:
\begin{align*}
\begin{bmatrix}[ccc|c]
4 & -2 & 1 & 11.00 \\
0 & 3 & -1.5 & -10.50 \\
0 & 0 & 3 & 9.00    
\end{bmatrix}
\end{align*}
\end{frame}
\subsection{Fase de sustitución hacia atrás}
\begin{frame}
\frametitle{Fase de sustitución hacia atrás}
Las incógnitas ahora se calculan por sustitución hacia atrás.
\\
\bigskip
\pause
Resolviendo las ecs. (\ref{eq:ecuacion_i}), (\ref{eq:ecuacion_h}) y (\ref{eq:ecuacion_g}) en ese orden, se obtiene la solución:
\begin{align*}
x_{3} &= 9/3 = 3 \\
x_{2} &= (-10.5 + 1.5 \, x-{3})/3 = -2 \\
x_{1} &= (11 + 2 \, x_{2} - x_{3})/4 = 1
\end{align*}
\end{frame}
\subsection{Algoritmo para el método de Gauss}
\begin{frame}
\frametitle{Algoritmo general}
El siguiente paso es considerar una matriz cuadrada de $n \cp n$ a la que la incorporamos el respectivo vector $\vb{b}$ de términos constantes:
\end{frame}
\begin{frame}[fragile]
\frametitle{Caso general}
\fontsize{10}{10}\selectfont
\begin{align*}
\begin{bmatrix}[ccccccccc|c]
A_{11} & A_{12} & A_{13} & \ldots & A_{1k} & \ldots & A_{ij} & \ldots & A_{1n} & b_{1} \\
0      & A_{22} & A_{23} & \ldots & A_{2k} & \ldots & A_{2j} & \ldots & A_{2n} & b_{2} \\
0      & 0      & A_{33} & \ldots & A_{3k} & \ldots & A_{3j} & \ldots & A_{3n} & b_{3} \\
\vdots & \vdots & \vdots & \vdots & \vdots & \vdots & \vdots & \vdots & \vdots & \vdots \\
0      & 0      & 0      & A_{kk} & \ldots & \ldots & A_{kj} & \ldots & A_{kn} & b_{k} \\
\vdots & \vdots & \vdots & \vdots & \vdots & \vdots & \vdots & \vdots & \vdots & \vdots \\
0      & 0      & 0      & A_{ik} & \ldots & \ldots & A_{ij} & \ldots & A_{in} & b_{i} \\
\vdots & \vdots & \vdots & \vdots & \vdots & \vdots & \vdots & \vdots & \vdots & \vdots \\
0      & 0      & 0      & A_{nk} & \ldots & \ldots & A_{nj} & \ldots & A_{nn} & b_{n} \\
\end{bmatrix}
\end{align*}
\end{frame}
\begin{frame}
\frametitle{Algoritmo general}
El procedimiento para construir el algoritmo es esencialmente el mismo que vimos en el ejemplo de la matriz de $3 \cp 3$.
\\
\bigskip
\pause
Con la finalidad de avanzar en el tema, la función \azulfuerte{scipy.linalg.solve} nos resuelve un sistema del tipo $\mathbf{A} \, x =  \mathbf{b}$.
\end{frame}
\subsection{La función \texttt{sla.solve}}
\begin{frame}
\frametitle{Uso de la función \texttt{sla.solve}}
La función \funcionazul{sla.solve} opera de manera similar que en el paquete \funcionazul{numpy,linalg}.
\\
\bigskip
Con el fin de unificar el criterio de uso con \python, estaremos utilizando \funcionazul{scipy} para resolver los sistemas de ecuaciones. 
\end{frame}
\begin{frame}
\frametitle{Parámetros de \texttt{sla.solve}}
La función \funcionazul{sla.solve} resuelve un sistema de ecuaciones $a * x = b$ para la incógnita $x$ de una matriz cuadrada.
\\
\bigskip
Los parámetros mínimos para la función \funcionazul{sla.solve} son:
\end{frame}
\begin{frame}[fragile]
\frametitle{Parámetros de \texttt{sla.solve}}
\verb|sla.solve(a, b, assume_a='gen')|
\\
\medskip
donde
\\
\textbf{a} : Es un arreglo de $N \cp N$. Matriz cuadrada.
\\
\textbf{b} : Es un arreglo de $N$. Vector del lado derecho de la expresión.
\\
\textbf{asume\_a} : cadena, opcional. Si se sabe que la matriz es de un tipo en particular, se 
sustituye por las siguientes opciones.
\end{frame}
\begin{frame}
\frametitle{Opciones para el tipo de matriz}
\begin{table}
\begin{tabular}{l l}
Tipo Matriz & Argumento \\ \hline
matriz genérica & 'gen' \\
simétrica & 'sym' \\
hermitiana & 'her' \\
definida positiva & 'pos'
\end{tabular}
\end{table}
Si se omite el parámetro, el valor por defecto es 'gen'
\end{frame}
\begin{frame}
\frametitle{Número de operaciones}
El tiempo de ejecución de un algoritmo depende en gran medida del número de operaciones  (multiplicaciones y divisiones) realizadas.
\\
\bigskip
Se puede demostrar que la \funcionazul{eliminación de Gauss} ocupa aproximadamente $n^{3}/3$ de tales operaciones ($n$ es el número de ecuaciones) en la fase de eliminación, y $n^{2}/2$ operaciones en sustitución inversa.
\end{frame}
\begin{frame}
\frametitle{Número de operaciones}
Estos números muestran que la mayor parte del tiempo de cálculo está en la fase de eliminación.
\\
\bigskip
Además, el tiempo aumenta muy rápidamente con el número de ecuaciones, por lo que hay que tener cuidado con sistemas grandes.
\end{frame}
\section{Métodos de descomposición LU}
\frame{\tableofcontents[currentsection, hideothersubsections]}
\subsection{Definición}
\begin{frame}
\frametitle{Métodos de descomposición \textbf{LU}}
Se puede demostrar que cualquier matriz cuadrada $\mathbf{A}$ se puede expresar como un producto de una matriz triangular inferior $\mathbf{L}$ y una matriz triangular superior $\mathbf{U}$:
\begin{align*}
\mathbf{A = L \: U}
\end{align*}
\end{frame}
\begin{frame}
\frametitle{Factorización \textbf{LU}}
El proceso de calcular $\mathbf{L}$ y $\mathbf{U}$ para un determinada matriz $\mathbf{A}$, se conoce como \funcionazul{descomposición LU} o \funcionazul{factorización LU}.
\end{frame}
\begin{frame}
\frametitle{Descomposición $LU$}
La descomposición $\mathbf{LU}$ no es única (las combinaciones de $\mathbf{L}$ y $\mathbf{U}$ para una determinada matriz A son infinitas), salvo ciertas restricciones de $\mathbf{L}$ o $\mathbf{U}$. 
\\
\bigskip
Estas limitaciones distinguen un tipo de descomposición de otro.
\end{frame}
\subsection{Métodos más utilizados}
\begin{frame}
\frametitle{Métodos $LU$ más utilizados}
Los tres métodos de descomposición $LU$ más utilizados son:
\begin{table}
\begin{tabular}{l | l}
    Nombre & Restricciones \\ \hline
    Doolittle & $L_{ii} = 1, \hspace{0.3cm} i= 1,2,\ldots,n$ \\
    Crout & $U_{ii} = 1, \hspace{0.3cm} i=1,2,\ldots,n$ \\
    Choleski & $L = U^{T}$
\end{tabular}
\end{table}
\end{frame}
\begin{frame}
\frametitle{Versatilidad de los métodos}
Después de descomponer la matriz $\mathbf{A}$, se facilita resolver las ecuaciones $A \: x = b$. 
\\
\bigskip
En primer lugar, volvemos a escribir las ecuaciones como $L \: U \: x = b$.
\end{frame}
\begin{frame}
\frametitle{Versatilidad de los métodos}
Al usar la notación $U \: x = y$, las ecuaciones quedan como
\begin{align*}
\mathbf{L \: y = b}
\end{align*}
que se puede resolver para $y$ por sustitución hacia delante. 
\end{frame}
\begin{frame}
\frametitle{Versatilidad de los métodos}
Entonces
\begin{align*}
\mathbf{U \: x = y}
\end{align*}
nos devolverá la solución $x$ con el proceso de sustitución hacia atrás.
\end{frame}
\begin{frame}
\frametitle{Versatilidad de los métodos}
La ventaja de la descomposición $LU$ sobre el método de eliminación de Gauss es que una vez descompuesta, podemos resolver $A \, x = b$ para muchos vectores constantes $b$.
\end{frame}
\begin{frame}
\frametitle{Versatilidad de los métodos}
El costo de cada solución adicional es relativamente pequeño, ya que el avance y las operaciones de sustitución están consumiendo mucho menos tiempo que el proceso de descomposición.
\end{frame}
\section{Descomposición de Doolittle}
\frame{\tableofcontents[currentsection, hideothersubsections]}
\subsection{Descripción del método}
\begin{frame}
\frametitle{Descomposición de Doolittle}
El método de descomposición de Doolittle está estrechamente relacionado con el proceso de eliminación de Gauss.
\\
\bigskip
Veamos el método con un ejemplo:
\end{frame}
\subsection{Fase de eliminación}
\begin{frame}[fragile]
\frametitle{Fase de eliminación}
Considera una matriz $A$ de $3 \cp 3$ y suponganmos que existen las matrices triangulares:
\begin{align*}
\mathbf{L} =
\begin{bmatrix}
1 & 0 & 0 \\
L_{21} & 1 & 0 \\
L_{31} & L_{32} & 1
\end{bmatrix}
\hspace{1.5cm} U =
\begin{bmatrix}
U_{11} & U_{12} & U_{13} \\
0 & U_{22} & U_{23} \\
0 & 0 & U_{33}
\end{bmatrix}
\end{align*}
tales que $\mathbf{A = L \, U}$.
\end{frame}
\begin{frame}
\frametitle{Fase de eliminación}
Después de realizar la multiplicación del lado derecho de $\mathbf{A = LU}$, tenemos que:
\fontsize{12}{12}\selectfont
\begin{align*}
\mathbf{A} =
\begin{array}{l l l}
U_{11} & U_{12} & U_{13} \\
U_{11}L_{21} & U_{12}L_{21} + U_{22} & U_{13}L_{21} + U_{23} \\
U_{11}L_{31} & U_{12}L_{31} + U_{22}L_{32} & U_{13}L_{31} + U_{23}L_{32} + U_{33}
\end{array}
\end{align*}
\end{frame}
\begin{frame}[fragile]
\frametitle{Fase de eliminación}
Apliquemos ahora la eliminación de Gauss a la ecuación anterior. 
\\
\bigskip
El primer paso de la eliminación consiste en la elección de la primera fila como la fila pivote y la aplicación de las operaciones elementales:
\end{frame}
\begin{frame}[fragile]
\frametitle{Fase de eliminación}
\fontsize{12}{12}\selectfont
\begin{align*}
\begin{array}{l l l l}
\text{renglón } 2 & \leftarrow & \text{renglón } 2 - L_{21} \times \text{renglón } 1 & (\text{elimina } A_{21}) \\
\text{renglón } 3 & \leftarrow & \text{renglón } 3 - L_{31} \times \text{renglón } 1 & (\text{elimina } A_{31})
\end{array}
\end{align*}
\fontsize{14}{14}\selectfont
El resultado es
\begin{align*}
\mathbf{A}^{\prime} =
\left[ \begin{array}{l l l}
U_{11} & U_{12} & U_{13} \\
0 & U_{22} & U_{23} \\
0 & U_{22}L_{32} & U_{23}L_{32}+U_{33}
\end{array} \right]
\end{align*}
\end{frame}
\begin{frame}
\frametitle{Fase de eliminación}
El siguiente paso es tomar la segunda fila como pivote y utilizar la operación:
\fontsize{12}{12}\selectfont
\begin{align*}
\begin{array}{l l l l}
\text{renglón } 3 & \leftarrow & \text{renglón } 3 - L_{32} \times \text{renglón } 2 & (\text{elimina } A_{32})
\end{array}
\end{align*}
\fontsize{14}{14}\selectfont
dejando al final:
\begin{align*}
\mathbf{A}^{\prime \prime} = \left[
\begin{array}{l l l}
U_{11} & U_{12} & U_{13} \\
0 & U_{22} & U_{23} \\
0 & 0 & U_{33}
\end{array} \right]
\end{align*}
\end{frame}
\begin{frame}
\frametitle{Fase de eliminación}
Del ejemplo vemos dos características importantes del proceso de descomposición de Doolittle:
\setbeamercolor{item projected}{bg=red!70!black,fg=white}
\setbeamertemplate{enumerate items}[circle]
\begin{enumerate}[<+->]
\item La matriz $\mathbf{U}$ es idéntica a la matriz triangular inferior que resulta del proceso de eliminación de Gauss.
\seti
\end{enumerate}
\end{frame}
\begin{frame}
\frametitle{Fase de eliminación}
\setbeamercolor{item projected}{bg=red!70!black,fg=white}
\setbeamertemplate{enumerate items}[circle]
\begin{enumerate}[<+->]
\conti
\item Los elementos fuera de la diagonal de $\mathbf{L}$ son multiplicadores de la ecuación pivote que se utilizan durante la eliminación de Gauss, es decir, $L_{ij}$ es el multiplicador que elimina el elemento $A_{ij}$.
\end{enumerate}
\end{frame}
\begin{frame}
\frametitle{Fase de eliminación}
Es una práctica habitual almacenar los multiplicadores en la parte triangular inferior de la matriz de coeficientes, reemplazando los coeficientes ya que se eliminaron ($L_{ij}$ sustituye a $A_{ij}$)
\end{frame}
\begin{frame}
\frametitle{Fase de eliminación}
Los elementos diagonales de $L$ no tienen que guardarse, ya que se entiende que cada uno de ellos es la unidad.
\\
\bigskip
La forma final de la matriz de coeficientes sería una mezcla de $L$ y $U$:
\begin{align*}
\qty[ \mathbf{L \backslash U} ] =
\begin{bmatrix}
U_{11} & U_{12} & U_{13} \\
L_{21} & U_{22} & U_{23} \\
L_{31} & L_{32} & U_{33}
\end{bmatrix}
\end{align*}
\end{frame}
\begin{frame}[fragile]
\frametitle{Fase de eliminación Doolittle}
El algoritmo para la descomposición de Doolittle es idéntico al procedimiento de eliminación de Gauss, excepto por que cada multiplicador $\lambda$ se almacena en la parte triangular inferior de $A$.
\end{frame}
\subsection{Fase de solución}
\begin{frame}
\frametitle{Fase de solución Doolittle}
Considera ahora la fase de solución $L \, y=b$ por sustitución hacia adelante. La forma escalar de las ecuaciones es (recuerda que $L_{ii}=1$):
\begin{align*}
y_{1} &= b_{1} \\
L_{21} \, y_{1} + y_{2} &=  b_{2} \\
\vdots \\
L_{k1} \, y_{1} + L_{k2} \, y_{2} + \ldots + L_{k, k+1} \, y_{k-1} + y_{k} &= b_{k} \\
\vdots
\end{align*}
\end{frame}
\begin{frame}
\frametitle{Fase de solución Doolittle}
Resolviendo en la $k$-ésima ecuación para $y_{k}$ tenemos que
\begin{align*}
y_{k} = b_{k} - \sum_{j=1}^{k-1} L_{k j} \, y_{j} , \hspace{0.4cm} k = 2, 3, \ldots, n
\end{align*}
\pause
Vemos que la fase de sustitución para resolver $\mathbf{U} \, x = \mathbf{y}$ es idéntica a la utilizada en el \funcionazul{método de Gauss}.
\end{frame}
\subsection{Algoritmo para Doolittle}
\begin{frame}
\frametitle{Elaborando un algoritmo}
Como vimos en el \funcionazul{método de Gauss}, es posible construir un algoritmo computacional para las fases de eliminación y de solución del \funcionazul{método de Doolittle}.
\\
\bigskip
Ocuparemos una serie de funciones de librería \funcionazul{scipy.linalg} que serían los equivalentes a el \funcionazul{método de Doolittle}.
\end{frame}
\begin{frame}
\frametitle{Funciones de utilidad}
Del módulo \funcionazul{scipy.linalg} ocuparemos las siguientes funciones:
\setbeamercolor{item projected}{bg=blue!70!black,fg=yellow}
\setbeamertemplate{enumerate items}[circle]
\begin{enumerate}[<+->]
\item \funcionazul{lu}:
\\
Calcula la descomposición de una matriz $A$ en la forma $LU$ con pivote: $A = P \, L \, U$.
\seti
\end{enumerate}
\end{frame}
\begin{frame}
\frametitle{Funciones de utilidad}
Del módulo \funcionazul{scipy.linalg} ocuparemos las siguientes funciones:
\setbeamercolor{item projected}{bg=blue!70!black,fg=yellow}
\setbeamertemplate{enumerate items}[circle]
\begin{enumerate}[<+->]
\conti
\item \funcionazul{lu\_factor}:
\\
Calcula la descomposición de una matriz $A$ en la forma $LU$ con pivote. Las matrices $L$ y $U$ están superpuestas en el mismo arreglo. \\
\seti
\end{enumerate}
\end{frame}
\begin{frame}
\frametitle{Funciones de utilidad}
Del módulo \funcionazul{scipy.linalg} ocuparemos las siguientes funciones:
\setbeamercolor{item projected}{bg=blue!70!black,fg=yellow}
\setbeamertemplate{enumerate items}[circle]
\begin{enumerate}[<+->]
\conti
\item \funcionazul{lu\_solve}:
\\
Resuelve un sistema $a * x = b$ dada una factorización $LU$ de la matriz $a$.
\end{enumerate}
\end{frame}
\begin{frame}[fragile]
\frametitle{La función \texttt{sla.lu}}
Los argumentos mínimos para la función son:
\\
\medskip
\verb|sla.lu(a)|
\\
\medskip
donde:
\textbf{a}: Es una matriz de $M \cp N$ a factorizar.
\end{frame}
\begin{frame}[fragile]
\frametitle{La función \texttt{sla.lu}}
La función devuelve lo siguiente:
\\
\medskip
\textbf{p}: Es la matriz de permtuación de tamaño $M \cp M$.
\\
\medskip
\textbf{l}: Es la matriz triangular inferior o trapezoidal con elementos unitarios en la diagonal de tamaño $M \cp K$.
\\
\medskip
\textbf{u}: Es la matriz triangular superior o trapezoidal de tamaño $K \cp N$.
\end{frame}
\begin{frame}[fragile]
\frametitle{Ejemplo de uso}
\begin{exampleblock}{Ejemplo de uso de la función \texttt{sla.lu}}
\textcolor{ao}{\texttt{In[x]: }} \texttt{import numpy as np} \\
\medskip
\pause
\textcolor{ao}{\texttt{In[x]: }} \texttt{import scipy.linalg as sla} \\
\medskip
\pause
\textcolor{ao}{\texttt{In[x]: }} \texttt{A = np.array([[2, 5, 8, 7], [5, 2, 2, 8], [7, 5, 6, 6], [5, 4, 4, 8]])} \\
\medskip
\pause
\textcolor{ao}{\texttt{In[x]: }} \texttt{p, l, u = lu(A)}
\end{exampleblock}
\end{frame}
\begin{frame}[fragile]
\frametitle{Ejemplo de uso}
\begin{exampleblock}{Ejemplo de uso de la función \texttt{sla.lu}}
\textcolor{ao}{\texttt{In[x]: }} \texttt{print(l)} \\
\medskip
\pause
\textcolor{red}{\texttt{Out[x]: }}
\begin{table}
\begin{tabular}{l l l l}
\texttt{[[ 1.} & \texttt{0.} & \texttt{0.} & \texttt{0.]} \\
\texttt{[ 0.28571429} & \texttt{1.} & \texttt{0.} & \texttt{0.]} \\
\texttt{[ 0.71428571} & \texttt{0.12} & \texttt{1.} & \texttt{0.]} \\
\texttt{[ 0.71428571} & \texttt{-0.44} & \texttt{-0.46153846} & \texttt{1.]]}
\end{tabular}
\end{table}
\end{exampleblock}
\end{frame}
\begin{frame}[fragile]
\frametitle{Ejemplo de uso}
\begin{exampleblock}{Ejemplo de uso de la función \texttt{sla.lu}}\textcolor{ao}{\texttt{In[x]: }} \texttt{print(u)} \\
\medskip    
\pause
\textcolor{red}{\texttt{Out[x]: }}
\fontsize{12}{12}\selectfont
\begin{table}
\begin{tabular}{l l l l}
\texttt{[[ 7.} & \texttt{5.} & \texttt{6.} & \texttt{6.]} \\
\texttt{[ 0.} & \texttt{3.57142857} & \texttt{6.28571429} & \texttt{5.28571429]} \\
\texttt{[ 0.} & \texttt{0.} & \texttt{-1.04} & \texttt{3.08]} \\
\texttt{[ 0.} & \texttt{0.} & \texttt{0.0} & \texttt{7.46153846]]}
\end{tabular}
\end{table}
\end{exampleblock}
\end{frame}
\begin{frame}
\frametitle{Verificando el resultado}
Comprobamos la factorización usando lo siguiente:
\begin{exampleblock}{Comprobando el resultado obtenido}
\textcolor{ao}{\texttt{In[x]: }} \texttt{np.allclose(A - p @ l @ u, np.zeros((4, 4)))} \\
\medskip
\pause
\textcolor{red}{\texttt{Out[x]: }} \texttt{True}
\end{exampleblock}
\end{frame}
\begin{frame}
\frametitle{La función \texttt{sla.lu\_factor}}
Esta función calcula la factorización $\mathbf{LU}$ con pivote.
\\
\bigskip
La factorización es
\begin{align*}
\mathbf{A} =  P \, L \, U 
\end{align*}
donde $P$ es la matriz de permutación, $L$ es la matriz triangular inferior con los elementos unitarios en la diagonal, y $U$ es la matriz triangular superior.
\end{frame}
\begin{frame}[fragile]
\frametitle{Argumentos mínimos}
Los argumentos mínimos son
\\
\medskip
\verb|sla.lu_factor(a)|
\\
\medskip
donde
\\
$\mathbf{a}$ : es un arreglo $M \cp M$. Es la matriz a factorizar.
\end{frame}
\begin{frame}[fragile]
\frametitle{Lo que devuelve la función}
La función devuelve lo siguiente:
\\
\medskip
$\mathbf{lu}$ : Arreglo de $N \cp N$. Es la matriz que contiene a $U$ y $L$, los elementos unitarios diagonales de $L$ no se almacenan.
\\
\medskip
$\mathbf{piv}$ : Arreglo de tamaño $N$. Los índices del pivote representan la matriz de permutación $P$.
\end{frame}
\begin{frame}
\frametitle{Ejemplo de uso}
\begin{exampleblock}{Uso de la función \texttt{lu\_factor}}
\textcolor{ao}{\texttt{In[x]: }} \texttt{A = np.array([[2, 5, 8, 7], [5, 2, 2, 8], [7, 5, 6, 6], [5, 4, 4, 8]])} \\
\medskip
\pause
\textcolor{ao}{\texttt{In[x]: }} \texttt{lu, piv = lu\_factor(A)} \\
\medskip
\pause
\textcolor{ao}{\texttt{In[x]: }} \texttt{piv} \\
\medskip
\pause
\textcolor{red}{\texttt{Out[x]: }} \\
\texttt{[2 2 3 3]}
\end{exampleblock}
\end{frame}
\begin{frame}
\frametitle{Ejemplo de uso}
\begin{exampleblock}{Uso de la función \texttt{lu\_factor}}
\textcolor{ao}{\texttt{In[x]: }} \texttt{print(lu)} \\
\medskip
\pause
\fontsize{10}{10}\selectfont
\textcolor{red}{\texttt{Out[x]: }} \\
\begin{table}
\begin{tabular}{l l l l}
\texttt{[[ 7.} & \texttt{5.} & \texttt{6.} & \texttt{6.]} \\
\texttt{[0.28571429} & \texttt{3.57142857} & \texttt{6.28571429} & \texttt{5.28571429]} \\
\texttt{[0.71428571} & \texttt{0.12} & \texttt{-1.04} & \texttt{3.08]} \\
\texttt{[0.71428571} & \texttt{0.44} & \texttt{-0.46153846} & \texttt{7.46153846]]}
\end{tabular}
\end{table}
\end{exampleblock}
\end{frame}
\begin{frame}
\frametitle{La función \texttt{lu\_solve}}
Esta función resuelve un sistema de ecuaciones del tipo
\begin{align*}
a \, x = b
\end{align*}
dada una factorización $LU$ de la matriz $a$.
\end{frame}
\begin{frame}[fragile]
\frametitle{Argumentos de la función}
Los argumentos mínimos de la función son:
\\
\medskip
\verb|lu_solve(lu_piv, b)|
\\
\medskip
donde
\textbf{lu, piv} : Es la factorización de la matriz $a$, dada por la función \funcionazul{lu\_factor}.
\\
\medskip
\textbf{b} : Arreglo del lado derecho de la ecuación.
\end{frame}
\begin{frame}
\frametitle{Lo que devuelve la función}
La función devuelve:
\\
\medskip
\textbf{x} : Arreglo. Corresponde a la solución del sistema de ecuaciones.
\end{frame}
\begin{frame}[fragile]
\frametitle{Ejemplo de uso}
En el siguiente ejemplo veamos el uso de la función \funcionazul{lu\_solve}
\begin{exampleblock}{Ejemplo de solución con la factorización $LU$ con comprobación}
\textcolor{ao}{\texttt{In[x]: }} \texttt{A = np.array([[2, 5, 8, 7], [5, 2, 2, 8], [7, 5, 6, 6], [5, 4, 4, 8]])} \\
\medskip
\pause
\textcolor{ao}{\texttt{In[x]: }} \texttt{b = np.array([1, 1, 1, 1])} \\
\medskip
\pause
\textcolor{ao}{\texttt{In[x]: }} \texttt{lu, piv = sla.lu\_factor(A)} \\
\medskip
\pause
\textcolor{ao}{\texttt{In[x]: }} \texttt{x = sla.lu\_solve((lu, piv), b)} \\
\end{exampleblock}
\end{frame}
\begin{frame}[fragile]
\frametitle{Ejemplo de uso}
\begin{exampleblock}{Ejemplo de solución con la factorización $LU$ con comprobación}
\textcolor{ao}{\texttt{In[x]: }} \texttt{print(x)} \\
\medskip
\pause
\textcolor{red}{\texttt{Out[x]: }} \texttt{[ 0.05154639 -0.08247423  0.08247423  0.09278351]}
\\
\medskip
\pause
\textcolor{ao}{\texttt{In[x]: }} \texttt{np.allclose(A @ x - b, np.zeros((4,)))} \\
\medskip
\pause
\textcolor{red}{\texttt{Out[x]: }} \texttt{True}
\end{exampleblock}
\end{frame}
\subsection{Ejercicio 1}
\begin{frame}
\frametitle{Ejercicio 1}
Resuelve con el método de Doolittle, $A \: x = b$ donde
\begin{align*}
A =
\begin{bmatrix}
-3 & 6 & -4 \\
9 & -8 & 24 \\
-12 & 24 & -26 
\end{bmatrix}
\hspace{1.5cm} b =
\begin{bmatrix}
-3 \\
65 \\
-42
\end{bmatrix}
\end{align*}
\end{frame}
\begin{frame}
\frametitle{Solución al ejercicio}
Como ya revisamos, para la solución de problemas con \python{} de sistemas de ecuaciones con matrices, contamos con funciones que nos van a simplificar la tarea.
\\
\bigskip
Pero la intención es aprovechar al máximo ese apoyo, por lo que para cada ejercicio se deberá de realizar el siguiente proceso en un solo archivo *.py:
\end{frame}
\begin{frame}
\frametitle{Solución a los ejercicios}
\setbeamercolor{item projected}{bg=blue!70!black,fg=yellow}
\setbeamertemplate{enumerate items}[circle]
\begin{enumerate}[<+->]
\item Presentar la matriz aumentada.
\item Explorar las propiedades del sistema: valor del determinante, número de condición, norma de la matriz.
\item Presentar las matrices $\mathbf{L}$ y $\mathbf{U}$ de la descomposición.
\item Resolver el sistema de ecuaciones.
\item Comprobar la solución obtenida.
\end{enumerate}
\end{frame}
\begin{frame}
\frametitle{Propiedades de la matriz}
El siguiente paso es revisar las propiedades de la matriz $\mathbf{A}$, tales como el determinante, el número de condición, la norma, ocupando las funciones de \funcionazul{scipy.linalg}:
\begin{table}
\begin{tabular}{l l}
Propiedad & Valor \\ \hline
Determinante & $300.00$\\ \hline
Norma & $46.6690$ \\ \hline
No. condición & $63.0749$ \\ \hline
\end{tabular}
\end{table}
\end{frame}
\begin{frame}
\frametitle{Las matrices $\mathbf{L}$ y $\mathbf{U}$}
El siguiente paso es mostrar las matrices que se obtienen en la descomposición $\mathbf{L \, U}$:
\begin{exampleblock}{Las matriz $\mathbf{L}$}
\textcolor{ao}{\texttt{In[x]: }} \texttt{p, l, u = sla.lu(A)} \\
\medskip
\pause
\textcolor{ao}{\texttt{In[x]: }} \texttt{print(l)} \\
\medskip
\pause
\textcolor{red}{\texttt{Out[x]: }} 
\begin{flushleft}
\begin{tabular}{l l l}
\texttt{[[1.} & \texttt{0.} & \texttt{ 0.]} \\
\texttt{[-0.75} & \texttt{1.} & \texttt{0.]} \\
\texttt{[ 0.25} & \texttt{0.} & \texttt{1.]]}    
\end{tabular}
\end{flushleft}
\end{exampleblock}
\end{frame}
\begin{frame}
\frametitle{Las matrices $\mathbf{L}$ y $\mathbf{U}$}
\begin{exampleblock}{Las matriz $\mathbf{U}$ que se obtiene}
\textcolor{ao}{\texttt{In[x]: }} \texttt{print(u)} \\
\medskip
\pause
\textcolor{red}{\texttt{Out[x]: }} \\
\begin{flushleft}
\begin{tabular}{l l l}
\texttt{[[-12.} & \texttt{24.} & \texttt{-26.]} \\
\texttt{[0.} & \texttt{10.} & \texttt{4.5]} \\
\texttt{[0.} & \texttt{0.} & \texttt{2.5]]}    
\end{tabular}
\end{flushleft}
\end{exampleblock}
\end{frame}
\begin{frame}
\frametitle{Solución al sistema}
Resolvemos el sistema
\begin{exampleblock}{Solución con \texttt{lu\_factor} y \texttt{lu\_solve}}
\textcolor{ao}{\texttt{In[x]: }} \texttt{lu, piv = sla.lu\_factor(A)} \\
\medskip
\pause
\textcolor{ao}{\texttt{In[x]: }} \texttt{x = sla.lu\_solve((lu, piv), b)} \\
\medskip
\pause
\textcolor{ao}{\texttt{In[x]: }} \texttt{print(x)} \\
\medskip
\pause
\textcolor{ao}{\texttt{Out[x]: }} \texttt{[1. 2. 3.]} \\
\medskip
\pause
\textcolor{ao}{\texttt{In[x]: }} \texttt{print(np.allclose(A @ x - b, np.zeros((3,))))} \\
\medskip
\pause
\textcolor{red}{\texttt{Out[x]: }} \texttt{True}
\end{exampleblock}
\end{frame}
\begin{frame}
\frametitle{Criterios para entregar la solución}
Se revisó el procedimiento para resolver un ejercicio con las funciones tanto de \funcionazul{numpy} y \funcionazul{scipy}.
\\
\bigskip
Esperamos que en su código elaboren de una manera más completa la salida en terminal, es decir, incluir textos que muestren lo que presentan, no sólo que anoten el código y ver la solución.
\end{frame}
\section{Descomposición de Choleski}
\frame{\tableofcontents[currentsection, hideothersubsections]}
\subsection{Descripción del método}
\begin{frame}
\frametitle{Descomposición de Choleski}
La descomposición de Choleski $\mathbf{A = L \: L}^{T}$ tiene dos limitaciones:
\setbeamercolor{item projected}{bg=red!70!black,fg=white}
\setbeamertemplate{enumerate items}[circle]
\begin{enumerate}[<+->]
\item Dado que $\mathbf{L \: L}^{T}$ devuelve siempre una matriz simétrica, la descomposición de Choleski requiere que la matriz $\mathbf{A}$ sea simétrica.
\seti
\end{enumerate}
\end{frame}
\begin{frame}
\frametitle{Descomposición de Choleski}
\setbeamercolor{item projected}{bg=red!70!black,fg=white}
\setbeamertemplate{enumerate items}[circle]
\begin{enumerate}[<+->]
\conti
\item El proceso de descomposición implica tomar raíces cuadradas de ciertas combinaciones de la elementos de $\mathbf{A}$.
\\
\bigskip
Se puede demostrar que, para evitar los valores negativos de las raíces, la matriz $\mathbf{A}$ debe de ser definita positiva.
\end{enumerate}
\end{frame}
\subsection*{Matriz definida positiva}
\begin{frame}
\frametitle{Definición de matriz definida positiva}
Se dice que una matriz $\mathbf{M}$ es definida positiva, si para todo vector no nulo $x$, se tiene que
\begin{align*}
x^{T} \: M \: x > 0
\end{align*}
cumpliendo esto, también se cumple que, todos los eigenvalores de $M$, son positivos (definición alterna).
\end{frame}
\subsection*{Desventaja con Choleski}
\begin{frame}
\frametitle{Desventaja del método de Choleski}
Aunque el número de multiplicaciones en todos los métodos de descomposición $LU$ es el mismo, la descomposición de Choleski no es tan popular en la solución de ecuaciones simultáneas, debido a las restricciones mencionadas anteriormente.
\end{frame}
\subsection{Factorización de Choleski}
\begin{frame}
\frametitle{El proceso de factorización de Choleski}
Consideremos la matriz $\mathbf{A}$ de $3 \cp 3$
\begin{align*}
\mathbf{A = L \: L^{T}}
\end{align*}
tal que
\begin{align*}
\begin{bmatrix}
A_{11} & A_{12} & A_{13} \\
A_{21} & A_{22} & A_{23} \\
A_{31} & A_{32} & A_{33}
\end{bmatrix}
= \begin{bmatrix}
L_{11} & 0      & 0 \\
L_{21} & L_{22} & 0 \\
L_{31} & L_{32} & L_{33}
\end{bmatrix}
\begin{bmatrix}
L_{11} & L_{21} & L_{31} \\
0      & L_{22} & L_{32} \\
0      & 0      & L_{33}
\end{bmatrix}
\end{align*}
\end{frame}
\begin{frame}[fragile]
\frametitle{Factorización de Choleski}
Luego de resolver la multiplicación del lado derecho, resulta
\fontsize{12}{12}\selectfont
\begin{align*}
&\begin{bmatrix}
A_{11} & A_{12} & A_{13} \\
A_{21} & A_{22} & A_{23} \\
A_{31} & A_{32} & A_{33}
\end{bmatrix} = \\
&=
\begin{bmatrix}
L_{11}^{2} & L_{11}L_{21} & L_{11}L_{31} \\
L_{11}L_{21} & L_{21}^{2} + L_{22}^{2} & L_{21}L_{31} + L_{22}L_{32} \\
L_{11}L_{31} & L_{21}L_{31} + L_{22}L_{32} & L_{31}^{2} + L_{32}^{2} + L_{33}^{2}
\end{bmatrix}
\end{align*}
\end{frame}
\begin{frame}[fragile]
\frametitle{Factorización de Choleski}
Vemos que la matriz del lado derecho es simétrica.

\begin{align*}
\begin{bmatrix}
L_{11}^{2} & L_{11}L_{21} & L_{11}L_{31} \\
L_{11}L_{21} & L_{21}^{2} + L_{22}^{2} & L_{21}L_{31} + L_{22}L_{32} \\
L_{11}L_{31} & L_{21}L_{31} + L_{22}L_{32} & L_{31}^{2} + L_{32}^{2} + L_{33}^{2}
\end{bmatrix}
\end{align*}
\end{frame}
\begin{frame}
\frametitle{Factorización de Choleski}
Al igualar las matrices $\mathbf{A}$ y $\mathbf{L \: L}^{T}$ elemento a elemento, tendremos seis ecuaciones (revisando que hay simetría en los elementos triangulares superiores e inferiores)
\end{frame}
\begin{frame}
\frametitle{Factorización de Choleski}
Consideremos la parte triangular inferior de cada matriz, al igualar los elementos de la primera columna, empezando por la primera fila y hacia abajo:
\end{frame}
\begin{frame}
\frametitle{Factorización de Choleski}
Calculando $L_{11}$, $L_{21}$ y $L_{31}$ en ese orden:
\begin{align*}
\begin{array}{l l l l l}
A_{11} & {=L_{11}^{2}}   & \rightarrow & L_{11} & {=\sqrt{A_{11}}} \\
&                 &             &        &  \\
A_{21} & {=L_{11}L_{21}} & \rightarrow & L_{21} & {=\dfrac{A_{21}}{L_{11}}} \\
&                 &             &        &  \\
A_{31} & {=L_{11}L_{31}} & \rightarrow & L_{31} & {=\dfrac{A_{31}}{L_{11}}}
\end{array}
\end{align*}
\end{frame}
\begin{frame}
\frametitle{Factorización de Choleski}
De la segunda columna, iniciamos con la segunda fila, para obtener $L_{22}$ y $L_{32}$:
\begin{align*}
 A_{22} &= L_{21}^{2} + L_{22}^{2} \hspace{0.5cm} \rightarrow L_{22} = \sqrt{A_{22}-L_{21}^{2}} \\[0.5em]
A_{32} &= L_{21} \, L_{31} + L_{22} \, L_{32} \hspace{0.25cm} \rightarrow L_{32} =\dfrac{A_{32} - L_{21} \, L_{31}}{L_{22}}
\end{align*}
\end{frame}
\begin{frame}
\frametitle{Factorización de Choleski}
Finalmente con la tercera columna y de la tercera fila, obtenemos $L_{33}$:
\begin{align*}
A_{33} &= L_{31}^{2} + L_{32}^{2} + L_{33}^{2} \hspace{0.5cm} \rightarrow \\[0.5em]
&\rightarrow \hspace{0.5cm} L_{33} =\sqrt{A_{33} - L_{31}^{2} - L_{32}^{2}}
\end{align*}
\end{frame}
\begin{frame}
\frametitle{Factorización de Choleski}
Podemos extrapolar los resultados para una matriz de $n \cp n$.
\\
\bigskip
\pause
Para un elemento en la parte triangular inferior de $\mathbf{LL}^{T}$ resulta:
\begin{align*}
\mathbf{LL}^{T}_{ij} &= L_{i1} \, L_{j1} + L_{i2} \, L_{j2} + \ldots + L_{ij} \, L_{jj} = \\
&= \sum_{k=1}^{j} L_{ik}\, L_{jk} \hspace{1.5cm} i \geq j
\end{align*}
\end{frame}
\begin{frame}
\frametitle{Factorización de Choleski}
Igualando los términos para los correspondientes elementos de $\mathbf{A}$:
\begin{align*}
A_{ij} = \sum_{k=1}^{j} L_{ik} \, K_{jk} \hspace{0.5cm} i &= j, j+1,\ldots, n, \\
j &= 1, 2, \ldots, n
\end{align*}
El intervalo de índices de los elementos mostrados limita a la parte triangular inferior. 
\end{frame}
\begin{frame}
\frametitle{Factorización de Choleski}
Para la primera columna ($j = 1$), obtenemos de la ecuación anterior:
\begin{align*}
L_{11} = \sqrt{A_{11}} \hspace{1.5cm} L_{i1} = \dfrac{A_{1j}}{L_{11}}, \hspace{0.5cm} i = 2, 3, \ldots, n
\end{align*}
\\
\bigskip
Continuando con las otras columnas, vemos que la incógnita en la ecuación es $L_{ij}$ (los otros elementos de $\mathbf{L}$ que aparecen en la ecuación, ya han sido calculados). 
\end{frame}
\begin{frame}
\frametitle{Factorización de Choleski}
Tomando el término que incluye a $L_{ij}$ fuera de la suma de la ecuación para los elementos de $\mathbf{A}$ tenemos:
\begin{align*}
A_{ij} = \sum_{k=1}^{j-1} L_{ik} \, L_{jk} + L_{ij} \, L_{jj}
\end{align*}
\end{frame}
\begin{frame}
\frametitle{Factorización de Choleski}
Si $i = j$ (un elemento de la diagonal), la solución es:
\begin{align*}
L_{jj} = \sqrt{A_{jj} - \sum_{k=1}^{j-1} L_{jk}^{2}} \hspace{1cm} j = 2, 3, \ldots, n
\end{align*}
\end{frame}
\begin{frame}
\frametitle{Factorización de Choleski}
Para un elemento que no está en la diagonal:
\begin{align*}
 L_{ij} = \dfrac{\left( A_{ij} - \sum\limits_{k = 1}^{j-1} \: L_{ik} \: L_{jk} \right)}{L_{jj}}
\end{align*}
donde
\begin{align*}
j =2, 3, \ldots, n-1, \hspace{0.5cm} i = j+1, j+2, \ldots, n
\end{align*}
\end{frame}
\subsection{Algoritmo para el método}
\begin{frame}
\frametitle{Algoritmo para Cholesky}
Al igual que con el \funcionazul{método de Doolittle}, podemos construir un algoritmo y por ende un código para reproducir la fase de factorización y de solución.
\\
\bigskip
Haremos uso nuevamente del módulo \funcionazul{scipy.linalg} para ocupar un conjunto de funciones disponible para este método.
\end{frame}
\begin{frame}
\frametitle{Funciones de utilidad}
Del módulo \funcionazul{scipy.linalg} ocuparemos las siguientes funciones:
\setbeamercolor{item projected}{bg=blue!70!black,fg=yellow}
\setbeamertemplate{enumerate items}[circle]
\begin{enumerate}[<+->]
\item \funcionazul{cholesky}: \\
Calcula la factorización de Cholesky para una matriz. Devuelve la factorización $\mathbf{A = L \, L^{T}}$ o $\mathbf{A = U^{T} \, U}$ para una matriz $\mathbf{A}$definida positiva.
\seti
\end{enumerate}
\end{frame}
\begin{frame}
\frametitle{Funciones de utilidad}
\setbeamercolor{item projected}{bg=blue!70!black,fg=yellow}
\setbeamertemplate{enumerate items}[circle]
\begin{enumerate}[<+->]
\conti
\item \funcionazul{cho\_factor}: \\
Calcula la factorización de Cholesky para una matriz para ser utilizada en la función \funcionazul{cho\_solve}.
\seti
\end{enumerate}
\end{frame}    
\begin{frame}
\frametitle{Funciones de utilidad}
\setbeamercolor{item projected}{bg=blue!70!black,fg=yellow}
\setbeamertemplate{enumerate items}[circle]
\begin{enumerate}[<+->]
\conti
\item \funcionazul{cho\_solve}: \\
Resuelve un sistema de ecuaciones $A \, x = b$ dada la factorización de Cholesky para una matriz $\mathbf{A}$
\end{enumerate}
\end{frame} 
\begin{frame}[fragile]
\frametitle{La función \texttt{cholesky}}
Los argumentos mínimos de la función son:
\\
\medskip
\verb|sla.cholesky(a, lower=False)|
\\
\medskip
donde:
\\
\textbf{a} : Es un arreglo de $M \cp M$, que es la matriz a factorizar.
\\
\medskip
\textbf{lower} : Booleano, opcional. Ya sea para calcular la factorización Cholesky usando la matriz triangular superior o inferior. El valor por defecto es triangular superior.
\end{frame}
\begin{frame}
\frametitle{Lo que devuelve la función}
La función devuelve:
\\
\medskip
\textbf{c} : Un arreglo de $M \cp M$. Es la matriz triangular superior o inferior de la factorización de la matriz $a$
\end{frame}
\begin{frame}[fragile]
\frametitle{La función \texttt{cho\_factor}}
Los argumentos mínimos son:
\\
\medskip
\verb|sla.cho_factor(a, lower=False)|
\\
\medskip
donde:
\\
\textbf{a} : Arreglo de $M \cp M$. Es la matriz a factorizar.
\\
\medskip
\textbf{lower} : Booleano, opcional.  Ya sea para calcular la factorización de Cholesky usando la matriz triangular superior o inferior. El valor por defecto es triangular superior.
\end{frame}
\begin{frame}
\frametitle{Lo que devuelve la función}
La función devuelve:
\\
\medskip
\textbf{c} : Un arreglo de $M \cp M$. Es la matriz triangular superior o inferior de la factorización de la matriz $a$. Algunas partes de la matriz contienen datos aleatorios.
\\
\medskip
\textbf{lower} : Booleano. Bandera que indica si el factor está en el triángulo inferior o superior.
\end{frame}
\begin{frame}[fragile]
\frametitle{La función \texttt{cho\_solve}}
Los argumentos mínimos son:
\\
\medskip
\verb|sla.cho_solve((c, lower), b)|
\\
\medskip
donde:
\\
\textbf{c, lower} : Es la factorización de Cholesky obtenida por la función \funcionazul{cho\_factor}.
\\
\medskip
\textbf{b} : Arreglo.  La parte derecha de la ecuación del sistema.
\end{frame}
\begin{frame}
\frametitle{Lo que devuelve la función}
La función devuelve:
\\
\medskip
\textbf{x }: Es la solución del sistema $A \, x =  b$
\end{frame}
\subsection{Ejercicio 1}
\begin{frame}
\frametitle{Ejercicio 1}
Resuelve el siguiente sistema mediante la factorización de Choleski:
\begin{align*}
\mathbf{A} =
\begin{bmatrix}
1 & 1 & 1 \\
1 & 2 & 2 \\
1 & 2 & 3
\end{bmatrix}
\hspace{1cm}
\mathbf{b}=
\begin{bmatrix}
1 \\
3/2 \\
3
\end{bmatrix}
\end{align*}
\end{frame}
\begin{frame}
\frametitle{Procedimiento de solución}
\setbeamercolor{item projected}{bg=blue!70!black,fg=yellow}
\setbeamertemplate{enumerate items}[circle]
\begin{enumerate}[<+->]
\item Se debe de revisar que la matriz $\mathbf{A}$, es una matriz definida positiva, para ello calculamos los valores propios de la matriz y revisamos que sean todos positivos.
\item Calculamos la matriz triangular inferior $\mathbf{L}$.
\item Resolvemos el sistema de ecuaciones.
\item Comprobamos la solución.
\end{enumerate}
\end{frame}
\begin{frame}[fragile]
\frametitle{Función adicional}
Para evaluar los valores propios de una matriz, ocuparemos la función \funcionazul{sla.eigvals}:
\\
\medskip
\verb|sla.eigvals(a)|
\\
\medskip
Que devuelve los valores propios de la matriz, revisa con cuidado que la respuesta se presenta en formato de número complejo.
\end{frame}
\begin{frame}[fragile]
\frametitle{Cálculo de los valores propios}
\begin{exampleblock}{Cálculo de los valores propios}
\textcolor{ao}{\texttt{In[x]: }} \texttt{sla.eigvals(A)} \\
\medskip
\pause
\textcolor{red}{\texttt{Out[x]: }} \\
\texttt{[5.04891734+0.j 0.64310413+0.j 0.30797853+0.j]}
\end{exampleblock}
Nótese que el formato de salida es de número complejo.
\end{frame}
\begin{frame}[allowframebreaks, fragile]
\frametitle{Código para resolver el problema}
\begin{lstlisting}[caption=Código para el ejercicio de factorización Cholesky, basicstyle=\linespread{1.1}\ttfamily=\small, columns=fullflexible]
import scipy.linalg as sla
import numpy as np
import scipy as sc

A = sc.array([[ 1, 1, 1], [1, 2, 2], [1, 2, 3]])
b = sc.array([1., 3./2, 3.])

L = sla.cholesky(A, lower=True)

print('\nLa matriz triangular inferior es:\n')
print(L)

print('\nSe comprueba que L multiplicada por la transpuesta L^T es la matriz A\n')
print(L @ L.T)

c, low = sla.cho_factor(A)

x =  sla.cho_solve((c, low), b)

print('\nEl vector solucion es:')
print(x)

print('\nLa comprobacion de la solucion es:')
print(np.allclose(A @ x - b, np.zeros(3)))
\end{lstlisting}
\end{frame}
\begin{frame}[fragile]
\frametitle{Solución del sistema}
La solución que se obtiene es:
\begin{exampleblock}{Solución con la factorización de Cholesky}
\texttt{El vector solución es:} \\
\texttt{[ 0.5 -1.   1.5]}
   
\end{exampleblock}
\end{frame}
\begin{frame}[fragile]
\frametitle{Requerimiento en la solución}
Deberás de incorporar en el código, una parte que indique si la matriz $\mathbf{A}$ es definida positiva:
\begin{exampleblock}{Parte de código que debe de incluirse}
\fontsize{12}{12}\selectfont
\texttt{Los valores propios de la matriz A son:} \\

\texttt{[5.04891734+0.j 0.64310413+0.j 0.30797853+0.j]} \\

\texttt{La matriz es definida positiva.}
\end{exampleblock}
\end{frame}
\begin{frame}[fragile]
\frametitle{Requerimiento en la solución}
Debe de haber una parte que evalúe la parte real de cada valor propio (\funcionazul{cmath.real}), si alguno de ellos no es positivo, entonces la matriz es \emph{no definida positiva}.
\\
\bigskip
En caso de que todos los valores propios sean positivos, debe indicarse un mensaje: \emph{la matriz es definida positiva.}
\end{frame}
\section{M. con coeficientes simétricos y en banda}
\frame[allowframebreaks]{\tableofcontents[currentsection, hideothersubsections]}
\subsection{Definición de matrices especiales}
\begin{frame}
\frametitle{Coeficientes simétricos y en banda}
Algunos problemas en física e ingeniería plantean la necesidad de trabajar con matrices \emph{escasamente pobladas}, en inglés \emph{sparse}, donde la gran mayoría de los elementos de la matriz, son cero.
\end{frame}
\begin{frame}
\frametitle{Coeficientes simétricos y en banda}
Si todos los elementos no nulos de la matriz se ubican sobre la diagonal principal, se dice entonces que la matriz es una \emph{matriz en banda}.
\end{frame}
\begin{frame}
\frametitle{Matriz tridiagonal}
Sea la matriz
\begin{align*}
A = \begin{bmatrix}
X & X & 0 & 0 & 0 \\
X & X & X & 0 & 0 \\
0 & X & X & X & 0 \\
0 & 0 & X & X & X \\
0 & 0 & 0 & X & X
\end{bmatrix}
\end{align*}
\end{frame}
\begin{frame}
\frametitle{Matriz tridiagonal}
En donde $X$ indica un elemento no nulo, dando la forma de una banda (considera que algunos de éstos elementos, aún así, pueden ser cero).
\\
\bigskip
Todos los demás elementos fuera de la banda, son nulos.
\end{frame}
\begin{frame}
\frametitle{Propiedades de las matrices \textbf{L} y \textbf{U}}
Si una matriz en banda se descompone de la forma $\mathbf{A} = \mathbf{L} \mathbf{U}$, donde $\mathbf{L}$ y $\mathbf{U}$ mantienen la estructura en banda de $\mathbf{A}$, por ejemplo, si descomponemos la matriz mostrada anteriormente, obtendremos:
\end{frame}
\begin{frame}
\frametitle{Propiedades de las matrices \textbf{L} y \textbf{U}}
\begin{align*}
\mathbf{L} = \begin{bmatrix}
X & 0 & 0 & 0 & 0 \\
X & X & 0 & 0 & 0 \\
0 & X & X & 0 & 0 \\
0 & 0 & X & X & 0 \\
0 & 0 & 0 & X & X
\end{bmatrix} \hspace{0.75cm}
\mathbf{U} = \begin{bmatrix}
X & X & 0 & 0 & 0 \\
0 & X & X & 0 & 0 \\
0 & 0 & X & X & 0 \\
0 & 0 & 0 & X & X \\
0 & 0 & 0 & 0 & X
\end{bmatrix}
\end{align*}
\end{frame}
\begin{frame}
\frametitle{Propiedades de las matrices \textbf{L} y \textbf{U}}
La estructura en banda de una matriz de coeficientes pues aprovecharse para guardar valores y reducir el tiempo de cálculo. 
\\
\medskip
Si la matriz es de coeficientes, y además es simétrica, la economía de operaciones sobre la misma, es mayor.
\end{frame}
\subsection{Matriz de coeficientes tridiagonal}
\begin{frame}
\frametitle{Matriz de coeficientes tridiagonal}
Considera la solución de un sistema $\mathbf{A} \, x = \mathbf{b}$ mediante la descomposición de Doolittle, donde $\mathbf{A}$ es una matriz triadiagonal de $n\times n$
\end{frame}
\begin{frame}
\frametitle{Matriz de coeficientes tridiagonal}
\begin{align*}
\mathbf{A} = 
 \begin{bmatrix}
d_{1} & e_{1} & 0 & 0 & \ldots & 0 \\
c_{1} & d_{2} & e_{2} & 0 & \ldots & 0 \\
0 & c_{2} & d_{3} & e_{3} & \ldots & 0 \\
0 & 0 & c_{3}& d_{4} & \ldots & 0 \\
\vdots & \vdots & \vdots & \vdots & \ddots & \vdots \\
0 & 0 & \ldots & 0 & c_{n-1} & d_{n}
\end{bmatrix}
\end{align*}
\end{frame}
\begin{frame}
\frametitle{Matriz de coeficientes tridiagonal}
Como la notación lo indica, podemos almacenar los elementos no nulos de $\mathbf{A}$ en los vectores
\begin{align*}
\mathbf{c} = \begin{bmatrix}
c_{1} \\
c_{2} \\
\vdots \\
c_{n-1}
\end{bmatrix}
\hspace{1cm}
\mathbf{d} = \begin{bmatrix}
d_{1} \\
d_{2} \\
\vdots \\
d_{n-1} \\
d_{n}
\end{bmatrix}
\hspace{1cm}
\mathbf{e} = \begin{bmatrix}
e_{1} \\
e_{2} \\
\vdots \\
e_{n-1}
\end{bmatrix}
\end{align*}
\end{frame}
\begin{frame}
\frametitle{Matriz de coeficientes tridiagonal}
El ahorro resultante de almacenamiento puede ser significativo.
\\
\bigskip
Por ejemplo, una matriz tridiagonal de $100 \times 100$, contiene $10000$ elementos, que pueden almacenarse en solo $99+100+99=298$ entradas, lo cual representa una compresión de $33:1$.
\end{frame}
\begin{frame}[plain]
\frametitle{Considera la siguiente matriz en banda}
\fontsize{12}{12}\selectfont
\begin{align*}
\begin{bmatrix}
1 & -5 & 3 & 0 & 0 & 0 & 0 & 0 \\
3 & 2 & -3 & 5 & 0 & 0 & 0 & 0 \\
0 & -2 & 1 & 5 & 9 & 0 & 0 & 0 \\
0 & 0 & 9 & -1 & 5 & 4 & 0 & 0 \\
0 & 0 & 0 & 0 & 2 & -3 & -2 & 0 \\
0 & 0 & 0 & 0 & 2 & 0 & 1 & -6 \\
0 & 0 & 0 & 0 & 0 & -3 & 2 & 7 \\
0 & 0 & 0 & 0 & 0 & 0 & 9 & 1 
\end{bmatrix}
\end{align*}
\end{frame}
\subsection{Representación de la matriz en banda}
\begin{frame}[plain]
\frametitle{Marcando la diagonal y las codiagonales}
\begin{figure}
    \centering
    \includestandalone[scale=0.8]{Figuras/matriz_marcada_04}    
\end{figure}
\end{frame}
\subsection{Arreglo de las codiagonales}
\begin{frame}
\frametitle{Arreglo de las codiagonales}
\begin{figure}
    \centering
    \includestandalone[scale=0.8]{Figuras/matriz_marcada_05}    
\end{figure}
\begin{itemize}
\item Las codiagonales superiores tienen ceros a la izquierda.
\item Las codiagonales inferiores tienen ceros al final.
\end{itemize}
\end{frame}
\begin{frame}
\frametitle{Nueva representación de la matriz}
La matriz en banda se puede representar por una matriz no cuadrada del tipo
\\
\bigskip
\pause
\begin{align*}
\begin{blockarray}{ccccccccc}
\begin{block}{c(cccccccc)}
g & 0 & 0 & 3 & 5 & 9 & 4 & -2 & -6 \\
c & 0 & -5 & -3 & 5 & 5 & -3 & 1 & 7 \\
d & 1 & 2 & 1 & -1 & 2 & 0 & 2 & 1 \\
f & 3 & -2 & 9 & 0 & 2 & -3 & 9 & 0 \\
\end{block}
\end{blockarray}
\end{align*}
\end{frame}
\begin{frame}
\frametitle{Solución a sistemas tridiagonales}
Para la solución de sistemas tridiagonales podemos ocupar el estudio previo para desarrollar un código que resuelva este tipo de matrices.
\\
\bigskip
La gran ventaja que tenemos es que el tipo de matrices especiales reduce y simplifica las operaciones que se tienen que implementar.
\end{frame}
\begin{frame}
\frametitle{Apoyo con \texttt{scipy.linalg}}
Ocuparemos una de las funciones incluidas en el módulo \funcionazul{sla} que resuelve sistemas en banda.
\\
\bigskip
Recordemos que los sistemas en banda pueden ser simétricos o no, pero la función que veremos resuelve una matriz en banda general.
\end{frame}
\begin{frame}[fragile]
\frametitle{Resolviendo matrices en banda}
Para resolver un sistema de ecuaciones con una matriz en banda, usaremos la función \funcionazul{sla.solve\_banded}.
\\
\bigskip
\pause
La sintaxis mínima para esta función es la siguiente
\begin{alltt}
\funcionazul{solve\_banded((l,u), cm, rhs)}
\end{alltt}
\end{frame}
\begin{frame}[fragile]
\frametitle{Resolviendo matrices en banda}
\begin{alltt}
\funcionazul{solve\_banded((l,u), cm, rhs)}
\end{alltt}
Donde:
\begin{itemize}[<+->]
\item \funcionazul{(l, u)} es una tupla, donde $l$ es el número de codiagonales inferiores no nulas, y $u$ es el número de codiagonales superiores no nulas.
\item \funcionazul{cm} es la matriz en banda de coeficientes.
\item \funcionazul{rhs} es el vector de constantes del lado derecho.
\end{itemize}
\end{frame}
\subsection{Ejercicio 1}
\begin{frame}[plain]
\frametitle{Ejercicio 1 - Matriz en banda}
Resuelve el siguiente sistema algebraico
\begin{align*}
\begin{bmatrix}
1 & -5 & 3 & 0 & 0 & 0 & 0 & 0 \\
3 & 2 & -3 & 5 & 0 & 0 & 0 & 0 \\
0 & -2 & 1 & 5 & 9 & 0 & 0 & 0 \\
0 & 0 & 9 & -1 & 5 & 4 & 0 & 0 \\
0 & 0 & 0 & 0 & 2 & -3 & -2 & 0 \\
0 & 0 & 0 & 0 & 2 & 0 & 1 & -6 \\
0 & 0 & 0 & 0 & 0 & -3 & 2 & 7 \\
0 & 0 & 0 & 0 & 0 & 0 & 9 & 1 
\end{bmatrix}
\begin{bmatrix}
x_{1} \\
x_{2} \\
x_{3} \\
x_{4} \\
x_{5} \\
x_{6} \\
x_{7} \\
x_{8}
\end{bmatrix}
=
\begin{bmatrix}
5 \\
3 \\
-3 \\
2 \\
0 \\
7 \\
8 \\
1
\end{bmatrix}
\end{align*}
\end{frame}
\begin{frame}
\frametitle{Nueva representación de la matriz}
Pasamos la matriz en banda inicial a una representación de una matriz no cuadrada:
\\
\bigskip
\pause
\begin{align*}
\begin{blockarray}{ccccccccc}
\begin{block}{c(cccccccc)}
g & 0 & 0 & 3 & 5 & 9 & 4 & -2 & -6 \\
c & 0 & -5 & -3 & 5 & 5 & -3 & 1 & 7 \\
d & 1 & 2 & 1 & -1 & 2 & 0 & 2 & 1 \\
f & 3 & -2 & 9 & 0 & 2 & -3 & 9 & 0 \\
\end{block}
\end{blockarray}
\end{align*}
\end{frame}
\begin{frame}
\frametitle{Valores para la función}
El acomodo de la matriz inicial nos permite identificar el valor de la tupla \funcionazul{(l, u)}:
\setbeamercolor{item projected}{bg=blue!70!black,fg=yellow}
\setbeamertemplate{enumerate items}[circle]
\begin{enumerate}[<+->]
\item Número de codiagonales inferiores no nulas $l =  1$
\item Número de codiagonales superiores no nulas $u = 2$
\end{enumerate}
\end{frame}
\begin{frame}[allowframebreaks, fragile, plain]
\frametitle{Orden de los vectores renglón}
Hay que tener cuidado al momento de organizar los vectores renglón que conformarán el arreglo \funcionazul{cm}:
\begin{lstlisting}[caption=Definición de los vectores renglón, basicstyle=\linespread{1.1}\ttfamily=\small, columns=fullflexible]
import scipy.linalg as sla
import numpy as np

g = [0. ,0, 3, 5, 9, 4, -2, -6]
c = [0., -5, -3, 5, 5, -3, 1, 7]
d = [1., 2, 1, -1, 2, 0, 2, 1]
e = [3., -2, 9, 0, 2, -3, 9, 0]

b = np.array([5., 3, -3, 2, 0, 7, 8, 1])

cm = np.array([g, c, d, e])

x = sla.solve_banded((1,2), cm, b)
\end{lstlisting}
\end{frame}
\begin{frame}
\frametitle{Solución al problema}
El vector $x$ contiene los $x_{i}$ que son solución al sistema de ecuaciones inicial.
\\
\bigskip
Implementado una pequeña rutina para visualizar el contenido tendremos la siguiente tabla:
\end{frame}
\begin{frame}
\frametitle{Solución al problema}
\fontsize{12}{12}\selectfont
\begin{table}
\begin{tabular}{c c}
$x_{i}$ & Solución \\ \hline
$x_{0}$ & $200.639937$ \\ \hline
$x_{1}$ & $-25.491352$ \\ \hline
$x_{2}$ & $-107.698899$ \\ \hline
$x_{3}$ & $-174.206761$ \\ \hline
$x_{4}$ & $102.750000$ \\ \hline
$x_{5}$ & $70.833333$ \\ \hline
$x_{6}$ & $-3.500000$ \\ \hline
$x_{7}$ & $32.5000$ \\
\end{tabular}
\end{table}
\end{frame}
\section{Ejercicios a cuenta}
\frame{\tableofcontents[currentsection, hideothersubsections]}
\subsection{Ejercicios a resolver}
\begin{frame}
\frametitle{Para resolver los ejercicios a cuenta}
Ya tienes los elementos necesarios para resolver los ejercicios a cuenta para este tema.
\\
\bigskip
Procura seguir el procedimiento de solución para cada ejercicio.
\end{frame}
\begin{frame}
\frametitle{Ejercicio 1}
Resolver la ecuación $\mathbf{Ax} = \mathbf{b}$ con el método de eliminación de Gauss, donde
\begin{align*}
\mathbf{A} =
\begin{pmatrix}
0 & 0 & 2 & 1 & 2 \\
0 & 1 & 0 & 2 & -1 \\
1 & 2 & 0 & -2 & 0 \\
0 & 0 & 0 & -1 & 1 \\
0 & 1 & -1 & 1 & -1
\end{pmatrix} \hspace{1.5cm}
\mathbf{b} =	
\begin{pmatrix}
1 \\
1 \\
-4 \\
-2 \\
-1
\end{pmatrix}
\end{align*}
\end{frame}
\begin{frame}
\frametitle{Ejercicio 2}
Encontrar $\mathbf{L}$ y $\mathbf{U}$ tales que:
\begin{align*}
\mathbf{A} = \mathbf{LU} =
\begin{pmatrix}
4 & -1 & 0 \\
-1 & 4 & -4 \\
0 & -1 & 4
\end{pmatrix}
\end{align*}
usando a) la descomposición de Doolittle y b) la descomposición de Choleski.
\end{frame}
\begin{frame}
\frametitle{Ejercicio 3}
Resolver $\mathbf{Ax} = \mathbf{b}$ por el método de descomposición de Doolittle, donde
\begin{align*}
\mathbf{A} =
\begin{pmatrix}
2.34 & -4.10 & 1.78 \\
-1.98 & 3.47 & -2.22 \\
2.36 & -15.17 & 6.18 \\
\end{pmatrix} \hspace{1.25cm}
\mathbf{b} =	
\begin{pmatrix}
0.02 \\
-0.73 \\
-6.63
\end{pmatrix}
\end{align*}
\end{frame}
\begin{frame}
\frametitle{Ejercicio 4}
Determinar $\mathbf{L}$ que resulta de la descomposición de Choleski para la matriz diagonal
\begin{align*}
\begin{pmatrix}
\alpha_{1} & 0 & 0 \ldots \\
0 & \alpha_{2} & 0 & \ldots \\
0 & 0 & \alpha_{3} & \ldots \\
\vdots & \vdots &\vdots & \ddots
\end{pmatrix}
\end{align*}
\fontsize{12}{12}\selectfont
Este ejercicio hay que resolverlo a mano y enviar la solución ya sea escanenado o tomando una foto de la(s) hoja(s) que ocupes.
\end{frame}
\begin{frame}
\frametitle{Ejercicio 5}
Un ejemplo clásico de una matriz mal condicionada, es la matriz de Hilbert
\begin{align*}
\mathbf{A} = 
\begin{bmatrix}
1 & 1/2 & 1/3 & \ldots \\
1/2 & 1/3 & 1/4 & \ldots \\
1/3 & 1/4 & 1/5 & \ldots \\
\vdots & \vdots & \vdots & \ddots
\end{bmatrix}
\end{align*}
\end{frame}
\begin{frame}
\frametitle{Ejercicio 5}
Escribe un programa en python que resuelva el sistema $\mathbf{A}\mathbf{x} = \mathbf{b}$ por el método de Doolittle, donde $\mathbf{A}$ es una matriz de Hilbert arbitraria de $n \times n$ y \[ b_{i} = \sum_{j=1}^{n} A_{ij} \]
\end{frame}
\begin{frame}
\frametitle{Ejercicio 5}
El programa no debe de utilizar un valor inicial para $n$, sino que en tiempo de ejecución, se determine para qué valor de $n$, la solución es exacta al menos hasta seis cifras significativas comparada con la solución exacta
\[\mathbf{x} = [1 \hspace{0.2cm} 1 \hspace{0.2cm} 1 \hspace{0.2cm} \ldots ]^{T} \]
\end{frame}
\begin{frame}
\frametitle{Ejercicio 6}
Resolver el sistema de ecuaciones simétricas tridiagonales:
\begin{align*}
4 x_{1} - x_{2} &= 9 \\
-x_{i-1} + 4x_{i} - x_{i+1} &= 5, \hspace{1cm} i=2,\ldots,n-1 \\
-x_{n-1} + 4 x_{n} &= 5
\end{align*}
con $n=10$.
\end{frame}
\begin{frame}
\frametitle{Ejercicio 7}
El sistema mostrado en la siguiente figura consiste en $n$ resortes lineales que soportan $n$ masas.
\\
\bigskip
La constante de los resortes se indican por $k_{i}$, mientras que el peso de las masas, es $W_{i}$ y $x_{i}$ son los desplazamientos de las masas (medidos de la posición donde el resorte no está deformado).
\end{frame}
\begin{frame}
\frametitle{Ejercicio 7}
\begin{figure}[H]
	\centering
	\includestandalone[scale=0.75]{Figuras/Sistema_Masas_Tarea}
\end{figure}
\end{frame}
\begin{frame}
\frametitle{Ejercicio 7}
La llamada \emph{formulación de desplazamiento} se obtiene escribiendo la ecuación de equilibrio para cada masa y sustituyendo $F_{i} = k_{i}(x_{i + 1} - x_{i})$ para la fuerza en los resortes. 
\end{frame}
\begin{frame}
\frametitle{Ejercicio 7}
El resultado es un conjunto de ecuaciones simétricas y tridiagonal:
\fontsize{12}{12}\selectfont
\begin{align*}
(k_{1} + k_{2}) \: x_{1} - k_{2} \: x_{2} &= \: W_{1} \\
-k_{i} \: x_{i-1} + (k_{i} + k_{i+1}) \: x_{i} - k_{i+1} \: x_{i+1} &= \:  W_{i}, \\
-k_{n} \:  x_{n-1} + k_{n} \: x_{n} &= \: W_{n} \\
i &= 2, 3, \ldots, n-1
\end{align*}
\end{frame}
\begin{frame}
\frametitle{Ejercicio 7}
Escribe un programa que resuelva este conjunto de ecuaciones para los valores dados de $n$, $k$ y $W$. Considera $n = 5$ y 
\begin{align*}
k_{1} &= k_{2} = k_{3} = 10 \mbox{ N/mm} \\
k_{4} &= k_{5} = 5 \mbox{ N/mm} \\
W_{1} &= W_{3} = W_{5} = 100 \mbox{ N} \\
W_{2} &= W_{4} = 50 \mbox{ N}
\end{align*}
\end{frame}
\end{document}