\documentclass[12pt]{beamer}
\newenvironment{ConCodigo}[1]
  {\begin{frame}[fragile,environment=ConCodigo]{#1}}
  {\end{frame}}
\graphicspath{{Imagenes/}{../Imagenes/}}
\usepackage[utf8]{inputenc}
\usepackage[spanish]{babel}
\usepackage{hyperref}
\usepackage{etex}
%\reserveinserts{28}
\usepackage{amsmath}
\usepackage{amsthm}
\usepackage{mathtools}
\usepackage{multicol}
\usepackage{multirow}
\usepackage{tabulary}
\usepackage{booktabs}
\usepackage{nccmath}
\usepackage{physics}
\usepackage{biblatex}
\usepackage[outdir=./]{epstopdf}
%\epstopdfsetup{outdir=./}
\usepackage{graphicx}
%\usepackage{enumitem,xcolor}
\usepackage{siunitx}
%\sisetup{scientific-notation=true}
%\usepackage{fontspec}
\usepackage{lmodern}
\usepackage{float}
\usepackage[format=hang, font=footnotesize, labelformat=parens]{caption}
\usepackage[autostyle,spanish=mexican]{csquotes}
\usepackage{standalone}
\usepackage{blkarray}
\usepackage{algorithm}
\usepackage{algorithmic}
\usepackage{tikz}
\usepackage[siunitx, RPvoltages]{circuitikz}
\usetikzlibrary{arrows,patterns,shapes}
\usetikzlibrary{decorations.markings}
\usetikzlibrary{arrows}
\usepackage{color}
\usepackage{xcolor}
%\usepackage{beton}
%\usepackage{euler}
%\usepackage[T1]{fontenc}
\usepackage[sfdefault]{roboto}  %% Option 'sfdefault' only if the base font of the document is to be sans serif
\usepackage[T1]{fontenc}
\renewcommand*\familydefault{\sfdefault}
\DeclareGraphicsExtensions{.pdf,.png,.jpg}
\usepackage{hyperref}
\renewcommand {\arraystretch}{1.5}
\newcommand{\python}{\texttt{python}}
\usefonttheme[onlymath]{serif}
\setbeamertemplate{navigation symbols}{}
\usetikzlibrary{patterns}
\usetikzlibrary{decorations.markings}
\tikzstyle{every picture}+=[remember picture,baseline]
%\tikzstyle{every node}+=[inner sep=0pt,anchor=base,
%minimum width=2.2cm,align=center,text depth=.15ex,outer sep=1.5pt]
%\tikzstyle{every path}+=[thick, rounded corners]
\setbeamertemplate{caption}[numbered]
\newcommand{\ptm}{\fontfamily{ptm}\selectfont}
%Se usa la plantilla Warsaw modificada con spruce
\mode<presentation>
{
  \usetheme{Warsaw}
  \setbeamertemplate{headline}{}
  \useoutertheme{default}
  \usecolortheme{albatross}
  \setbeamercovered{invisible}
}
% \AtBeginSection[]
% {
% \begin{frame}<beamer>{Contenido}
% \normalfont\mdseries
% \tableofcontents[currentsection]
% \end{frame}
% }

\include{pre_codigo}
\makeatletter
\setbeamertemplate{footline}
%\setbeamercolor{title in head/foot}{fg=Green}
{
  \leavevmode%
  \hbox{%
  \begin{beamercolorbox}[wd=.333333\paperwidth,ht=2.25ex,dp=1ex,center]{author in head/foot}%
    \usebeamerfont{author in head/foot} \textcolor{white}{\insertsection}
  \end{beamercolorbox}}%
  \begin{beamercolorbox}[wd=.333333\paperwidth,ht=2.25ex,dp=1ex,center]{title in head/foot}%
    \usebeamerfont{title in head/foot} \textcolor{white}\insertsubsection
  \end{beamercolorbox}%
  \begin{beamercolorbox}[wd=.333333\paperwidth,ht=2.25ex,dp=1ex,right]{date in head/foot}%
    \usebeamerfont{date in head/foot} \textcolor{white}\insertshortdate{}\hspace*{2em}
    \textcolor{white}\insertframenumber{} / \textcolor{white}\inserttotalframenumber\hspace*{2ex} 
  \end{beamercolorbox}}%
  \vskip0pt%
\makeatother
\normalfont
\usepackage{ccfonts}% http://ctan.org/pkg/{ccfonts}
\usepackage[T1]{fontenc}% http://ctan.or/pkg/fontenc
\renewcommand{\rmdefault}{cmr}% cmr = Computer Modern Roman
\linespread{1.3}
\title{Algebra matricial 2}
\subtitle{Curso de Física Computacional}
\author[]{M. en C. Gustavo Contreras Mayén}
\newcounter{saveenumi}
\newcommand{\seti}{\setcounter{saveenumi}{\value{enumi}}}
\newcommand{\conti}{\setcounter{enumi}{\value{saveenumi}}}
\resetcounteronoverlays{saveenumi}

\begin{document}
\newcommand{\localtextbulletone}{\textcolor{gray}{\raisebox{.45ex}{\rule{.6ex}{.6ex}}}}
\maketitle
\fontsize{14}{14}\selectfont
\spanishdecimal{.}
\begin{frame}{Contenido}
\tableofcontents[pausesections]
\end{frame}
\section{Métodos iterativos}
\subsection{El método de Jacobi}
\begin{frame}
\frametitle{El método de Jacobi}
El método de Jacobi es un procedimiento iterativo relativamente simple que extrae todos los valores propios y vectores propios de una matriz simétrica.
\\
\bigskip
Es útil cuando se tienen matrices pequeñas (digamos, menos de $50 \times 50$), porque el esfuerzo computacional se incrementa muy rápidamente con el tamaño de la matriz.
\\
\medskip
\emph{La fuerza principal del método es su robustez.}
\end{frame}
\subsubsection{Transformación de Similitud y Diagonalización}
\begin{frame}
\frametitle{Transformación de Similitud y Diagonalización}
Consideremos el problema estándar de valores propios de la matriz
\begin{equation}
\mathbf{A \; x} = \lambda \; \mathbf{x}
\label{eq:ecuacion_09_04}
\end{equation}
donde la matriz $\mathbf{A}$ es simétrica. Apliquemos la transformación
\begin{equation}
\mathbf{x} = \mathbf{P \; x^{*}}
\label{eq:ecuacion_09_05}
\end{equation}
donde $\mathbf{P}$ es una matriz no singular.
\end{frame}
\begin{frame}
Sustituyendo la ec. (\ref{eq:ecuacion_09_05}) en la ec. (\ref{eq:ecuacion_09_04}) y multiplicando por la izquierda de cada lado de la igualdad por $\mathbf{P}^{-1}$, se obtiene
\[ \mathbf{P}^{-1} \; \mathbf{A \; P \; x^{*}} = \lambda \; \mathbf{P}^{-1} \; \mathbf{P \; x}^{*} \]
\pause
o equivalentemente
\begin{equation}
\mathbf{A}^{*} \; \mathbf{x}^{*} = \lambda \; \mathbf{x}^{*}
\label{eq:ecuacion_09_06}
\end{equation}
donde $\mathbf{A}^{*} = \mathbf{P}^{-1} \; \mathbf{A \; P}$
\end{frame}
\begin{frame}
Dado que $\lambda$ no se modifica/altera durante la transformación, los valores propios de $\mathbf{A}$ son los mismos valores propios de $\mathbf{A}^{*}$.
\\
\bigskip
Las matrices que tienen los mismos valores propios se consideran similares, y la transformación entre ellas se llama transformación de similitud.
\end{frame}
\begin{frame}
Las transformaciones de similitud se utilizan con frecuencia para cambiar un problema de valores propios a una forma que es más fácil de resolver.
\end{frame}
\begin{frame}
Supongamos que nos arreglamos por algo para encontrar una matriz $\mathbf{P}$ que diagonaliza $\mathbf{A}^{*}$.
\\
\bigskip
Las ecuaciones (\ref{eq:ecuacion_09_06}) son entonces
\[  \begin{bmatrix}
A_{11}^{*} - \lambda & 0          & \ldots & 0 \\
0          & A_{22}^{*} - \lambda & \ldots & 0 \\
\vdots 	   & \vdots & \ddots & \vdots \\
0 & 0 & \ldots & A_{nn}^{*} - \lambda
\end{bmatrix} = 
\begin{bmatrix}
x_{1}^{*} \\
x_{2}^{*} \\
\vdots \\
x_{n}^{*} 
\end{bmatrix} = 
\begin{bmatrix}
0 \\
0 \\
\vdots \\
0
\end{bmatrix}\]
\end{frame}
\begin{frame}
Que tiene por soluciones
\begin{equation}
\lambda_{1} =  A_{11}^{*} \hspace{1cm} \lambda_{2} =  A_{22}^{*} \hspace{1cm} \lambda_{n} =  A_{nn}^{*}
\label{eq:ecuacion_09_07}
\end{equation}
\[ \mathbf{x}_{1}^{*} = \begin{bmatrix}
1 \\
0 \\
\vdots \\
0
\end{bmatrix} \hspace{1.5cm}
\mathbf{x}_{2}^{*} = \begin{bmatrix}
0 \\
1 \\
\vdots \\
0
\end{bmatrix} \ldots \hspace{1.5cm}
\mathbf{x}_{n}^{*} = \begin{bmatrix}
0 \\
0 \\
\vdots \\
1
\end{bmatrix} \]
\pause
o
\[ \mathbf{X^{*}} = \left[ \mathbf{x}_{1}^{*} \hspace{0.4cm} \mathbf{x}_{2}^{*} \hspace{0.4cm} \ldots \hspace{0.4cm} \mathbf{x}_{n}^{*} \right] = \mathbf{I} \]
\end{frame}
\begin{frame}
De acuerdo a la ec. (\ref{eq:ecuacion_09_05}), los valores propios de $\mathbf{A}$ son
\begin{equation}
\mathbf{X} = \mathbf{P \; X}^{*} = \mathbf{P \; I} = \mathbf{P}
\label{eq:ecuacion_09_08}
\end{equation}
Por lo tanto la matriz de transformación $\mathbf{P}$ contiene los vectores propios de $\mathbf{A}$, y los valores propios de $\mathbf{A}$ son los elementos de la diagonal de $\mathbf{A}^{*}$.
\end{frame}
\subsubsection{Rotación de Jacobi}
\begin{frame}
\frametitle{Rotación de Jacobi}
Una transformación especial de similitud es la rotación plana
\begin{equation}
\mathbf{x} = \mathbf{R \; x}^{*}
\label{eq:ecuacion_09_09}
\end{equation}
donde
\end{frame}
\begin{frame}
\begin{equation}
\mathbf{R} = 
\begin{blockarray}{ccccccccc}
 &  & k &  & \ell &  & & \\
\begin{block}{(cccccccc)c}
  1 & 0 & 0  & 0 & 0 & 0 & 0 & 0 &   \\
  0 & 1 & 0  & 0 & 0 & 0 & 0 & 0 &   \\
  0 & 0 & c  & 0 & 0 & s & 0 & 0 & k \\
  0 & 0 & 0  & 1 & 0 & 0 & 0 & 0 &   \\
  0 & 0 & 0  & 0 & 1 & 0 & 0 & 0 &   \\
  0 & 0 & -s & 0 & 0 & c & 0 & 0 & \ell   \\
  0 & 0 & 0  & 0 & 0 & 0 & 1 & 0 &   \\
  0 & 0 & 0  & 0 & 0 & 0 & 0 & 1 &   \\
\end{block}
\end{blockarray}
\end{equation}
\end{frame}
\begin{frame}
La matriz $\mathbf{R}$ se llama \emph{matriz de rotación de Jacobi}.
\\
\bigskip
Nótese que $\mathbf{R}$ es una matriz de identidad modificada por los términos $c = \cos \theta$ y $s = \sin \theta$ que aparecen en las intersecciones de columnas/filas $k$ y $\ell$, donde $\theta$ es el ángulo de rotación.
\end{frame}
\begin{frame}
La matriz de rotación tiene la propiedad útil de ser ortogonal, lo que significa que
\begin{equation}
\mathbf{R}^{-1} = \mathbf{R}^{T}
\label{eq:ecuacion_09_11}
\end{equation}
Una consecuencia de la ortogonalidad es que la transformación en la ecuación (\ref{eq:ecuacion_09_05}) tiene la característica esencial de una rotación: \emph{conserva la magnitud del vector}, es decir $\vert \mathbf{x} \vert = \vert \mathbf{x}^{*} \vert$
\end{frame}
\begin{frame}
La transformación de similitud correspondiente al plano de rotación en la ec. (\ref{eq:ecuacion_09_09}) es
\begin{equation}
\mathbf{A}^{*} = \mathbf{R}^{-1} \; \mathbf{A \; R} = \mathbf{R}^{T} \; \mathbf{A \; R}
\label{eq:ecuacion_09_12}
\end{equation}
\pause
La matriz $\mathbf{A}^{*}$ no solo tiene los mismos valores propios de la matriz original $\mathbf{A}$, sino que también por la ortogonalidad de $\mathbf{R}$, es simétrica.
\end{frame}
\begin{frame}
La transformación en la ec. (\ref{eq:ecuacion_09_12}) cambia solo los renglones/columna $k$ y $\ell$ de $\mathbf{A}$.
\\
\bigskip
Las fórmulas para esos cambios son
\fontsize{12}{12}\selectfont
\begin{equation}
\begin{aligned}
A_{k k}^{*} &= c^{2} \; A_{k k} + s^{2} \; A_{\ell \ell} - 2 \; c \; s \; A_{k \ell} \\
A_{\ell \ell}^{*} &= c^{2} \; A_{\ell \ell} + s^{2} \; A_{k k} + 2 \; c \; s \; A_{k \ell} \\
A_{k \ell}^{*} &= A_{\ell k}^{*} = (c^{2} - s^{2}) \; A_{k \ell} + c \; s \; (A_{k k} - A_{\ell \ell}) \\
A_{k i}^{*} &= A_{i k}^{*} = c \; A_{k i } - s \; A_{k i} \hspace{1cm} i \neq k, i \neq \ell \\
A_{\ell i}^{*} &= A_{i \ell}^{*} = c \; A_{\ell i } + s \; A_{k i} \hspace{1cm} i \neq k, i \neq \ell
\end{aligned}
\label{eq:ecuacion_09_13}
\end{equation}
\end{frame}
\subsubsection{Diagonalización de Jacobi}
\begin{frame}
\frametitle{Diagonalización de Jacobi}
El ángulo $\theta$ en la matriz de rotación de Jacobi se puede elegir de modo que $A_{k \ell}^{*} = A_{\ell k}^{*} = 0$.
\\
\bigskip
Esto sugiere la siguiente idea: ¿Por qué no diagonalizar $\mathbf{A}$ haciendo un bucle a través de todos los términos por fuera de la diagonal y cero uno por uno?
\end{frame}
\begin{frame}
Esto es exactamente lo que hace el método de Jacobi.
\\
\bigskip
Sin embargo, hay un inconveniente importante: la transformación que aniquila un término fuera de la diagonal deshace algunos de los ceros previamente creados.
\end{frame}
\begin{frame}
Afortunadamente, resulta que los términos fuera de la diagonal que reaparecen serán menores que antes.
\\
\bigskip
Así, el método de Jacobi es un procedimiento iterativo que aplica repetidamente las rotaciones de Jacobi hasta que los términos fuera de la diagonal se anulan.
\end{frame}
\begin{frame}
La matriz final de transformación $\mathbf{P}$ es la acumulación de rotaciones individuales $\mathbf{R}_{i}$
\begin{equation}
\mathbf{P} = \mathbf{R}_{1} \cdot \mathbf{R}_{2} \cdot \mathbf{R}_{2} \ldots
\label{eq:ecuacion_09_14}
\end{equation}
Las columnas de $\mathbf{P}$ terminan siendo los vectores propios de $\mathbf{A}$, y los elementos diagonales de $\mathbf{A}^{*} = \mathbf{P}^{T} \; \mathbf{A \; P}$ se convierten en los vectores propios.
\end{frame}
\begin{frame}
Veamos un poco más a detalle la rotación de Jacobi.
\\
\bigskip
De la ec. (\ref{eq:ecuacion_09_13}) vemos que $A_{k \ell}^{*} = 0$ si
\begin{equation}
(c^{2} - s^{2}) \; A_{k \ell} + c \; s \; (A_{k k} - A_{\ell \ell}) = 0
\label{eq:ecuacion_a}
\end{equation}
Usando las identidades trigonométricas
\begin{align*}
c^{2} - s^{2} &= \cos^{2} \theta - \sin^{2} \theta = \cos 2 \theta \\
c \: s &= \cos \theta \; \sin \theta = \dfrac{1}{2} \sin 2 \: \theta
\end{align*}
\end{frame}
\begin{frame}
La ec. (\ref{eq:ecuacion_a}) queda como
\begin{equation}
\tan 2 \theta = - \dfrac{2 A_{k \ell}}{A_{k k} - A_{\ell \ell}}
\label{eq:ecuacion_b}
\end{equation}
la cual podría ser resuelta para $\theta$, seguida por el cálculo de $c = \cos \theta$ y $s = \sin \theta$.
\end{frame}
\begin{frame}
Sin embargo, el procedimiento descrito a continuación conduce a un mejor algoritmo.
\\
\bigskip
Introducimos la notación
\begin{equation}
\phi = \cot 2 \theta = - \dfrac{A_{k k} - A_{\ell \ell}}{2 \; A_{k \ell}}
\label{eq:ecuacion_09_15}
\end{equation}
y usando la identidad trigonométrica
\[ \tan 2 \theta = \dfrac{2}{(1 - t^{2})} \]
donde $t = \tan \theta$
\end{frame}
\begin{frame}
La ec. (\ref{eq:ecuacion_b}) puede escribirse como
\[ t^{2} +  2 \phi \; t - 1 = 0 \]
la cual tiene raíces
\[ t =  - \phi \pm \sqrt{\phi^{2} + 1} \]
\end{frame}
\begin{frame}
Se ha encontrado que la raíz $\vert t \vert \leq 1$, que corresponde a $\vert \theta \vert \leq 45\si{\degree}$, conduce a la transformación más estable.
\\
\bigskip
Por lo tanto, elegimos el signo más si $\phi > 0$  y el signo menos si $\phi \leq 0$.
\end{frame}
\begin{frame}
Lo que equivale a usar
\[ t = sgn(\phi) \left( - \vert \phi \vert + \sqrt{\phi^{2} + 1} \right) \]
\pause
Para evitar un error excesivo por el redondeo si $\phi$ es grande, multiplicamos ambos lados de la ecuación por $\vert \phi \vert + \sqrt{\phi^{2} + 1}$, lo que nos lleva a
\begin{equation}
t = \dfrac{sgn(\phi)}{\vert \phi \vert + \sqrt{\phi^{2} + 1}}
\label{eq:ecuacion_09_16a}
\end{equation}
\end{frame}
\begin{frame}
En el caso de que $\phi$ sea muy grande, debemos de sustituir la ec. (\ref{eq:ecuacion_09_16a}) por la aproximación
\begin{equation}
t = \dfrac{1}{2 \; \phi}
\label{eq:ecuacion_09_16b}
\end{equation}
\pause
para prevenir un desborde en el cálculo de $\phi^{2}$.
\\
\bigskip
Una vez calculado $t$, podemos usar la relación trigonométrica $\tan \theta = \sin \theta / \cos \theta = \sqrt{1 - \cos^{2}} / \cos \theta$
\end{frame}
\begin{frame}
Para obtener
\begin{equation}
c = \dfrac{1}{\sqrt{1 + t^{2}}},\hspace{1cm} s = t \: c
\label{eq:ecuacion_09_17}
\end{equation}
\end{frame}
\begin{frame}
Ahora podemos mejorar las fórmulas de transformación en las ecs. (\ref{eq:ecuacion_09_13}).
\\
\bigskip
Resolviendo de la ec. (\ref{eq:ecuacion_a}) para $A_{\ell \ell}$, se obtiene
\begin{equation}
A_{\ell \ell} = A_{k k} + A_{k \ell} \: \dfrac{c^{2} - s^{2}}{c \: s}
\label{eq:ecuacion_c}
\end{equation}
\end{frame}
\begin{frame}
Re-emplazando todas las veces que aparece $A_{\ell \ell}$ de la ec. (\ref{eq:ecuacion_c}) y simplificando, las fórmulas de transformación de la ec. (\ref{eq:ecuacion_09_13}) se pueden escribir como
\fontsize{12}{12}\selectfont
\begin{align}
\begin{aligned}
A_{k k}^{*} &= A_{k k} - t \; A_{k \ell} \\
A_{\ell \ell}^{*} &= A_{\ell \ell} - t \; A_{k \ell} \\
A_{k \ell}^{*} &= A_{\ell k}^{*} = 0 \\
A_{k i}^{*} &= A_{i k}^{*} = A_{k i} - s \; A_{\ell i} + \tau A_{k i} \hspace{0.5cm} i \neq k, i \neq \ell \\
A_{\ell i}^{*} &= A_{i \ell}^{*} = A_{\ell i} - s \; A_{k i} + \tau A_{\ell i} \hspace{0.5cm} i \neq k, i \neq \ell \\
\end{aligned}
\label{eq:ecuacion_09_18}
\end{align}
\fontsize{14}{14}\selectfont
donde
\fontsize{12}{12}\selectfont
\begin{equation}
\tau = \dfrac{s}{1 + c}
\label{eq:ecuacion_09_19}
\end{equation}
\end{frame}
\begin{frame}
La introducción de $\tau$ nos permitió expresar cada fórmula en la forma, (valor original) $+$ (cambio), que es útil para reducir el error por redondeo.
\\
\bigskip
Al inicio del proceso de diagonalización de Jacobi, la matriz de transformación $\mathbf{P}$ se inicializa a la matriz de identidad.
\end{frame}
\begin{frame}
Cada rotación de Jacobi cambia esta matriz de $\mathbf{P}$ a $\mathbf{P}^{*} = \mathbf{P \: R}$.
\\
\bigskip
Los cambios correspondientes en los elementos de $\mathbf{P}$ puede demostrarse que son (sólo las columnas $k$ y $\ell$ son afectadas) 
\begin{equation}
\begin{aligned}
P_{i k}^{*} &= P_{i k} - s (P_{i \ell} + \tau P_{i k}) \\
P_{i \ell}^{*} &= P_{i \ell} - s (P_{i k} + \tau P_{i \ell})
\end{aligned}
\label{eq:ecuacion_09_20}
\end{equation}
\end{frame}
\begin{frame}
Todavía tenemos que decidir el orden en que los elementos fuera de la diagonal de $\mathbf{A}$ deben ser eliminados.
\\
\bigskip
La idea original de Jacobi era atacar el elemento más grande porque hacerlo resulta en el menor número de rotaciones.
\end{frame}
\begin{frame}
El problema aquí es que $\mathbf{A}$ se tiene que buscar para el elemento más grande antes de cada rotación, que es un proceso que consume mucho tiempo.
\\
\bigskip
Si la matriz es grande, es más rápido recorrerla por filas o columnas y anular todos los elementos por encima de un valor \textcolor{blue}{umbral}. En el siguiente barrido, de disminuye el umbral y el proceso se repite.
\end{frame}
\begin{frame}
Hay varias maneras de elegir el \textcolor{blue}{umbral}.
\\
\bigskip
Calculamos la suma $S$ de los elementos por encima de la diagonal principal de $\mathbf{A}$:
\begin{equation}
S = \sum_{i = 1}^{n - 1} \sum_{j = i + 1}^{n} \vert A_{i j} \vert
\label{eq:ecuacion_(a)}
\end{equation}
\pause
Dado que hay $n(n - 1)/2$ de tales elementos, la magnitud promedio de los elementos fuera de la diagonal es
\[ \dfrac{2 \: S}{n(n - 1)} \]
\end{frame}
\begin{frame}
El \textcolor{blue}{umbral} que usaremos es
\begin{equation}
\mu = \dfrac{0.5 \: S}{n(n - 1)}
\label{eq:ecuacion_(b)}
\end{equation}
que representa la cuarta parte de la magnitud promedio de los elementos que están por fuera de la diagonal.
\end{frame}
\begin{frame}
En resumen, el procedimiento de barrido de Jacobi (usa solamente la parte superior de la matriz), es el siguiente:
\\
\fontsize{11}{11}\selectfont
Calcular el \textcolor{blue}{umbral} $\mu$ usando las ecs. (\ref{eq:ecuacion_(a)}) y (\ref{eq:ecuacion_(b)})
\\
Hacer un barrido de los elementos por fuera de la diagonal de $\mathbf{A}$:
\\
\hspace{0.5cm} Si $\vert A_{ij} \vert \geq \mu$:
\\
\hspace{1cm} Calcular $\phi$, $t$, $c$ y $s$ de las ecs. (\ref{eq:ecuacion_09_15})-(\ref{eq:ecuacion_09_17})
\\
\hspace{1cm} Calcular $\tau$ de la ec. (\ref{eq:ecuacion_09_19})
\\
\hspace{1cm} Modificar los elementos de $\mathbf{A}$ de acuerdo a las ecs. (\ref{eq:ecuacion_09_18})
\\
\hspace{1cm} Actualizar la matriz de transformación $\mathbf{P}$ usando las ecs. (\ref{eq:ecuacion_09_20})
\\
\\
El barrido se realiza hasta que $\mu \leq \varepsilon$, donde $\varepsilon$ es la tolerancia. Normalmente se requieren de 6-10 barridos para alcanzar la convergencia.
\end{frame}

\end{document}