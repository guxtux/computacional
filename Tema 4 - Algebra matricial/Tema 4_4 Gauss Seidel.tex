\documentclass[12pt]{beamer}
\newenvironment{ConCodigo}[1]
  {\begin{frame}[fragile,environment=ConCodigo]{#1}}
  {\end{frame}}
\graphicspath{{Imagenes/}{../Imagenes/}}
\usepackage[utf8]{inputenc}
\usepackage[spanish]{babel}
\usepackage{hyperref}
\usepackage{etex}
%\reserveinserts{28}
\usepackage{amsmath}
\usepackage{amsthm}
\usepackage{mathtools}
\usepackage{multicol}
\usepackage{multirow}
\usepackage{tabulary}
\usepackage{booktabs}
\usepackage{nccmath}
\usepackage{physics}
\usepackage{biblatex}
\usepackage[outdir=./]{epstopdf}
%\epstopdfsetup{outdir=./}
\usepackage{graphicx}
%\usepackage{enumitem,xcolor}
\usepackage{siunitx}
%\sisetup{scientific-notation=true}
%\usepackage{fontspec}
\usepackage{lmodern}
\usepackage{float}
\usepackage[format=hang, font=footnotesize, labelformat=parens]{caption}
\usepackage[autostyle,spanish=mexican]{csquotes}
\usepackage{standalone}
\usepackage{blkarray}
\usepackage{algorithm}
\usepackage{algorithmic}
\usepackage{tikz}
\usepackage[siunitx, RPvoltages]{circuitikz}
\usetikzlibrary{arrows,patterns,shapes}
\usetikzlibrary{decorations.markings}
\usetikzlibrary{arrows}
\usepackage{color}
\usepackage{xcolor}
%\usepackage{beton}
%\usepackage{euler}
%\usepackage[T1]{fontenc}
\usepackage[sfdefault]{roboto}  %% Option 'sfdefault' only if the base font of the document is to be sans serif
\usepackage[T1]{fontenc}
\renewcommand*\familydefault{\sfdefault}
\DeclareGraphicsExtensions{.pdf,.png,.jpg}
\usepackage{hyperref}
\renewcommand {\arraystretch}{1.5}
\newcommand{\python}{\texttt{python}}
\usefonttheme[onlymath]{serif}
\setbeamertemplate{navigation symbols}{}
\usetikzlibrary{patterns}
\usetikzlibrary{decorations.markings}
\tikzstyle{every picture}+=[remember picture,baseline]
%\tikzstyle{every node}+=[inner sep=0pt,anchor=base,
%minimum width=2.2cm,align=center,text depth=.15ex,outer sep=1.5pt]
%\tikzstyle{every path}+=[thick, rounded corners]
\setbeamertemplate{caption}[numbered]
\newcommand{\ptm}{\fontfamily{ptm}\selectfont}
%Se usa la plantilla Warsaw modificada con spruce
\mode<presentation>
{
  \usetheme{Warsaw}
  \setbeamertemplate{headline}{}
  \useoutertheme{default}
  \usecolortheme{albatross}
  \setbeamercovered{invisible}
}
% \AtBeginSection[]
% {
% \begin{frame}<beamer>{Contenido}
% \normalfont\mdseries
% \tableofcontents[currentsection]
% \end{frame}
% }

\include{pre_codigo}
\makeatletter
\setbeamertemplate{footline}
%\setbeamercolor{title in head/foot}{fg=Green}
{
  \leavevmode%
  \hbox{%
  \begin{beamercolorbox}[wd=.333333\paperwidth,ht=2.25ex,dp=1ex,center]{author in head/foot}%
    \usebeamerfont{author in head/foot} \textcolor{white}{\insertsection}
  \end{beamercolorbox}}%
  \begin{beamercolorbox}[wd=.333333\paperwidth,ht=2.25ex,dp=1ex,center]{title in head/foot}%
    \usebeamerfont{title in head/foot} \textcolor{white}\insertsubsection
  \end{beamercolorbox}%
  \begin{beamercolorbox}[wd=.333333\paperwidth,ht=2.25ex,dp=1ex,right]{date in head/foot}%
    \usebeamerfont{date in head/foot} \textcolor{white}\insertshortdate{}\hspace*{2em}
    \textcolor{white}\insertframenumber{} / \textcolor{white}\inserttotalframenumber\hspace*{2ex} 
  \end{beamercolorbox}}%
  \vskip0pt%
\makeatother
\normalfont
\usepackage{ccfonts}% http://ctan.org/pkg/{ccfonts}
\usepackage[T1]{fontenc}% http://ctan.or/pkg/fontenc
\renewcommand{\rmdefault}{cmr}% cmr = Computer Modern Roman
\linespread{1.3}
\title{Algebra matricial - El método de Gauss-Seidel}
\subtitle{Curso de Física Computacional}
\author[]{M. en C. Gustavo Contreras Mayén}
\newcounter{saveenumi}
\newcommand{\seti}{\setcounter{saveenumi}{\value{enumi}}}
\newcommand{\conti}{\setcounter{enumi}{\value{saveenumi}}}
\resetcounteronoverlays{saveenumi}
\begin{document}
\newcommand{\localtextbulletone}{\textcolor{gray}{\raisebox{.45ex}{\rule{.6ex}{.6ex}}}}
\maketitle
\fontsize{14}{14}\selectfont
\spanishdecimal{.}
\begin{frame}{Contenido}
\tableofcontents[pausesections]
\end{frame}
\section{Método Gauss-Seidel}
\subsection{Introducción}
\begin{frame}
\frametitle{Método Gauss-Seidel}
Las ecuaciones $\mathbf{A \; x} = \mathbf{b}$ en su forma escalar, se escriben de la siguiente manera:
\[ \sum_{j=1}^{n} A_{ij} \;  x_{j} = b_{i} \hspace{1cm} i = 1,2,\ldots,n \]
Despejando el término que contiene a $x_{i}$ de la suma, obtenemos
\[ A_{ii} \; x_{i} + \sum_{\substack{j=1 \\ j \neq i}}^{n} A_{ij} \; x_{j} = b_{i} \hspace{1cm} i= 1, 2,\ldots,n \]
\end{frame}
\begin{frame}
Resolviendo para $x_{i}$, resulta
\[ x_{i} = \dfrac{1}{A_{ii}} \left( b_{i} - \sum_{\substack{ j=1 \\ j \neq i}}^{n} A_{ij} \; x_{j} \right) \hspace{1cm} i = 1, 2, \ldots, n \]
\end{frame}
\begin{frame}
La ecuación anterior sugiere el siguiente esquema iterativo:
\[ x_{i}\leftarrow \dfrac{1}{A_{ii}} \left( b_{i} - \sum_{\substack{ j=1 \\ j \neq i}}^{n} A_{ij}  \; x_{j} \right) \hspace{1cm} i = 1, 2, \ldots, n \]
Iniciamos eligiendo un vector $\mathbf{x}$. Si la suposición inicial no fue buena, podemos elegir de manera aleatoria a $\mathbf{x}$.
\end{frame}
\begin{frame}
La ecuación anterior se utiliza nuevamente para calcular cada uno de los elementos de $x$, utilizando siempre los últimos valores disponibles de $x_{j}$.
\\
\bigskip
Esto completa un ciclo de iteración.
\end{frame}
\begin{frame}
El procedimiento se repite hasta que los cambios en $\mathbf{x}$ con cada iteración sucesiva se  vuelven lo suficientemente pequeños.
\\
\medskip
Es posible mejorar la convergencia del método de Gauss-Seidel con una técnica conocida como \emph{\textcolor{blue}{relajación}}.
\end{frame}
\begin{frame}
\frametitle{La relajación}
La idea es tomar un nuevo valor de $x_{i}$ como un promedio ponderado de su valor anterior y el valor predicho por la ecuación anterior. La correspondiente fórmula iterativa, es
\[  x_{i}\leftarrow \dfrac{\omega}{A_{ii}} \left( b_{i} - \sum_{\substack{ j=1 \\ j \neq i}}^{n} A_{ij} x_{j} \right) + (1 - \omega) x_{i} \hspace{0.75cm} i=1,2,\ldots,n \]
donde $\omega$ es el \emph{factor de relajación}.
\end{frame}
\begin{frame}
Nótese que
\setbeamercolor{item projected}{bg=red!70!black,fg=white}
\setbeamertemplate{enumerate items}[circle]
\begin{enumerate}[<+->]
\item  Si $\omega=1$, no se presenta la relajación.
\item Si $\omega<1$, la ecuación de relajación, representa una interpolación entre los valores anteriores de $x_{i}$ y el valor dado por la ecuación inicial. A esto se le llama \emph{subrelajación}.
\item Si $\omega>1$, tenemos una extrapolación o sobrerelajación.
\end{enumerate}
\end{frame}
\begin{frame}
No hay ningún método práctico para determinar el valor óptimo de $\omega$ de antemano, sin embargo, una buena estimación se puede calcular en tiempo de ejecución.
\end{frame}
\begin{frame}
Sea
\[ \Delta x^{k} = \vert \mathbf{x}^{(k-1)} - \mathbf{x}^{(k)} \vert \]
la magnitud del cambio de $\mathbf{x}$ durante la $k$-ésima iteración (sin relajación, i.e. $\omega=1$).
\\
\medskip
Si $k$ es lo suficientemente grande ($k\geqslant5$), se puede demostrar que una aproximación para el valor óptimo de $\omega$ es
\[ \omega_{\mbox{opt}} \simeq \dfrac{2}{1 + \sqrt{1- (\Delta x^{(k+p)} / \Delta x^{(k)})^{1/p}}} \]
donde $p$ es un entero positivo.
\end{frame}
\begin{frame}
\frametitle{Método de GS con relajación}
\setbeamercolor{item projected}{bg=red!70!black,fg=white}
\setbeamertemplate{enumerate items}[circle]
\begin{enumerate}[<+->]
\item Realizar $k$ iteraciones con $\omega=1$ ($k=10$ es razonable).
\item Luego de la $k$-ésima iteración, guardar $\Delta x^{(k)}$.
\item Ejecutar $p$ iteraciones adicionales.
\item Guardar $\Delta x^{(k+p)}$ en la última iteración.
\item Ejecutar las demás iteraciones con $\omega = \omega_{\mbox{opt}}$, donde $\omega_{\mbox{opt}}$ se calcula como se indicó anteriormente.
\end{enumerate}
\end{frame}
\subsubsection{Algorimto para Gauss-Seidel}
\begin{frame}
\frametitle{Algorimto para Gauss-Seidel}
La función \texttt{gaussSeidel} es una implementación del método de Gauss-Seidel con relajación. Se calcula automáticamente el valor de$\omega_{\mbox{opt}}$ utilizando $k = 10$ y $p = 1$.
\end{frame}
\begin{frame}
El usuario debe proporcionar la función \texttt{iterEqs} que calcula la mejora de $\mathbf{x}$ a partir de las fórmulas iterativas.
\\
\bigskip
La función devuelve el vector solución $\mathbf{x}$, el número de iteraciones llevadas a cabo y el valor de $\omega_{\mbox{opt}}$ utilizado.
\end{frame}
\begin{frame}[fragile]
\begin{lstlisting}[basicstyle=\linespread{0.9}\ttfamily\small, columns=fullflexible]
def gaussSeidel(iterEqs, x, tol = 1.0e-9):
    omega = 1.0
    k = 10
    p = 1
    
    for i in range(1,501):
        xVieja = x.copy()
        x = iterEqs(x, omega)
        dx = np.sqrt(np.dot(x - xVieja, x - xVieja))
        if dx < tol: return x, i, omega
        
# se calcula el relajamiento luego de k +p iteraciones
        if i == k: dx1 = dx
        if i == k + p:
            dx2 = dx
            omega = 2.0/(1.0 + np.sqrt(1.0 - (dx2/dx1)**(1.0/p)))
	print 'No converge Gauss-Seidel'
\end{lstlisting} 
\end{frame}
\begin{frame}[fragile]
\frametitle{La función \texttt{iterEqs}}
\begin{lstlisting}[basicstyle=\linespread{0.9}\ttfamily\small, columns=fullflexible]
def iterEqs(x, omega):
    n = len(x)
    x[0] = omega*(x[1] - x[n-1])/2.0 + (1.0 - omega)*x[0]
    
    for i in range(1, n - 1):
        x[i] = omega*(x[i-1] + x[i+1])/2.0 + (1.0 - omega)*x[i]
    
    x[n-1] = omega*(1.0 - x[0] + x[n-2])/2.0 + (1.0 - omega)*x[n-1]

    return x
\end{lstlisting}
\end{frame}
\begin{frame}
\frametitle{Ejercicio}
\fontsize{12}{12}\selectfont
Resolver el siguiente sistema de $n$ ecuaciones simultáneas por el método de Gauss-Seidel con relajación (el programa deberá resolver para cualquier valor de $n$)
\fontsize{11}{11}\selectfont
\[ \begin{bmatrix}
2 & -1 & 0 & 0 & \ldots & 0 & 0 & 0 & 1 \\
-1 & 2 & -1 & 0 & \ldots & 0 & 0 & 0 & 0 \\
0 & -1 & 2 & -1 & \ldots & 0 & 0 & 0 & 0 \\
\vdots & \vdots & \vdots & \vdots & & \vdots & \vdots & \vdots & \vdots \\
0 & 0 & 0 & 0 & \ldots & -1 & 2 & -1 & 0 \\
0 & 0 & 0 & 0 & \ldots & 0 & -1 & 2 & -1 \\
1 & 0 & 0 & 0 & \ldots & 0 & 0 & -1 & 2 \\
\end{bmatrix}
\begin{bmatrix}
x_{1} \\
x_{2} \\
x_{3} \\
\vdots \\
x_{n-2} \\
x_{n-1} \\
x_{n} 
\end{bmatrix} = 
\begin{bmatrix}
0 \\
0 \\
0 \\
\vdots \\
0 \\
0 \\
1 
\end{bmatrix} \]
\end{frame}
\begin{frame}
\frametitle{Ejercicio}
Ejecutar el programa con $n=20$, la solución exacta es
\[ x_{i} = - \dfrac{n}{4} + \dfrac{i}{2} \hspace{1cm} \mbox{ con } i = 1, 2, \ldots, n \]
\end{frame}
\begin{frame}
\frametitle{¿Qué necesitamos?}
Necesitamos desarrollar las fórmulas iterativas a partir de:
\[  x_{i}\leftarrow \dfrac{\omega}{A_{ii}} \left( b_{i} - \sum_{\substack{ j=1 \\ j \neq i}}^{n} A_{ij} x_{j} \right) + (1 - \omega) x_{i} \hspace{1cm} i=1,2,\ldots,n \]
Para $x_{1}$, tenemos
\[ \begin{split}
x_{1} =& \dfrac{\omega}{2} \left( (1)(x_{2})-(1)(x_{n}) \right) + (1-\omega)x_{1} \\
x_{1} =& \dfrac{\omega(x_{2}-x_{n})}{2} + (1 - \omega)x_{1}
\end{split} \]
\end{frame}
\begin{frame}
\[ \begin{split}
x_{1} =& \dfrac{\omega(x_{2}-x_{n})}{2} + (1 - \omega)x_{1} \\
\pause
x_{i} =& \dfrac{\omega(x_{i-1}+x_{i+1})}{2} + (1 - \omega)x_{i} \hspace{1cm} i=2,3,\ldots,n-1 \\
x_{n} =& \dfrac{\omega(1-x_{1} + x_{n-1})}{2} + (1-\omega)x_{n}
\end{split} \]
Estas expresiones son las que serán evaluadas en la función \texttt{iterEqs}.
\end{frame}
\begin{frame}
\emph{Nota: } Cada matriz $\mathbf{A}$ requiere de la construcción de sus ecuaciones de iteración, revisa que se necesita conocer los valores de la diagonal principal $\mathbf{d}$, así como del vector de coeficientes $\mathbf{b}$.
\end{frame}
\begin{frame}[fragile]
\frametitle{Código principal}
\begin{lstlisting}[basicstyle=\linespread{0.9}\ttfamily\small, columns=fullflexible]
n = eval(input('Numero de ecuaciones ==> '))

x = np.zeros((n), dtype='float64')

x, numIter, omega = gaussSeidel(iterEqs, x)

print ('\nNumero de iteraciones =', numIter)

print ('\nFactor de relajacion =', omega)

print ('\nLa solucion es :\n', x)
\end{lstlisting}
\end{frame}
\begin{frame}{fragile}
\frametitle{Solución al problema}
La solución que nos devuelve el algoritmo, con $n=20$ es:
\\
\medskip
\texttt{
Número de ecuaciones = 20 
\\
\medskip
Número de iteraciones = 259 
\\
\medskip
Factor de relajación = 1.70545231071
}
\end{frame}
\begin{frame}
\frametitle{Explorando la solución I.}
Como podemos ver en la solución, el número de iteraciones es elevado, ¿a qué se debe?
\\
\medskip
Revisando la configuración del arreglo, notamos que no es dominante diagonal, por lo que en gran medida, el procedimiento de iteración es elevado, ahora: ¿qué podemos hacer?
\end{frame}
\begin{frame}
Podemos reconfigurar el arreglo de tal manera en que sea \textcolor{blue}{dominante diagonal} y podríamos revisar cuántas iteraciones requiere y compararlas contra la manera inicial.
\end{frame}
\begin{frame}
La convergencia es muy lenta, porque la matriz de coeficientes carece de dominancia diagonal.
\\
\bigskip
Si cambiamos cada término diagonal del coeficiente de $2$ a $4$, la matriz $\mathbf{A}$ sería diagonalmente dominante y la solución convergería en...¿?
\end{frame}
\end{document}