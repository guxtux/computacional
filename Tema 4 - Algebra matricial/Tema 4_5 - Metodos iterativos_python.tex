    \documentclass[12pt]{beamer}
\newenvironment{ConCodigo}[1]
  {\begin{frame}[fragile,environment=ConCodigo]{#1}}
  {\end{frame}}
\graphicspath{{Imagenes/}{../Imagenes/}}
\usepackage[utf8]{inputenc}
\usepackage[spanish]{babel}
\usepackage{hyperref}
\usepackage{etex}
%\reserveinserts{28}
\usepackage{amsmath}
\usepackage{amsthm}
\usepackage{mathtools}
\usepackage{multicol}
\usepackage{multirow}
\usepackage{tabulary}
\usepackage{booktabs}
\usepackage{nccmath}
\usepackage{physics}
\usepackage{biblatex}
\usepackage[outdir=./]{epstopdf}
%\epstopdfsetup{outdir=./}
\usepackage{graphicx}
%\usepackage{enumitem,xcolor}
\usepackage{siunitx}
%\sisetup{scientific-notation=true}
%\usepackage{fontspec}
\usepackage{lmodern}
\usepackage{float}
\usepackage[format=hang, font=footnotesize, labelformat=parens]{caption}
\usepackage[autostyle,spanish=mexican]{csquotes}
\usepackage{standalone}
\usepackage{blkarray}
\usepackage{algorithm}
\usepackage{algorithmic}
\usepackage{tikz}
\usepackage[siunitx, RPvoltages]{circuitikz}
\usetikzlibrary{arrows,patterns,shapes}
\usetikzlibrary{decorations.markings}
\usetikzlibrary{arrows}
\usepackage{color}
\usepackage{xcolor}
%\usepackage{beton}
%\usepackage{euler}
%\usepackage[T1]{fontenc}
\usepackage[sfdefault]{roboto}  %% Option 'sfdefault' only if the base font of the document is to be sans serif
\usepackage[T1]{fontenc}
\renewcommand*\familydefault{\sfdefault}
\DeclareGraphicsExtensions{.pdf,.png,.jpg}
\usepackage{hyperref}
\renewcommand {\arraystretch}{1.5}
\newcommand{\python}{\texttt{python}}
\usefonttheme[onlymath]{serif}
\setbeamertemplate{navigation symbols}{}
\usetikzlibrary{patterns}
\usetikzlibrary{decorations.markings}
\tikzstyle{every picture}+=[remember picture,baseline]
%\tikzstyle{every node}+=[inner sep=0pt,anchor=base,
%minimum width=2.2cm,align=center,text depth=.15ex,outer sep=1.5pt]
%\tikzstyle{every path}+=[thick, rounded corners]
\setbeamertemplate{caption}[numbered]
\newcommand{\ptm}{\fontfamily{ptm}\selectfont}
%Se usa la plantilla Warsaw modificada con spruce
\mode<presentation>
{
  \usetheme{Warsaw}
  \setbeamertemplate{headline}{}
  \useoutertheme{default}
  \usecolortheme{albatross}
  \setbeamercovered{invisible}
}
% \AtBeginSection[]
% {
% \begin{frame}<beamer>{Contenido}
% \normalfont\mdseries
% \tableofcontents[currentsection]
% \end{frame}
% }

%Se usa la plantilla Warsaw modificada con spruce
\mode<presentation>
{
  \usetheme{Warsaw}
  \setbeamertemplate{headline}{}
  %\useoutertheme{infolines}
  \usecolortheme{crane}
  \setbeamercovered{invisible}
  

\setbeamertemplate{section in toc}[sections numbered]
\setbeamertemplate{subsection in toc}[subsections numbered]
\setbeamertemplate{subsection in toc}{\leavevmode\leftskip=3.2em\rlap{\hskip-2em\inserttocsectionnumber.\inserttocsubsectionnumber}\inserttocsubsection\par}
\setbeamercolor{section in toc}{fg=blue}
\setbeamercolor{subsection in toc}{fg=blue}
\setbeamerfont{subsection in toc}{size=\small}



\setbeamertemplate{navigation symbols}{}
\setbeamertemplate{caption}[numbered]
% \setbeamercolor{frametitle}{fg=yellow,bg=blue!70!white}
%\setbeamercolor{section in head}{bg=green,fg=white}
%\setbeamercolor{subsection in head/foot}{bg=gray!30,fg=black}
%\setbeamercolor{author in head/foot}{fg=yellow}
%\setbeamercolor{date in head/foot}{fg=blue}

%\mode<presentation>
%{
%  \usetheme{Warsaw}
%  \setbeamertemplate{headline}{}
%  %\useoutertheme{infolines}
%  \useoutertheme{default}
%  \setbeamercovered{invisible}
%  % or whatever (possibly just delete it)
%}
}

\input{../Preambulos/pre_codigo}
\makeatletter
%\setbeamercolor{section in foot}{bg=gray!30, fg=black!90!orange}
%\setbeamercolor{subsection in foot}{bg=blue!30!yellow, fg=red}
\setbeamertemplate{footline}
{
  \leavevmode%
  \hbox{%
  \begin{beamercolorbox}[wd=.333333\paperwidth,ht=2.25ex,dp=1ex,center]{author in foot}%
    \usebeamerfont{author in head/foot} \textcolor{red}{\insertsection}
  \end{beamercolorbox}}%
  \begin{beamercolorbox}[wd=.333333\paperwidth,ht=2.25ex,dp=1ex,center]{title in foot}%
    \usebeamerfont{title in head/foot}\insertsubsection
  \end{beamercolorbox}%
  \begin{beamercolorbox}[wd=.333333\paperwidth,ht=2.25ex,dp=1ex,right]{date in head/foot}%
    \usebeamerfont{date in head/foot} \insertshortdate{} \hspace*{2em}
    \insertframenumber{} / \inserttotalframenumber \hspace*{2ex} 
  \end{beamercolorbox}}%
  \vskip0pt%
\makeatother	
\normalfont
\usepackage{ccfonts}% http://ctan.org/pkg/{ccfonts}
\usepackage[T1]{fontenc}% http://ctan.or/pkg/fontenc
\renewcommand{\rmdefault}{cmr}% cmr = Computer Modern Roman
\linespread{1.3}
\title{Métodos iterativos de solución para matrices}
\subtitle{Curso de Física Computacional}
\author{M. en C. Gustavo Contreras Mayén}
\date{\today}
\institute{Facultad de Ciencias - UNAM}
\titlegraphic{\includegraphics[width=1.75cm]{Imagenes/escudo-facultad-ciencias}\hspace*{4.75cm}~%
   \includegraphics[width=1.75cm]{Imagenes/escudo-unam}
}
\begin{document}
\maketitle
\fontsize{14}{14}\selectfont
\spanishdecimal{.}
\section*{Contenido}
\frame{\tableofcontents[currentsection, hideallsubsections]}
\section{Métodos iterativos}
\frame{\tableofcontents[currentsection, hideothersubsections]}
\subsection{Métodos más comunes}
\begin{frame}
\frametitle{Métodos iterativos}
Hasta ahora, hemos discutido sólo los métodos directos de solución. La característica común de estos métodos es que calculan la solución con un número finito de operaciones.
\\
\medskip
Por otra parte, sabemos que si la computadora tuviera una precisión infinita (no hay errores de redondeo), la solución sería exacta.
\end{frame}
\begin{frame}
\frametitle{Definición}
Los métodos indirectos o iterativos, comienzan con una estimación inicial de la solución de $x$ y luego mejoran repetidamente la solución hasta que el cambio en $x$ se hace insignificante.
\end{frame}
\begin{frame}
\frametitle{Definición}
Dado que el número requerido de iteraciones puede ser grande, los métodos indirectos son, en general, más lentos que sus homólogos directos. 
\\
\bigskip
Sin embargo, los métodos iterativos tienen las siguientes ventajas que los hacen atractivos para ciertos problemas:
\end{frame}
\begin{frame}
\frametitle{Consideraciones}
\begin{enumerate}
\item Es factible para almacenar sólo los elementos distintos de cero de la matriz de coeficientes. 
\\
\bigskip
Esto hace que sea posible para hacer frente a matrices muy grandes que son escasas, pero no necesariamente en bandas. En muchos problemas, no hay necesidad de almacenar la matriz de coeficientes en absoluto.
\seti
\end{enumerate}
\end{frame}
\begin{frame}
\frametitle{Consideraciones}
\begin{enumerate}
\conti
\item Hay procedimientos iterativos de auto-corrección, es decir, que los errores de redondeo (o incluso errores aritméticos) en un ciclo iterativo se corrigen en los ciclos posteriores.
\end{enumerate}
\end{frame}
\begin{frame}
Un serio inconveniente de los métodos iterativos es que no siempre convergen a la solución.
\\
\bigskip
Se puede demostrar que la convergencia está garantizada sólo si la matriz de coeficientes es diagonal dominante.
\end{frame}
\begin{frame}
La estimación inicial de $x$ no juega ningún papel en la determinación de si la convergencia se produce, si el procedimiento converge para un vector de partida, lo haría para cualquier vector de partida. 
\\
\bigskip
La estimación inicial afecta sólo el número de iteraciones que son necesarias para la convergencia.
\end{frame}
\begin{frame}
\frametitle{Definición de dominancia diagonal}
Una matriz $\mathbf{A}$ de $n \times n$ se dice que es \emph{diagonal dominante} si cada elemento de la diagonal es más grande que la suma de los otros elementos de la misma fila (estamos hablando aquí de valor absoluto.
\\
\medskip
Por lo tanto dominancia diagonal que requiere
\[ \vert A_{ii} \vert > \sum_{\substack{j=1 \\ j \neq i}}^{n} \vert A_{ij} \vert \hspace{1cm} (i=1,2,\ldots,n) \]
\end{frame}
\begin{frame}
\frametitle{Ejemplo de dominancia diagonal}
La matriz
\[ \begin{bmatrix}
-2 & 4 & -1 \\
1 & -1 & 3 \\
4 & 2 & 1
\end{bmatrix}\]
No es dominante diagonal.
\end{frame}
\begin{frame}
\frametitle{Ejemplo de dominancia diagonal}
Pero si reordenamos los renglones de la siguiente manera
\[ \begin{bmatrix}
4 & -2 & 1 \\
-2 & 4 & -1 \\
1 & -1 & 3
\end{bmatrix}\]
resulta ser diagonal dominante.
\end{frame}
\subsection{Método Gauss-Seidel}
\begin{frame}
\frametitle{Método Gauss-Seidel}
Las ecuaciones $\mathbf{A}x = \mathbf{b}$ en su forma escalar, son:
\[ \sum_{j=1}^{n} A_{ij} x_{j} = b_{i} \hspace{1cm} i=1,2,\ldots,n \]
Despejando el término que contiene a $x_{i}$ de la suma, obtenemos
\[ A_{ii} x_{i} + \sum_{\substack{j=1 \\ j \neq i}}^{n} A_{ij}x_{j} = b_{i} \hspace{1cm} i,2,\ldots,n \]
\end{frame}
\begin{frame}
\frametitle{Operación del método G-S}
Resolviendo para $x_{i}$, resulta
\[ x_{i} = \dfrac{1}{A_{ii}} \left( b_{i} - \sum_{\substack{ j=1 \\ j \neq i}}^{n} A_{ij} x_{j} \right) \hspace{1cm} i=1,2,\ldots,n \]
\end{frame}
\begin{frame}
\frametitle{Operación del método G-S}
La ecuación anterior sugiere el siguiente esquema iterativo:
\[ x_{i}\leftarrow \dfrac{1}{A_{ii}} \left( b_{i} - \sum_{\substack{ j=1 \\ j \neq i}}^{n} A_{ij} x_{j} \right) \hspace{1cm} i=1,2,\ldots,n \]
Iniciamos eligiendo un vector $x$. Si la suposición inicial no fue buena, podemos elegir de manera aleatoria a $x$.
\end{frame}
\begin{frame}
\frametitle{Operación del método G-S}
La ecuación anterior se utiliza nuevamente para calcular cada uno de los elementos de $x$, utilizando siempre los últimos valores disponibles de $x_{j}$: esto completa un ciclo de iteración.
\\
\bigskip
El procedimiento se repite hasta que los cambios en $x$ con cada iteración sucesiva vuelven lo suficientemente pequeños.
\end{frame}
\begin{frame}
\frametitle{Operación del método G-S}
Es posible mejorar la convergencia del método de Gauss-Seidel con una técnica conocida como \emph{relajación}.
\end{frame}
\begin{frame}
\frametitle{Operación del método G-S}
La idea es tomar un nuevo valor de $x_{i}$ como un promedio ponderado de su valor anterior y el valor predicho por la ecuación anterior.
\\
\bigskip
La correspondiente fórmula iterativa, es
\[  x_{i}\leftarrow \dfrac{\omega}{A_{ii}} \left( b_{i} - \sum_{\substack{ j=1 \\ j \neq i}}^{n} A_{ij} x_{j} \right) + (1 - \omega) x_{i} \hspace{0.75cm} i=1,2,\ldots,n \]
donde $\omega$ es el \emph{factor de relajación}.
\end{frame}
\subsection{Factor de relajación}
\begin{frame}
\frametitle{Factor de relajación}
Nótese que
\setbeamercolor{item projected}{bg=red!70!black,fg=white}
\setbeamertemplate{enumerate items}[circle]
\begin{enumerate}[<+->]
\item  Si $\omega=1$, no se presenta la relajación.
\item Si $\omega<1$, la ecuación de relajación, representa una interpolación entre los valores anteriores de $x_{i}$ y el valor dado por la ecuación inicial. A esto se le llama \emph{subrelajación}.
\item Si $\omega>1$, tenemos una extrapolación o sobrerelajación.
\end{enumerate}
\end{frame}
\begin{frame}
\frametitle{Factor de relajación}
No hay ningún método práctico para determinar el valor óptimo de $\omega$ de antemano, sin embargo, una buena estimación se puede calcular en tiempo de ejecución.
\end{frame}
\begin{frame}
\frametitle{Estimación del factor de relajación}
Sea
\[ \Delta x^{k} = \vert \mathbf{x}^{(k-1)} - \mathbf{x}^{(k)} \vert \]
la magnitud del cambio de $x$ durante la $k$-ésima iteración (sin relajación, i.e. $\omega=1$).
\\
\medskip
\pause
Si $k$ es lo suficientemente grande ($k\geqslant5$), se puede demostrar que una aproximación para el valor óptimo de $\omega$ es
\[ \omega_{\mbox{opt}} \simeq \dfrac{2}{1 + \sqrt{1- (\Delta x^{(k+p)} / \Delta x^{(k)})^{1/p}}} \]
donde $p$ es un entero positivo.
\end{frame}
\begin{frame}
\frametitle{Elementos esenciales para el método de GS con relajación}
\setbeamercolor{item projected}{bg=red!70!black,fg=white}
\setbeamertemplate{enumerate items}[circle]
\begin{enumerate}[<+->]
\item Realizar $k$ iteraciones con $\omega=1$ ($k=10$ es razonable). Luego de la $k$-ésima iteración, guardar $\Delta x^{(k)}$.
\item Ejecutar $p$ iteraciones adicionales y guardar $\Delta x^{(k+p)}$ en la última iteración.
\item Ejecutar las demás iteraciones con $\omega = \omega_{\mbox{opt}}$, donde $\omega_{\mbox{opt}}$ se calcula como se indicó anteriormente.
\end{enumerate}
\end{frame}
\subsubsection{Algoritmo para Gauss-Seidel}
\begin{frame}
\frametitle{Algorimto para Gauss-Seidel}
La función \texttt{gaussSeidel} es una implementación del método de Gauss-Seidel con relajación. Se calcula automáticamente $\omega_{\mbox{opt}}$ utilizando $k = 10$ y $p = 1$.
\end{frame}
\begin{frame}
\frametitle{Algorimto para Gauss-Seidel}
El usuario debe proporcionar la función \texttt{iterEqs} que calcula la mejora de $x$ a partir de las fórmulas iterativas. 
\\
\bigskip
La función devuelve el vector solución $x$, el número de iteraciones llevadas a cabo y el valor de $\omega_{\mbox{opt}}$ utilizado.
\end{frame}
\begin{frame}[plain, allowframebreaks, fragile]
\begin{lstlisting}[caption=Función Gauss-Seidel, style=FormattedNumber, basicstyle=\linespread{1.1}\ttfamily=\small, columns=fullflexible]
def gaussSeidel(iterEqs,x,tol = 1.0e-9):
    omega = 1.0
    k = 10
    p = 1
    
    for i in range(1, 501):
        xVieja = x.copy()
        x = iterEqs(x,omega)
        dx = sqrt(dot(x - xVieja, x - xVieja))
        if dx < tol: return x, i, omega
        
        if i == k: dx_1_ = dx
        if i == k + p:
            dx_2_ = dx
            omega = 2.0/(1.0 + sqrt(1.0 - (dx_2_/dx_1_)**(1.0/p)))
	print ('No converge Gauss-Seidel')
\end{lstlisting} 
\end{frame}
\begin{frame}[plain]
\frametitle{Ejercicio}
Resuelve el siguiente sistema de $n$ ecuaciones simultáneas por el método de Gauss-Seidel con relajación (el programa deberá resolver para cualquier valor de $n$)
\fontsize{12}{12}\selectfont
\[ \begin{bmatrix}
2 & -1 & 0 & 0 & \ldots & 0 & 0 & 0 & 1 \\
-1 & 2 & -1 & 0 & \ldots & 0 & 0 & 0 & 0 \\
0 & -1 & 2 & -1 & \ldots & 0 & 0 & 0 & 0 \\
\vdots & \vdots & \vdots & \vdots & & \vdots & \vdots & \vdots & \vdots \\
0 & 0 & 0 & 0 & \ldots & -1 & 2 & -1 & 0 \\
0 & 0 & 0 & 0 & \ldots & 0 & -1 & 2 & -1 \\
1 & 0 & 0 & 0 & \ldots & 0 & 0 & -1 & 2 \\
\end{bmatrix}
\begin{bmatrix}
x_{1} \\
x_{2} \\
x_{3} \\
\vdots \\
x_{n-2} \\
x_{n-1} \\
x_{n} 
\end{bmatrix} = 
\begin{bmatrix}
0 \\
0 \\
0 \\
\vdots \\
0 \\
0 \\
1 
\end{bmatrix} \]
\end{frame}
\begin{frame}[plain]
\frametitle{Ejercicio}
Ejecuta el programa con $n=20$, la solución exacta es $x_{i}= - \frac{n}{4} + \frac{i}{2}$, con $i=1,2,\ldots,n$.
\end{frame}
\begin{frame}
\frametitle{¿Qué necesitamos?}
Necesitamos desarrollar las fórmulas iterativas a partir de:
\[  x_{i}\leftarrow \dfrac{\omega}{A_{ii}} \left( b_{i} - \sum_{\substack{ j=1 \\ j \neq i}}^{n} A_{ij} x_{j} \right) + (1 - \omega) x_{i} \hspace{1cm} i=1,2,\ldots,n \]
\end{frame}
\begin{frame}
\frametitle{¿Qué necesitamos?}
Para $x_{1}$, tenemos
\[ \begin{split}
x_{1} =& \dfrac{\omega}{2} \left( (1)(x_{2})-(1)(x_{n}) \right) + (1-\omega)x_{1} \\
x_{1} =& \dfrac{\omega(x_{2}-x_{n})}{2} + (1 - \omega)x_{1}
\end{split} \]
\end{frame}
\begin{frame}
\[ \begin{split}
x_{1} =& \dfrac{\omega(x_{2}-x_{n})}{2} + (1 - \omega)x_{1} \\
\pause
x_{i} =& \dfrac{\omega(x_{i-1}+x_{i+1})}{2} + (1 - \omega)x_{i} \hspace{1cm} i=2,3,\ldots,n-1 \\
x_{n} =& \dfrac{\omega(1-x_{1} + x_{n-1})}{2} + (1-\omega)x_{n}
\end{split} \]
\emph{Nota: } Cada matriz $\mathbf{A}$ requiere de la construcción de sus ecuaciones de iteración, revisa que se necesita conocer los valores de la diagonal principal $\mathbf{d}$, así como del vector de coeficientes $\mathbf{b}$.
\end{frame}
\begin{frame}[fragile]
\frametitle{La función \texttt{iterEqs}}
\begin{lstlisting}[caption=Código para el ejercicio GS, style=FormattedNumber, basicstyle=\linespread{1.1}\ttfamily=\small, columns=fullflexible]
def iterEqs(x,omega):
    n = len(x)
    x[_0_] = omega*(x[_1_] - x[n-_1_])/2.0 + (1.0 - omega)*x[_0_]
    
    for i in range(1, n-1):
        x[i] = omega*(x[i-_1_] + x[i+_1_])/2.0 + (1.0 - omega)*x[i]
    
    x[n-_1_] = omega*(1.0 - x[_0_] + x[n-_2_])/2.0 + (1.0 - omega)*x[n-_1_]

    return x
\end{lstlisting}
\end{frame}
\begin{frame}[plain, allowframebreaks, fragile]
\frametitle{Código principal}
\begin{lstlisting}[caption=Código principal, style=FormattedNumber, basicstyle=\linespread{1.1}\ttfamily=\small, columns=fullflexible]
n = eval(input('Numero de ecuaciones '))

x = zeros((n),dtype='float_6_ _4_')

x,numIter,omega = gaussSeidel(iterEqs, x)

print ('\nNumero de iteraciones =', numIter)

print ('\nFactor de relajacion =',omega)

print ('\nLa solucion es :',x)
\end{lstlisting}
\end{frame}
\begin{frame}{fragile}
\frametitle{Solución al problema}
La solución que nos devuelve el algoritmo, con $n=20$ es:
\\
\medskip
\texttt{
Número de ecuaciones = 20 
\\
\medskip
Número de iteraciones = 259 
\\
\medskip
Factor de relajación = 1.70545231071
}
\end{frame}
\begin{frame}
\frametitle{Explorando la solución I.}
Como podemos ver en la solución, el número de iteraciones es elevado, ¿a qué se debe?
\\
\medskip
Revisando la configuración del arreglo, notamos que no es dominante diagonal, por lo que en gran medida, el procedimiento de iteración es elevado, ahora: ¿qué podemos hacer?
\\
\medskip
Podemos reconfigurar el arreglo de tal manera en que sea dominante diagonal y podríamos revisar cuántas iteraciones requiere y compararlas contra la manera inicial.
\end{frame}
\begin{frame}
\frametitle{Explorando la solución II.}
¿Y si aumentamos el tamaño del arreglo?
\\
\medskip
El algoritmo que se propuso para el ejercicio, considera la solución para un sistema de $n$ ecuaciones, ya lo tenemos para $n=20$, completa la siguiente tabla:
\begin{center}
\begin{tabular}{c | c | c }
\hline
$n$ & iteraciones & $\omega_{opt}$ \\ \hline
20 & & \\ \hline
25 & & \\ \hline
50 & & \\ \hline
75 & & \\ \hline
100 & & \\ \hline
\end{tabular}
\end{center}
Interpreta tus resultados.
\end{frame}
\end{document}