\documentclass[11pt]{article}
\usepackage[utf8]{inputenc}
%\usepackage[latin1]{inputenc}
\usepackage[spanish]{babel}
\decimalpoint
\usepackage{anysize}
\usepackage{graphicx} 
\usepackage{amsmath}
\usepackage{booktabs}
\usepackage{tabulary}
\usepackage{nccmath}
\usepackage{float}
\usepackage{tikz}
\usepackage{color}
\usepackage{listings}
\lstset{ %
language=Python,                % choose the language of the code
basicstyle=\normalsize,       % the size of the fonts that are used for the code
numbers=left,                   % where to put the line-numbers
numberstyle=\footnotesize,      % the size of the fonts that are used for the line-numbers
stepnumber=1,                   % the step between two line-numbers. If it is 1 each line will be numbered
numbersep=5pt,                  % how far the line-numbers are from the code
backgroundcolor=\color{white},  % choose the background color. You must add \usepackage{color}
showspaces=false,               % show spaces adding particular underscores
showstringspaces=false,         % underline spaces within strings
showtabs=false,                 % show tabs within strings adding particular underscores
frame=single,   		% adds a frame around the code
tabsize=4,  		% sets default tabsize to 2 spaces
captionpos=b,   		% sets the caption-position to bottom
breaklines=true,    	% sets automatic line breaking
breakatwhitespace=false,    % sets if automatic breaks should only happen at whitespace
escapeinside={\#}{)}          % if you want to add a comment within your code
}
\marginsize{1.5cm}{1.5cm}{2cm}{2cm}  
\title{Ejercicio interpolaci\'{o}n de Newton. \\ Curso de Física Computacional}
\author{M. en C. Gustavo Contreras Mayén}
\date{ }
\begin{document}
\maketitle
\fontsize{14}{14}\selectfont
\section{Problema}
Los datos que se muestran en la siguiente tabla se obtuvieron de la funci\'{o}n
\[ f(x) = 4.8 \cos \left( \dfrac{\pi x}{20} \right)\]
Con ese conjunto de datos, interpola mediante el polinomio de Newton en $x=0, 0.5, 1.0, \ldots, 8.0$ y compara los resultados con el valor ''exacto" de los valores $y_{i} = f(x_{i})$
\begin{table}[htbp]
\centering \Large
\begin{tabulary}{15cm}{c | c | c | c | c | c | c}
$x$ & $0.15$ & $2.30$ & $3.15$ & $4.85$ & $6.25$ & $7.95$ \\
\midrule $y$ & $4.79867$ & $4.49013$ & $4.2243$ & $3.47313$ & $2.66674$ & $1.51900$
\end{tabulary}
\end{table}
\section{C\'{o}digo}
Ocuparemos \texttt{numpy} para facilitar el manejo de los datos en arreglos.
\subsection{C\'{a}lculo de los coeficientes}
\begin{lstlisting}
def coeficientes(xDatos, yDatos):
    m = len(xDatos)
    a = yDatos.copy()
    
    for k in range(1,m):
        a[k:m] = (a[k:m] - a[k-1])/(xDatos[k:m] - xDatos[k-1])
    
    return a
\end{lstlisting}
\subsection{Evaluaci\'{o}n del Polinimio}
Una vez obtenido el arreglo de diferencias divididas, corresponde evaluar el polinomio en cada punto.
\begin{lstlisting}
def evaluaPoli(a,xDatos,x):
    n = len(xDatos)-1
    
    p = a[n]
    
    for k in range(1,n+1):
        p = a[n-k] + (x - xDatos[n-k])*p
    
    return p
\end{lstlisting}
\subsection{Datos iniciales}
Se necesitan los arreglos con los valores tanto de $x$, como de $y$, luego se indican los valores en donde se requiere evaluar y se muestra en pantalla, los resultados.
\begin{lstlisting}
resultados = open('DatosNewton.dat','w')

xDatos = array([0.15,2.30,3.15,4.85,6.25,7.95])
yDatos = array([4.79867,4.49013,4.2243,3.47313,2.66674,1.51909])

a = coeficientes(xDatos, yDatos)

print 'x    yInterpol  yExacta'
print '---------------------------'


for x in arange(0.0,8.1,0.5):
    y = evaluaPoli(a,xDatos,x)
    yExacta = 4.8 * cos(pi*x/20.0)
    print '%3.1f %9.5f %9.5f'   %(x, y, yExacta)
    resultados.write ('%3.1f %9.5f %9.5f \n' %(x, y, yExacta))

resultados.close()
\end{lstlisting}
\end{document}