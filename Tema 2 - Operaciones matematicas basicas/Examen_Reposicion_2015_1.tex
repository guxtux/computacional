\documentclass[11pt]{article}
\usepackage[utf8]{inputenc}
%\usepackage[latin1]{inputenc}
\usepackage[spanish]{babel}
\decimalpoint
\usepackage{anysize}
\usepackage{graphicx} 
\usepackage{amsmath}
\usepackage{booktabs}
\usepackage{tabulary}
\usepackage{nccmath}
\usepackage{float}
\usepackage{tikz}
\usetikzlibrary{patterns}
\usetikzlibrary{decorations.markings}
\usepackage{pgfplots}
\usepackage{etex}
\usepackage{color}
\usepackage{listings}
\renewcommand{\arraystretch}{1.5}
\lstset{ %
language=Python,                % choose the language of the code
basicstyle=\normalsize,       % the size of the fonts that are used for the code
numbers=left,                   % where to put the line-numbers
numberstyle=\footnotesize,      % the size of the fonts that are used for the line-numbers
stepnumber=1,                   % the step between two line-numbers. If it is 1 each line will be numbered
numbersep=5pt,                  % how far the line-numbers are from the code
backgroundcolor=\color{white},  % choose the background color. You must add \usepackage{color}
showspaces=false,               % show spaces adding particular underscores
showstringspaces=false,         % underline spaces within strings
showtabs=false,                 % show tabs within strings adding particular underscores
frame=single,   		% adds a frame around the code
tabsize=4,  		% sets default tabsize to 2 spaces
captionpos=b,   		% sets the caption-position to bottom
breaklines=true,    	% sets automatic line breaking
breakatwhitespace=false,    % sets if automatic breaks should only happen at whitespace
escapeinside={\#}{)}          % if you want to add a comment within your code
}
\marginsize{1.5cm}{1.5cm}{1cm}{2cm}  
\title{Examen Reposicion Tema 2 \\ Curso de Física Computacional}
\author{M. en C. Gustavo Contreras May\'{e}n}
\date{ }
\begin{document}
\maketitle
\fontsize{14}{14}\selectfont
\begin{enumerate}
\item Dados los puntos
\begin{table}[H]
\centering 
\begin{large}
\begin{tabulary}{15cm}{c | c | c | c | c | c  }\normalsize
$x$ & $0$ & $0.5$ & $1$ & $1.5$ & $2$ \\
\midrule
$y$ & $-0.7854$ & $0.6529$ & $1.7390$ & $2.2071$ & $1.9425$
\end{tabulary}
\end{large}
\end{table}
Calcula $y$ en $x=\pi/4$ y en $\pi/2$ usa (justifica) el método que consideres más conveniente.
\item El calor específico $c_{p}$ del alumnio depende de la temperatura $T$ como se muestra en la tabla
\begin{table}[H]
\centering 
\begin{large}
\begin{tabulary}{15cm}{c | c | c | c | c | c | c }
$T (^{\circ}C)$ & $-250$ & $-200$ & $-100$ & $0$ & $100$ & $300$ \\
\midrule
$c_{p} (k J/kg \cdot K)$ & $0.0163$ & $0.318$ & $0.699$ & $0.870$ & $0.941$ & $1.04  $
\end{tabulary}
\end{large}	
\end{table}
Determina $c_{p}$ para $T=200^{\circ}C$ y para $T=400^{\circ}C$.
\item La velocidad $v$ del cohete Saturno V en vuelo vertical cerca de la superficie de la Tierra, puede aproximarse por
\[ v = u ln \dfrac{M_{0}}{M_{0}- \dot{m}t}- gt\]
donde
\begin{itemize}
\item $u=2510$ m/s = velocidad de escape relativa al cohete.
\item $M_{0}= 2.8 \times 10^{6}$ kg = masa del cohete al despegue.
\item $\dot{m}=13.3 \times 10^{3}$ kg/s = tasa de consumo de combustible.
\item $g=9.81 \mbox{ }m/s^{2}$ = aceleración gravitacional.
\item  $t$ = tiempo medido desde el despegue.
\end{itemize}
Calcula el tiempo que tarda el cohete en alcanzar la velocidad del sonido ($335$ m/s).
\item Un pico de energía en un circuito eléctrico se debe a la corriente que circula por la resistencia.
\[ i(t) = i_{0} e^{-t/t_{0}} \sin(2t/t_{0}) \]
La energía $E$ disipada por la resistencia es
\[ E = \int_{0}^{\infty} R [i(t)]^{2}  dt \]
Calcula el valor de E con los siguientes valores: $i_{0}=100 \mbox{ A}$, $R=0.5 \mbox{ } \Omega$ y $t_{0}=0.01$ s.
\end{enumerate}
\end{document}