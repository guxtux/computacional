\documentclass[pdf, azure]{prosper}
\usepackage[utf8]{inputenc}
%\usepackage[latin1]{inputenc}
\usepackage[spanish]{babel}
\title{Curso de Física Computacional}
\subtitle{Diferenciación numérica}
\author{M. en C. Gustavo Contreras Mayén}
\email{curso.fisica.comp@gmail.com}
\ptsize{10}
\begin{document}
\maketitle
\begin{slide}{Diferenciación numérica}
Consideremos una función $f(x)$, nuestro interés será evaluar la primera derivada de esta función en el punto $x = x_{0}$
\\
Si conocemos los valores de $f$ en $x_{0}-h$, $x_{0}+h$, donde $h$ es el tamaño del intervalo entre dos puntos consecutivos en el eje $x$, entonces podremos aproximar $f'(x_{0})$ mediante un gradiente de interpolación lineal A, B o C.
\end{slide}
\end{document}