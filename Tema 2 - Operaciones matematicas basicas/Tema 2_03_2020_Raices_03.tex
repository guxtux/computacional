\documentclass[12pt]{beamer}
\usepackage[utf8]{inputenc}
\usepackage[spanish]{babel}
\usepackage{color}
\usepackage{hyperref}
\usepackage{amsmath}
\usepackage{amsthm}
\usepackage{multicol}
\usepackage{graphicx}
\usepackage{tikz}
\usepackage[autostyle,spanish=mexican]{csquotes}
%\usepackage[sfdefault]{roboto}  %% Option 'sfdefault' only if the base font of the document is to be sans serif
\renewcommand{\arraystretch}{1.5}
\renewcommand{\rmdefault}{cmr}% cmr = Computer Modern Roman
\usefonttheme[onlymath]{serif}

\newcommand{\python}{\texttt{python}}
\newcommand{\textoazul}[1]{\textcolor{blue}{#1}}
\newcommand{\azulfuerte}[1]{\textcolor{blue}{\textbf{#1}}}
\newcounter{saveenumi}
\newcommand{\seti}{\setcounter{saveenumi}{\value{enumi}}}
\newcommand{\conti}{\setcounter{enumi}{\value{saveenumi}}}

\linespread{1.5}
\beamertemplatenavigationsymbolsempty
\usefonttheme{professionalfonts}
\usefonttheme{serif}
\DeclareGraphicsExtensions{.pdf,.png,.jpg}
\renewcommand {\arraystretch}{1.25}
\mode<presentation>
{
  \usetheme{Warsaw}
  \setbeamertemplate{headline}{}
  %\useoutertheme{infolines}
  \useoutertheme{default}
  \setbeamercovered{invisible}
  % or whatever (possibly just delete it)
  \setbeamertemplate{section in toc}[sections numbered]
  \setbeamertemplate{subsection in toc}[subsections numbered]
  \setbeamertemplate{subsection in toc}{\leavevmode\leftskip=3.2em\rlap{\hskip-2em\inserttocsectionnumber.\inserttocsubsectionnumber}\inserttocsubsection\par}
  \setbeamercolor{section in toc}{fg=blue}
  \setbeamercolor{subsection in toc}{fg=blue}
  \setbeamercolor{frametitle}{fg=yellow}

  \setbeamertemplate{footline} 
{
  \leavevmode%
  \hbox{%
  \begin{beamercolorbox}[wd=.333333\paperwidth,ht=2.25ex,dp=1ex,center]{author in head/foot}%
    \usebeamerfont{author in head/foot}\insertsection
  \end{beamercolorbox}%
  \begin{beamercolorbox}[wd=.333333\paperwidth,ht=2.25ex,dp=1ex,center]{title in head/foot}%
    \usebeamerfont{title in head/foot}\textcolor{yellow}{\insertsubsection}
  \end{beamercolorbox}%
  \begin{beamercolorbox}[wd=.333333\paperwidth,ht=2.25ex,dp=1ex,right]{date in head/foot}%
    \usebeamerfont{date in head/foot}\insertshortdate{}\hspace*{2em}
    \insertframenumber{} / \inserttotalframenumber\hspace*{2ex} 
  \end{beamercolorbox}}%
  \vskip0pt%
}
}
\makeatother

\makeatletter
\patchcmd{\beamer@sectionintoc}
  {\vfill}
  {\vskip\itemsep}
  {}
  {}
\makeatother
\title{Cálculo de raíces en sistemas de ecuaciones- 2}
\subtitle{Tema 2 - Operaciones matemáticas básicas}
\author{M. en C. Gustavo Contreras Mayén}
\date{\today}
\institute{Facultad de Ciencias - UNAM}
\titlegraphic{\includegraphics[width=1.75cm]{Imagenes/escudo-facultad-ciencias}\hspace*{4.75cm}~%
   \includegraphics[width=1.75cm]{Imagenes/escudo-unam}
}
\begin{document}
\maketitle
\fontsize{14}{14}\selectfont
\spanishdecimal{.}
\section*{Contenido}
\frame{\tableofcontents[currentsection, hideallsubsections]}
\section{Método de Newton-Raphson generalizado}
\frame{\tableofcontents[currentsection, hideothersubsections]}
\subsection{Sistemas de ecuaciones lineales}
%Referencia: Beu 6.48 Newtons' method for systems of no-linear equations
\begin{frame}
\frametitle{Método de N-R generalizado}
El método de Newton-Raphson se puede generalizar para sistemas de ecuaciones no lineales.
\\
\bigskip
Las condiciones de convergencia para el algoritmo multidimensional son considerablemente más exigentes que para el caso unidimensional.
\end{frame}
\begin{frame}
\frametitle{Método de N-R generalizado}
El método multidimensional es generalmente muy sensible a la aproximación inicial, siendo eficiente sólo si uno comienza lo suficientemente cerca de la solución exacta.
\\
\bigskip
Además, bien puede divergir rápidamente en lugar de aproximaciones iniciales inadecuadas, debido a la convergencia global bastante pobre.
\end{frame}
\begin{frame}
\frametitle{Método de N-R generalizado}
Consideremos el sistema de ecuaciones no lineales:
\begin{align}
f_{i} (x_{1}, x_{2}, \ldots, x_{n}) = 0 \hspace{1cm} i = 1, 2, \ldots, n
\label{eq:ecuacion_6_43}
\end{align}
con respecto a las incógnitas: $x_{1}, x_{2}, \ldots$
\end{frame}
\begin{frame}
\frametitle{En notación matricial}
El sistema en notación matricial toma la forma:
\begin{align}
\mathbf{f(x)} = \mathbf{0}
\label{eq:ecuacion_6_44}
\end{align}
donde $\mathbf{x}$ es el vector de incógnitas, cuyas componentes son $x_{i}$, y las componentes de $\mathbf{f}$ son las funciones $f_{i}(x_{1}, x_{2}, \ldots, x_{n})$.
\end{frame}
\end{document}