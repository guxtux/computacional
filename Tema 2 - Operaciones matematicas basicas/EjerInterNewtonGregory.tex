\documentclass[11pt]{article}
\usepackage[utf8]{inputenc}
%\usepackage[latin1]{inputenc}
\usepackage[spanish]{babel}
\decimalpoint
\usepackage{anysize}
\usepackage{graphicx} 
\usepackage{amsmath}
\usepackage{booktabs}
\usepackage{tabulary}
\usepackage{nccmath}
\usepackage{float}
\usepackage{tikz}
\usepackage{color}
\usepackage{listings}
\lstset{ %
language=Python,                % choose the language of the code
basicstyle=\normalsize,       % the size of the fonts that are used for the code
numbers=left,                   % where to put the line-numbers
numberstyle=\footnotesize,      % the size of the fonts that are used for the line-numbers
stepnumber=1,                   % the step between two line-numbers. If it is 1 each line will be numbered
numbersep=5pt,                  % how far the line-numbers are from the code
backgroundcolor=\color{white},  % choose the background color. You must add \usepackage{color}
showspaces=false,               % show spaces adding particular underscores
showstringspaces=false,         % underline spaces within strings
showtabs=false,                 % show tabs within strings adding particular underscores
frame=single,   		% adds a frame around the code
tabsize=4,  		% sets default tabsize to 2 spaces
captionpos=b,   		% sets the caption-position to bottom
breaklines=true,    	% sets automatic line breaking
breakatwhitespace=false,    % sets if automatic breaks should only happen at whitespace
escapeinside={\#}{)}          % if you want to add a comment within your code
}
\marginsize{1.5cm}{1.5cm}{2cm}{2cm}  
\title{Ejercicio interpolaci\'{o}n de Newton-Gregory. \\ Curso de Física Computacional}
\author{M. en C. Gustavo Contreras Mayén}
\date{ }
\begin{document}
\maketitle
\fontsize{14}{14}\selectfont
\section{Problema}
\begin{frame}
Con la t\'{e}cnica de interpolaci\'{o}n de Newton-Gregory, a partir del siguiente conjunto de datos:
\begin{table}[H]
\centering \Large
\begin{tabulary}{15cm}{c | c | c | c | c }
$x_{i}$ & $0.4$ & $0.6$ & $0.8$ & $1.0$ \\
\midrule $f_{i}$ & $0.423$ & $0.684$ & $1.030$ & $1.557$ 
\end{tabulary}
\end{table}
estima el valor de interpolaci\'{o}n para:\\ $x=0.43,0.49,0.5,0.55,0.73, 0.75, 0.91, 0.95$
\end{frame}
\section{C\'{o}digo}
\begin{lstlisting}
from numpy import *

def NewtonGregory(x1,deltaX,f,xt):
   n = len(f)-1
   deltaF = zeros([n+1,n+1])
   deltaF[:,0] = f
   for j in range(1,n+1):
      for i in range(n-j+1):
         deltaF[i,j] = deltaF[i+1,j-1]-deltaF[i,j-1]
   deltaF = deltaF[0:n,1:n+1]

   s = (xt-x1)/deltaX

   yt = []
   for t in range(len(xt)):
      sum = f[0]
      for i in range(n):
         sum += combinaciones(s[t],i+1)*deltaF[0,i]
      yt += [sum]
   return yt


def combinaciones(s,k):
   res = 1.0
   if k!=0:
      for i in range(1,k+1):
         res *= (s-i+1)/i
   return res

x = array([0.4,0.6,0.8,1.0],float)
f = array([0.423,0.684,1.030,1.557],float)

xt = array([0.43, 0.49, 0.5, 0.55, 0.73, 0.75, 0.91, 0.95])

ft = NewtonGregory(x[0],x[1]-x[0],f,xt)

print 'xt        ft'
print '---------------'
for i in range(len(xt)):
    print '%4.2f %9.5f' %(xt[i], ft[i])
\end{lstlisting}
\end{document}