\documentclass[12pt]{beamer}
\newenvironment{ConCodigo}[1]
  {\begin{frame}[fragile,environment=ConCodigo]{#1}}
  {\end{frame}}
\graphicspath{{Imagenes/}{../Imagenes/}}
\usepackage[utf8]{inputenc}
\usepackage[spanish]{babel}
\usepackage{hyperref}
\usepackage{etex}
%\reserveinserts{28}
\usepackage{amsmath}
\usepackage{amsthm}
\usepackage{mathtools}
\usepackage{multicol}
\usepackage{multirow}
\usepackage{tabulary}
\usepackage{booktabs}
\usepackage{nccmath}
\usepackage{physics}
\usepackage{biblatex}
\usepackage[outdir=./]{epstopdf}
%\epstopdfsetup{outdir=./}
\usepackage{graphicx}
%\usepackage{enumitem,xcolor}
\usepackage{siunitx}
%\sisetup{scientific-notation=true}
%\usepackage{fontspec}
\usepackage{lmodern}
\usepackage{float}
\usepackage[format=hang, font=footnotesize, labelformat=parens]{caption}
\usepackage[autostyle,spanish=mexican]{csquotes}
\usepackage{standalone}
\usepackage{blkarray}
\usepackage{algorithm}
\usepackage{algorithmic}
\usepackage{tikz}
\usepackage[siunitx, RPvoltages]{circuitikz}
\usetikzlibrary{arrows,patterns,shapes}
\usetikzlibrary{decorations.markings}
\usetikzlibrary{arrows}
\usepackage{color}
\usepackage{xcolor}
%\usepackage{beton}
%\usepackage{euler}
%\usepackage[T1]{fontenc}
\usepackage[sfdefault]{roboto}  %% Option 'sfdefault' only if the base font of the document is to be sans serif
\usepackage[T1]{fontenc}
\renewcommand*\familydefault{\sfdefault}
\DeclareGraphicsExtensions{.pdf,.png,.jpg}
\usepackage{hyperref}
\renewcommand {\arraystretch}{1.5}
\newcommand{\python}{\texttt{python}}
\usefonttheme[onlymath]{serif}
\setbeamertemplate{navigation symbols}{}
\usetikzlibrary{patterns}
\usetikzlibrary{decorations.markings}
\tikzstyle{every picture}+=[remember picture,baseline]
%\tikzstyle{every node}+=[inner sep=0pt,anchor=base,
%minimum width=2.2cm,align=center,text depth=.15ex,outer sep=1.5pt]
%\tikzstyle{every path}+=[thick, rounded corners]
\setbeamertemplate{caption}[numbered]
\newcommand{\ptm}{\fontfamily{ptm}\selectfont}
%Se usa la plantilla Warsaw modificada con spruce
\mode<presentation>
{
  \usetheme{Warsaw}
  \setbeamertemplate{headline}{}
  \useoutertheme{default}
  \usecolortheme{albatross}
  \setbeamercovered{invisible}
}
% \AtBeginSection[]
% {
% \begin{frame}<beamer>{Contenido}
% \normalfont\mdseries
% \tableofcontents[currentsection]
% \end{frame}
% }

\input{../Preambulos/pre_plantilla_Warsaw_spruce}
\input{../Preambulos/pre_codigo}
\makeatletter
\setbeamercolor{section in foot}{bg=gray!30, fg=black!90!orange}
\setbeamercolor{subsection in foot}{bg=blue!30!yellow, fg=red}
\setbeamertemplate{footline}
{
  \leavevmode%
  \hbox{%
  \begin{beamercolorbox}[wd=.333333\paperwidth,ht=2.25ex,dp=1ex,center]{section in foot}%
    \usebeamerfont{section in foot} \insertsection
  \end{beamercolorbox}}%
  \begin{beamercolorbox}[wd=.333333\paperwidth,ht=2.25ex,dp=1ex,center]{subsection in foot}%
    \usebeamerfont{subsection in foot}  \insertsubsection
  \end{beamercolorbox}%
  \begin{beamercolorbox}[wd=.333333\paperwidth,ht=2.25ex,dp=1ex,right]{date in head/foot}%
    \usebeamerfont{date in head/foot} \insertshortdate{} \hspace*{2em}
    \insertframenumber{} / \inserttotalframenumber \hspace*{2ex} 
  \end{beamercolorbox}}%
  \vskip0pt%
\makeatother
\title{\large{Diferenciación numérica}}
\subtitle{Tema 2 - Operaciones matemáticas básicas}
\author{M. en C. Gustavo Contreras Mayén}
\date{\today}
\institute{Facultad de Ciencias - UNAM}
\titlegraphic{\includegraphics[width=1.75cm]{Imagenes/escudo-facultad-ciencias}\hspace*{4.75cm}~%
   \includegraphics[width=1.75cm]{Imagenes/escudo-unam}
}
\begin{document}
\maketitle
\fontsize{14}{14}\selectfont
\spanishdecimal{.}
\section*{Contenido}
\frame{\tableofcontents[currentsection, hideallsubsections]}
\section{Diferenciación numérica}
\frame{\tableofcontents[currentsection, hideothersubsections]}
\subsection{Problema inicial}
\begin{frame}
\frametitle{Problema inicial}
Dada una función $f(x)$ y un valor $x$, queremos calcular
\[\dfrac{d^{n}f}{d x^{n}} \]
\\
\bigskip
Que representa la $n-$ésima derivada de la función $f$, con respecto a $x$.
\end{frame}
\begin{frame}
\frametitle{Problema inicial}
Usamos la diferenciación numérica para el siguente problema: se nos da una función $y = f(x)$ y deseamos obtener una de sus derivadas en el punto $x = x_{k}$.
\end{frame}
\begin{frame}
\frametitle{Problema inicial}
Cuando decimos \enquote{dada} significa que, o bien tenemos un algoritmo para calcular la función, o contamos con un conjunto de puntos discretos $(x_{i}, y_{i})$, $i = 0, 1,\ldots,N$. 
\\
\bigskip
En cualquier caso, tenemos acceso a un número finito de pares de datos $(x, y)$ a partir de ellos, podemos calcular la derivada.
\end{frame}
\begin{frame}
\frametitle{Diferenciación e interpolación}
Si estás pensando en que la diferenciación numérica se relaciona con interpolación, tienes toda la razón.
\\
\bigskip
Ya que es un medio para encontrar la derivada a partir de una aproximación con un polinomio, y luego diferenciar.
\end{frame}
\begin{frame}
\frametitle{Diferenciación e interpolación}
Una herramienta igualmente eficaz es el desarrollo en serie de Taylor de $f(x)$ sobre el punto de $x_{k}$, lo que representa como ventaja de que nos proporciona información acerca del error cometido en la aproximación.
\end{frame}
\begin{frame}
\frametitle{Punto críticamente importante}
\emph{La diferenciación numérica no es un proceso particularmente exacto}: se presentan un conflicto entre los errores de redondeo (debido a la precisión de la máquina) y los errores inherentes a la interpolación.
\\
\bigskip
\pause
Por esta razón, la derivada de una función no debe de ser calculada con la misma precisión que la propia función.
\end{frame}
\section{Métodos de aproximación}
\frame{\tableofcontents[currentsection, hideothersubsections]}
\subsection{Aproximación por diferencias finitas}
\begin{frame}
\frametitle{Desarrollo por diferencias finitas}
La derivación por aproximación usando diferencias finitas de las derivadas de $f(x)$, se basa en el desarrollo de series de Taylor hacia adelante y hacia atrás de $f(x)$ alrededor de $x$:
\fontsize{11}{11}\selectfont
\begin{align*}
f(\textcolor{red}{x + h}) &= f(x) + h \: f^{\prime}(x) + \dfrac{h^{2}}{2!} \: f^{\prime \prime}(x) + \dfrac{h^{3}}{3!} \: f^{\prime \prime \prime}(x) + \dfrac{h^{4}}{4!} \: f^{4}(x) + \ldots \\
\visible<2->{f(\textcolor{red} {x - h}) &= f(x) - h \: f^{\prime}(x) + \dfrac{h^{2}}{2!} \: f^{\prime \prime} (x) - \dfrac{h^{3}}{3!} \: f^{\prime \prime \prime}(x) + \dfrac{h^{4}}{4!} \: f^{4}(x) - \ldots} \\
\visible<3->{f(\textcolor{blue}{x + 2h}) &= f(x) + 2h \: f^{\prime}(x) + \dfrac{(2h)^{2}}{2!} \: f^{\prime \prime}(x) + \dfrac{(2h)^{3}}{3!} \: f^{\prime \prime \prime}(x) + \dfrac{(2h)^{4}}{4!} \: f^{4}(x) + \ldots} \\
\visible<4->{f(\textcolor{blue}{x - 2h}) &= f(x) - 2h \: f^{\prime}(x) + \dfrac{(2h)^{2}}{2!} \: f^{\prime \prime}(x) - \dfrac{(2h)^{3}}{3!} \: f^{\prime \prime \prime}(x) + \dfrac{(2h)^{4}}{4!} \: f^{4}(x) - \ldots}
\end{align*}
\end{frame}
\begin{frame}
\frametitle{Suma de las diferencias}
\fontsize{12}{12}\selectfont
\begin{align*}
f(\textcolor{red}{x + h}) + f(\textcolor{red}{x - h}) &= 2 \: f(x) + h^{2} \: f^{\prime \prime}(x) + \dfrac{h^{4}}{12} \: f^{4}(x) + \ldots \\
\visible<2->{f(\textcolor{red}{x + h}) - f(\textcolor{red}{x - h}) &= 2 \: h \: f^{\prime}(x) + \dfrac{h^{3}}{3}  \: f^{\prime \prime \prime}(x) + \ldots} \\
\visible<3->{f(\textcolor{blue}{x + 2h}) + f(\textcolor{blue}{x - 2h}) &= 2 \: f(x) + 4\: h^{2} \: f^{\prime \prime}(x) + \dfrac{4h^{4}}{3} \: f^{4}(x) +\ldots} \\
\visible<4->{f(\textcolor{blue}{x + 2h})-f(\textcolor{blue}{x - 2h}) &= 4 \: h \: f^{\prime}(x) + \dfrac{8h^{3}}{3} \: f^{\prime \prime \prime}(x) + \ldots }
\end{align*}
\end{frame}
\subsection{Aproximación por diferencias centrales}
\begin{frame}
\frametitle{Primera derivada por diferencias centrales}
Tomando la segunda suma de la lista anterior, de donde despejamos $f^{\prime}(x)$, tenemos
\[ f^{\prime}(x) = \dfrac{f(x + h) - f(x - h)}{2h} - \dfrac{h^{2}}{6} \: f^{\prime \prime \prime}(x) - \ldots \]
\pause
equivalentemente
\begin{empheq}[box={\mybluebox[5pt]}]{equation*}
   f^{\prime}(x) = \dfrac{f(x + h)-f(x - h)}{2h} + O(h^{2})
\end{empheq}
que es llamada \textcolor{blue}{aproximación por diferencias centrales} de $f^{\prime}(x)$.
\end{frame}
\begin{frame}
\frametitle{Segunda derivada por diferencias centrales}
De manera similar, obtenemos la aproximación por diferencias centrales de $f^{\prime \prime}(x)$:
\[  f^{\prime \prime}(x) =  \dfrac{f(x + h)- 2 \: f(x) + f(x - h)}{h^{2}} + \dfrac{h^{2}}{12} \: f^{4}(x) + \ldots\]
\pause
es decir:
\begin{empheq}[box={\mybluebox[5pt]}]{equation*}
   f^{\prime \prime}(x) =  \dfrac{f(x+h)- 2 \: f(x) + f(x - h)}{h^{2}} + O(h^{2})
\end{empheq}
\end{frame}
\subsection{Aprox. por diferencias adelante/atrás}
\begin{frame}
\frametitle{Aprox. por diferencias hacia adelante/atrás}
Las aproximaciones por diferencias centrales  no siempre son útiles. 
\\
\bigskip
Por ejemplo, considera la situación en que se da la función en los  $n$ puntos discretos $x_{0}, x_{1}, \ldots,x_{n}$.
\\
\bigskip
Dado que las diferencias centrales utilizan valores de la función en cada lado de $x$, no sería posible calcular las derivadas en $x_{0}$ y $x_{n}$.
\end{frame}
\begin{frame}
\frametitle{Aprox. por diferencias hacia adelante/atrás}
Es cierto que necesitamos una expresión para diferencias finitas que evalúe la función en un solo lado de $x$. 
\\
\bigskip
Estas expresiones son llamadas aproximaciones por diferencias finitas \textcolor{red}{hacia adelante} y \textcolor{blue}{hacia atrás}.
\end{frame}
\begin{frame}
Despejamos $f^{\prime}(x)$ de la primera lista de expresiones por diferencias, para obtener:
\[ f^{\prime}(x) = \dfrac{f(\textcolor{red}{x + h}) - f(x)}{h} - \dfrac{h}{2} f^{\prime\prime}(x) - \dfrac{h^{2}}{6} f^{\prime \prime \prime}(x) - \ldots \]
Manteniendo los primeros términos, tenemos la aproximación por diferencias centrales hacia adelante:
\begin{empheq}[box={\mybluebox[5pt]}]{equation*}
   f^{\prime}(x) = \dfrac{f(\textcolor{red}{x + h}) - f(x)}{h} + O(h)
\end{empheq}
\end{frame}
\begin{frame}
\frametitle{Aproximación hacia atrás}
De manera análoga, obtenemos la aproximación por diferencias hacia atrás:
\begin{empheq}[box={\mybluebox[5pt]}]{equation*}
   f^{\prime}(x) = \dfrac{f(x) - f(x - h)}{h} + O(h)
\end{empheq}
Hay que hacer notar que el error es $O(h)$, que no es tan bueno como $O(h^{2})$ en la aproximación por diferencias centrales.
\end{frame}
\subsection{Segunda aprox. por diferencias centrales}
\begin{frame}
\frametitle{Segunda aprox. por diferencias centrales}
Las aproximaciones por diferencias centrales del tipo $O(h)$ no son tan populares debido a que es más común usar expresiones del tipo $O(h^{2})$.
\\
\bigskip
Para obtener una fórmula con diferencias centrales con ese orden de error, tenemos que incluir más términos de la serie de Taylor.
\end{frame}
\begin{frame}
\frametitle{Desarrollo}
Consideremos
\fontsize{11}{11}\selectfont
\begin{align*}
f(\textcolor{red}{x + h}) &= f(x) + h \: f^{\prime}(x) + \dfrac{h^{2}}{2!}f^{\prime \prime}(x) + \dfrac{h^{3}}{3!}f^{\prime \prime \prime}(x) + \dfrac{h^{4}}{4!}f^{4}(x) + \ldots \\
f(\textcolor{red}{x + 2h}) &= f(x) + 2h \: f^{\prime}(x) + \dfrac{(2h)^{2}}{2!}f^{\prime \prime}(x) + \dfrac{(2h)^{3}}{3!}f^{\prime \prime \prime}(x) + \\
&+ \dfrac{(2h)^{4}}{4!}f^{4}(x) + \ldots
\end{align*}
\end{frame}
\begin{frame}
\frametitle{Desarrollo}
Eliminamos $f^{\prime \prime} (x)$ multiplicando la primera ecuación y restándola de la segunda, para obtener:
\fontsize{11}{11}\selectfont
\[ f(x + 2h) - 4 \: f(x + h) = -3 \: f(x) - 2 \: h \: f^{\prime}(x) + \dfrac{2 \: h^{3}}{3} f^{\prime \prime \prime}(x) + \ldots\]
\end{frame}
\begin{frame}
\frametitle{Segunda aprox. hacia adelante}
Por lo tanto
\[ f^{\prime}(x) = \dfrac{-f(\textcolor{red}{x + 2 \: h}) + 4 \: f(\textcolor{red}{x + h}) - 3 \: f(x)}{2 \: h} + \dfrac{h^{2}}{3} \: f^{\prime \prime \prime}(x) + \ldots\]
entonces
\pause
\begin{empheq}[box={\mybluebox[5pt]}]{equation*}
   f^{\prime}(x) = \dfrac{-f(\textcolor{red}{x + 2h}) + 4 \: f(\textcolor{red}{x + h}) - 3 \: f(x)}{2h} + O(h^{2})	
\end{empheq}
es la expresión para la segunda aproximación de la derivada por diferencias finitas hacia adelante.
\end{frame}
\subsection{Errores en las aprox. por diferencias finitas}
\begin{frame}
\frametitle{Errores en las aprox. por diferencias finitas}
El efecto del error por redondeo puede ser profundo, si $h$ es muy pequeña, valores de $f(x)$, $f(x \pm h)$, $f(x \pm 2 \: h)$, etc. serán aproximadamente iguales.
\end{frame}
\begin{frame}
\frametitle{Errores en las aprox. por diferencias finitas}
Cuando se multiplican por los coeficientes y se suman, podemos perder un número grande de términos. 
\\
\bigskip
Por otro lado, no debemos de hacer muy grande el valor de $h$, ya que el error debido al truncamiento, sería excesivo.
\end{frame}
\begin{frame}
\frametitle{Control de los errores}
Para manejar esta situación que siempre se va a presentar, podemos apoyarnos con las siguientes opciones:
\setbeamercolor{item projected}{bg=purple!70!black,fg=white}
\setbeamertemplate{enumerate items}[circle]
\begin{enumerate}[<+->]
\item Aumentar la precisión en el tipo de dato.
\item Usar diferencias finitas en donde se alcance el orden de $O(h^{2})$
\end{enumerate}
\end{frame}
\subsection*{Primer ejercicio}
\begin{frame}
\frametitle{Hagamos un ejercicio:}
A partir de la fórmula de diferencias centrales, calcula la derivada de $f(x)=exp(-x)$ en $x = 1$.
\\
\bigskip
Inicia con $h = 0.64$ y divide a la mitad su valor en diez ocasiones ($h = 0.64, 0.32, 0.16, \ldots)$. 
\\
\bigskip
Calcula el error relativo para cada $h$, el valor exacto es $f^{\prime}(1) = exp(-1) = 0.36787944$
\end{frame}
\begin{frame}
\frametitle{El código en \python}
Usando el módulo \funcionazul{moduloDerivadas}, genera una rutina que itere en $10$ ocasiones para dividir a la mitad el valor de $h$.
\end{frame}
\begin{frame}[allowframebreaks, plain, fragile]
\frametitle{Código}
\begin{lstlisting}[caption=Usando el módulo de diferenciación, style=FormattedNumber, basicstyle=\linespread{1.1}\ttfamily=\small, columns=fullflexible]
from moduloDerivadas import ddifcentral
from math import exp

def funcion(x):
    return exp(-x)
    
h = 0.64
x = 1.
ddexacta = 0.36787944


for i in range(1, 11):
    print ('{} \t {} \t {:1.5e}'.format(h, ddifcentral(funcion, x, h), \
           (abs(ddexacta - ddifcentral(funcion, x, h))) * 100/ddexacta))
    h = h * 0.5
\end{lstlisting}
\end{frame}
\begin{frame}
\fontsize{12}{12}\selectfont
\begin{center}
\begin{tabular}{c | l | l}
\hline
h & Priimera derivada & Error \\ \hline
$0.64$ & $0.380609096726$ & $3.46028e+00$ \\ \hline
$0.32$ & $0.371029413951$ & $8.56252e-01$ \\ \hline
$0.16$ & $0.368664920656$ & $2.13516e-01$ \\ \hline
$0.08$ & $0.368075685401$ & $5.33450e-02$ \\ \hline
$0.04$ & $0.36792849438$ & $1.33344e-02$ \\ \hline
$0.02$ & $0.367891703983$ & $3.33370e-03$ \\ \hline
$0.01$ & $0.367882506844$ & $8.33655e-04$ \\ \hline
$0.005$ & $0.367880207588$ & $2.08652e-04$ \\ \hline
$0.0025$ & $0.367879632774$ & $5.24015e-05$ \\ \hline
$0.00125$ & $0.36787948904$ & $1.33306e-05$ \\ \hline
\end{tabular}
\end{center}
\end{frame}
\subsection*{Segundo ejercicio}
\begin{frame}
\frametitle{Un ejercicio más complicado}
El siguiente arreglo tiene por dimensiones $a=100$ mm, $b=120$ mm, $c=150$ mm y $d=180$ mm
\begin{figure}
	\centering
	\includestandalone[scale=1.3]{Figuras/diferenciacion_01}
	\caption{Sistema de barras articulado}
\end{figure}
\end{frame}
\begin{frame}
\frametitle{Segundo ejercicio}
\begin{figure}
  \centering
  \includestandalone[scale=1.3]{Figuras/diferenciacion_01}
\end{figure}
Las barras son rígidas y están articuladas en los puntos de unión, por lo que al moverse una de ellas, las otras también de desplazan.
\end{frame}
\begin{frame}
\frametitle{Consideración geométrica}
Se puede demostrar a partir de la geometría del problema que la relación entre los ángulos $\alpha$ y $\beta$ es:
\[ (d - a \cos \alpha - b \cos \beta)^{2} + (a \sin \alpha + b \sin \beta)^{2} - c^{2} = 0\]
\pause
Para un valor dado de $\alpha$, podemos resolver la ecuación para $\beta$, mediante alguno de los métodos para encontrar raíces que ya hemos visto. 
\end{frame}
\begin{frame}
\frametitle{Valores de $\alpha$ y de $\beta$}
Haciendo para $\alpha=0^{\circ}, 5^{\circ}, 10^{\circ},\ldots, 30^{\circ}$, los resultados son:
\begin{center}
\fontsize{10}{10}\selectfont
\begin{tabular}{c | c | c | c | c | c | c | c}
$\alpha$ (grados) & $0$ & $5$ & $10$ & $15$ & $20$ & $25$ & $30$  \\ \hline
$\beta$ (rad) & $1.6595$ & $1.5434$ & $1.4186$ & $1.2925$ & $1.1712$ & $1.0585$ & $0.9561$
\end{tabular}
\end{center}
\end{frame}
\begin{frame}
\frametitle{Enunciado del ejercicio}
Si el segmento AB gira con velocidad angular constante de $\SI{25}{\radian\per\second}$.
\\
\bigskip
Con el método de diferencias finitas de orden $O(h^{2})$, calcula la velocidad angular $d\beta / dt$ del segmento BC contra el ángulo $\alpha$.
\end{frame}
\begin{frame}
\frametitle{Solución al ejercicio}
La velocidad angular de BC es:
\[ \dfrac{d\beta}{dt} = \dfrac{d\beta}{d\alpha}\dfrac{d\alpha}{dt} = 25\dfrac{d\beta}{d\alpha} \si{\radian\per\second}\]
donde $d\beta/d\alpha$ se puede estimar con una aproximación de diferencias finitas, tomando los datos de la tabla anterior.
\end{frame}
\begin{frame}
\frametitle{Solución al ejercicio}
Para los puntos extremos se usarían las diferencias hacia adelante y hacia atrás de orden $O(h^{2})$, mientras que para los puntos de en medio, el cálculo se hace con las diferencias centrales.
\end{frame}
\begin{frame}
\frametitle{Solución al ejercicio}
Nótese que el incremento de $\alpha$ es
\[h = (\SI{5}{\degree}) \left( \frac{\pi}{180} \si{\radian} / \si{\degree} \right) = \SI{0.087266}{\radian}\]
Así tenemos que:
\fontsize{12}{12}\selectfont
\begin{align*}
\dot{\beta}(\SI{0}{\degree}) &=  25 \dfrac{-3 \beta(\SI{0}{\degree})+ 4 \beta(\SI{5}{\degree})-\beta(\SI{10}{\degree})}{2h} = -\SI{32.01}{\radian\per\second} \\
\dot{\beta}(\SI{5}{\degree}) &= \dfrac{\beta(\SI{10}{\degree})- \beta(\SI{0}{\degree})}{2h} = -\SI{34.51}{\radian\per\second}
\end{align*}
\end{frame}
\begin{frame}
\frametitle{Ejercicio para entregar: Completa la tabla}
\begin{center}
\fontsize{12}{12}\selectfont
\begin{tabular}{c | c | c | c | c | c | c | c}
$\alpha$ (grados) & $\SI{0}{\degree}$ & $\SI{5}{\degree}$ & $\SI{10}{\degree}$ & $\SI{15}{\degree}$ & $\SI{20}{\degree}$ & $\SI{25}{\degree}$ & $\SI{30}{\degree}$  \\ \hline
$\dot{\beta}$ $(\si{\radian\per\second})$ & $-32.01$ & $-34.51$ &  &  &  &  & 
\end{tabular}
\end{center}
\textcolor{red}{Debes de entregar la tabla y el código en \python.}
\end{frame}
\begin{frame}
\frametitle{Solución usando lo que tenemos}
La primera pregunta que planteamos es: \emph{¿podemos utilizar las funciones de las aproximaciones por diferencias finitas?}
\\
\bigskip
\pause
La respuesta es no, de manera directa, pero podemos hacer un arreglo para utilizar el desarrollo por diferencias finitas.
\end{frame}
\begin{frame}
\frametitle{Diferencia entre función y pares de datos}
En el ejemplo anterior con la función exponencial, tenemos una expresión que se evalúa para el dato (o conjunto de datos) en particular.
\\
\bigskip
Cuando tenemos un conjunto de datos de una tabla, propiamente la expresión para la función no la tenemos de manera explícita.
\end{frame}
\begin{frame}
\frametitle{Ajuste a las expresiones}
Es por ello que debemos de hacer un ajuste en las expresiones de las diferencias finitas para el conjunto de datos de la tabla.
\end{frame}
\begin{frame}
\frametitle{¿Cómo lo resolvemos?}
Con los tips que nos dan en el enunciado, para utilizar en los extremos una aproximación de la derivada hacia adelante y hacia atrás, y para los puntos centrales, con una aproximación por diferencias centrales.
\\
\bigskip
Debemos de adaptar el algoritmo que revisamos, ya que tenemos un conjunto de puntos discretos de la tabla.
\end{frame}
\begin{frame}[fragile]
\frametitle{Diferencias hacia adelante con puntos}
El algoritmo ahora considera el conjunto de puntos de la lista:
\begin{lstlisting}[caption=Algoritmo ajustado para el conjunto de puntos, style=FormattedNumber, basicstyle=\linespread{1.1}\ttfamily=\small, columns=fullflexible]
def seg-aprox-dif-adelante-puntos(f_2_h, fh, f, h):
    valor = (-f_2_h + 4 * fh - 3 * f)/(2 * h)
    return valor
\end{lstlisting}
\end{frame}
\begin{frame}[fragile]
\frametitle{Comentario del algoritmo}
En el código anterior se expresa
\begin{verbatim}
f2h --> f(x + 2h)
fh  --> f(x + h)
\end{verbatim}
que sería el equivalente para el caso que se utilice la función $f$ con una expresión.
\end{frame}
\begin{frame}[fragile]
\frametitle{Diferencias centrales con puntos}
Para el caso de las diferencias centrales, tendremos
\begin{lstlisting}[caption=Algoritmo ajustado para el conjunto de puntos, style=FormattedNumber, basicstyle=\linespread{1.1}\ttfamily=\small, columns=fullflexible]
def ddifcentralespuntos(fh, hf, h):
    valor = (fh - hf)/(2 * h)
    return valor
\end{lstlisting}
\end{frame}
\begin{frame}[fragile]
\frametitle{Comentario del algoritmo}
Ahora en este código anterior se expresa
\begin{verbatim}
fh --> f(x + h)
hf --> f(x - h)
\end{verbatim}
Esta aclaración es conveniente para evitar que haya un embrollo al momento de usar los puntos de la tabla.
\end{frame}
\begin{frame}
\frametitle{La expresión pendiente}
Nos restaría encontrar una expresión a partir de la lista de $f(\textcolor{red}{x - 2h})$ y de $f(\textcolor{red}{x - h})$, de tal manera que la aproximación de la primera derivada, sea del orden de $h^{2}$.
\end{frame}
\begin{frame}
\frametitle{El código completo}
Una vez con los tres algoritmos necesarios, procedemos a utilizarlos para obtener el resultado que nos pide el ejercicio.
\\
\bigskip
El manejo de los índices de la lista de valores, será de gran utilidad para los cálculos.
\end{frame}
\begin{frame}[allowframebreaks, plain, fragile]
\frametitle{Propuesta de código}
\begin{lstlisting}[caption=Propuesta de código completo, style=FormattedNumber, basicstyle=\linespread{1.1}\ttfamily=\small, columns=fullflexible]
from moduloDerivadas import seg_aprox_dif_adelante_puntos, ddif_centrales_puntos, seg_aprox_dif_atras_puntos

alfas = [0., 5., 10., 15., 20., 25., 30.]

betas = [1.6595, 1.5434, 1.4186, 1.2925, 1.1712, 1.0585, 0.9561]

h = 0.087266

print('{:2.3f}'.format(25 * seg_aprox_dif_adelante_puntos(betas[_2_], betas[_1_], betas[_0_], h)))

for i in range(1, len(betas) - 1):
    print(('{:2.3f}'.format(25 * ddif_centrales_puntos(betas[i + _1_], betas[i - _1_], h))))
    
print('{:2.3f}'.format(25*seg_aprox_dif_atras_puntos(betas[-1 - 2], betas[-1 - 1], betas[-1], h)))
\end{lstlisting}    
\end{frame}
\begin{frame}[fragile]
\frametitle{Comentarios del código}
Habrás notado que el código evalúa directamente el primer valor de la lista así como el último, debido a que se usan las diferencias hacia adelante y hacia atrás.
\end{frame}
\begin{frame}[fragile]
\frametitle{Comentarios del código}
Para los puntos intermedios necesitamos iterar para leer los datos centrales, checa que en la secuencia \verb|range(1, len(betas) - 1)|, iniciamos la secuencia en $1$, tomamos el tamaño completo de la lista y le restamos un $-1$, así garantizamos que no nos excedemos en los índices.
\end{frame}
\begin{frame}
\frametitle{Ejercicio resuelto}
\begin{center}
\fontsize{12}{12}\selectfont
\begin{tabular}{c | c | c | c | c | }
$\alpha$ (grados) & $\SI{0}{\degree}$ & $\SI{5}{\degree}$ & $\SI{10}{\degree}$ & $\SI{15}{\degree}$ \\ \hline
$\dot{\beta}$ $(\si{\radian\per\second})$ & $-32.01$ & $-34.51$ & $-35.94$ & $-35.44$ 
\end{tabular}
\end{center}
\begin{center}
\fontsize{12}{12}\selectfont
\begin{tabular}{c | c | c | c }
$\alpha$ (grados) & $\SI{20}{\degree}$ & $\SI{25}{\degree}$ & $\SI{30}{\degree}$  \\ \hline
$\dot{\beta}$ $(\si{\radian\per\second})$ & $-33.52$ & $-30.81$  & $-27.86$ 
\end{tabular}
\end{center}
\textcolor{red}{Debe de entregarse la tabla y el código en python.}
\end{frame}
\section{Extrapolación de Richardson}
\frame{\tableofcontents[currentsection, hideothersubsections]}
\subsection{Fundamento de la extrapolación}
\begin{frame}
\frametitle{Extrapolación de Richardson}
La \emph{Extrapolación de Richardson} es un método sencillo para aumentar la precisión de ciertos procedimientos numéricos, incluyendo las aproximaciones por diferencias finitas.
\end{frame}
\begin{frame}
\frametitle{Extrapolación de Richardson}
Supongamos que tenemos la forma de calcular una cantidad $G$. 
\\
\bigskip
Por otra parte, si consideramos que el resultado depende de un parámetro $h$, hagamos la aproximación por $g(h)$, tenemos que $G = g(h) + E(h)$, donde $E(h)$ representa el error.
\end{frame}
\begin{frame}
\frametitle{Extrapolación de Richardson}
La extrapolación de Richardson puede remover el error, siempre que tenga la forma $E(h) = c \: h^{p}$, donde $c$ y $p$ son constantes.
\\
\bigskip
\pause
Iniciamos el cálculo para varios valores de $h$, digamos $h = h_{1}$, así
\[ G = g(h_{1}) + c \; h_{1}^{p}\]
Repetimos el cálculo con $h = h_{2}$, por tanto:
\[ G = g(h_{2}) + c \; h_{2}^{p} \]
\end{frame}
\begin{frame}
\frametitle{Fórmula de la extrapolación}
Eliminando $c$ y resolviendo para $G$, tenemos:
\begin{empheq}[box={\mybluebox[5pt]}]{equation*}
   G = \dfrac{(h_{1}/h_{2})^{p} g(h_{2}) - g(h_{1})}{(h_{1}/h_{2})^{p}-1}
\end{empheq}
que es la fórmula de Extrapolación de Richardson. 
\end{frame}
\begin{frame}
\frametitle{Fórmula de la extrapolación}
En la práctica se usa $h_{2} = h_{1}/2$, quedando
\begin{empheq}[box={\mybluebox[5pt]}]{equation*}
   G = \dfrac{2^{p} g(h_{1}/2) - g(h_{1})}{2^{p} - 1}
\end{empheq}
\end{frame}
\begin{frame}
\frametitle{Ejercicio}
Usemos el ejemplo de $(exp(-x))^{\prime \prime}$ en $x = 1$, consideremos los valores de la tabla con seis dígitos.
\\
\bigskip
Dado que la extrapolación contine errores por truncamiento, debemos limitarnos a los valores de $h$ que producen redondeo insignificante.
\end{frame}
\begin{frame}
\frametitle{Ejercicio}
De la tabla anterior que calculamos, tenemos que:
\[ g(0.64) = 0.380609 \hspace{2cm} g(0.32) = 0.371029\]
El error de truncamiento en la aproximación central por diferencias finitas es:
\[E(h) = O(h^{2}) = c_{1} \: h^{2} + c_{2} \: h^{4} + c_{3} \: h^{6} + \ldots\]
\end{frame}
\begin{frame}
\frametitle{Ejercicio}
Por lo que podemos eliminar el primer término del error (dominante), si usamos $p = 2$ y $h_{1} = 0.64$, así
\[ \begin{split}
G =& \dfrac{2^{2}g(0.32)- g(0.64)}{2^{2}-1} = \dfrac{4(0.371035)-0.380610}{3} \\
 =& 0.367843
\end{split} \]
\end{frame}
\begin{frame}
\frametitle{Ejercicio}
Que es una aproximación de $(exp(-x))^{\prime \prime}$ con un error $=O(h^{4})$.
\\
\bigskip
Este valor representa el mejor valor obtenido en comparación de los obtenidos previamente.
\end{frame}
\begin{frame}
\frametitle{Segundo Ejercicio}
Teniendo en cuenta los siguientes puntos de datos uniformemente espaciados:
\begin{center}
\begin{tabular}{c | c | c | c | c | c }
x & $0$ & $0.1$ & $0.2$ & $0.3$ & $0.4$ \\ \hline
f(x) & $0.0000$ & $0.0819$ & $0.1341$ & $0.1646$ & $0.1797$
\end{tabular}
\end{center}
Calcula $f^{\prime}(x)$ y $f^{\prime\prime}(x)$ en $x = 0$ y $x = 0.2$, usando la aproximación por diferencias finitas de orden $O(h^{2})$.
\end{frame}
\begin{frame}
\frametitle{Solución}
Usando la aproximación por diferencias finitas de orden $O(h^{2})$, de la lista de diferencias hacia adelante, tenemos:
\[ \begin{split} 
f^{\prime}(0) =& \dfrac{-3 \: f(0) + 4 \: f(0.1) - f(0.2)}{2 \: (0.1)} = 0.967 \\
f^{\prime \prime}(0) =& \dfrac{2 \: f(0) - 5 \: f(0.1) + 4 \: f(0.2) - f(0.3)}{(0.1)^{2}} = -3.77
\end{split} \]
\end{frame}
\begin{frame}
Si usamos ahora la aproximación por diferencias centrales:
\[ \begin{split}
f^{\prime}(0.2) =& \dfrac{-f(0.1) + f(0.3)}{2(0.1)} = 0.4135 \\
f^{\prime \prime}(0.2) =& \dfrac{f(0.1) - 2 \: f(0.2) + f(0.3)}{(0.1)^{2}} = -2.17
\end{split} \]
\end{frame}
\begin{frame}
\frametitle{Tercer Ejemplo}
Usando los siguientes datos(del ejemplo anterior):
\begin{center}
\begin{tabular}{c | c | c | c | c | c }
x & $0$ & $0.1$ & $0.2$ & $0.3$ & $0.4$ \\ \hline
f(x) & $0.0000$ & $0.0819$ & $0.1341$ & $0.1646$ & $0.1797$
\end{tabular}
\end{center}
Calcula el valor de  $f^{\prime}(0)$ con la mayor precisión posible.
\end{frame}
\begin{frame}
\frametitle{Tercer Ejemplo}
Una solución es usar el método de extrapolación de Richardson con aproximación de diferencias finitas.
\end{frame}
\begin{frame}
\frametitle{Solución}
Iniciamos con la segunda aproximación por diferencias hacia adelante de orden $O(h^{2})$ para $f^{\prime}(0)$: usamos en una $h = 0.2$ y en otra $h = 0.1$
\[ \begin{split}
g(0.2) =& \dfrac{-3 \: f(0) + 4 \: f(0.2) - f(0.4)}{2 \: (0.2)} = 0.8918 \\
g(0.1) =& \dfrac{-3 \: f(0) + 4 \: f(0.1) - f(0.2)}{2 \: (0.1)} = 0.9675
\end{split} \]
\end{frame}
\begin{frame}
\frametitle{Solución}
Recordemos que el error en ambas aproximaciones, es de la forma $E(h) = c_{1} \: h^{2} + c_{2} \: h^{4} + c_{3} \: h^{5} + \ldots$. 
\end{frame}
\begin{frame}
\frametitle{Solución}
Usamos la extrapolación de Richardson para eliminar el término dominante. Con $p = 2$, resulta
\[ f^{\prime}(0) \simeq G = \dfrac{2^{2} \: g(0.1) - g(0.2)}{2^{2} - 1} = 0.9927 \]
la cual es una aproximación por diferencias finitas de orden $O(h^{4})$.
\end{frame}
\end{document}
