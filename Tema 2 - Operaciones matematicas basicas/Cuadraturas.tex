\documentclass[12pt]{article}
\usepackage[utf8]{inputenc}
\usepackage[spanish]{babel}
\usepackage{amsmath}
\usepackage{amsthm}
\usepackage{anysize}
\marginsize{2cm}{2cm}{0cm}{2cm}
\author{M. en C. Gustavo Contreras Mayén.}
\title{Ejercicios para el Tema 2.\\ Operaciones matemáticas básicas\\ \begin{large}Curso Física Computacional\end{large}}
\date{ }
\begin{document}
\maketitle
\section{Cuadraturas de Gauss-Legendre.}
Las cuadraturas de Gauss-Legendre son métodos de integración numérica que utilizan puntos de Legendre (raíces de polinomios de Legendre); las cuadraturas de Gauss no se pueden utilizar para integrar una función dad de la forma de tabla con intervalos de separación uniforme debido a que los puntos de Legendre no están separados de esa manera, sin embargo, son más adecuados para integrar funciones analíticas.
\\
La cuadratura de Gauss que se extiende en el intervalo $[-1,1]$ está dada por
\begin{equation}{\label{eq:integral}}
\int_{-1}^{1} f(x) dx = \sum_{k=1}^{N}w_{k}f(x_{k})
\end{equation}
donde $N$ es el número de puntos de Gauss, los $w_{i}$ son los pesos y las $x_{i}$ son los puntos de Gauss, que se indican en la siguiente tabla: \\
\\
\begin{tabular}{c l l}
 & $\pm x_{i}$ & $w_{i}$ \\
\hline N=2 & 0.577350269  & 1,000000000 \\
\hline N=3 & 0            & 0.888888889 \\
    & 0.7745986669 & 0.555555556 \\
\hline N=4 & 0.339981043  & 0.652145155 \\
    & 0.861136312  & 0.347854845 \\
\hline N=5 & 0            & 0.568888889 \\
    & 0.538469310  & 0.478628670 \\
    & 0.906179846  & 0.236926885 \\
\hline N=6 & 0.238619186  & 0.467913935 \\
    & 0.661209387  & 0.360761573 \\
    & 0.932469514  & 0.171324492 \\
\hline N=8 & 0.183434642  & 0.362683783 \\
    & 0.525532410  & 0.313706646 \\
    & 0.796666478  & 0.222381034 \\
    & 0.960289857  & 0.101228536 \\
\hline N=10& 0.148874339  & 0.295524225 \\
    & 0.433395394  & 0.269266719 \\
    & 0.865063367  & 0.149451349 \\
    & 0.973906528  & 0.066671344 \\
\end{tabular}
\\
\\
Los signos $\pm$ en la tabla significan que los valores de $x$ de los puntos de Gauss aparecen son pares, uno de los cuales es positivo y el otro negativo.
\\
\\
La fórmula de integración de Gauss puede aplicarse a cualquier intervalo arbitrario $[a,b]$ con la transformación
\begin{equation}{\label{eq:lasx}}
 x = \dfrac{2z-a-b}{b-a}
 \end{equation}
donde $z$ es la coordenada original en $a<z<b$ y x es la coordenada normalizada en $-1<x<1$. La transformación de $x$ en $z$ es
\begin{equation}{\label{eq:lasz}}
 z = \dfrac{(b-a)x+a+b}{2}
\end{equation}
Por medio de ésta transformación, la integral puede escribirse como
\begin{equation}{\label{eq:cambiovar}}
\int_{a}^{b} f(z) dz = \int_{-1}^{1} {f(z)} \dfrac{dz}{dx} dx  = \dfrac{b-a}{2} \sum_{k=1}^{N} w_{k}f(z_{k})
\end{equation}
donde $\dfrac{dz}{dx}=\dfrac{b-a}{2}$. Los valores de $x_{k}$ se obtienen al sustituir $x$ en la ecuación (\ref{eq:lasz}) por los puntos de Gauss, a saber:
\begin{equation}
z_{k} = \dfrac{(b-a)x_{k}+a+b}{2}
\end{equation}
\section{Otras cuadraturas de Gauss.}
Las cuadraturas de Gauss analizadas en la sección anterior se llaman cuadraturas de Gauss-Legendre por que se basan en la ortogonalidad de los polinomios de Legendre. Existen cuadraturas análogas con base en polinomos de Hermite, de Laguerre y de Chebyshev, que reciben por nombre cuadraturas de Gauss-Hermite, Gauss-Laguerre y Gauss-Chebyshev, respectivamente.
\\
Las cuadraturas de Gauss-Hermite son adecuadas para
\begin{equation}
\int_{-\infty}^{\infty} exp(-x^{2}) f(x) dx
\end{equation}
y están dadas por
\begin{equation} {\label{eq:hermite}}
\int_{-\infty}^{\infty} exp(-x^{2}) f(x) dx = \sum_{k=1}^{N}w_{k}f(x_{k})
\end{equation}
\\
En la ecuación (\ref{eq:hermite}) los $x_{k}$ son raíces del polinomio de Hermite de orden $N$ y los $w_{k}$ son los pesos (\textbf{Nota:} no son los mismos puntos de la tabla que se presentaron en la Sección 1. Consulta una referencia de tablas matemáticas.)
\\
\\
Las cuadraturas de Gauss-Laguerre son adecuadas para
\begin{equation}
\int_{0}^{\infty} exp(-x) f(x) dx
\end{equation}
y están dadas por
\begin{equation}
\int_{0}^{\infty} exp(-x) f(x) dx = \sum_{k=1}^{N}w_{k}f(x_{k})
\end{equation}
donde los $x_{k}$ son raíces del polinomio de Laguerre de orden $N$ y los $w_{k}$ son los pesos.
\\
\\
Las cuadraturas de Gauss-Chebyshev son adecuadas para
\begin{equation}
\int_{-1}^{1} \dfrac{1}{\sqrt{1-x^{2}}} f(x) dx
\end{equation}
y están dadas por
\begin{equation} {\label{eq:cheby}}
\int_{-1}^{1} \dfrac{1}{\sqrt{1-x^{2}}} f(x) dx = \sum_{k=1}^{N} w_{k} f(x_{k})
\end{equation}
En la ecuación (\ref{eq:cheby}) los $x_{k}$ son las raíces de los polinomios de Chebyshev de orden $N$ y los $w_{k}$ son los pesos. Las raíces de los polinomios de Chebyshev de orden $N$ son
\begin{equation}
x_{k} = \cos \dfrac{k-1/2}{N} \pi, \hspace{1cm} k=1,2,\ldots,N
\end{equation}
Los pesos son
\begin{equation}
w_{k} = \dfrac{\pi}{N} \hspace{1cm} \text{para toda k}
\end{equation}
Así la ecuación (\ref{eq:cheby}) se reduce a
\begin{equation}
\int_{-1}^{1} \dfrac{1}{\sqrt{1-x^{2}}} f(x) dx = \dfrac{\pi}{N} \sum_{k=1}^{N} f(x_{k})
\end{equation}
Los límites de integración de $[-1,1]$ se pueden cambiar a un dominio arbitrario $[a,b]$, mediante la ecuación (\ref{eq:cambiovar}).
\end{document}