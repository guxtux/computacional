\documentclass[12pt]{article}
\usepackage[utf8]{inputenc}
\usepackage[spanish]{babel}
\usepackage{amsmath}
\usepackage{amsthm}
\usepackage{graphicx}
\usepackage{float}
\usepackage{anysize}
\usepackage{color}
\usepackage{listings}
\lstset{ %
language=C++,                % choose the language of the code
basicstyle=\small,       % the size of the fonts that are used for the code
numbers=left,                   % where to put the line-numbers
numberstyle=\footnotesize,      % the size of the fonts that are used for the line-numbers
stepnumber=1,                   % the step between two line-numbers. If it is 1 each line will be numbered
numbersep=5pt,                  % how far the line-numbers are from the code
backgroundcolor=\color{white},  % choose the background color. You must add \usepackage{color}
showspaces=false,               % show spaces adding particular underscores
showstringspaces=false,         % underline spaces within strings
showtabs=false,                 % show tabs within strings adding particular underscores
frame=single,   		% adds a frame around the code
tabsize=4,  		% sets default tabsize to 2 spaces
captionpos=b,   		% sets the caption-position to bottom
breaklines=true,    	% sets automatic line breaking
breakatwhitespace=false,    % sets if automatic breaks should only happen at whitespace
escapeinside={\%}{)}          % if you want to add a comment within your code
}
\marginsize{1cm}{1cm}{2cm}{2cm}
\author{M. en C. Gustavo Contreras Mayén.}
\title{Curso de Física Computacional \\ Integración por método secante y regla de Simpson}
\date{ }
\begin{document}
\maketitle
\fontsize{14}{14}\selectfont
\section{Descripción.}
El programa integra una función analítica, ya sea mediante la regla extendida del trapecio o mediante la regla extendida de Simpson, según la elección del usuario.
\\
Antes de ejecutar el programa, el usuario debe de definir el integrando en el subprograma \texttt{FUNC}. El usuario puede dar como entrada la elección de un método de integración, los límites de integración y el número de intervalos, en forma interactiva desde el teclado. Si el número de intervalos para la regla de Simpson es impar, se utiliza la regla de 3/8 para los tres primero interalos de la retícula y después se utiliza la regla extendida de 1/3 para el resto del dominio.
\section{Variables.}
\texttt{ISIMP}: elección del método (0 = Regla del trapecio, 1 = Regla de Simpson) \\
\texttt{A, B}: límite inferior y superior de integración, respectivamente. \\
\texttt{N}: número de intervalos en la retícula.
\texttt{H}: espaciamiento, $H=(B-A)/N$ \\
\texttt{W}: valores de los pesos en las fórmulas de integración.\\
\texttt{S, SS}: integral.\\
\texttt{LS}: último punto de la retícula para la regla de 3/8.
\section{Código.}
\begin{lstlisting}
PROGRAM integracion
	
	COMMON A, B, H
!	ISIMP = 0 ejecuta regla del trapecio, ISIMP = 1 ejecuta regla de Simpson
!	A,B limite inferior y superior de integracion
!	N = numero de intervalos de la reticula
!	H = espaciamiento, H = (B-A)/N
!	W = valores de los pesos en las formulas de integracion
!	S, SS = integral
!	LS = ultimo punto para la regla de 3/8

	PRINT *
	PRINT *, 'Regla del trapecio y de Simpson'
	PRINT *
	PRINT *, 'La funcion a integrar se debe de codificar en el Subprograma Func'
	PRINT *
10	PRINT *, 'Oprime 0 para la regla del trapecio, 1 para la regla de Simpson'
	READ *, ISIMP
	PRINT *, 'Numero de intervalos?'
	READ *, N
135	IF (N .GT. 0 .AND. ISIMP .EQ. 0) GOTO 140
	IF (ISIMP .EQ. 1 .AND. N .GT. 1) GOTO 140
		PRINT *, 'El dato que tecleaste no es valido'
		GOTO 10
140	PRINT *, 'Limite inferior de integracion?'
	READ *, A
150	PRINT *, 'Limite superior de integracion?'
	READ *, B
160	H = (B-A)/N
	IF (ISIMP .EQ. 0) THEN
		CALL TRAPZ (S,N)
		GOTO 200
	ELSE
		CALL SIMPS(S,N)
	END IF
200	PRINT *, '------------------------------------------------------------------------------'
210	PRINT *, 'Resultado final',S
220	PRINT *, '------------------------------------------------------------------------------'
	PRINT *
	PRINT *, 'Oprime 1 para continuar, 0 para terminar'
	READ *, K
	IF (K .EQ. 1) GOTO 10
	PRINT *
	END PROGRAM integracion

!++++++++++++++++++++++++++++++++
	SUBROUTINE TRAPZ(S,N)
	COMMON A, B, H
	S = 0
	DO 10 I = 0, N
		X = A+I*H
		W= 2
		IF (I .EQ. 0 .OR. I .EQ. N) W= 1
		S = S+W*FUNC(X)
		PRINT *, I, X, H, S, FUNC(X), W
10	CONTINUE
	S = S*H/2
	RETURN
	END
!+++++++++++++++++++++++++++++++++
	SUBROUTINE SIMPS(SS,N)
	COMMON A, B, H
	S=0
	SS=0


	IF (N/2*2 .EQ. N) THEN
		LS=0
		GOTO 35
	END IF
	LS=3
	DO 30 I=0,3
		X=A+H*I
		W=3
		IF (I .EQ. 0 .OR.I .EQ.3) W=1
		SS=SS+W*FUNC(X)
30	CONTINUE
	SS=SS*H*3/8
	IF (N .EQ. 3) RETURN
35	DO 40 I=0, N-LS
		X=A+H*(I+LS)
		W=2
		IF (INT(I/2)*2+1 .EQ. I) W=4
		IF (I .EQ. 0 .OR. I .EQ. N-LS) W=1
		S=S+W*FUNC(X)
40	CONTINUE
	SS=SS+S*H/3
	RETURN
	END
!+++++++++++++++++++++++++++++++++
	FUNCTION FUNC(X)
	FUNC = (1+(X/2)**2)**2*3.14159
	RETURN
	END	
\end{lstlisting}
\end{document}