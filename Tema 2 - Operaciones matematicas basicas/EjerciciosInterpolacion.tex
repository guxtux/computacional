\documentclass[11pt]{article}
\usepackage[utf8]{inputenc}
%\usepackage[latin1]{inputenc}
\usepackage[spanish]{babel}
\decimalpoint
\usepackage{anysize}
\usepackage{graphicx} 
\usepackage{amsmath}
\usepackage{booktabs}
\usepackage{tabulary}
\usepackage{nccmath}
\usepackage{multicol}
\usepackage{multirow}
\usepackage{graphicx}
\usepackage{tikz}
\usepackage{color}
\usepackage{float}
\usepackage{tikz}
\usepackage{color}
\marginsize{1cm}{2cm}{2cm}{2cm}  
\title{Ejercicios Interpolaci\'{o}n. \\ Curso de Física Computacional}
\author{M. en C. Gustavo Contreras Mayén}
\date{ }
\begin{document}
\maketitle
\fontsize{14}{14}\selectfont
\begin{enumerate}
\item Usa el m\'{e}todo de interpolaci\'{o}n de Neville para calcular $y$ en $x=\pi/4$ del conjunto de datos
\begin{table}[htbp]
\centering 
\begin{tabulary}{15cm}{c | c | c | c | c | c }
$x$ & $0$ & $0.5$ & $1$ & $1.5$ & $2$ \\
\midrule
$y$ & $-1.00$ & $1.75$ & $4.00$ & $5.75$ & $7.00$ 
\end{tabulary}
\end{table}
\item Dados los puntos
\begin{table}[htbp]
\centering 
\begin{tabulary}{15cm}{c | c | c | c | c | c }
$x$ & $0$ & $0.5$ & $1$ & $1.5$ & $2$ \\
\midrule
$y$ & $-0.7854$ & $0.6529$ & $1.7390$ & $2.2071$ & $1.9425$ 
\end{tabulary}
\end{table}
calcular $y$ en $x=\pi/4$ y en $x=\pi/2$. Usa el m\'{e}todo que consideres m\'{a}s conveniente.
\item Los puntos 
\begin{table}[htbp]
\centering 
\begin{tabulary}{15cm}{c | c | c | c | c | c | c |}
$x$ & $-2$ & $1$ & $4$ & $-1$ & $3$ & $-4$ \\
\midrule
$y$ & $-1$ & $2$ & $59$ & $4$ & $24$ & $-53$ 
\end{tabulary}
\end{table}
se obtuvieron de un polinomio. Usando las diferencias divididas de la tabla del m\'{e}todo de Newton, determina el grado del polinomio.
\item Escribe un programa que use el m\'{e}todo de interpolaci\'{o}n de Neville para que realice el c\'{a}lculo de varios valores de $x$. Determina $y$ en $x=1.1,1.2,1.3$ de los siguientes datos:
\begin{table}[htbp]
	\centering 
	\begin{tabulary}{15cm}{c | c | c | c | c |}
		$x$ & $-2.0$ & $-0.1$ & $-1.5$ & $0.5$  \\
		\midrule
		$y$ & $2.2796$ & $1.0025$ & $1.6467$ & $1.0635$ \\
		\midrule
		$x$ & $-0.6$ & $2.2$ & $1.0$ & $1.8$  \\
		\midrule
		$y$ & $1.0920$ & $2.6291$ & $1.2661$ & $1.9896$
	\end{tabulary}
\end{table}
Respuesta: $y=1.3262,1.3938,1.4693$
\end{enumerate}
\end{document}