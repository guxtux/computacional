\documentclass[12pt]{article}
\usepackage[utf8]{inputenc}
\usepackage[spanish]{babel}
\usepackage{amsmath}
\usepackage{amsthm}
\usepackage{graphicx}
\usepackage{float}
\usepackage{anysize}
\usepackage{color}
\usepackage{listings}
\lstset{ %
language=C++,                % choose the language of the code
basicstyle=\small,       % the size of the fonts that are used for the code
numbers=left,                   % where to put the line-numbers
numberstyle=\footnotesize,      % the size of the fonts that are used for the line-numbers
stepnumber=1,                   % the step between two line-numbers. If it is 1 each line will be numbered
numbersep=5pt,                  % how far the line-numbers are from the code
backgroundcolor=\color{white},  % choose the background color. You must add \usepackage{color}
showspaces=false,               % show spaces adding particular underscores
showstringspaces=false,         % underline spaces within strings
showtabs=false,                 % show tabs within strings adding particular underscores
frame=single,   		% adds a frame around the code
tabsize=4,  		% sets default tabsize to 2 spaces
captionpos=b,   		% sets the caption-position to bottom
breaklines=true,    	% sets automatic line breaking
breakatwhitespace=false,    % sets if automatic breaks should only happen at whitespace
escapeinside={\%}{)}          % if you want to add a comment within your code
}
\marginsize{1cm}{1cm}{-1cm}{2cm}
\author{M. en C. Gustavo Contreras Mayén.}
\title{Curso de Física Computacional \\ Método de la falsa posición modificada}
\date{ }
\begin{document}
\maketitle
\fontsize{14}{14}\selectfont
\section{Descripción.}
Este programa encuentra la raíz de una función mediante el método de la falsa posición modificada. 
El usuario debe definir la ecuación a resolver en el subprograma \texttt{FUNC}. El flujo de cálculo es muy similar al programa de bisección, excepto que \texttt{YA} o \texttt{YC} se dividen a la mitad cuando los contadores de extremos fijos, \texttt{KI} o \texttt{KD}, respectivamente, se vuelven mayores que uno.
\section{Variables.}
\textbf{A, C}: valores de $x$ en los puntos extremos actuales. \\
\textbf{EP}: tolerancia.\\
\textbf{IL}: máximo número de pasos de iteración permitido. \\
\textbf{IT}: contador de pasos de iteración.\\
\textbf{YA, YC}: valores de la función en dos puntos extremos actuales.\\
\textbf{F}: valor de la función en $x$. \\
\textbf{KI}: contador del extremo fijo izquierdo. \\
\textbf{KD}: contador del extremos fijo derecho.
\section{Código.}
\begin{lstlisting}
PROGRAM FalsaModificada
	PRINT *
	PRINT *, 'Metodo de la Falsa posicion modificada'
	PRINT *, 'Ingresa la cota inferior A'
	READ *, A
	PRINT *, 'Ingresa la cota superior C'
	READ *, C
	PRINT *, 'Indica la tolerancia EP'
	READ *, EP
	PRINT *, 'Indica el numero de iteraciones IL'
	READ *, IL

	PRINT *
	YA=FUNC(A)
	YC=FUNC(C)
	PRINT 10
10	FORMAT ('IT. No.', 6X, 'A', 11X, 'B', 11X, 'C', 11X,  'YA',10X,'YB',10X,'YC')

	IT=0
	KI=1
	KD=1
30	IT=IT+1
	IF (IT .GT. IL) THEN
		PRINT *, 'Se ha excedido el limite de iteraciones'
		GOTO 110
	END IF

	GR=(YC-YA)/(C-A)
	BB=B
	B=-YA/GR+A
	YB=FUNC(B)
	PRINT 70, IT, A, B, C, YA, YB, YC
70	FORMAT (I3, 3X, 1P6E12.4)

	IF (abs(BB-B) .LT. EP) GOTO 100
	IF (YA*YB .LE. 0) THEN
		YC=YB
		C=B
		KI=KL + 1
		KD=0
		IF (KL .GT. 1) YA = YA/2
			GOTO 30
		ELSE
			YA=YB
			A=B
			KD=KD + 1
			KI=0
			IF (KD .GT. 1) YC = YC/2
			GOTO 30
		END IF

	PRINT *
100	PRINT *, 'Se ha satisfecho la tolerancia'
	PRINT *
	PRINT *,'----------------------------------------------------'
110	PRINT *,'	Raiz aproximada = ', B
	PRINT *,'----------------------------------------------------'
	PRINT *
	STOP
END PROGRAM FalsaModificada
!******************************************************************
FUNCTION FUNC(x)
	FUNC=tan(x)-x-0.5
	RETURN
END
\end{lstlisting}
\end{document}