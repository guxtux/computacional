\documentclass[12pt]{article}
\usepackage[utf8]{inputenc}
\usepackage[spanish]{babel}
\usepackage{amsmath}
\usepackage{amsthm}
\usepackage{graphicx}
\usepackage{float}
\usepackage{anysize}
\usepackage{color}
\usepackage{listings}
\lstset{ %
language=C++,                % choose the language of the code
basicstyle=\small,       % the size of the fonts that are used for the code
numbers=left,                   % where to put the line-numbers
numberstyle=\footnotesize,      % the size of the fonts that are used for the line-numbers
stepnumber=1,                   % the step between two line-numbers. If it is 1 each line will be numbered
numbersep=5pt,                  % how far the line-numbers are from the code
backgroundcolor=\color{white},  % choose the background color. You must add \usepackage{color}
showspaces=false,               % show spaces adding particular underscores
showstringspaces=false,         % underline spaces within strings
showtabs=false,                 % show tabs within strings adding particular underscores
frame=single,   		% adds a frame around the code
tabsize=4,  		% sets default tabsize to 2 spaces
captionpos=b,   		% sets the caption-position to bottom
breaklines=true,    	% sets automatic line breaking
breakatwhitespace=false,    % sets if automatic breaks should only happen at whitespace
escapeinside={\%}{)}          % if you want to add a comment within your code
}
\marginsize{1.5cm}{1.5cm}{-1cm}{1.5cm}
\author{M. en C. Gustavo Contreras Mayén.}
\title{Curso de Física Computacional \\ Método de la secante}
\date{ }
\begin{document}
\maketitle
\fontsize{14}{14}\selectfont
\section{Descripción.}
El siguiente código calcula la raíz de una función mediante el método de la secante, es necesario indicar dos valores iniciales de aproximación, para el problema del proyectil, son x0=30.0 y x1=30.1\\
Es importante considerar que los valores iniciales juegan un papel importante para la convergencia y velocidad de aproximación a la raíz exacta.
\section{Código}
\begin{lstlisting}
PROGRAM metsecante

	x0=30.0
	x1=30.1
	x2=x1-fx(x1)/secfx(x0,x1)

	DO WHILE (abs(fx(x2)) .gt. 1E-8)
		x0=x1
		x1=x2
		x2=x1-fx(x1)/secfx(x0,x1)
		WRITE *,'La raiz obtenida es= ', x2
	END DO

END PROGRAM metsecante
!***********************************
FUNCTION fx(x)
	fx=(2*9.81)/1000-1.4E-5*x**1.5-1.15E-5*x**2
	RETURN
END FUNCTION
!***********************************
FUNCTION secfx(x0,x1)
	secfx=(fx(x1)-fx(x0))/(x1-x0)
	RETURN
END FUNCTION

\end{lstlisting}
\end{document}