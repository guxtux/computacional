\documentclass[11pt]{article}
\usepackage[utf8]{inputenc}
%\usepackage[latin1]{inputenc}
\usepackage[spanish]{babel}
\decimalpoint
\usepackage{anysize}
\usepackage{graphicx} 
\usepackage{amsmath}
\usepackage{booktabs}
\usepackage{tabulary}
\usepackage{nccmath}
\usepackage{float}
\usepackage{tikz}
\usetikzlibrary{patterns}
\usetikzlibrary{decorations.markings}
\usepackage{pgfplots}
\usepackage{etex}
\usepackage{color}
\usepackage{listings}
\renewcommand{\arraystretch}{1.5}
\lstset{ %
language=Python,                % choose the language of the code
basicstyle=\normalsize,       % the size of the fonts that are used for the code
numbers=left,                   % where to put the line-numbers
numberstyle=\footnotesize,      % the size of the fonts that are used for the line-numbers
stepnumber=1,                   % the step between two line-numbers. If it is 1 each line will be numbered
numbersep=5pt,                  % how far the line-numbers are from the code
backgroundcolor=\color{white},  % choose the background color. You must add \usepackage{color}
showspaces=false,               % show spaces adding particular underscores
showstringspaces=false,         % underline spaces within strings
showtabs=false,                 % show tabs within strings adding particular underscores
frame=single,   		% adds a frame around the code
tabsize=4,  		% sets default tabsize to 2 spaces
captionpos=b,   		% sets the caption-position to bottom
breaklines=true,    	% sets automatic line breaking
breakatwhitespace=false,    % sets if automatic breaks should only happen at whitespace
escapeinside={\#}{)}          % if you want to add a comment within your code
}
\marginsize{1.5cm}{1.5cm}{1cm}{2cm}  
\title{Examen Tema 2 - Operaciones matem\'{a}ticas b\'{a}sicas. \\ Curso de Física Computacional}
\author{M. en C. Gustavo Contreras May\'{e}n}
\date{ }
\begin{document}
\maketitle
\fontsize{14}{14}\selectfont
Resuelve los siguientes problemas mediante un c\'{o}digo con Python.
\begin{enumerate}
\item La trayectoria de un sat\'{e}lite que orbita la Tierra es
\[ R = \dfrac{C}{1 + e \sin(\theta + \alpha)}\]
donde $(R,\theta)$ son las coordenadas polares del sat\'{e}lite, $C$, e, y $\alpha$ son constantes (e se conoce como la excentricidad de la \'{o}rbita).
\begin{center}
\begin{tikzpicture}
\draw (0,0) circle (1cm);
\draw (-0.2,0) node {O};\draw [dashed](0,0) -- node [midway, above] {R} (2,1.5);
\draw (2,1.5) circle (0.05cm);
\draw (0,0) -- (0.8,0);
\draw (0.7,0.2) node {$\theta$};
\end{tikzpicture}
\end{center}
Si el sat\'{e}lite fue observado en las siguientes tres posiciones:
\\
\begin{center}
\begin{tabular}{c | c | c | c}
$\theta$ & $-30^{\circ}$ & $0^{\circ}$ & $30^{\circ}$ \\ \hline
R (km) & 6870 & 6728 & 6615
\end{tabular}
\end{center}
Determinar el valor m\'{a}s pequeño de $R$ en la trayectoria y el respectivo valor de $\theta$.
\item La ecuaci\'{o}n de equilibrio qu\'{i}mico en la producci\'{o}n de metanol a partir de CO y $H_{2}$, es
\[ \dfrac{\xi (3-2 \xi)^{2}}{(1-\xi)^{3}} = 249.2\]
donde $\xi$ es el grado de equilibrio de la reacción. Determinar $\xi$.
\item Obt\'{e}n la aproximaci\'{o}n por diferencias centrales de $f''(x)$ de orden $O(h^{4})$ aplicando la extrapolaci\'{o}n de Richardson a la aproximaci\'{o}n por diferencias centrales de orden $O(h^{2})$.
\item Obt\'{e}n la primera aproximaci\'{o}n por diferencias centrales para $f^{4}(x)$ a partir de la serie de Taylor.
\item Evaluar
\[ \int_{-1}^{1} \cos(2 \cos^{-1} x) dx\]
con la regla de $1/3$ de Simpson, usando 2, 4 y 6 bloques. Explicar los resultados.
\item Determina el valor de 
\[ \int_{1}^{\infty} (1+x^{4})^{-1} dx\]
con la regla del trapecio, usando cinco bloques y compara el resultado con la integral exacta $0.24375$. Tip: usa la transformaci\'{o}n $x^{3}= 1/t$.
\item El per\'{i}odo de un p\'{e}ndulo simple de longitud $L$ es $\tau = 4 \sqrt{L/g} h(\theta_{0})$, donde $g$ es la aceleraci\'{o}n debida a la gravedad, $\theta_{0}$ representa la amplitud angular y
\[ h(\theta_{0}) = \int_{0}^{\pi/2} \dfrac{d\theta}{\sqrt{1 - \sin^{2}(\theta_{0}/2)\sin^{2} \theta}} \]
Calcular $h(15^{\circ})$,$h(30^{\circ})$ y $h(45^{\circ})$, compara esos valores con $h(0)=\pi/2$ (que es el valor aproximado para pequeñas amplitudes).
\item La f\'{o}rmula de Debye para la capacidad calor\'{i}fica $C_{v}$ de un s\'{o}lido, es \\ $C_{v} = 9 Nkg(u)$, donde
\[g(u) = u^{3} \int_{0}^{1/u} \dfrac{x^{4}e^{x}}{(e^{x}-1)}dx\]
los t\'{e}rminos de la ecuaci\'{o}n son:
\begin{eqnarray*}
N &=& \text{N\'{u}mero de part\'{i}culas en el s\'{o}lido} \\
k &=& \text{Constante de Boltzmann} \\
u &=& \frac{T}{\Theta_{D}} \\
T &=& \text{temperatura absoluta} \\
\Theta_{D} &=& \text{Temperatura de Debye}
\end{eqnarray*}
Calcular $g(u)$ para $u=0$ a $1.0$ en intervalos de $0.05$, grafica los resultados.
\end{enumerate}
\end{document}