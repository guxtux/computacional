\documentclass[11pt]{article}
\usepackage[utf8]{inputenc}
%\usepackage[latin1]{inputenc}
\usepackage[spanish]{babel}
\decimalpoint
\usepackage{anysize}
\usepackage{graphicx} 
\usepackage{amsmath}
\usepackage{booktabs}
\usepackage{tabulary}
\usepackage{nccmath}
\usepackage{float}
\usepackage{tikz}
\usetikzlibrary{patterns}
\usetikzlibrary{decorations.markings}
\usepackage{pgfplots}
\usepackage{etex}
\usepackage{color}
\usepackage{listings}
\renewcommand{\arraystretch}{1.5}
\lstset{ %
language=Python,                % choose the language of the code
basicstyle=\normalsize,       % the size of the fonts that are used for the code
numbers=left,                   % where to put the line-numbers
numberstyle=\footnotesize,      % the size of the fonts that are used for the line-numbers
stepnumber=1,                   % the step between two line-numbers. If it is 1 each line will be numbered
numbersep=5pt,                  % how far the line-numbers are from the code
backgroundcolor=\color{white},  % choose the background color. You must add \usepackage{color}
showspaces=false,               % show spaces adding particular underscores
showstringspaces=false,         % underline spaces within strings
showtabs=false,                 % show tabs within strings adding particular underscores
frame=single,   		% adds a frame around the code
tabsize=4,  		% sets default tabsize to 2 spaces
captionpos=b,   		% sets the caption-position to bottom
breaklines=true,    	% sets automatic line breaking
breakatwhitespace=false,    % sets if automatic breaks should only happen at whitespace
escapeinside={\#}{)}          % if you want to add a comment within your code
}
\marginsize{1.5cm}{1.5cm}{1cm}{1cm}  
\title{Examen Tema 2 - Operaciones matem\'{a}ticas b\'{a}sicas - 2a. Parte \\ Curso de Física Computacional}
\author{M. en C. Gustavo Contreras May\'{e}n}
\date{ }
\begin{document}
\maketitle
\fontsize{14}{14}\selectfont
\begin{enumerate}
	\item (\textbf{2 puntos.}) La funci\'{o}n gamma $\Gamma (x)$, se define como la siguiente integral
	\[ \Gamma (x) = \int_{0}^{\infty} t^{x-1} e^{-t} dt\]
	que converge para todo $x$ positivo, pese a que para $0<x<1$ el integrando tiene una divergencia en $t=0$.
	\\
	Calcula num\'{e}ricamente a partir de la definci\'{o}n anterior, la $\Gamma$ para $x=10$ y $x=1/2$, valores para los cuales se conocen los resultados anal\'{i}ticos:
	\begin{eqnarray*}
		\Gamma(10) &=& 9! = 362880 \\
		\Gamma(1/2) &=& \sqrt{\pi}
	\end{eqnarray*}
	En cada caso debes:
	\begin{enumerate}
		\item indicar el cambio de variable utilizado.
		\item el n\'{u}mero de puntos utilizados en la discretizaci\'{o}n.
		\item el m\'{e}todo de integraci\'{o}n.
		\item el resultado obtenido.
		\item el error cometido respecto al valor anal\'{i}tico.
	\end{enumerate}
	\item (\textbf{2 puntos.}) Eval\'{u}a num\'{e}ricamente las siguientes integrales:
	\begin{eqnarray*}
		I_{1} &=& \int_{0}^{\infty} e^{-x} ln x dx \\
		I_{2} &=& \int_{0}^{1} \dfrac{1+x}{1-x^{3}} ln\dfrac{1}{x} dx
	\end{eqnarray*}
	El problema consiste en resolver las integrales con alg\'{u}n cambio de variable para tener un integrando suave en un intervalo finito.
	\\
	Se debe de obtener un resultado razonablemente bueno, teniendo que evaluar el integrando final con el menor n\'{u}mero ($N$) de veces que sea posible. Como criterio de convergencia debes de usar alguna cantidad como
	\[ \epsilon = \dfrac{I - I_{N}}{I} < 10^{-n}\]
	con n = $2,3,4,5,6$
	En cada caso debes:
	\begin{enumerate}
		\item indicar el(los) cambio(s) de variable utilizado(s).
		\item el n\'{u}mero de puntos utilizados en la discretizaci\'{o}n.
		\item el m\'{e}todo de integraci\'{o}n.
		\item el resultado obtenido.
		\item el error cometido respecto al valor de I.
	\end{enumerate}
	Nota: no se vale usar integraci\'{o}n por partes.
	\item (\textbf{6 puntos.}) Para muchos efectos la fuerza entre \'{a}tomos puede ser tratada exitosamente con el potencial central, llamado de Lennard-Jones
	\[ V = 4 V_{0} \left[ \left( \dfrac{a}{r} \right)^{12} - \left( \dfrac{a}{r} \right)^{6} \right] \]
	cuyo valor m\'{i}nimo es $V_{0}$ y se anula cuando $r$ coindice con el radio de Bohr. Una part\'{i}cula atrapada en este potencial (energ\'{i}a menor que cero), tiene un movimiento en el intervalo (rmin, rmax) donde ambos radios son
mayores que $a$. Cu\'{a}nticamente solo hay un conjunto discreto de energ\'{i}as $E_{n}$ posibles. Cl\'{a}sicamente $E= \frac{p^{2}}{2m}+ V(r)$ o equivalentemente, la magnitud del momento depende de $r$ en la forma $p(r) = \sqrt{2m(E-V(r))}$.
	\\
	Una forma aproximada de plantear el problema para encontrar los valores de los niveles cu\'{a}nticos $E_{n}$ consiste en exiger la \textit{condici\'{o}n de Bohr-Sommerfeld}
	\[ \oint \dfrac{p(r)}{\hbar} dr = 2 \pi \left( n + \dfrac{1}{2} \right) \]
	con $n$ entero no negativo. La integral es sobre un ciclo completo de oscilaci\'{o}n. El problema se adimensionaliza haciendo las sustituciones
	\[ E = V_{0} \mathcal{E}, \hspace{0.5cm} r = a \rho, \hspace{0.5cm} V_{0} = \dfrac{\gamma^{2} \hbar^{2}}{2 a^{2} m}\]
	Para la mol\'{e}cula de hidr\'{o}geno el valor de $\gamma=21.7$, para el O$_{2}$, $\gamma \sim 150$.
	\\
	La condici\'{o}n integral de arriba se convierte en la exigencia que se anule la funci\'{o}n
	\[ F_{n}(\mathcal{E}_{n}) = \gamma  \int_{\rho_{min}}^{\rho^{max}} \sqrt{\mathcal{E}_{n} - 4 \left( \dfrac{1}{\rho^{12}} - \dfrac{1}{\rho^{6}} \right)} d\rho - \pi \left( n + \dfrac{1}{2} \right) \]
	Es decir, el problmea consiste en encontrar los ceros de $F_{n}$ dados $\gamma=150$ y $n=0,1,2$, con $-1<\mathcal{E}_{n} < 0$ sabiendo que
	\[ \rho_{min} = \left( \dfrac{2-2\sqrt{\delta_{n}}}{1-\delta_{n}} \right)^{1/6}, \hspace{1cm}  \rho_{max} = \left( \dfrac{2+2\sqrt{\delta_{n}}}{1-\delta_{n}} \right)^{1/6}\]
	donde $\delta_{n} = 1 + \mathcal{E}_{n}$.
	El programa que realices para resolver este problema, deber de ser \'{u}til para otros potenciales $V(r)$, como por ejemplo el potencial de Yukawa
	\[ V(r) = \dfrac{\kappa}{r} e^{-r/a}\]
	En la b\'{u}squeda de los ceros debes usar el m\'{e}todo de la secante (indicando, entre otras cosas, la tolerancia usada y cu\'{a}ntas iteraciones fueron necesarias)
\end{enumerate}
\end{document}