\documentclass[11pt]{article}
\usepackage[utf8]{inputenc}
%\usepackage[latin1]{inputenc}
\usepackage[spanish]{babel}
\usepackage{anysize}
\usepackage{graphicx} 
\usepackage{amsmath}
\marginsize{1cm}{2cm}{0cm}{2cm}  
\title{Ejercicios Interpolación de Lagrange \\ \begin{large}Curso de Física Computacional\end{large}}
\author{M. en C. Gustavo Contreras Mayén}
\date{ }
\begin{document}
\maketitle
\begin{enumerate}
\item Ajusta $x \sin(x)$ en $[0, \dfrac{\pi}{2}]$ con un polinomio de interpolación de Lagrange de orden 4, utilizando puntos con igual separación. Calcula el error de cada interpolación en cada incremento de $\dfrac{\pi}{16}$, muestra una gráfica.
\item Ajusta $\sin(x)$ en $[0, 2\pi]$ con el polinomio de interpolación de Lagrange de orden 4 y 8, utilizando puntos con igual separación (5 y 9 puntos respectivamente). Grafica los polinomios de interpolación junto con $\sin(x)$ y las distribuciones de sus errores.
\end{enumerate}
\end{document}