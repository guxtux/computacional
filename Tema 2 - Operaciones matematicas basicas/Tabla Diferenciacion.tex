\documentclass[12pt]{article}
\usepackage[latin1]{inputenc}
\usepackage[spanish]{babel}
\usepackage{amsmath}
\usepackage{amsthm}
\usepackage{anysize}
\marginsize{2cm}{2cm}{0cm}{2cm}
\author{M. en C. Gustavo Contreras May�n.}
\title{Tabla para el ejercicio de diferenciaci�n \\ \begin{large}Curso F�sica Computacional\end{large}}
\date{ }
\begin{document}
\maketitle
Calcula la primera derivada de $\tan(x)$ en $x=1$ mediante las cinco aproximaciones por diferencias, utilizando $h=0.1,0.05,0.02$.\\
Estima el error relativo de cada aproximaci�n, usando el valor exacto.
\begin{center}
\begin{tabular}{l c c c c}
Algoritmo & & $h=0.1$ & $h=0.05$ & $h=0.02$ \\
\hline
dif. hacia atr�s & valor & & & \\
 & error rel. & & & \\
 \hline
dif. hacia adelante &valor & & & \\
 & error rel. & & & \\
 \hline
dif. centrales &valor & & & \\
 & error rel. & & & \\
 \hline
dif. hacia atr�s 3 puntos & valor& & & \\
 & error rel. & & & \\
 \hline
dif. hacia adelante 3 puntos & valor& & & \\
 & error rel. & & & \\
 \hline
\end{tabular}
\end{center}
\end{document}