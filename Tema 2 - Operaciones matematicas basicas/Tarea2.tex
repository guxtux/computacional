\documentclass[12pt]{article}
\usepackage[latin1]{inputenc}
\usepackage[spanish]{babel}
\usepackage{amsmath}
\usepackage{amsthm}
\usepackage{anysize}
\marginsize{2cm}{2cm}{0cm}{2cm}
\author{M. en C. Gustavo Contreras May�n.}
\title{Tarea 2 \\ \begin{large}Curso F�sica Computacional\end{large}}
\date{ }
\begin{document}
\maketitle
\textbf{Fecha de entrega: Martes 23 de marzo de 2010.} \\
\begin{enumerate}
\item Si se ajusta un polinomio de interpolaci�n de Lagrange a cuatro datos en $x=1,2,3,4$, aparecen los siguientes polinomios c�bicos en la f�rmula de interpolaci�n:
\begin{enumerate}
\item $\dfrac{(x-2)(x-3)(x-4)}{(1-2)(1-3)(1-4)}$
\item $\dfrac{(x-1)(x-3)(x-4)}{(2-1)(2-3)(2-4)}$
\item $\dfrac{(x-1)(x-2)(x-4)}{(3-1)(3-2)(3-4)}$
\item $\dfrac{(x-1)(x-2)(x-3)}{(4-1)(4-2)(4-3)}$
\end{enumerate}
Grafica las cuatro funciones anteriores y analiza las implicaciones de cada una.
\item El polinomio de interpolaci�n de Newton hacia atr�s ajustado a los puntos $x_{0}$, $x_{1}$ y $x_{2}$ se escribe como
\[ g(x) = f_{2} + s \nabla f_{2} + \frac{1}{2} s(s+1) \nabla^{2} f_{2}, \hspace{1cm} -2 \leq s \leq 0 \]
donde $s = \frac{(x-x_{2})}{h}$\\
Por otro lado, el polinomio de interpolaci�n de Newton hacia adelante ajustado a los mismos datos es
\[ g(x) = f_{0} + s \Delta f_{0} + \frac{1}{2} s(s-1) \Delta^{2} f_{0}, \hspace{1cm} 0 \leq s \leq 2 \]
donde $s = \frac{(x-x_{0})}{h}$\\
Verifica la equivalencia de las ecuaciones.
\item La funci�n de transferencia para un sistema est� dada por
\[ F(s) = \dfrac{H(s)}{1+G(s)H(s)} \]
donde
\[ G(s)= \dfrac{1}{s} exp(-0.1s), \hspace{0.5cm} H(s)= K \]
Busca las ra�ces de la ecuaci�n caracter�stica $1+G(s)H(s)$ para $K=1,2,3$ mediante el m�todo gr�fico y eval�alas posteriomente, mediante el m�todo de la falsa posici�n modificada.
\item La longitud de una curva definida por $x=\theta(t)$ y $\psi(t)$, $a<t<b$, est� dada por
\[ s=\int_{a}^{b}\left( [\theta'(t)]^{2}+[\psi(t)]^{2}\right)^{1/2} dt \]
Usando las cuadraturas de Gauss con $N=2,4,6$ para encontrar la longitud de la cicloide definida por
\[ x=3[t-\sin(t)], \hspace{1cm}  y=2-2\cos(t), \hspace{1cm} 0<t<2\pi\]
\item Considera una varilla uniforme de $1$ metro de longitud apoyada en dos extremos; el momento de doblamiento est� dado por la siguiente f�rmula:
\[ y''= \dfrac{M(x)}{EI} \]
donde $y(x)$ es la deflexi�n, $M(x)$ es el momento de doblamiento y $EI$ es la rigidez en la uni�n. Calcula el momento de doblamiento en cada punto de la ret�cula -incluyendo los extremos- suponiendo que la distribuci�n de la deflexi�n tiene los siguientes valores:
\begin{center}
\begin{tabular}{c c c}
i & $x_{i}$ & $f(x_{i})$ \\
\hline
0 & 0.0(m) & 0.0(cm) \\
1 & 0.2 & 7.78 \\
2 & 0.4 & 10.68 \\
3 & 0.6 & 8.37 \\
4 & 0.8 & 3.97 \\
5 & 1.0 & 0.0
\end{tabular}
\end{center}
Supongamos que $EI=1.2$ $Nm^{2}$. Utiliza la aproximaci�n por diferencias centrales para los puntos de la ret�cula distintos de los extremos. Para �stos, utiliza la aproximaci�n por diferencias hacia adelante o hacia atr�s utilizando cuatro puntos.
\end{enumerate}
\end{document}