\documentclass[12pt]{beamer}
\newenvironment{ConCodigo}[1]
  {\begin{frame}[fragile,environment=ConCodigo]{#1}}
  {\end{frame}}
\graphicspath{{Imagenes/}{../Imagenes/}}
\usepackage[utf8]{inputenc}
\usepackage[spanish]{babel}
\usepackage{hyperref}
\usepackage{etex}
%\reserveinserts{28}
\usepackage{amsmath}
\usepackage{amsthm}
\usepackage{mathtools}
\usepackage{multicol}
\usepackage{multirow}
\usepackage{tabulary}
\usepackage{booktabs}
\usepackage{nccmath}
\usepackage{physics}
\usepackage{biblatex}
\usepackage[outdir=./]{epstopdf}
%\epstopdfsetup{outdir=./}
\usepackage{graphicx}
%\usepackage{enumitem,xcolor}
\usepackage{siunitx}
%\sisetup{scientific-notation=true}
%\usepackage{fontspec}
\usepackage{lmodern}
\usepackage{float}
\usepackage[format=hang, font=footnotesize, labelformat=parens]{caption}
\usepackage[autostyle,spanish=mexican]{csquotes}
\usepackage{standalone}
\usepackage{blkarray}
\usepackage{algorithm}
\usepackage{algorithmic}
\usepackage{tikz}
\usepackage[siunitx, RPvoltages]{circuitikz}
\usetikzlibrary{arrows,patterns,shapes}
\usetikzlibrary{decorations.markings}
\usetikzlibrary{arrows}
\usepackage{color}
\usepackage{xcolor}
%\usepackage{beton}
%\usepackage{euler}
%\usepackage[T1]{fontenc}
\usepackage[sfdefault]{roboto}  %% Option 'sfdefault' only if the base font of the document is to be sans serif
\usepackage[T1]{fontenc}
\renewcommand*\familydefault{\sfdefault}
\DeclareGraphicsExtensions{.pdf,.png,.jpg}
\usepackage{hyperref}
\renewcommand {\arraystretch}{1.5}
\newcommand{\python}{\texttt{python}}
\usefonttheme[onlymath]{serif}
\setbeamertemplate{navigation symbols}{}
\usetikzlibrary{patterns}
\usetikzlibrary{decorations.markings}
\tikzstyle{every picture}+=[remember picture,baseline]
%\tikzstyle{every node}+=[inner sep=0pt,anchor=base,
%minimum width=2.2cm,align=center,text depth=.15ex,outer sep=1.5pt]
%\tikzstyle{every path}+=[thick, rounded corners]
\setbeamertemplate{caption}[numbered]
\newcommand{\ptm}{\fontfamily{ptm}\selectfont}
%Se usa la plantilla Warsaw modificada con spruce
\mode<presentation>
{
  \usetheme{Warsaw}
  \setbeamertemplate{headline}{}
  \useoutertheme{default}
  \usecolortheme{albatross}
  \setbeamercovered{invisible}
}
% \AtBeginSection[]
% {
% \begin{frame}<beamer>{Contenido}
% \normalfont\mdseries
% \tableofcontents[currentsection]
% \end{frame}
% }

\include{pre_codigo}
\begin{document}
\title{Tema 2 - Operaciones matemáticas básicas}
\subtitle{Diferenciación numérica}
\author[]{M. en C. Gustavo Contreras Mayén}
\maketitle
\fontsize{14}{14}\selectfont
\spanishdecimal{.}
\begin{frame}{Contenido}
\tableofcontents[pausesections]
\end{frame}
\section{Problema inicial}
\begin{frame}
\frametitle{Problema inicial}
Dada una función $f(x)$ y un valor $x$, queremos calcular
\[\dfrac{d^{n}f}{dx^{n}} \]
\end{frame}
\begin{frame}
Usamos la diferenciación numérica para el siguente problema: se nos da una función $y=f(x)$ y deseamos obtener una de sus derivadas en el punto $x=x_{k}$.
\\
\bigskip
Cuando decimos ''dada" significa que, o bien tenemos un algoritmo para calcular la función, o contamos con un conjunto de puntos discretos $(x_{i}, y_{i})$, $i = 0, 1,\ldots,N$. En cualquier caso, tenemos acceso a un número finito de pares de datos $(x, y)$ a partir del cual, podemos calcular la derivada.
\end{frame}
\begin{frame}
Si estás pensando en que la diferenciación numérica se relaciona con interpolación, tienes toda la razón. Ya que es un medio para encontrar la derivada a partir de una aproximación con un polinomio, y luego diferenciar.
\end{frame}
\begin{frame}
Una herramienta igualmente eficaz es el desarrollo en serie de Taylor de $f(x)$ sobre el punto de $x_{k}$, lo que representa como ventaja de que nos proporciona información acerca del error cometido en la aproximación.
\\
\bigskip
La diferenciación numérica no es un proceso particularmente exacto: se presentan un conflicto entre los errores de redondeo (debido a la precisión de la máquina) y los errores inherentes a la interpolación. Por esta razón, la derivada de una función no debe de ser calculada con la misma precisión que la propia función.
\end{frame}
\section{Aproximación por diferencias finitas}
\begin{frame}
\frametitle{Aproximación por diferencias finitas}
La derivación por aproximación por diferencias finitas de las derivadas de $f(x)$, se basa en el desarrollo de series de Taylor hacia adelante y hacia atrás de $f(x)$ alrededor de $x$:
\fontsize{11}{11}\selectfont
\begin{eqnarray*}
f(x+h) &=& f(x) + hf'(x) + \dfrac{h^{2}}{2!}f''(x) + \dfrac{h^{3}}{3!}f'''(x) + \dfrac{h^{4}}{4!}f^{4}(x) + \ldots \\
\visible<2->{f(x-h) &=& f(x) - hf'(x) + \dfrac{h^{2}}{2!}f''(x) - \dfrac{h^{3}}{3!}f'''(x) + \dfrac{h^{4}}{4!}f^{4}(x) - \ldots} \\
\visible<3->{f(x+2h) &=& f(x) + 2hf'(x) + \dfrac{(2h)^{2}}{2!}f''(x) + \dfrac{(2h)^{3}}{3!}f'''(x) + \dfrac{(2h)^{4}}{4!}f^{4}(x) + \ldots} \\
\visible<4->{f(x-2h) &=& f(x) - 2hf'(x) + \dfrac{(2h)^{2}}{2!}f''(x) - \dfrac{(2h)^{3}}{3!}f'''(x) + \dfrac{(2h)^{4}}{4!}f^{4}(x) - \ldots}
\end{eqnarray*}
\end{frame}
\begin{frame}
\frametitle{Suma de las diferencias}
\fontsize{12}{12}\selectfont
\begin{eqnarray*}
f(x+h)+f(x-h) &=& 2f(x) + h^{2}f''(x) + \dfrac{h^{4}}{12}f^{4}(x) + \ldots \\
\visible<2->{f(x+h) - f(x-h) &=& 2h f'(x) + \dfrac{h^{3}}{3} f'''(x) + \ldots} \\
\visible<3->{f(x+2h) + f(x-2h) &=& 2f(x)+4h^{2}f''(x) + \dfrac{4h^{4}}{3}f^{4}(x) +\ldots} \\
\visible<4->{f(x+2h)-f(x-2h) &=& 4hf'(x) + \dfrac{8h^{3}}{3} f'''(x) + \ldots }
\end{eqnarray*}
\end{frame}
\subsection{Aproximación por diferencias centrales}
\begin{frame}
\frametitle{Aproximación por diferencias centrales}
Tomando la segunda suma de la lista anterior, de donde despejamos $f'(x)$, tenemos
\[ f'(x) = \dfrac{f(x+h) - f(x-h)}{2h} - \dfrac{h^{2}}{6} f'''(x) - \ldots \]
equivalentemente
\[ f'(x) = \dfrac{f(x+h)-f(x-h)}{2h} + O(h^{2})\]
que es llamada \textcolor{blue}{aproximación por diferencias centrales} de $f'(x)$.
\end{frame}
\begin{frame}
De manera similar, obtenemos la aproximación por diferencias centrales de $f''(x)$:
\[  f''(x) =  \dfrac{f(x+h)- 2f(x) + f(x-h)}{h^{2}} + \dfrac{h^{2}}{12}f^{4}(x) + \ldots\]
es decir:
\[ f''(x) =  \dfrac{f(x+h)- 2f(x) + f(x-h)}{h^{2}} + O(h^{2})\]
\end{frame}
\section{Aproximación por diferencias hacia adelante/atrás}
\begin{frame}
\frametitle{Aproximación por diferencias hacia adelante/atrás}
Las aproximaciones por diferencias centrales  no siempre son útiles. Por ejemplo, considera la situación en que se da la función en los  $n$ puntos discretos $x_{0}, x_{1}, \ldots,x_{n}$.  Dado que las diferencias centrales utilizan valores de la función en cada lado de $x$, no sería posible calcular las derivadas en $x_{0}$ y $x_{n}$.
\\
\bigskip
Es cierto, que necesitamos una expresión para diferencias finitas que evalúe la función en un solo lado de $x$. Estas expresiones son llamadas aproximaciones por diferencias finitas \textcolor{red}{hacia adelante} y \textcolor{blue}{hacia atrás}.
\end{frame}
\begin{frame}
Despejamos $f'(x)$ de la primera lista de expresiones por diferencias, para obtener:
\[ f'(x) = \dfrac{f(x+h)-f(x)}{h} - \dfrac{h}{2} f''(x) - \dfrac{h^{2}}{6} f'''(x) - \ldots \]
Manteniendo los primeros términos, tenemos la aproximación por diferencias centrales hacia adelante:
\[ f'(x) = \dfrac{f(x+h)-f(x)}{h} + O(h) \]
\end{frame}
\begin{frame}
De manera análoga, obtenemos la aproximación por diferencias hacia atrás:
\[ f'(x) = \dfrac{f(x)-f(x-h)}{h} + O(h) \]
Hay que hacer notar que el error es $O(h)$, que no es tan bueno como $O(h^{2})$ en la aproximación por diferencias centrales.
\end{frame}
\section{Segunda aproximación por diferencias centrales}
\begin{frame}
\frametitle{Segunda aproximación por diferencias centrales}
Las aproximaciones por diferencias centrales del tipo $O(h)$ no son tan populares debido a que es más común usar expresiones del tipo $O(h^{2})$.
\\
\bigskip
Para obtener una fórmula con diferencias centrales con ese orden de error, tenemos que incluir más términos de la serie de Taylor.
\end{frame}
\begin{frame}
\fontsize{11}{11}\selectfont
Consideremos
\begin{eqnarray*}
f(x+h) &=& f(x) + hf'(x) + \dfrac{h^{2}}{2!}f''(x) + \dfrac{h^{3}}{3!}f'''(x) + \dfrac{h^{4}}{4!}f^{4}(x) + \ldots \\
f(x+2h) &=& f(x) + 2hf'(x) + \dfrac{(2h)^{2}}{2!}f''(x) + \dfrac{(2h)^{3}}{3!}f'''(x) + \dfrac{(2h)^{4}}{4!}f^{4}(x) + \ldots
\end{eqnarray*}
Eliminamos $f''(x)$ multiplicando la primera ecuación y restándola de la segunda, para obtener:
\[ f(x+2h) - 4f(x+h) = -3 f(x) - 2 hf'(x) + \dfrac{2h^{3}}{3} f'''(x) + \ldots\]
\end{frame}
\begin{frame}
Por lo tanto
\[ f'(x) = \dfrac{-f(x+2h) + 4 f(x+h) - 3 f(x)}{2h} + \dfrac{h^{2}}{3}f'''(x) + \ldots\]
que es
\[ f'(x) = \dfrac{-f(x+2h) + 4 f(x+h) - 3 f(x)}{2h} + O(h^{2})\]
siendo la expresión para la segunda aproximación por diferencias finitas hacia adelante.
\end{frame}
\section{Errores en las aproximaciones por diferencias finitas}
\begin{frame}
\frametitle{Errores en las aproximaciones por diferencias finitas}
El efecto del error por redondeo puede ser profundo, si $h$ es muy pequeña, valores de $f(x)$, $f(x \pm h)$, $f(x \pm 2h)$, etc. serán aproximadamente iguales.
\\
\bigskip
Cuando se multiplican por los coeficientes y se suman, podemos perder un número grande de términos. Por otro lado, no debemos de hacer muy grande el valor de $h$, ya que el error debido al truncamiento, sería excesivo.
\end{frame}
\begin{frame}
Para manejar esta situación que siempre se va a presentar, podemos apoyarnos con las siguientes opciones:
\begin{enumerate}
\item Usar doble precisión.
\item Usar diferencias finitas en donde se alcance el orden de $O(h^{2})$
\end{enumerate}
\end{frame}
\begin{frame}
\frametitle{Veamos un ejemplo:}
Calcular la segunda derivada de $f(x)=exp(-x)$ en $x=1$ a partir de la fórmula de diferencias centrales; calcula el error relativo, el valor exacto es $f''(1) = exp(-1) = 0.36787944$
\end{frame}
\begin{frame}
\fontsize{12}{12}\selectfont
\begin{center}
\begin{tabular}{c | l | l}
\hline
h & Segunda derivada & Error \\ \hline
$0.64$ & $0.380609096726$ & $0.0127296567262$ \\ \hline
$0.32$ & $0.371029413951$ & $0.00314997395074$ \\ \hline
$0.16$ & $0.368664920656$ & $0.000785480656265$ \\ \hline
$0.08$ & $0.368075685401$ & $0.000196245401341$ \\ \hline
$0.04$ & $0.36792849438$ & $4.90543797212e-05$ \\ \hline
$0.02$ & $0.367891703983$ & $1.22639830198e-05$ \\ \hline
$0.01$ & $0.367882506844$ & $3.0668437197e-06$ \\ \hline
$0.005$ & $0.367880207588$ & $7.67587933936e-07$ \\ \hline
$0.0025$ & $0.367879632774$ & $1.92774403829e-07$ \\ \hline
$0.00125$ & $0.36787948904$ & $4.90404901687e-08$ \\ \hline
\end{tabular}
\end{center}
\end{frame}
\begin{frame}
\frametitle{Ejercicio}
El siguiente arreglo tiene por dimensiones $a=100$ mm, $b=120$ mm, $c=150$ mm y $d=180$ mm
\begin{center}
\begin{tikzpicture}[font=\small]
\draw (0,0) node [above left] {A} circle (0.05);
\draw (1.8,2) node [above left] {B} circle (0.05);
\draw (3.5,2.5)node [above right] {C} circle (0.05);
\draw (6,0) node [above right] {D} circle (0.05);
\draw [dashed] (0,0) -- node [midway, below] {d} (6,0);
\draw [thick] (0,0) -- node [midway, above] {a} +(1.8,2) -- node [midway, above] {b}(3.5,2.5) -- node [midway, above] {c} (6,0);
\draw [pattern= north east lines](-0.2,-0.2) rectangle (0.2,0);
\draw [pattern= north east lines](5.8,-0.2) rectangle (6.2,0);
\draw [->](0.5,0) arc (0:27:1cm);
\draw (0.7,0.3) node {$\alpha$};
\draw (2,2) -- (3.1,2);
\draw [->](2.8,2) arc (0:17:1cm);
\draw (3.3,2.1) node {$\beta$};
\end{tikzpicture}
\end{center}
\end{frame}
\begin{frame}
Se puede demostrar a partir de la geometría del problema que la relación entre los ángulos $\alpha$ y $\beta$ es:
\[ (d - a \cos \alpha - b \cos \beta)^{2} + (a \sin \alpha + b \sin \beta)^{2} - c^{2} = 0\]
Para un valor dado de $\alpha$, podemos resolver la ecuación para $\beta$, mediante alguno de los métodos para encontrar raíces que ya hemos visto. Haciendo para $\alpha=0^{\circ}, 5^{\circ}, 10^{\circ},\ldots, 30^{\circ}$, los resultados son:
\begin{center}
\fontsize{10}{10}\selectfont
\begin{tabular}{c | c | c | c | c | c | c | c}
$\alpha$ (grados) & $0$ & $5$ & $10$ & $15$ & $20$ & $25$ & $30$  \\ \hline
$\beta$ (rad) & $1.6595$ & $1.5434$ & $1.4186$ & $1.2925$ & $1.1712$ & $1.0585$ & $0.9561$
\end{tabular}
\end{center}
\end{frame}
\begin{frame}
Si el segmento AB gira con velocidad angular constante de 25 rad/s, con el método de diferencias finitas de orden $O(h^{2})$, calcula la velocidad angular $d\beta / dt$ del segmento BC contra el ángulo $\alpha$
\end{frame}
\begin{frame}
\frametitle{Solución}
La velocidad angular de BC es:
\[ \dfrac{d\beta}{dt} = \dfrac{d\beta}{d\alpha}\dfrac{d\alpha}{dt} = 25\dfrac{d\beta}{d\alpha} \text{rad/s}\]
donde $d\beta/d\alpha$ se puede estimar con una aproximación de diferencias finitas, tomando los datos de la tabla anterior. Para los puntos extremos se usarían las diferencias hacia adelante y hacia atrás de orden $O(h^{2})$, mientras que para los puntos de en medio, el cálculo se hace con las diferencias centrales.
\end{frame}
\begin{frame}
Nótese que el incremento de $\alpha$ es
\[h = (5 \text{ grados}) \left( \frac{\pi}{180} \text{ rad} / \text{grados} \right) = 0.087266 \text{ rad}\]
Así tenemos que:
\begin{eqnarray*}
\dot{\beta}(0^{\circ}) &=&  25 \dfrac{-3 \beta(0^{\circ})+ 4 \beta(5^{\circ})-\beta(10^{\circ})}{2h} = -32.01 \text{rad} \\
\dot{\beta}(5^{\circ}) &=& \dfrac{\beta(10^{\circ})- \beta(0^{\circ})}{2h} = -34.51 \text{rad/s}
\end{eqnarray*}
\end{frame}
\begin{frame}
\frametitle{Ejercicio para entregar: Completa la tabla}
\begin{center}
\fontsize{12}{12}\selectfont
\begin{tabular}{c | c | c | c | c | c | c | c}
$\alpha$ (grados) & $0$ & $5$ & $10$ & $15$ & $20$ & $25$ & $30$  \\ \hline
$\dot{\beta}$ (rad/s) & $-32.01$ & $-34.51$ &  &  &  &  & 
\end{tabular}
\end{center}
\textcolor{red}{Debes de entregar la tabla y el código en \python.}
\end{frame}
%\begin{frame}
%\frametitle{Ejercicio resuelto}
%\begin{center}
%\fontsize{12}{12}\selectfont
%\begin{tabular}{c | c | c | c | c | c | c | c}
%$\alpha$ (grados) & 0 & 5 & 10 & 15 & 20 & 25 & 30  \\ \hline
%$\dot{\beta}$ (rad/s) & -32.01 & -34.51 & -35.94 & -35.44 & -33.52 & -30.81  & -27.86 
%\end{tabular}
%\end{center}
%\textcolor{red}{Debe de entregarse la tabla y el código en python.}
%\end{frame}
%
\section{Extrapolación de Richardson}
\begin{frame}
\frametitle{Extrapolación de Richardson}
La Extrapolación de Richardson es un método sencillo para aumentar la precisión de ciertos procedimientos numéricos, incluyendo las aproximaciones por diferencias finitas.
\\
\bigskip
Supongamos que tenemos la forma de calcular una cantidad $G$. Por otra parte, si consideramos que el resultado depende de un parámetro $h$, hagamos la aproximación por $g(h)$, tenemos que $G=g(h)+E(h)$, donde $E(h)$ representa el error.
\end{frame}
\begin{frame}
La extrapolación de Richardson puede remover el error, siempre que tenga la forma $E(h) = ch^{p}$, donde $c$ y $p$ son constantes. Iniciamos el cálculo para varios valores de $h$, digamos $h=h_{1}$, así
\[ G = g(h_{1}) + c h_{1}^{p}\]
Repetimos el cálculo con $h=h_{2}$, por tanto:
\[ G = g(h_{2}) + c h_{2}^{p} \]
\end{frame}
\begin{frame}
Eliminando $c$ y resolviendo para $G$, tenemos:
\[ G = \dfrac{(h_{1}/h_{2})^{p} g(h_{2}) - g(h_{1})}{(h_{1}/h_{2})^{p}-1}\]
que es la fórmula de Extrapolación de Richardson. En la práctica se usa $h_{2} = h_{1}/2$, quedando
\[ G = \dfrac{2^{p} g(h_{1}/2) - g(h_{1})}{2^{p}-1}\]
\end{frame}
\begin{frame}
\frametitle{Ejemplo}
Usemos el ejemplo de $(exp(-x))''$ en $x=1$, consideremos los valores de la tabla con seis dígitos.
\\
\bigskip
Dado que la extrapolación contine errores por truncamiento, debemos limitarnos a los valores de $h$ que producen redondeo insignificante.
\end{frame}
\begin{frame}
De la tabla anterior que calculamos, tenemos que:
\[ g(0.64) = 0.380609 \hspace{2cm} g(0.32) = 0.371029\]
El error de truncamiento en la aproximación central por diferencias finitas es:
\[E(h) = O(h^{2}) = c_{1}h^{2} + c_{2}h^{4} + c_{3}h^{6} + \ldots\]
\end{frame}
\begin{frame}
Por lo que podemos eliminar el primer término del error (dominante), si usamos $p=2$ y $h_{1}=0.64$, así
\[ \begin{split}
G =& \dfrac{2^{2}g(0.32)- g(0.64}{2^{2}-1} = \dfrac{4(0.371035)-0.380610}{3} \\
 =& 0.367843
\end{split} \]
Que es una aproximación de $(exp(-x))''$ con un error $=O(h^{4})$. Que es el mejor valor obtenido en comparación de los obtenidos con precisión de ocho dígitos.
\end{frame}
\begin{frame}
\frametitle{Ejemplo}
Teniendo en cuenta los puntos de datos uniformemente espaciados:
\begin{center}
\begin{tabular}{c | c | c | c | c | c }
x & $0$ & $0.1$ & $0.2$ & $0.3$ & $0.4$ \\ \hline
f(x) & $0.0000$ & $0.0819$ & $0.1341$ & $0.1646$ & $0.1797$
\end{tabular}
\end{center}
Calcular $f'(x)$ y $f''(x)$ en $x=0$ y $x=0.2$, usando la aproximación por diferencias finitas de orden $O(h^{2})$.
\end{frame}
\begin{frame}
\frametitle{Solución}
Usando la aproximación por diferencias finitas de orden $O(h^{2})$, de la lista de diferencias hacia adelante, tenemos:
\[ \begin{split} 
f'(0) =& \dfrac{-3f(0) + 4f(0.1) - f(0.2)}{2(0.1)} = 0.967 \\
f''(0) =& \dfrac{2 f(0) - 5f(0.1) + 4f(0.2) - f(0.3)}{(0.1)^{2}} = -3.77
\end{split} \]
\end{frame}
\begin{frame}
Si usamos ahora la aproximación por diferencias centrales:
\[ \begin{split}
f'(0.2) =& \dfrac{-f(0.1) + f(0.3)}{2(0.1)} = 0.4135 \\
f''(0.2) =& \dfrac{f(0.1)-2f(0.2) + f(0.3)}{(0.1)^{2}} = -2.17
\end{split} \]
\end{frame}
\begin{frame}
\frametitle{Otro Ejemplo}
Usando los siguientes datos(del ejemplo anterior):
\begin{center}
\begin{tabular}{c | c | c | c | c | c }
x & $0$ & $0.1$ & $0.2$ & $0.3$ & $0.4$ \\ \hline
f(x) & $0.0000$ & $0.0819$ & $0.1341$ & $0.1646$ & $0.1797$
\end{tabular}
\end{center}
Calcular $f'(0)$ con la mayor precisión posible.
\\
\medskip
Una solución es usar el método de extrapolación de Richardson con aproximación de diferencias finitas.
\end{frame}
\begin{frame}
\frametitle{Solución}
Iniciamos con la segunda aproximación por diferencias hacia adelante de orden $O(h^{2})$ para $f'(0)$: usamos en una $h=0.2$ y en otra $h=0.1$
\[ \begin{split}
g(0.2) =& \dfrac{-3f(0) + 4f(0.2) - f(0.4)}{2(0.2)} = 0.8918 \\
g(0.1) =& \dfrac{-3f(0) + 4f(0.1) - f(0.2)}{2(0.1)} = 0.9675
\end{split} \]
Recordemos que el error en ambas aproximaciones, es de la forma $E(h) = c_{1}h^{2} + c_{2} h^{4} + c_{3}h^{5} + \ldots$. 
\end{frame}
\begin{frame}
Usamos la extrapolación de Richardson para eliminar el término dominante. Con $p=2$, resulta
\[ f'(0) \simeq G = \dfrac{2^{2}g(0.1)-g(0.2)}{2^{2}-1}=0.9927 \]
la cual es una aproximación por diferencias finitas de orden $O(h^{4})$.
\end{frame}
\end{document}
