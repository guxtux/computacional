\documentclass[11pt]{article}
\usepackage[utf8]{inputenc}
%\usepackage[latin1]{inputenc}
\usepackage[spanish,es-noshorthands]{babel}
\decimalpoint
\usepackage{anysize}
\usepackage{graphicx} 
\usepackage{amsmath}
\usepackage{booktabs}
\usepackage{tabulary}
\usepackage{nccmath}
\usepackage{float}
\usepackage{tikz}
\usetikzlibrary{patterns}
\usepackage{color}
\usepackage{listings}
\renewcommand{\arraystretch}{1.5}
\lstset{ %
language=Python,                % choose the language of the code
basicstyle=\normalsize,       % the size of the fonts that are used for the code
numbers=left,                   % where to put the line-numbers
numberstyle=\footnotesize,      % the size of the fonts that are used for the line-numbers
stepnumber=1,                   % the step between two line-numbers. If it is 1 each line will be numbered
numbersep=5pt,                  % how far the line-numbers are from the code
backgroundcolor=\color{white},  % choose the background color. You must add \usepackage{color}
showspaces=false,               % show spaces adding particular underscores
showstringspaces=false,         % underline spaces within strings
showtabs=false,                 % show tabs within strings adding particular underscores
frame=single,   		% adds a frame around the code
tabsize=4,  		% sets default tabsize to 2 spaces
captionpos=b,   		% sets the caption-position to bottom
breaklines=true,    	% sets automatic line breaking
breakatwhitespace=false,    % sets if automatic breaks should only happen at whitespace
escapeinside={\#}{)}          % if you want to add a comment within your code
}
\marginsize{1.5cm}{1.5cm}{1cm}{2cm}  
\title{Tarea Interpolaci\'{o}n y C\'{a}lculo de ra\'{i}ces. \\ Curso de F\'{i}sica Computacional}
\author{M. en C. Gustavo Contreras May\'{e}n}
\date{ }
\begin{document}
\maketitle
\fontsize{14}{14}\selectfont
\begin{enumerate}
\item La densidad del aire $\rho$ var\'{i}a con la altura de la siguiente manera:
\begin{table}[H]
\centering \Large
\begin{tabulary}{15cm}{c | c | c | c}
$h (km)$ & $0$ & $3$ & $6$ \\
\midrule $\rho (kg/m^{3})$ & $1.225$ & $0.905$ & $0.652$
\end{tabulary}
\end{table}
Define $\rho(h)$ como una funci\'{o}n cuadr\'{a}tica a partir del m\'{e}todo de Lagrange.
\item Usando el m\'{e}todo de Newton, encuentra un polinomio que se ajuste a los siguientes puntos:
\begin{table}[H]
\centering \Large
\begin{tabulary}{15cm}{c | c | c | c | c | c}
$x$ & $-3$ & $2$ & $-1$ & $3$ & $1$ \\
\midrule $y$ & $0$ & $-5$ & $-4$ & $12$ & $0$
\end{tabulary}
\end{table}
\item El calor espec\'{i}fico del alumino $c_{p}$ depende de la temperatura $T$ como sigue:
\begin{table}[H]
\centering \Large
\begin{tabulary}{15cm}{c | c | c | c | c | c | c}
$T(^{\circ} C)$ & $-250$ & $-200$ & $-100$ & $0$ & $100$ & $300$ \\
\midrule $c_{p} (kJ/kgK)$ & $0.0163$ & $0.318$ & $0.699$ & $0.870$ & $0.941$ & $1.04$
\end{tabulary}
\end{table}
Calcula $c_{p}$ en $T=200^{\circ}$C y $T=400^{\circ}$C
\item La velocidad $v$ de un cohete Saturno V en vuelo vertical cercano a la superficie de la Tierra, puede aproximarse por
\[ v = u ln \dfrac{M_{0}}{M_{0} - \dot{m}t} - gt\]
donde
\begin{eqnarray*}
u &=& 2510 m/s = \text{velocidad de escape del cohete} \\
M_{0} &=& 2.8 \times 10^{6} kg = \text{masa del cohete al despegue} \\
\dot{m} &=& 13.3 \times 10^{3} kg/s = \text{tasa de consumo de combustible} \\ 
g &=& 9.81 m/s^{2} \text{aceleraci\'{o}n debida a la gravedad} \\
t &=& \text{tiempo medido desde el despegue}
\end{eqnarray*}
Calcula el tiempo en el cual el cohete alcanza la velocidad del sonido (335 m/s)
\item La energ\'{i}a libre de Gibbs en un mol de hidr\'{o}geno a una temperatura $T$ es:
\[ G = -RT ln [(T/T_{0})^{5/2}] J \]
donde la constante del gas es $R=8.311441$ J/K y $T_{0}=4.44418$ K. Calcula la temperatura en la cual $G=-10^{5}$ J.
\item La ecuaci\'{o}n de equilibrio qu\'{i}mico en la producci\'{o}n de metanol a partir de CO y $H_{2}$, es
\[ \dfrac{\xi (3-2 \xi)^{2}}{(1-\xi)^{3}} = 249.2\]
donde $\xi$ es el grado de equilibrio de la reacción. Determinar $\xi$.
\item Un cable de acero de longitud $s$ est\'{a} suspendido como se muestra en la figura:
\begin{center}
\begin{tikzpicture}[font=\small, scale=1.3]
	\draw [dashed] (0,0) -- (4,0);
	\draw [dashed] (2,-1) -- (2,0.5);
	\draw (0,-0.2) -- (0,0.2);
	\draw (4,-0.2) -- (4,0.2);
	\draw [pattern=north east lines] (-0.2,-0.2) rectangle (0,0.2);
	\draw [pattern=north east lines] (4,-0.2) rectangle (4.2,0.2);
	\draw (0,0) .. controls (1,-1) and (3,-1) .. (4,0);
	\draw (2.2,-1) node {O};
	\draw (1,0.5) node {$L/2$};
	\draw (3,0.5) node {$L/2$};
	\draw (0,-1.3) node {Longitud = s};
	\draw [->] (0.5,-1) -- (1,-0.65);
\end{tikzpicture}
\end{center}
La tensi\'{o}n de tracci\'{o}n m\'{a}xima en el cable, que se produce en los soportes, es
\[ \sigma_{max} = \sigma_{0} \cosh \beta \]
donde:
\begin{eqnarray*}
\beta &=& \dfrac{\gamma L}{2 \sigma_{0}} \\
\sigma_{0} &=& \mbox{la tensi\'{o}n de tracci\'{o}n en el cable en O.} \\
\gamma &=& \mbox{peso del cable por unidad de volumen.} \\
L &=& \mbox{extensi\'{o}n horizontal del cable.} 
\end{eqnarray*}
La relaci\'{o}n entre la extensi\'{o}n y la longitud del cable, est\'{a} relacionada con $\beta$ por:
\[ \dfrac{s}{L} = \dfrac{1}{\beta} \sinh \beta\]
Calcular $\sigma_{max}$ si $\gamma = 77 \times 10^{3}$ $N/m^{3}$ (para el acero), $L=1000$ m y $s=1100$ m
\end{enumerate}
\end{document}