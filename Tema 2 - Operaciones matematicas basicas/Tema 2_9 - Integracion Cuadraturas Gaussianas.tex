\documentclass[12pt]{beamer}
\newenvironment{ConCodigo}[1]
  {\begin{frame}[fragile,environment=ConCodigo]{#1}}
  {\end{frame}}
\graphicspath{{Imagenes/}{../Imagenes/}}
\usepackage[utf8]{inputenc}
\usepackage[spanish]{babel}
\usepackage{hyperref}
\usepackage{etex}
%\reserveinserts{28}
\usepackage{amsmath}
\usepackage{amsthm}
\usepackage{mathtools}
\usepackage{multicol}
\usepackage{multirow}
\usepackage{tabulary}
\usepackage{booktabs}
\usepackage{nccmath}
\usepackage{physics}
\usepackage{biblatex}
\usepackage[outdir=./]{epstopdf}
%\epstopdfsetup{outdir=./}
\usepackage{graphicx}
%\usepackage{enumitem,xcolor}
\usepackage{siunitx}
%\sisetup{scientific-notation=true}
%\usepackage{fontspec}
\usepackage{lmodern}
\usepackage{float}
\usepackage[format=hang, font=footnotesize, labelformat=parens]{caption}
\usepackage[autostyle,spanish=mexican]{csquotes}
\usepackage{standalone}
\usepackage{blkarray}
\usepackage{algorithm}
\usepackage{algorithmic}
\usepackage{tikz}
\usepackage[siunitx, RPvoltages]{circuitikz}
\usetikzlibrary{arrows,patterns,shapes}
\usetikzlibrary{decorations.markings}
\usetikzlibrary{arrows}
\usepackage{color}
\usepackage{xcolor}
%\usepackage{beton}
%\usepackage{euler}
%\usepackage[T1]{fontenc}
\usepackage[sfdefault]{roboto}  %% Option 'sfdefault' only if the base font of the document is to be sans serif
\usepackage[T1]{fontenc}
\renewcommand*\familydefault{\sfdefault}
\DeclareGraphicsExtensions{.pdf,.png,.jpg}
\usepackage{hyperref}
\renewcommand {\arraystretch}{1.5}
\newcommand{\python}{\texttt{python}}
\usefonttheme[onlymath]{serif}
\setbeamertemplate{navigation symbols}{}
\usetikzlibrary{patterns}
\usetikzlibrary{decorations.markings}
\tikzstyle{every picture}+=[remember picture,baseline]
%\tikzstyle{every node}+=[inner sep=0pt,anchor=base,
%minimum width=2.2cm,align=center,text depth=.15ex,outer sep=1.5pt]
%\tikzstyle{every path}+=[thick, rounded corners]
\setbeamertemplate{caption}[numbered]
\newcommand{\ptm}{\fontfamily{ptm}\selectfont}
%Se usa la plantilla Warsaw modificada con spruce
\mode<presentation>
{
  \usetheme{Warsaw}
  \setbeamertemplate{headline}{}
  \useoutertheme{default}
  \usecolortheme{albatross}
  \setbeamercovered{invisible}
}
% \AtBeginSection[]
% {
% \begin{frame}<beamer>{Contenido}
% \normalfont\mdseries
% \tableofcontents[currentsection]
% \end{frame}
% }

%Se usa la plantilla Madrid modificada con beaver
\mode<presentation>
{
  \usetheme{Madrid}
  \setbeamertemplate{headline}{}
  %\useoutertheme{infolines}
  \usecolortheme{beaver}
  \setbeamercovered{invisible}
  

\setbeamertemplate{section in toc}[sections numbered]
\setbeamertemplate{subsection in toc}[subsections numbered]
\setbeamertemplate{subsection in toc}{\leavevmode\leftskip=3.2em\rlap{\hskip-2em\inserttocsectionnumber.\inserttocsubsectionnumber}\inserttocsubsection\par}
\setbeamercolor{section in toc}{fg=blue}
\setbeamercolor{subsection in toc}{fg=blue}
\setbeamerfont{subsection in toc}{size=\small}

\setbeamertemplate{navigation symbols}{}
\setbeamertemplate{caption}[numbered]

}

\input{../Preambulos/pre_codigo}
\input{../Preambulos/pre_define_footers_Madrid_beaver_2}
\title{Integración numérica}
\subtitle{Tema 2 - Operaciones matemáticas básicas}
\author{M. en C. Gustavo Contreras Mayén}
\date{\today}
\institute{Facultad de Ciencias - UNAM}
\titlegraphic{\includegraphics[width=1.75cm]{Imagenes/escudo-facultad-ciencias}\hspace*{4.75cm}~%
   \includegraphics[width=1.75cm]{Imagenes/escudo-unam}
}
\newtheorem{miteorema}{Teorema}
\begin{document}
\maketitle
\fontsize{14}{14}\selectfont
\spanishdecimal{.}
\setbeamercolor{frametitle}{bg=blue!30!white}
\section*{Contenido}
\frame{\tableofcontents[currentsection, hideallsubsections]}
\section{Cuadraturas Gaussianas}
\frame{\tableofcontents[currentsection, hideothersubsections]}
\subsection{Definición de cuadraturas}
\begin{frame}
\frametitle{Cuadraturas Gaussianas}
Hemos visto que las fórmulas de Newton-Cotes para aproximar la intregral
\[ \int_{a}^{b} f(x) dx\]
trabajan muy bien si $f(x)$ es una función suave, como los polinomios.
\end{frame}
\begin{frame}
\frametitle{Cuadraturas Gaussianas}
También aplica para las cuadraturas Gaussianas, ya que son buenas para estimar integrales de la forma:
\[ \int_{a}^{b} w(x) f(x) dx\]
donde $w(x)$ se denomina \emph{función de ponderación} que puede contener singularidades, siempre y cuando sean integrables.
\end{frame}
\begin{frame}
\frametitle{Cuadraturas Gaussianas}
Un ejemplo de este tipo, es la integral
\[ \int_{0}^{1} \left(1+x^{2} \right) ln x \; dx \]
\\
\bigskip
\pause
Cuando tenemos límites de integración infinitos
\[ \int_{0}^{\infty} \exp(-x) \sin x dx \]
éstos se pueden reacomodar para calcular la integral.
\end{frame}
\begin{frame}
\frametitle{Fórmulas de integración Gaussianas}
Las fórmulas de integración Gaussianas tiene la misma forma de las reglas de Newton-Cotes:
\[I = \sum_{i=0}^{n} A_{i} f(x_{i})\]
donde $I$ representa la aproximación al valor de la integral, la diferencia radica en la forma en que se determinan los pesos $A_{i}$ y abscisas nodales $x_{i}$.
\end{frame}
\begin{frame}
\frametitle{Fórmulas de integración Gaussianas}
En la integración de Newton-Cotes los nodos se espacian uniformemente en $(a, b)$, es decir, estaban ya predeterminadas sus ubicaciones.
\\
\bigskip
En la cuadratura de Gauss, se eligen los nodos y los pesos de modo que la ecuación para $I$, se obtiene la integral exacta si $f(x)$ es un polinomio de grado $2n + 1$ o menor, es decir:
\end{frame}
\begin{frame}
\[ \int_{a}^{b} w(x) P_{m}(x) dx = \sum_{i=0}^{n} A_{i}P_{m}(x_{i}), \hspace{1cm} m \leq 2n+1\]
\pause
Una manera de determinar los pesos y las abscisas es sustituir $P_{0}(x)=1$, $P_{1}(x)=x,\ldots$, $P_{2n+1}(x) = x^{2n+1}$ en la ecuación anterior y resolver el sistema de $2n+2$ ecuaciones:
\[ \int_{a}^{b} w(x) x^{j}dx = \sum_{i=0}^{n} A_{i}x_{i}^{j}, \hspace{1cm} j=0,1,\ldots,2n+1 \]
para las incógnitas $A_{i}$ y $x_{i}$.
\end{frame}
\begin{frame}
Como ejemplo veamos: sea $w(x)= \exp(-x)$, $a=0$, $b=\infty$ y $n=1$. Las cuatro ecuaciones que determinan $x_{0}$, $x_{1}$, $A_{0}$ y $A_{1}$ son:
\begin{eqnarray*}
\int_{0}^{\infty} \exp(-x) dx &=& A_{0}+A_{1} \\
\pause
\int_{0}^{\infty} \exp(-x) x dx &=& A_{0} x_{0} + A_{1} x_{1} \\
\pause
\int_{0}^{\infty} \exp(-x) x^{2} dx &=& A_{0} x_{0}^{2}+ A_{1} x_{1}^{2} \\
\pause
\int_{0}^{\infty} \exp(-x) x^{3} dx &=& A_{0} x_{0}^{3} + A_{1} x_{1}^{3} \\
\end{eqnarray*}
\end{frame}
\begin{frame}
Evaluando las integrales, obtenemos
\begin{eqnarray*}
A_{0} + A_{1} &=& 1 \\
A_{0} x_{0} + A_{1} x_{1} &=& 1 \\
A_{0} x_{0}^{2} + A_{1} x_{1}^{2} &=& 2 \\
A_{0} x_{0}^{3} + A_{1} x_{1}^{3} &=& 6
\end{eqnarray*}
Cuya solución es:
\begin{eqnarray*}
x_{0} = 2 - \sqrt{2} \hspace{0.75cm} A_{0} \dfrac{\sqrt{2}+1}{2\sqrt{2}} \\
x_{1} = 2 + \sqrt{2} \hspace{0.75cm} A_{1} \dfrac{\sqrt{2}-1}{2\sqrt{2}}
\end{eqnarray*}
\end{frame}
\begin{frame}
Por tanto la fórmula de integración obtenida es:
\begin{eqnarray*}
\int_{0}^{\infty} \exp(-x) f(x) dx & \simeq \dfrac{1}{2\sqrt{2}} \left[ \left( \sqrt{2}+1 \right) f \left(2 - \sqrt{2}\right) + \right. \\ 
&+ \left. \left( \sqrt{2}-1 \right) f\left(2 +\sqrt{2} \right) \right]
\end{eqnarray*}
\pause
Debido a la no linealidad de las ecuaciones, este enfoque no va a funcionar bien para valores grandes de $n$.
\end{frame}
\begin{frame}
Hay métodos prácticos para calcular $x_{i}$ y $A_{i}$ que requieren un cierto conocimiento de los polinomios ortogonales y su relación con la cuadratura de Gauss.
\\
\bigskip
Hay, sin embargo, varias fórmulas \enquote{clásicas} de integración Gaussianas para los cuales, las abscisas y pesos han sido calculados y tabulados con gran precisión.
\end{frame}
\begin{frame}
Estas fórmulas se pueden utilizar sin conocer la teoría detrás de ellas, ya que todo lo que uno necesita para la integración de Gauss son los valores de $x_{i}$ y $A_{i}$.
\end{frame}
\subsection{Polinomios ortogonales}
\begin{frame}
\frametitle{Polinomios ortogonales}
Los polinomios ortogonales se utilizan en muchas áreas de la física, de la matemática y del análisis numérico; se han estudiado a fondo y muchas de sus propiedades ya son conocidas. 
\end{frame}
\subsection{Polinomios ortogonales}
\begin{frame}
\frametitle{Polinomios ortogonales}
Los polinomios $\varphi_{n}(x)$, con $n = 0, 1, 2,\ldots$ ($n$ es el grado del polinomio) se dice que forman un conjunto ortogonal en el intervalo $(a, b)$ con respecto a la función de peso $w(x)$ si
\[ \int_{a}^{b} w(x) \varphi_{m}(x) \varphi_{n}(x) dx = 0, \hspace{0.5cm} m \neq n \]
\end{frame}
\begin{frame}
El conjunto se determina (con excepción de un factor constante) por: la elección de la función de peso y los límites de integración.
\\
\bigskip
Es decir, cada conjunto de polinomios ortogonales se asocia con ciertos $w(x)$, $a$ y $b$. El factor constante se especifica de manera estandarizada.
\end{frame}
\begin{frame}
A continuación se enlistan algunos de los polinomios ortogonales clásicos, la última columna indica la estandarización usada.
\end{frame}
\begin{frame}
\frametitle{Polinomios ortogonales}
\fontsize{12}{12}\selectfont
\begin{tabular}{| l | c | c | c | c | c |}
\hline
Nombre & Símbolo & a & b & $w(x)$ & $\int_{a}^{b} \left[ \varphi_{n} (x)\right]^{2} dx $ \\ \hline
Legendre & $p_{n}(x)$ & $-1$ & $1$ & $1$ & $2/(2n+1)$ \\
Chebyshev & $T_{n}(x)$ & $-1$ & $1$ & $(1-x^{2})^{-1/2}$ & $\pi/2 \hspace{0.2cm} (n>0)$ \\
Laguerre & $L_{n}(x)$ & $0$ & $\infty$ & $e^{-x}$ & $1$ \\
Hermite & $H_{n}(x)$ & $-\infty$ & $\infty$ & $e^{x^{2}}$ & $\sqrt{\pi} 2^{n} n!$ \\ \hline
\end{tabular}
\end{frame}
\begin{frame}
Los polinomios ortogonales cumplen relaciones de recurrencia de la forma
\[ a_{n} \varphi_{n+1} (x) = (b_{n} + c_{n} x) \varphi_{n} (x) - d_{n} \varphi_{n-1} (x)  \]
Si los dos primeros polinomios del conjunto se conocen, los otros elementos del conjunto pueden calcularse de la ecuación anterior. 
\end{frame}
\begin{frame}
Los coeficientes en la fórmula de recurrencia, junto con $\varphi_{0}(x)$ y $\varphi_{1}(x)$ son:
\fontsize{12}{12}\selectfont
\begin{tabular}{| l | c | c | c | c | c | c |}
\hline
Nombre & $\varphi_{0}(x)$ & $\varphi_{1}(x)$ & $a_{n}$ & $b_{n}$ & $c_{n}$ & $d_{n}$ \\ \hline
Legendre & $1$ & $x$ & $n+1$ & $0$ & $2n+1$ & n \\
Chebyshev & $1$ & $x$ & $1$ & $0$ & $2$ & $1$ \\
Laguerre & $1$ & $1-x$ & $n+1$ & $2n+1$ & $-1$ & $n$ \\
Hermite & $1$ & $2x$ & $1$ & $0$ & $2$ & $2$ \\ \hline
\end{tabular}
\end{frame}
\begin{frame}
\frametitle{Otra manera para obtener los polinomios}
Los polinomios ortogonales clásicos también se pueden obtener de las fórmulas:
\begin{eqnarray*}
p_{n}(x) &=& \dfrac{(-1)^{n}}{2^{n}n!} \dfrac{d^{n}}{dx^{n}} \left[ \left( 1-x^{2} \right)^{n} \right] \\
T_{n}(x) &=& \cos(n\cos^{-1}x), \hspace{0.3cm} n>0 \\
L_{n}(x) &=& \dfrac{e^{x}}{n!} \dfrac{d^{n}}{dx^{n}} \left( x^{n} e^{-x} \right) \\
H_{n}(x) &=& (-1)^{n} e^{x^{2}} \dfrac{d^{n}}{dx^{n}} \left(e^{x^{2}} \right)
\end{eqnarray*}
\end{frame}
\begin{frame}
\frametitle{Derivadas de los polinomios ortogonales}
Las derivadas de los polinomios anteriores se pueden calcular de:
\begin{eqnarray*}
(1-x^{2}) p'_{n}(x) &=& n[-x p_{n}(x) + p_{n-1}(x) ] \\
(1-x^{2}) T'_{n}(x) &=& n[-x T_{n}(x) + n p T{n-1}(x) ] \\
x L'_{n} (x) &=& n [ L_{n}(x) - L_{n-1}(x) ] \\
H'_{n}(x) &=& 2 n H_{n-1}(x)
\end{eqnarray*}
\end{frame}
\begin{frame}
\frametitle{Propiedades de los polinomios}
Algunas propiedades los polinomios ortogonales que son relevantes para la preceso de integración Gaussiana son:
\setbeamercolor{item projected}{bg=yellow!70!black,fg=black}
\setbeamertemplate{enumerate items}[circle]
\begin{enumerate}[<+->]
\item $\varphi(x)$ tiene $n$ distintos ceros en el intervalo $(a,b)$
\item Los ceros de $\varphi_{n}(x)$ están entre los ceros de $\varphi_{n+1}(x)$
\item Cualquier polinomio $P_{n}(x)$ de grado $n$ puede expresarse de la forma:
\[ P_{n}(x) =  \sum_{i=0}^{n} c_{i} \varphi_{i} (x) \]
\seti
\end{enumerate}
\end{frame}
\begin{frame}
\frametitle{Propiedades de los polinomios}
\setbeamercolor{item projected}{bg=yellow!70!black,fg=black}
\setbeamertemplate{enumerate items}[circle]
\begin{enumerate}
\conti
\item Se deduce de la ecuación anterior y de la propiedad de ortogonalidad que:
\fontsize{12}{12}\selectfont
\[ \int_{a}^{b} w(x) P_{n}(x) \varphi_{n+m} (x) dx = 0, \hspace{0.4cm} m \geq 0 \]
\end{enumerate}
\end{frame}
\subsection{Deduciendo las abscisas nodales y los pesos}
\begin{frame}
\frametitle{Deduciendo las abscisas nodales y los pesos}
Hay dos teoremas importantes que son de gran utilidad para apoyarnos y tomar sus resultados para la integración Gaussiana, la demostración es relativamente sencilla, pero no los demostraremos aquí, puede ser un buen ejercicio fuera de clase.
\end{frame}
\begin{frame}
\frametitle{Teorema 1}
\begin{miteorema}
Las abscisas nodales $x_{0},x_{1},\ldots,x_{n}$ son los ceros del polinomio $\varphi_{n+1}(x)$  que pertenece al conjunto ortogonal definido por
\[ \int_{a}^{b} w(x) \varphi_{m}(x) \varphi_{n}(x) dx = 0, \hspace{0.5cm} m \neq n \]
\end{miteorema}
\end{frame}
\begin{frame}
\frametitle{Teorema 2}
\begin{miteorema}
\[ A_{i} = \int_{a}^{b} w(x) \mathcal{L}_{i} (x) dx, \hspace{1cm} i=0,1,\ldots,n \]
donde $\mathcal{L}_{i} (x)$ son las funciones cardinales de Lagrange que abarcan los nodos $x_{0},x_{1},\ldots,x_{n}$.
\end{miteorema}
\end{frame}
\begin{frame}
No es difícil calcular los ceros $x_{i}$, $i=0,1,\ldots,n$ de un polinomio $\varphi_{n+1} (x)$ que pertenece a un conjunto ortogonal, podemos usar alguno de los métodos discutidos en la parte de cálculo de raíces.
\\
\bigskip
Una vez conocidos los ceros, los pesos $A_{i}$, $i=0,1,\ldots,n$ pueden calcularse de la ecuación anterior.
\end{frame}
\begin{frame}
\frametitle{Fórmulas para calcular los pesos}
Se puede demostrar que los pesos se pueden calcular a partir de:
\begin{eqnarray*}
\mbox{Gauss-Legendre   }  A_{i} &=& \dfrac{2}{(1-x^{2}_{i})\left[P'_{n+1} (x_{i}) \right]^{2}} \\
\mbox{Gauss-Laguerre   } A_{i} &=& \dfrac{1}{x_{i} \left[L'_{n+1} (x_{i}) \right]^{2}} \\
\mbox{Gauss-Hermite   } A_{i} &=& \dfrac{2^{n+2}(n+1)! \sqrt{\pi}}{\left[ H'_{n+1} (x_{i}) \right]^{2}}
\end{eqnarray*}
\end{frame}
\begin{frame}
\frametitle{Abscisas y pesos para cuadraturas gaussianas}
Vamos a mencionar la expresión para algunas fórmulas de integración por cuadraturas gaussianas.
\\
\bigskip
La tabla de abscisas y pesos que se presenta a continuación, cubre para $n=1$ a $5$, y se redondea a seis decimales.
\end{frame}
\begin{frame}
Las operaciones con estos valores, se considera que funcionan bien si se hacen las cuentas a mano, en caso de requerir una mayor precisión o incluir un número  mayor de nodos, será necesario usar la computadora.
\end{frame}
\subsection{Error en la cuadratura gaussiana}
\begin{frame}
\frametitle{Error en la cuadratura gaussiana}
El error debido al truncamiento
\[ E = \int_{a}^{b} w(x) f(x) dx  - \sum_{i=0}^{n} A_{i} f(x_{i})\]
es de la forma $E= K(n) f^{2n+2} (c) $, donde $a<c<b$, el valor de $c$ no se conoce, solamente los extremos.
\end{frame}
\begin{frame}
\frametitle{Error en la cuadratura gaussiana}
La expresión para $K(n)$ depende de la cuadratura en particular que se esté utilizando.
\\
\bigskip
Si las derivadas de $f(x)$ se pueden evaluar, el error de las fórmulas es útil para estimar el error en el intervalo.
\end{frame}
\begin{frame}[plain]
\frametitle{Cuadratura de Gauss-Legendre}
\[ \int_{-1}^{1} f(\xi) d \xi \approx \sum_{i=0}^{n} A_{i} f(\xi_{i}) \]
\fontsize{10}{10}\selectfont
\begin{center}
\begin{tabular}{|c c c | c c c|}
\hline
$\pm \xi_{i}$ &       & $A_{i}$    & $\pm \xi_{i}$ &       & $A_{i}$ \\ \hline
             & $n=1$ &            &               & $n=4$ &         \\ %\hline
$0.577350$    &       & $1.000000$ & $0.000000$    &       & $0.568889$ \\ %\hline
             & $n=2$ &            & $0.538469$    &       & $0.478629$ \\ %\hline
$0.000000$    &       & $0.888889$ & $0.906180$    &       & $0.236927$ \\ %\hline
$0.774597$    &       & $0.555556$ &               & $n=5$ &            \\ %\hline
             & $n=3$ &            & $0.238619$    &       & $0.467914$ \\ %\hline
$0.339981$    &       & $0.652145$ & $0.661209$    &       & $0.360762$ \\ %\hline
$0.861136$    &       & $0.347855$ & $0.932470$    &       & $0.171324$ \\ \hline
\end{tabular}
\end{center}
\end{frame}
\begin{frame}
\frametitle{Cuadratura de Gauss-Legendre}
La cuadratura de Gauss-Legendre es la más utilizada. 
\\
\bigskip
Nótese que los nodos están colocados simétricamente sobre $\xi=0$, y los pesos asociados al par de nodos simétricos, son iguales, i.e. para $n=1$, tenemos que $\xi_{0} = - \xi_{1}$ y $A_{0} = A_{1}$.
\end{frame}
\begin{frame}
\frametitle{Error en la cuadratura}
El error de truncamiento está dado por:
\[ E = \dfrac{2^{2n+3} [(n+1)!]^{4}}{(2n+3)[(2n+2)!]^{3}} f^{2n+2} (c), \hspace{1cm} -1 < c < 1 \]
\end{frame}
\begin{frame}
Para usar la cuadratura de Gauss-Legendre en la integral $\int_{a}^{b} f(x) dx$, primero hay que \enquote{mapear} el intervalo de integración $(a,b)$ al intervalo \enquote{estándar} $(-1,1)$, para ello, usamos la transformación
\[ x = \dfrac{b+a}{2} + \dfrac{b-a}{2} \xi \]
\end{frame}
\begin{frame}
Ahora $dx = d \xi (b-a)/2$, y la cuadratura toma la expresión
\[ \int_{a}^{b} f(x) dx \approx \dfrac{b-a}{2} \sum_{i=0}^{n} A_{i} f(x_{i}) \]
\end{frame}
\begin{frame}
El error por truncamiento, se expresa como
\[ E = \dfrac{(b-a)^{2n+3} [(n+1)!]^{4}}{(2n+3)[(2n+2)!]^{3}} f^{(2n+2)} (c), \hspace{0.7cm} a<c<b \]
\end{frame}
% \begin{frame}
% \frametitle{Ejercicio para el examen}
% Te pedimos que entregues una lista con los pesos y nodos para las siguientes cuadraturas:
% \begin{enumerate}
% \item Gauss-Chebyshev
% \[\int_{1}^{1} (1-x^{2})^{-1/2} f(x) \approx \dfrac{\pi}{n+1} \sum_{i=0}^{n} f(x_{i}) \]
% \item Gauss-Laguerre
% \[  \int_{0}^{\infty} e^{-x} f(x) dx \approx \sum_{i=0}^{n} A_{i} f(x_{i}) \]
% \end{enumerate}
% \end{frame}
% \begin{frame}
% \frametitle{Ejercicio para el examen}
% \begin{enumerate}
% \item Gauss-Hermite
% \[ \int_{-\infty}^{\infty} e^{-x^{2}} f(x) dx \approx \sum_{i=0}^{n} A_{i} f(x_{i}) \]
% \end{enumerate}
% \end{frame}
% \begin{frame}
% \frametitle{¿Aquí termina todo respecto a la integración numérica?}
% Como sabemos de los cursos de cálculo, podemos considerar ahora un proceso de integración para calcular integrales de funciones con dos y hasta tres variables, recurriendo a las técnicas mostradas.
% \\
% \medskip
% ¿Cómo construimos un algoritmo para ello?
% \\
% \medskip
% El proceso no es complicado pero requiere una revisión cuidadosa, con \python{} podemos hacer el proceso más sencillo para calcular una integral doble o triple, pero no está demás que te las ingenies para desarrollar un código!!
% \end{frame}
\begin{frame}
\frametitle{Ejemplo}
Evaluar la integral
\[ \int_{-1}^{1} (1 - x^{2})^{3/2} \; dx \]
con la mayor precisión posible, usando una cuadratura gaussiana.
\end{frame}
\begin{frame}
\frametitle{Modo clásico}
El modo normal de resolver mediante una cuadratura gaussiana, es calcuar los nodos para una cuadratura de tipo Gauss-Legendre.
\\
\bigskip
Por lo que tendríamos que ocupar las expresiones que nos devuelvan los ceros de los polinomios. 
\end{frame}
\begin{frame}[fragile]
\frametitle{Usando \texttt{scipy.integrate.quadrature}}
Para usar la función \azulfuerte{integrate.cuadrature} contenida en el módulo \azulfuerte{scipy}, hay que considerar lo siguiente:
\begin{verbatim}
scipy.integrate.quadrature(func, a, b, 
tol=1.49e-08, maxiter=50)
\end{verbatim}
\pause
Esta función calcula la integral definida usando una cuadratura gaussina con tolerancia fija.
\end{frame}
\begin{frame}
\frametitle{Argumentos de \texttt{quadrature}}
% \newcommand{\localtextbulletone}{\textcolor{red}{\raisebox{.45ex}{\rule{.6ex}{.6ex}}}}
% \renewcommand{\labelitemi}{\localtextbulletone}
Los argumentos son:
\begin{itemize}
\setbeamercolor{local structure}{fg=red}
\item \texttt{func} : una función, ya sea una función de \python{} o una expresión.
\item \texttt{a} : dato de tipo \texttt{float}, que representa el límite inferior de integración.
\item \texttt{b} : dato de tipo \texttt{float}, que representa el límite superior de integración.
\item \texttt{maxiter} : dato de tipo \texttt{int}, este argumento es opcional, representa el orden máximo para la quadratura gaussiana.
\end{itemize}
\end{frame}
\begin{frame}
% \newcommand{\localtextbulletone}{\textcolor{blue}{\raisebox{.45ex}{\rule{.6ex}{.6ex}}}}
% \renewcommand{\labelitemi}{\localtextbulletone}
\frametitle{Valores que devuelve la función}
Devuelve:
\begin{itemize}
	\setbeamercolor{local structure}{fg=blue}
\item \texttt{val} : dato de tipo \texttt{float}, que representa la aproximación a la integral.
\item \texttt{err} : dato de tipo \texttt{float}, que representa el error en las últimas dos estimaciones de la integral.
\end{itemize}
\end{frame}
\begin{frame}[fragile]
\frametitle{Código}
\begin{lstlisting}[caption=Código para cuadratura gaussiana, style=FormattedNumber, basicstyle=\linespread{1.1}\ttfamily=\small, columns=fullflexible]
from scipy.integrate import quadrature

def g(x):
    return (1 - x**2)**(3/2.)

print(quadrature(g, -1., 1)[0])
\end{lstlisting}
\end{frame}
\begin{frame}
\frametitle{Solución}
El valor de la integral es $1.1781$
\begin{figure}
	\centering
	\includegraphics[scale=0.5]{Imagenes/cuadratura_01.eps}
	\caption{El área debajo de la curva, representa el valor de la integral.}
\end{figure}
\end{frame}
\end{document}

