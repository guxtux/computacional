\documentclass[12pt]{beamer}
\newenvironment{ConCodigo}[1]
  {\begin{frame}[fragile,environment=ConCodigo]{#1}}
  {\end{frame}}
\graphicspath{{Imagenes/}{../Imagenes/}}
\usepackage[utf8]{inputenc}
\usepackage[spanish]{babel}
\usepackage{hyperref}
\usepackage{etex}
%\reserveinserts{28}
\usepackage{amsmath}
\usepackage{amsthm}
\usepackage{mathtools}
\usepackage{multicol}
\usepackage{multirow}
\usepackage{tabulary}
\usepackage{booktabs}
\usepackage{nccmath}
\usepackage{physics}
\usepackage{biblatex}
\usepackage[outdir=./]{epstopdf}
%\epstopdfsetup{outdir=./}
\usepackage{graphicx}
%\usepackage{enumitem,xcolor}
\usepackage{siunitx}
%\sisetup{scientific-notation=true}
%\usepackage{fontspec}
\usepackage{lmodern}
\usepackage{float}
\usepackage[format=hang, font=footnotesize, labelformat=parens]{caption}
\usepackage[autostyle,spanish=mexican]{csquotes}
\usepackage{standalone}
\usepackage{blkarray}
\usepackage{algorithm}
\usepackage{algorithmic}
\usepackage{tikz}
\usepackage[siunitx, RPvoltages]{circuitikz}
\usetikzlibrary{arrows,patterns,shapes}
\usetikzlibrary{decorations.markings}
\usetikzlibrary{arrows}
\usepackage{color}
\usepackage{xcolor}
%\usepackage{beton}
%\usepackage{euler}
%\usepackage[T1]{fontenc}
\usepackage[sfdefault]{roboto}  %% Option 'sfdefault' only if the base font of the document is to be sans serif
\usepackage[T1]{fontenc}
\renewcommand*\familydefault{\sfdefault}
\DeclareGraphicsExtensions{.pdf,.png,.jpg}
\usepackage{hyperref}
\renewcommand {\arraystretch}{1.5}
\newcommand{\python}{\texttt{python}}
\usefonttheme[onlymath]{serif}
\setbeamertemplate{navigation symbols}{}
\usetikzlibrary{patterns}
\usetikzlibrary{decorations.markings}
\tikzstyle{every picture}+=[remember picture,baseline]
%\tikzstyle{every node}+=[inner sep=0pt,anchor=base,
%minimum width=2.2cm,align=center,text depth=.15ex,outer sep=1.5pt]
%\tikzstyle{every path}+=[thick, rounded corners]
\setbeamertemplate{caption}[numbered]
\newcommand{\ptm}{\fontfamily{ptm}\selectfont}
%Se usa la plantilla Warsaw modificada con spruce
\mode<presentation>
{
  \usetheme{Warsaw}
  \setbeamertemplate{headline}{}
  \useoutertheme{default}
  \usecolortheme{albatross}
  \setbeamercovered{invisible}
}
% \AtBeginSection[]
% {
% \begin{frame}<beamer>{Contenido}
% \normalfont\mdseries
% \tableofcontents[currentsection]
% \end{frame}
% }

\include{pre_codigo}
\begin{document}
\title{Tema 2 - Operaciones matemáticas básicas}
\subtitle{Integración numérica}
\author{M. en C. Gustavo Contreras Mayén}
\maketitle
\fontsize{14}{14}\selectfont
\spanishdecimal{.}
\begin{frame}{Contenido}
\tableofcontents[pausesections]
\end{frame}
\section{Problema inicial}
\begin{frame}
\frametitle{Problema inicial}
Calcular
\[\int_{a}^{b} f(x) dx\]
donde $f(x)$ es una función dada.
\end{frame}
\section{Introducción}
\begin{frame}
\frametitle{Introducción}
La integración numérica (también conocida como \textcolor{blue}{cuadratura}) es un procedimiento con mayor precisión que la diferenciación numérica.
\\
\bigskip
La cuadratura aproxima la integral definida
\[\int_{a}^{b} f(x) dx\]
mediante la suma
\[ I = \sum_{i=0}^{n} A_{i}f(x_{i})\]
\fontsize{12}{12}\selectfont
donde las \textit{abscisas nodales} $x_{i}$ y los pesos $A_{i}$ dependen de una regla en particular usada para la cuadratura.
\end{frame}
\begin{frame}
Todas las reglas de cuadratura se dividen en dos grupos:
	\begin{enumerate}
		\item Fórmulas de \textcolor{red}{Newton-Cotes}.
		\item Fórmulas de \textcolor{blue}{Cuadraturas Gaussianas}.
	\end{enumerate}
\end{frame}
\section{Fórmulas de Newton-Cotes}
\begin{frame}
\frametitle{Fórmulas de Newton-Cotes}
Estas fórmulas se caracterizan por usar un espaciamiento uniforme y constante en las abscisas, aquí se consideran los métodos del trapecio y la regla de Simpson.
\\
\bigskip
Son útiles si $f(x)$ se ha evaluado en intervalos iguales; dado que las fórmulas Newton-Cotes se basan en una interpolación local, se requiere de una porción del dominio para ajustarla al polinomio.
\end{frame}
\begin{frame}
Consideremos la integral definida
\[\int_{a}^{b} f(x)dx\]
Dividimos el intervalo de integración $[a,b]$ en $n$ intervalos de igual longitud $h=(b-a)/n$, y hacemos que las abscisas sean $x_{0},x_{1},\ldots,x_{n}$.
\end{frame}
\begin{frame}
\frametitle{Aproximación polinomial de $f(x)$}
	\begin{center}
		\begin{tikzpicture}[font=\small]
			\draw [->] (0,0) -- node [near end, left]{$f(x)$} (0,3);
			\draw [->] (0,0) -- (7,0);
			\draw (7.4,0) node {x};
			\draw [blue] (1,0.5) .. controls(2.5,2) .. (6,0.4);
			\foreach \x in {1,1.5,...,6} \draw (\x,0) circle (0.03cm);
			\draw (1,-0.3) node {$x_{0}$};
			\draw (1,-0.7) node {a};
			\draw (1.5,-0.3) node {$x_{1}$};
			\draw (2,-0.3) node {$x_{2}$};
			\draw (2.5,-0.3) node {$x_{3}$};
			\draw (3,-0.3) node {$x_{4}$};
			\draw (5.5,-0.3) node {$x_{n-1}$};
			\draw (6,-0.3) node {$x_{n}$};
			\draw (6,-0.7) node {b};
			\draw [dashed] (1,0) -- (1,0.5);
			\draw [dashed] (1.5,0) -- (1.5,1.05);
			\draw [dashed] (2,0) -- (2,1.37);
			\draw [dashed] (2.5,0) -- (2.5,1.6);
			\draw [dashed] (3,0) -- (3,1.56);
			\draw [dashed] (3.5,0) -- (3.5,1.48);
			\draw [dashed] (4,0) -- (4,1.27);
			\draw [dashed] (4.5,0) -- (4.5,1.09);
			\draw [dashed] (5,0) -- (5,0.85);
			\draw [dashed] (5.5,0) -- (5.5,0.65);
			\draw [dashed] (6,0) -- (6,0.43);
			\draw [<->] (2.5,1) -- node [midway, below] {h} (3,1);
		\end{tikzpicture}
	\end{center}
\end{frame}
\begin{frame}
Ahora aproximamos $f(x)$ con un polinomio de orden $n$ que intersecta todos los nodos. La expresión para el polinomio de Lagrange es:
\[P_{n}(x) = \sum_{i=0}^{n} f(x_{i})\mathcal{L}_{i}(x) \]
donde $\mathcal{L}_{i}(x)$ son las funciones definidas en el tema de interpolación. 
\end{frame}
\begin{frame}
Por tanto, un aproximación a la integral es
\fontsize{12}{12}\selectfont
\[I = \int_{a}^{b} P_{n}(x) dx = \sum_{i=0}^{n} \left[ f(x_{i}) \int_{a}^{b} \mathcal{L}_{i}(x) dx \right] = \sum_{i=0}^{n} A_{i} f(x_{i})\]
\fontsize{14}{14}\selectfont
donde
\[A_{i} = \int_{a}^{b} \mathcal{L}_{i} dx, \hspace{1cm} i=0,1,\ldots,n\]
\end{frame}
\begin{frame}
Las ecuaciones
\fontsize{12}{12}\selectfont
\[I = \int_{a}^{b} P_{n}(x) dx = \sum_{i=0}^{n} \left[ f(x_{i}) \int_{a}^{b} \mathcal{L}_{i}(x) dx \right] = \sum_{i=0}^{n} A_{i} f(x_{i})\]
\fontsize{14}{14}\selectfont
se conocen como las fórmulas de Newton-Cotes. Siendo los casos:
	\begin{enumerate}
		\item $n=1$, Regla del trapecio.
		\item $n=2$, Regla de Simpson.
		\item $n=3$, Regla de Simpson de $3/8$.
	\end{enumerate}
\end{frame}
\begin{frame}
La más importante es la regla del trapecio, ya que se puede combinar con la extrapolación de Richardson, en un algoritmo eficiente llamado: \textcolor{blue}{Integración de Romberg}.
\end{frame}
\subsection{Regla del trapecio}
\begin{frame}
\frametitle{Regla del trapecio}
	\begin{center}
		\begin{tikzpicture}[font=\small,scale=0.6]
			\draw [->] (0,0) -- node [near end, left] {$f(x)$}(0,4);
			\draw [->] (0,0) -- (5,0);
			\draw [blue] (1,2.3) .. controls (2.5,2.8) ..  (4,2.8);
			\draw (1,2.3) -- (4,2.8);
			\draw [dashed] (1,0) -- (1,2.3);
			\draw [dashed] (4,0) -- (4,2.8);
			\draw [<->] (1,0.5) -- node [midway, above] {h} (4,0.5);
			\draw (0.7,-0.3) node {$x_{0}=a$};
			\draw (3.7,-0.3) node {$x_{1}=b$};
			\draw (2.2,1.7) node {Area=I};
			\draw (2,3) node {E};
			\draw [->] (2.3,2.9) -- (2.5,2.7);
		\end{tikzpicture}
	\end{center}
Si $n=1$ (un bloque), tenemos que $l_{0} = (x-x_{1})/(x_{0}-x_{1})= (x-b)/h$ por tanto:
\[A_{0} = \dfrac{1}{h} \int_{a}^{b} (x-b) dx = \dfrac{1}{2h} (b-a)^{2}= \dfrac{h}{2}\]
\end{frame}
\begin{frame}
Para $l_{1} = (x-x_{0})/(x_{1}-x_{0}) = (x-a)/h$ tenemos
\[A_{1} = \dfrac{1}{h} \int_{a}^{b} (x-a) dx = \dfrac{1}{2h} (b-a)^{2}= \dfrac{h}{2}\]
Sustituyendo:
\[I = \left[ f(a) + f(b) \right] \dfrac{h}{2}\]
Siendo la regla del trapecio. Representa el área del trapecio que se muestra en la figura anterior.
\end{frame}
\subsection{Error en la regla del trapecio}
\begin{frame}
\frametitle{Error en la regla del trapecio}
El error viene dado por
\[ E = \int_{a}^{b} f(x) dx - I \]
que es diferencia entre el área debajo de la curva de $f(x)$ y el la integral obtenida. 
\end{frame}
\begin{frame}
Integrando el error de interpolación:
\[ \begin{split} 
E =& \dfrac{1}{2!} \int_{a}^{b} (x-x_{0})(x-x_{1}) f''(\xi) dx  \\
=& \dfrac{1}{2} f''(\xi) \int_{a}^{b} (x-a)(x-b) dx = \\
=& -\dfrac{1}{12}(b-a)^{3} f''(\xi) \\
=& -\dfrac{h^{3}}{12}f''(\xi) \\
\end{split} \]
\end{frame}
\subsection{Regla extendida del trapecio}
\begin{frame}
\frametitle{Regla extendida del trapecio}
En la práctica la regla del trapecio se usa con una división en el dominio. La siguiente figura muestra la región $[a,b]$ dividida en $n$ bloques, de longitud $h$.
	\begin{center}
		\begin{tikzpicture}[font=\small]
			\draw [->] (0,0) -- node [near end, left]{$f(x)$} (0,3);
			\draw [->] (0,0) -- (7,0);
			\draw (7.4,0) node {x};
			\draw [blue] (1,0.5) .. controls(2.5,2) .. (6,0.4);
			\foreach \x in {1,1.5,...,6} \draw (\x,0) circle (0.03cm);
			\draw (1,-0.3) node {$x_{0}$};
			\draw (1,-0.7) node {a};
			\draw (2.5,-0.3) node {$x_{i}$};
			\draw (3,-0.3) node {$x_{i+1}$};
			\draw (5.5,-0.3) node {$x_{n-1}$};
			\draw (6,-0.3) node {$x_{n}$};
			\draw (6,-0.7) node {b};
			\draw [dashed] (1,0) -- (1,0.5);
			\draw [dashed] (1.5,0) -- (1.5,1.05);
			\draw [dashed] (2,0) -- (2,1.37);
			\draw [dashed] (2.5,0) -- (2.5,1.6);
			\draw [dashed] (3,0) -- (3,1.56);
			\draw [dashed] (3.5,0) -- (3.5,1.48);
			\draw [dashed] (4,0) -- (4,1.27);
			\draw [dashed] (4.5,0) -- (4.5,1.09);
			\draw [dashed] (5,0) -- (5,0.85);
			\draw [dashed] (5.5,0) -- (5.5,0.65);
			\draw [dashed] (6,0) -- (6,0.43);
			\draw [<->] (2.5,1) -- node [midway, below] {h} (3,1);
			\draw (1,0.5) -- (1.5,1.05) -- (2,1.37) -- (2.5,1.6) --(3,1.56) -- (3.5,1.48) -- (4,1.27) -- (4.5,1.09) -- (5,0.85) --(5.5,0.65) -- (6,0.43);
		\end{tikzpicture}
	\end{center}
\end{frame}
\begin{frame}
La función $f(x)$ se integrará con una aproximación lineal en cada panel. De la regla del trapecio, tenemos una que para el i-ésimo panel:
\[ I_{i} = [ f(x_{i}) + f(x_{i+1}) ] \dfrac{h}{2}\]
y como el área total, repre|sentada por la integral:
\fontsize{12}{12}\selectfont
\[ I \simeq \sum_{i=0}^{n-1} [f(x_{0}) + 2f(x_{1}) + 2f(x_{2}) + \ldots +2 f(x_{n-1}) + f(x_{n})] \dfrac{h}{2}\]
\fontsize{14}{14}\selectfont
que es la regla del extendida del trapecio.
\end{frame}
\subsection{Regla recursiva del trapecio}
\begin{frame}
\frametitle{Regla recursiva del trapecio}
Sea $I_{k}$ la integral evaluada con la regla compuesta del trapecio, usando $2^{k-1}$ bloques. Con la notación $H=b-a$,de la regla compuesta del trapecio, para $k=1,2,3$
\[
	\begin{split}
		k=1 \text{ (1 bloque) }: \\
		I_{1} = [f(a) + f(b)] \frac{H}{2}
	\end{split}
\]
\end{frame}
\begin{frame}
\[
	\begin{split}
		k=2 \text{ (2 bloques) }: \\
		I_{2} =& \left[ f(a) + 2f\left(a+\frac{H}{2} \right) + f(b) \right] \frac{H}{4} \\
		=& \frac{1}{2} I_{1} + f \left( a + \frac{H}{2} \right) \frac{H}{2}
	\end{split}
\]
\end{frame}
\begin{frame}
\[
	\begin{split}
		k=&3 \text{ (4 bloques) }: \\
		I_{3} =& \left[ f(a) + 2f \left(a + \frac{H}{4} \right) + 2 f \left( a + \frac{H}{2} \right) + \right.\\
		 +& \left. 2 f \left( a + \frac{3H}{4} \right)+ f(b) \right] \frac{H}{8} \\
		=& \frac{1}{2} I_{2} \left[f\left(a+\frac{H}{4} \right) + f\left(a+\frac{3H}{4}\right) \right] \frac{H}{4}
	\end{split}
\]
\end{frame}
\begin{frame}
\frametitle{Regla recursiva del trapecio}
Para un $k>1$ arbitrario, tenemos
\[ I_{k} = \dfrac{1}{2} I_{k-1} + \dfrac{H}{2^{k-1}} \sum_{i=1}^{2^{k-2}} f \left[ a + \dfrac{(2i-1)H}{2^{k-1}} \right], \hspace{0.6cm} k=2,3,\ldots\]
Otra forma de la misma ecuación es:
\[ I(h) = \dfrac{1}{2}I(2h) + h \sum f(x_{\text{nuevo}}) \]
\end{frame}
\begin{frame}
\frametitle{Ejercicio}
El cuerpo de revolución que se muestra en la figura, se obtiene al girar la curva dada por
\[ y= 1 + \left( \dfrac{x}{2} \right)^{2}, \hspace{0.5cm} 0 \leq x \leq 2\]
en torno al eje $x$. Calcula el volumen, usando la regla extendida del trapecio con $N=2,4,8,16,32,64,128$.
\\
\bigskip
El valor exacto es $I=11.7286$. Evalúa el error para cada $N$.
\end{frame}
\begin{frame}
	\begin{center}
		\begin{tikzpicture}[domain=0:2,font=\small, scale=1.4]
			\draw [<->] (0,-2) -- (0,2);
			\draw [->] (0,0) -- (3,0);
			\draw (1.7,2.3) node {$y=1+\left(\dfrac{x}{2}\right)^{2}$};
			\draw(3.4,0) node {$x$};
			\draw [color=red] plot(\x,{1+(\x/2)^2});
			\draw [color=red] plot(\x,{-(1+(\x/2)^2)});
			\draw [red] (2,0) ellipse [x radius=2,y radius=0.5, rotate=90];
		\end{tikzpicture}
	\end{center}
\end{frame}
\begin{frame}
\frametitle{Resolviendo el problema}
Hay que definir inicialmente la función que queremos integrar, por tanto
\[ I = \int_{a}^{b} f(x) dx\]
donde
\[ f(x) = \pi  \left( 1 + \left( \dfrac{x}{2} \right)^{2} \right)^{2}  \]
\end{frame}
\begin{frame}[fragile]
\begin{lstlisting}
def trapecios(f,a,b,n):
   h = (b-a)/float(n)
   x = a
   suma = 0
   for i in range(1,n):
      x = x + h
      suma = suma + funcion(x)
   return (h/2.)*(funcion(a) + funcion(b) + 2*suma)
\end{lstlisting}
\end{frame}
\begin{frame}
\begin{center}
	\begin{tabular}{r | l | l} 
		i & Integral & Error \\ \hline
		2 & 12.762720155 & 8.81708094e-02 \\ \hline
		4 & 11.989593838 & 2.22527700e-02 \\ \hline
		8 & 11.794011288 & 5.57707550e-03 \\ \hline
		16 & 11.744971839 & 1.39589033e-03 \\ \hline
		32 & 11.732702989 & 3.49827695e-04 \\ \hline
		64 & 11.729635215 & 8.82641387e-05 \\ \hline
		128 & 11.728868236 & 2.28702562e-05
	\end{tabular}
\end{center}
\end{frame}
\section{Librería Scipy}
\begin{frame}
\frametitle{Librería Scipy}
SciPy es un conjunto de algoritmos matemáticos y funciones de conveniencia construidos en la extensión NumPy para Python.
\\
\bigskip
Se agrega un poder significativo a la sesión interactiva de Python mediante el manejo por parte del  usuario con comandos de alto nivel y clases, para la manipulación y visualización de datos.
\end{frame}
\subsection{Organización de Scipy}
\begin{frame}
\frametitle{Organización de Scipy}
SciPy está organizada en sub-paquetes que cubren diferentes áreas de computación científica. Estos se resumen en la siguiente tabla:
\fontsize{12}{12}\selectfont
\begin{tabular}{l | l}
	Subpaquete	&	Descripción \\ \hline
	cluster		&	Algortimos para clusters \\ \hline
	constants	&	Constantes físicas y matemáticas \\ \hline
	fftpack 	&	Rutinas para la Transformada Rápida de Fourier \\ \hline
	integrate	&	Integración y EDO \\ \hline
	interpolate	&	Interpolación y uso de splines \\ \hline
	io			&	Rutinas de entrada y salida
\end{tabular}
\end{frame}
\begin{frame}
\fontsize{12}{12}\selectfont
\begin{tabular}{l | l}
	Subpaquete	&	Descripción \\ \hline
	linalg		&	Algebra lineal \\ \hline
	ndimage		&	Procesamiento N-dimensional de imagenes \\ \hline
	odr			&	Regresión de distancias ortogonales \\ \hline
	optimize	&	Optimizació y rutinas para encontrar raíces \\ \hline
	signal		&	Procesamiento de señales \\ \hline
	sparse		&	Matrices sparse y rutinas asociadas \\ \hline
	spatial		&	Estructura de datos espaciales \\ \hline
	special		&	Funciones especiales \\ \hline
	stats		&	Distribuciones estadísticas \\ \hline
	weave		&	Integración con C/C++
\end{tabular}
\end{frame}
\subsection{Integración (\texttt{scipy.integrate})}
\begin{frame}
\frametitle{Integración (\texttt{scipy.integrate})}
El subpaquete \texttt{scipy.integrate} proporciona varias técnicas de integración.
\fontsize{12}{12}\selectfont
	\begin{tabular}{l | p{8cm}}
	quad 		& Integración en general. \\ \hline
	dblquad 	& Integración doble en general. \\ \hline
	tplquad 	& Integración triple en general. \\ \hline
	fixed-quad 	& Integración de f(x) usando cuadraturas gaussianas de orden n. \\ \hline
	quadrature 	& Integra con tolerancia dada usando cuadratura gaussiana. \\ \hline
	romberg 	& Integra una función mediante la integración de Romberg.
\end{tabular}
\end{frame}
\begin{frame}
\frametitle{Funciones lambda}
El operador \texttt{lambda} sirve para crear funciones anónimas en línea. Al ser funciones anónimas, es decir, sin nombre, éstas no podrán ser referenciadas más tarde.
\\
\medskip
Las funciones \texttt{lambda} se construyen mediante el operador \texttt{lambda}, los parámetros de la función separados por comas (atención, SIN paréntesis), dos puntos (:) y el código de la función.
\end{frame}
\begin{frame}[fragile]
\frametitle{Ejemplo del uso de \texttt{integrate.quad}}
\begin{verbatim}
>>> from scipy import integrate
>>> x2 = lambda x: x**2
>>> integrate.quad(x2,0.,4.)
(21.333333333333336, 2.368475785867001e-13)
\end{verbatim}
\end{frame}
\begin{frame}[fragile]
\frametitle{El problema del volumen con \texttt{integrate.quad}}
Para comparar el resultado que nos da el módulo \texttt{scipy.integrate.quad}, veamos cómo implementar la solución del problema del sólido de revolución.
\begin{lstlisting}
from numpy import pi
from scipy.integrate import quad

def f(x):
    return pi*(1+(x/2)**2)**2
   
print quad(f,0,2)
\end{lstlisting}
\pause
El resultado que nos devuelve es: $(11.728612573401893, 1.302137572589889e-13)$.
\end{frame}
\begin{frame}[fragile]
El valor posterior al resultado de la integral es el error asociado al algoritmo que usa \texttt{integrate.quad}, para que no lo reporte en el resultado, basta con indicar que queremos sólo el primer elemento de la lista:
\\
\bigskip
\verb|print quad(f,0,2)[0]|
\end{frame}
\end{document}