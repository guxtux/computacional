\documentclass[11pt]{article}
\usepackage[utf8]{inputenc}
%\usepackage[latin1]{inputenc}
\usepackage[spanish,es-noshorthands]{babel}
\decimalpoint
\usepackage{anysize}
\usepackage{graphicx} 
\usepackage{amsmath}
\usepackage{booktabs}
\usepackage{tabulary}
\usepackage{nccmath}
\usepackage{float}
\usepackage{tikz}
\usetikzlibrary{patterns}
\usepackage{color}
\usepackage{listings}
\renewcommand{\arraystretch}{1.5}
\lstset{ %
language=Python,                % choose the language of the code
basicstyle=\normalsize,       % the size of the fonts that are used for the code
numbers=left,                   % where to put the line-numbers
numberstyle=\footnotesize,      % the size of the fonts that are used for the line-numbers
stepnumber=1,                   % the step between two line-numbers. If it is 1 each line will be numbered
numbersep=5pt,                  % how far the line-numbers are from the code
backgroundcolor=\color{white},  % choose the background color. You must add \usepackage{color}
showspaces=false,               % show spaces adding particular underscores
showstringspaces=false,         % underline spaces within strings
showtabs=false,                 % show tabs within strings adding particular underscores
frame=single,   		% adds a frame around the code
tabsize=4,  		% sets default tabsize to 2 spaces
captionpos=b,   		% sets the caption-position to bottom
breaklines=true,    	% sets automatic line breaking
breakatwhitespace=false,    % sets if automatic breaks should only happen at whitespace
escapeinside={\#}{)}          % if you want to add a comment within your code
}
\marginsize{1.5cm}{1.5cm}{0cm}{2cm}  
\title{C\'{o}digos del tema C\'{a}lculo de ra\'{i}ces. \\ Curso de F\'{i}sica Computacional}
\author{M. en C. Gustavo Contreras May\'{e}n}
\date{ }
\begin{document}
\maketitle
\fontsize{14}{14}\selectfont
\section{Incrementos sucesivos}
\begin{lstlisting}
def buscaraiz(f,a,b,dx):
    x1 = a; f1 = f(a)
    x2 = a + dx; f2 = f(x2)
    while f1*f2 > 0.0:
        if x1 >= b: return None
        x1 = x2; f1 = f2
        x2 = x1 + dx; f2 = f(x2)
    else:
        return x1,x2
\end{lstlisting}
\section{Bisecci\'{o}n}
\begin{lstlisting}
	def bisect(f,x1,x2,switch,epsilon=1.0e-9):
    f1 = f(x1)
    if f1 == 0.0: return x1
    f2 = f(x2)
    if f2 == 0.0: return x2
    
    if f1*f2 > 0.0: print 'La raiz no se ha identificado en un intervalo'

    n = ceil(log(abs(x2 - x1)/epsilon)/log(2.0))
	
	for i in np.arange(n):
        x3 = 0.5*(x1 + x2); f3 = f(x3)
        if (switch == 0) and (abs(f3) > abs(f1)) \
                        and (abs(f3) > abs(f2)):
            return None
        if f3 == 0.0: return x3
        if f2*f3 < 0.0:
            x1 = x3; f1 = f3
        else:
            x2 =x3; f2 = f3
    return (x1 + x2)/2.0
\end{lstlisting}
\subsection{Estrategia de soluci\'{o}n}
\begin{lstlisting}
def f(x): return x**3-10*x**2+5

a,b,dx = (-2.0,11.0,0.02)
	
print 'Intervalo (x1,x2)   raiz'
while 1:
    try:
        x1, x2 = buscaraiz(f,a,b,dx)
    except Exception, e:
        print e; break
    if x1 != None:
        a = x2
        root = bisect(f,x1,x2,0)
        if raiz != None: print '(%2.4f, %2.4f) %2.8f' %(x1, x2, raiz)
    else:
        print '\nHecho'
        break
\end{lstlisting}
\section{Newton-Raphson}
\begin{lstlisting}
def newtonRaphson(f,df,a,b,tol=1.0e-9):
    fa = (f(a)
    if fa == 0.0: return a
    fb = f(b)
    if f(b) == 0.0: return b
    if fa*fb > 0.0: print 'La raiz no esta en el intervalo'
    x = 0.5 * (a + b)
    
    for i in range(30):
        fx = f(x)
        if abs(fx) < tol: return x

        if fa*fx < 0.0:
            b = x
        else:
            a = x; fa = fx
    	dfx = df(x)
        
        try: dx = -fx/dfx
        except ZeroDivisionError: dx = b - a
        x =  x + dx
        
        if(b - x)*(x - a) < 0.0:
            dx = 0.5*(b-a)
            x = a + dx
        
        if abs(dx) < tol*max(abs(b),1.0): return x
    
    print 'Son demasiadas iteraciones'
\end{lstlisting}
%\section{Falsa posici\'{o}n}
%Aqu\'{i} va el c\'{o}digo para el m\'{e}todo de la falsa posici\'{o}n y falsa posici\'{o}n modificada.
%\section{M\'{e}todo de la secante}
%Aqu\'{i} va el m\'{e}todo de la secante.
\end{document}