\documentclass[12pt]{article}
\usepackage[utf8]{inputenc}
\usepackage[spanish]{babel}
\usepackage{amsmath}
\usepackage{amsthm}
\usepackage{booktabs}
\usepackage{tabulary}
\usepackage{nccmath}
\usepackage{anyfontsize}
\usepackage{anysize}
\decimalpoint
\marginsize{1.5cm}{1.5cm}{0cm}{1.5cm}
\author{M. en C. Gustavo Contreras Mayén.}
\title{\begin{center}
Examen Reposición Tema 5 \\ Curso Física Computacional
\end{center}}
\date{ }
\begin{document}
\maketitle
\fontsize{13}{13}\selectfont
Del tema de Ecuaciones Diferenciales Parabólicas, presentamos la solución con el mismo ejercicio de la barra metálica a temperatura uniforme, excepto en los extremos donde vale $0^{\circ}C$. Haciendo la misma referencia al problema, resuelve lo siguiente:
\begin{enumerate}
\item \textbf{Prueba de estabilidad}: Revisa que la temperatura diverge con el tiempo si la constante $C$ en la ecuación, se hace más grande que $0.5$.
\item \textbf{Dependencia del material}. Repite el cálculo pero ahora para el aluminio, $C=0.217$ cal/(g ${}^{\circ}$C), $K=0.49$ cal/(g ${}^{\circ}$C), $\rho = 2.7$ g/cc 
\\ 
\\
Considera que la condición de estabilidad necesita que cambies el tamaño del paso en la variable temporal.
\\
\\
¿A qué se refiere la indicación sobre la condición de estabilidad?
\\
\\
A partir del método de diferencias finitas, obtuvimos una expresión discreta:
\[ T(i,j+1)=T(i,j) + \dfrac{K \Delta t [T(i+1,j) + T(i-1,j) - 2T(i,j)]}{C \rho (\Delta x)^{2}} \]
donde $x = i \Delta x$ y $ t= t \Delta t$
\\
\\
Normalmente las EDP se resuelven transformándolas a ecuaciones en diferencias finitas y entonces, apoyarnos con la computadora para encontrar la solución numérica de esas ecuaciones en diferencias finitas. Así mismo, es posible encontrar soluciones analíticas de las EDP y comparar los resultados de la solución numérica contra las soluciones analíticas, aunque en problemas reales, o con condiciones de frontera más complejas, encontrar las soluciones analíticas es difícil o imposible. Lo que se hace en esos casos, es ajustar una solución analítica ''parecida'' a nuestro problema y contrastar las soluciones.
\\
\\
La ecuación de calor es un caso interesante ya que es posible obtener una solución analítica a partir de la ecuación en diferencias finitas:
\[  T(i,j) = A \left[  1 - 4 \dfrac{K}{C \rho} \dfrac{\Delta t}{(\Delta x)^{2}}  \sin^{2} \left( \dfrac{i \pi \Delta x}{2l} \right) \right]^{j} \sin \left( \dfrac{i \pi \Delta x}{l} \right) \]
\textbf{Nota: } Esta solución analítica a la ecuación de diferencias finitas NO es válida como solución de la EDP, pero nos ayuda a entender el algoritmo. Si comparamos la solución analítica de la EDP:
\[ T(x,t) = \sum_{n=1,3,\ldots}^{\infty} \dfrac{4T_{0}}{n \pi} \exp(-n^{2} \pi^{2}Kt/(L^{2} C \rho)) \sin \left( \dfrac{n \pi x}{L} \right)  \]
vemos que la solución decae exponencialmente con el tiempo, pero esto no se cumple para la solución numérica, a menos que:
\[ \dfrac{K}{C \rho} \dfrac{\Delta t}{(\Delta x)^{2}} \leq \dfrac{1}{4} \]
Si esta condición no se cumple, la solución numérica no decaerá en el tiempo y por tanto, estará mal, así mismo, nos dice que si hacemos un cambio más pequeño en el tiempo, mejoraremos la convergencia, pero si reducimos el valor en el paso de posición sin un incremento cuadrático simultáneo al paso de tiempo, la convergencia empeora; lo que hay que hacer es intentar con diferentes combinaciones de $\Delta x$ y $\Delta t$ hasta obtener una solución estable y razonable.
\item \textbf{Escala}: La curva de temperatura contra tiempo puede ser la misma para diferentes materiales, pero no en escala.
\\
\\
¿cuál de las dos barras anteriores se enfría más rápido?
\end{enumerate}
\end{document}