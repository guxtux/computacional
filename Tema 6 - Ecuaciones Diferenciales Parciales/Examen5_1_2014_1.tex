\documentclass[letterpaper]{article}
\usepackage[utf8]{inputenc}
%\usepackage[latin1]{inputenc}
\usepackage[spanish]{babel}
\usepackage{geometry}
\usepackage{anysize}
\usepackage{graphicx} 
\usepackage{amsmath}
\usepackage{tikz}
\usetikzlibrary{patterns}
\usetikzlibrary{decorations.markings}
\usetikzlibrary{matrix}
\usepackage{xy}
\usepackage{siunitx}
\usepackage[american,cuteinductors,smartlabels]{circuitikz}
\usetikzlibrary{calc}
\usepackage{color}
%\numberwithin{equation}{list}
\marginsize{1cm}{2cm}{0cm}{2cm}  
\title{Tarea Examen 1/3 - Ecuaciones diferenciales parciales \\ \begin{large}Curso de Física Computacional\end{large}}
\author{M. en C. Gustavo Contreras Mayén}
\date{ }
\begin{document}
\maketitle
\fontsize{14}{14}\selectfont
\spanishdecimal{.}
\begin{enumerate}
\item Desarrolla un esquema num\'{e}rico para resolver la ecuaci\'{o}n de Poisson
\[ \nabla^{2} \phi (r,\theta) = - \rho (r, \theta) / \epsilon_{0} \]
en coordenadas polares. Considera que la geometr\'{i}a en la frontera es un anillo circular con potenciales dados, para el radio interno es $\phi(a,\theta)$, y para el radio externo $\phi(b,\theta)$. Prueba tu soluci\'{o}n asignando valores de potencial al problema.
\\
\\
\textbf{Tips}
La ecuaci\'{o}n de Poisson en coordenadas polares resulta ser:
\[ \nabla^{2} \phi (r,\theta) = \dfrac{1}{r} \dfrac{\partial}{\partial r} \left( r \dfrac{\partial V}{\partial r} \right) + \dfrac{1}{r^{2}} \dfrac{\partial^{2} V}{\partial \theta^{2}} = - \dfrac{\rho(r,\theta)}{\epsilon_{0}} \]
donde $0 \leq r \leq R $ y $0 \leq \theta \leq 2 \pi$
\\
\\
Usando la malla:
\begin{eqnarray*}
r_{i} &=& i \Delta R \\
\theta &=& j \Delta \theta
\end{eqnarray*}
Se aproxima la ecuaci\'{o}n por
\[ \begin{split}
\dfrac{1}{r_{i}} \left( r_{i+\frac{1}{2}} \dfrac{V_{i+1,j} - V_{ij}}{\Delta r} - r_{j+\frac{1}{2}} \dfrac{V_{ij}-V_{i-1,j}}{\Delta r} \right) \dfrac{1}{\Delta r} + \\
+ \dfrac{1}{r^{2}} \dfrac{V_{i,j+1}-2V_{ij}+V_{i,j-1}}{\Delta \theta^{2}} = - \dfrac{\rho(r,\theta)}{\epsilon_{0}}
\end{split} \]
donde $V_{ij}$ y $r_{ij}$ son funciones
\[ (r_{i}, \theta_{j}) = (i \Delta r, j \Delta \theta) \]
Las funciones son peri\'{o}dicas de $j$ en la malla, con per\'{i}odo $j=\dfrac{2 \pi}{\Delta \theta}$  y $V_{ij}$ es independiente de $j$.
\end{enumerate}
\end{document}