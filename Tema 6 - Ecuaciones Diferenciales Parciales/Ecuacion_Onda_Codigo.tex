\documentclass[12pt]{article}
%\usepackage[latin1]{inputenc}
\usepackage[utf8x]{inputenc}
\usepackage[spanish]{babel}
\usepackage{amsmath}
\usepackage{amsthm}
\usepackage{graphicx}
\usepackage{tabularx}
\usepackage{float}
\usepackage{anysize}
\usepackage{listings} 
\lstset{ %
language=Python,                % choose the language of the code
basicstyle=\small,       % the size of the fonts that are used for the code
numbers=left,                   % where to put the line-numbers
numberstyle=\footnotesize,      % the size of the fonts that are used for the line-numbers
stepnumber=1,                   % the step between two line-numbers. If it is 1 each line will be numbered
numbersep=5pt,                  % how far the line-numbers are from the code
%backgroundcolor=\color{white},  % choose the background color. You must add \usepackage{color}
showspaces=false,               % show spaces adding particular underscores
showstringspaces=false,         % underline spaces within strings
showtabs=false,                 % show tabs within strings adding particular underscores
frame=single,   		% adds a frame around the code
tabsize=2,  		% sets default tabsize to 2 spaces
captionpos=b,   		% sets the caption-position to bottom
breaklines=true,    	% sets automatic line breaking
breakatwhitespace=false,    % sets if automatic breaks should only happen at whitespace
escapeinside={\#}          % if you want to add a comment within your code
}
\numberwithin{equation}{section}
\marginsize{1.5cm}{1.5cm}{0cm}{2cm}
\author{M. en C. Gustavo Contreras Mayén.}
\title{Código para resolver la ecuación de onda \\ \begin{Large}Curso de Fí­sica Computacional\end{Large}}
\date{ }
\begin{document}
\maketitle
\begin{lstlisting}

from visual import *
# Detalle de la cuerda

g = display(width=600,height=300, title = 'Cuerda oscilante')
vibst=curve(x=list(range(0,100)),color=color.yellow)
ball1=sphere(pos=(100,0),color=color.red,radius=2)
ball2=sphere(pos=(-100,0),color=color.red,radius=2)
ball1.pos
ball2.pos
vibst.radius=1.0


rho = 0.01              # Densidad de la cuerda
ten=40.                 # Tension de la cuerda
c=sqrt(ten/rho)         # Velocidad de propagacion
c1 = c                  # Criterio CFL
ratio = c*c/(c1*c1)

# Inicializacion
xi=zeros((101,3), dtype=float)          # 101 x’s y 3 t’s
for i in range(0,81): xi[i,0]= 0.00125*i

for i in range(81,101): xi[i,0]=0.1-0.005*(i-80) # IC

for i in range(0,100):
    vibst.x[i]=2.0*i-100.0      # asignado escala x
    vibst.y[i]=300.*xi[i,0]     # asignando escala y
vibst.pos                       # dibujando la cuerda

# Pasos posteriores de tiempo
for i in range(1,100): xi[i,1] = xi[i,0] + 0.5*ratio*(xi[i+1,0]+ xi[i-1,0]-2*xi[i,0])
while 1:
    rate(50)
    for i in range(1,100):
        xi[i,2]=2.*xi[i,1]- xi[i,0] + ratio*(xi[i+1,1]+ xi[i-1,1]-2*xi[i,1])
    for i in range(1,100):
        vibst.x[i] = 2.*i-100.0
        vibst.y[i] = 300.*xi[i,2]
    vibst.pos
    for i in range(0,101):
        xi[i,0] = xi[i,1]
        xi[i,1] = xi[i,2]
\end{lstlisting}

\end{document}