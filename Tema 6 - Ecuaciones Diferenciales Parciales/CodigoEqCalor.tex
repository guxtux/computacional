\documentclass[12pt]{article}
%\usepackage[latin1]{inputenc}
\usepackage[utf8x]{inputenc}
\usepackage[spanish]{babel}
\usepackage{amsmath}
\usepackage{amsthm}
\usepackage{graphicx}
\usepackage{tabularx}
\usepackage{color}
\usepackage{float}
\usepackage{anysize}
\usepackage{listings}
\lstset{ %
language=C++,                % choose the language of the code
basicstyle=\footnotesize,       % the size of the fonts that are used for the code
numbers=left,                   % where to put the line-numbers
numberstyle=\footnotesize,      % the size of the fonts that are used for the line-numbers
stepnumber=1,                   % the step between two line-numbers. If it is 1 each line will be numbered
numbersep=5pt,                  % how far the line-numbers are from the code
backgroundcolor=\color{white},  % choose the background color. You must add \usepackage{color}
showspaces=false,               % show spaces adding particular underscores
showstringspaces=false,         % underline spaces within strings
showtabs=false,                 % show tabs within strings adding particular underscores
frame=single,   		% adds a frame around the code
tabsize=2,  		% sets default tabsize to 2 spaces
captionpos=b,   		% sets the caption-position to bottom
breaklines=true,    	% sets automatic line breaking
breakatwhitespace=false,    % sets if automatic breaks should only happen at whitespace
escapeinside={\%}{)}          % if you want to add a comment within your code
}
\numberwithin{equation}{section}
\marginsize{1.5cm}{1.5cm}{-1cm}{2cm}
\author{M. en C. Gustavo Contreras Mayén.}
\title{EDP Parab\'{o}licas \\ \begin{Large}Curso de Fí­sica Computacional\end{Large}}
\date{ }
\begin{document}
\maketitle
\begin{lstlisting}
Program eqcalor
		Implicit None       
		Double Precision cons, ro, sph, thk, u(101,2)
		Integer i, k, max
		Open(9,FILE='eqcalor.dat',STATUS='UNKNOWN')

! se definen el calor especifico, la conductividad termica y la densidad del acero
		sph=0.113
		thk=0.12
		ro=7.8
		cons = thk/(sph*ro)

!numero de iteraciones
		max=30000

! en t=0 (i=1) todos los puntos estan an 100 C
		Do 10 i=1,100
			u(i,1) = 100.0
10		Continue

! excepto los extremos donde estan a cero grados
		Do 20 i=1,2
			u(1,i)   = 0.0
			u(101,i) = 0.0
20		Continue

! se inicia el loop para resolver la ecuacion
		Do 100 k=1,max
! loop sobre la posicion, los extremos quedan fijos
			Do 30 i=2,100
				u(i,2) = u(i,1) + cons*(u(i+1,1) + u(i-1,1)-2*u(i,1))
30			Continue

! calculamos las temperaturas cada 1000 pasos
			If((MOD(k,1000).eq.0).or.(k.eq.1)) Then
				Do 40 i=1,101,2
					Write(9,22)u(i,2)
40				Continue
				Write (9,22)
			EndIf
! los valores nuevos, ahora son "viejos"
			Do 50 i=2,100
				u(i,1) = u(i,2)
50			Continue
100		Continue
22		Format (f10.6)
		Close(9)
		Stop 'los datos se guardaron en eqcalor.dat'
End Program eqcalor
\end{lstlisting}
\end{document}