\documentclass[12pt]{article}
\usepackage[utf8]{inputenc}
\usepackage[spanish]{babel}
\usepackage{amsmath}
\usepackage{amsthm} 
\usepackage{anysize}
\marginsize{2cm}{2cm}{2cm}{2cm}  
\title{Ecuaciones en diferencias para la ecuación de Poisson}
\author{M. en C. Gustavo Contreras Mayén}
\date{Noviembre de 2008}
\begin{document}
\maketitle
Con la geometría y la retícula que aparecen en la figura, determinar las ecuaciones en diferencias para la ecuación de Poisson:
\begin{equation}\label{eq:Poisson}
- \nabla^{2}\phi=S
\end{equation}
Las condiciones de frontera son:
%\begin{eqnarray}
\begin{align}
\frac {\partial \phi}{\partial x} & = \phi \text{\hspace{1.5cm} para la frontera izquierda} \nonumber \\
\nonumber \\
\frac {\partial \phi}{\partial y} & = \phi - 2 \text{\hspace{0.7cm} para la frontera inferior}\nonumber \\
\nonumber \\
\phi & = 5 \text{\hspace{1.5cm} para la frontera derecha}\nonumber \\
\nonumber \\
\phi & = 7 \text{\hspace{1.5cm} para la frontera superior}\nonumber
\end{align}
%\end{eqnarray}
Los intervalos de la malla son unitarios en ambas direcciones.\\
\\
\textbf{Solución:}\\
Dado que las condiciones en las fronteras superior e inferior son del tipo de valor fijo, obtenemos las ecuaciones en diferencias sólo para los siguientes cuatro puntos de la malla: (1,1), (2,1), (1,2), (2,2).\\
\\
\textit{Punto (1,1)}. Aproximamos la derivada parcial con respecto a $x$ por:
\begin{eqnarray}\label{eq:par11x}
\left( \frac {\partial \phi^{2}}{\partial x^{2}}\right)_{1,1} & = & \dfrac{\left( \dfrac {\partial \phi}{\partial x}\right)_{1+\frac{1}{2},1} - \ \left( \dfrac {\partial \phi}{\partial x}\right)_{1,1}}{\dfrac{1}{2}}  \nonumber \\
& = & \dfrac {(\phi_{2,1}-\phi_{1,1})-\phi_{1,1}}{\dfrac{1}{2}} \nonumber \\
& = & -4\phi_{1,1}+2\phi_{2,1} %\tag*{(2)}
\end{eqnarray}
en donde utilizaremos la condición de frontera izquierda para eliminar $\left( \frac{\partial \phi}{\partial x}\right)_{1,1}$. La derivada parcial con respecto a $y$ se aproxima por:
\begin{eqnarray}\label{eq:par11y}
\left( \frac {\partial \phi^{2}}{\partial y^{2}}\right)_{1,1} & = & \dfrac{\left( \dfrac {\partial \phi}{\partial y}\right)_{1,1+\frac{1}{2}} - \ \left( \dfrac {\partial \phi}{\partial y}\right)_{1,1}}{\dfrac{1}{2}}  \nonumber \\
& = & \dfrac {(\phi_{1,2}-\phi_{1,1})-\phi_{1,1}-2}{\dfrac{1}{2}} \nonumber \\
& = & -4\phi_{1,1}+2\phi_{1,2}+4 %\tag*{(2)}
\end{eqnarray}
donde utilizamos la condición de frontera inferior para eliminar $\left( \frac{\partial \phi}{\partial y}\right)_{1,1}$. Sustituimos \ref{eq:par11x} y \ref{eq:par11y} en la ecuación de Poisson
\end{document}