\documentclass[12pt]{article}
\usepackage[utf8]{inputenc}
\usepackage[spanish]{babel}
\usepackage{amsmath}
\usepackage{amsthm}
\usepackage{multicol,multienum}
\usepackage{graphicx}
\usepackage{standalone}
\usepackage[outdir=../]{epstopdf}
\usepackage[binary-units=true]{siunitx}
\usepackage{float}
\DeclareGraphicsExtensions{.pdf,.png,.jpg}
\usepackage{tikz}
\usetikzlibrary{patterns}
\usetikzlibrary{decorations.pathmorphing,patterns}
\usetikzlibrary{arrows,calc,patterns,decorations.markings}
\usetikzlibrary{positioning}
\usepackage{color}
\usepackage{anysize}
\usepackage[spanish=mexican]{csquotes}
\usepackage{anyfontsize}
\usepackage[os=win]{menukeys}
\usepackage{pbox}
%Este paquete permite manejar los encabezados del documento
\usepackage{fancyhdr}
%hay que definir el ambiente de la página
\pagestyle{fancy}
%aqui va el texto para todas las paginas l--> izquierda, r--> derecha, hay un C--> para centrar el texto deseado
%\lhead{Curso de Física Computacional}
\fancyhead[R]{\nouppercase{\leftmark}}
%define el ancho de la linea que separa el encabezado del cuerpo del texto
\renewcommand{\headrulewidth}{0.5pt}
\setlength{\parskip}{1em}
\renewcommand{\baselinestretch}{1.25}
\newcommand{\python}{\texttt{python}}
\newcommand{\funcionazul}[1]{\textcolor{blue}{\textbf{\texttt{#1}}}}
\interfootnotelinepenalty=8000
\usepackage{hyperref}
%esta parte define el color del marco que aparece en las hiperreferencias.
\definecolor{links}{HTML}{2A1B81}
\hypersetup{colorlinks,linkcolor=,urlcolor=links}
\spanishdecimal{.}
\marginsize{1.5cm}{1.5cm}{1.5cm}{1.5cm}
\numberwithin{equation}{section}
\date{}
\title{Metodología de programación con \texttt{python} \\ \begin{Large}Curso de Física Computacional - Guía de apoyo 3 \end{Large}}
\author{M. en C. Gustavo Contreras Mayén.}
%\usepackage{listings}
\usepackage{xcolor}
\usepackage{textcomp}
\usepackage{color}
\definecolor{deepblue}{rgb}{0,0,0.5}
\definecolor{brown}{rgb}{0.59, 0.29, 0.0}
\definecolor{OliveGreen}{rgb}{0,0.25,0}
% \usepackage{minted}

% \DeclareCaptionFont{white}{\color{white}}
% \DeclareCaptionFormat{listing}{\colorbox{gray}{\parbox{0.98\textwidth}{#1#2#3}}}
% \captionsetup[lstlisting]{format=listing,labelfont=white,textfont=white}
\renewcommand{\lstlistingname}{Código}


\definecolor{Code}{rgb}{0,0,0}
\definecolor{Keywords}{rgb}{255,0,0}
\definecolor{Strings}{rgb}{255,0,255}
\definecolor{Comments}{rgb}{0,0,255}
\definecolor{Numbers}{rgb}{255,128,0}



\lstset{ 
language=Python,                % choose the language of the code
basicstyle=\normalsize\ttfamily,       % the size of the fonts that are used for the code
numbers=left,                   % where to put the line-numbers
numberstyle=\scriptsize,      % the size of the fonts that are used for the line-numbers
stepnumber=1,                   % the step between two line-numbers. If it is 1 each line will be numbered
numbersep=5pt,                  % how far the line-numbers are from the code
backgroundcolor=\color{white},  % choose the background color. You must add \usepackage{color}
showspaces=false,               % show spaces adding particular underscores
showstringspaces=false,         % underline spaces within strings
showtabs=false,                 % show tabs within strings adding particular underscores
frame=single,   		% adds a frame around the code
tabsize=2,  		% sets default tabsize to 2 spaces
captionpos=t,   		% sets the caption-position to bottom
breaklines=true,    	% sets automatic line breaking
breakatwhitespace=false,    % sets if automatic breaks should only happen at whitespace
escapeinside={\#},  % if you want to add a comment within your code
stringstyle =\color{OliveGreen},
%otherkeywords={{as}},             % Add keywords here
keywordstyle = \color{blue},
commentstyle = \color{black},
identifierstyle = \color{black},
literate=%
         {á}{{\'a}}1
         {é}{{\'e}}1
         {í}{{\'i}}1
         {ó}{{\'o}}1
         {ú}{{\'u}}1
%
%keywordstyle=\ttb\color{deepblue}
%fancyvrb = true,
}

\lstdefinestyle{FormattedNumber}{%
    literate={0}{{\textcolor{red}{0}}}{1}%
             {1}{{\textcolor{red}{1}}}{1}%
             {2}{{\textcolor{red}{2}}}{1}%
             {3}{{\textcolor{red}{3}}}{1}%
             {4}{{\textcolor{red}{4}}}{1}%
             {5}{{\textcolor{red}{5}}}{1}%
             {6}{{\textcolor{red}{6}}}{1}%
             {7}{{\textcolor{red}{7}}}{1}%
             {8}{{\textcolor{red}{8}}}{1}%
             {9}{{\textcolor{red}{9}}}{1}%
             {.0}{{\textcolor{red}{.0}}}{2}% Following is to ensure that only periods
             {.1}{{\textcolor{red}{.1}}}{2}% followed by a digit are changed.
             {.2}{{\textcolor{red}{.2}}}{2}%
             {.3}{{\textcolor{red}{.3}}}{2}%
             {.4}{{\textcolor{red}{.4}}}{2}%
             {.5}{{\textcolor{red}{.5}}}{2}%
             {.6}{{\textcolor{red}{.6}}}{2}%
             {.7}{{\textcolor{red}{.7}}}{2}%
             {.8}{{\textcolor{red}{.8}}}{2}%
             {.9}{{\textcolor{red}{.9}}}{2}%
             {\ }{{ }}{1}% handle the space
         ,%
          %mathescape=true
          escapeinside={__}
}

\begin{document}
\maketitle
\fontsize{14}{14}\selectfont
\section{Metodología de programación}
De manera paralela a los conceptos importantes del curso de Física Computacional, es necesario iniciar el trabajo de plantear algoritmos de solución a problemas de la física.
\par
La mejor manera de aprender, es sentarse a programar. Debe de tenerse la calma para ello, la idea es ir perfeccionando las propuestas de solución, es importante señalar que la inspiración divina, no se da siempre.
\par
A continuación se presenta una metodoĺogía de trabajo para la programación, como en todo proceso que se incorpora en nuestras actividades, nos daremos cuenta de que poco a poco cubrimos los puntos mencionados. Un buen trabajo en programación, contempla necesariamente todos los puntos indicados, quedarnos hasta el punto de programación, nos va a resolver la necesidad de contar con una solución numérica, pero no basta con ello, debemos de extender con los otros dos puntos, una metodología de trabajo consistente y completa.
\subsection{Proceso de programación}
El proceso de programación consta de las actividades necesarias para escribir programas que funcionen adecuadamente como solución a un problema particular. \footnote{Amparo López Gaona, \textit{Introducción al desarrollo de programas con Java}, 3a. Ed., La Prensa de Ciencias, México D.F. 2013.}
\begin{enumerate}
\item Definición del problema.
\item Diseño de la solución.
\item Codificación.
\item Depuración.
\item Mantenimiento.
\end{enumerate}
\subsection{Definición del problema}
Aquí se especifica qué es lo que debe de hacer el programa.
\par
Este primer paso puede parecer trivial aunque no lo es. La comprensión exacta de lo que se necesita hacer es requisito indispensable para crear una solución funcional.
\par
En ocasiones, se ignora esta fase y se comienza a escribir un programa sin tener en claro el problema a resolver.
\subsection{Diseño de la solución}
En esta fase se indica una forma de satisfacer, mediante un programa, los requerimientos establecidos en la etapa anterior.
\par
El diseño de un programa es un proceso al que muchas veces no se le da la importancia y de ahí que en las etapas posteriores se tengan muchos problemas.
\par
En el diseño es necesario identificar los principales componentes de la solución y la relación entre ellos.
\subsection{Codificación}
Una vez que se tiene el diseño de la solución, se procede a traducirlo a un lenguaje de programación.
\par
Esta tarea se conoce como codificación o implementación. En muchas ocasiones, uno se centra únicamente en esta etapa aunque, como se puede ver, el proceso de programar es mucho más complejo y creativo.
\par
Es recomendable acostumbrarse desde el inicio a escribir programas que sean fácilmente entendibles por otras personas; podemos apoyarnos con lo siguiente:
\begin{enumerate}
\item Los programas deben de tener una estructura clara.
\item El código debe estar organizado y presentado de manera que sea fácil su lectura.
\item El código debe de estar documentado.
\end{enumerate}
\subsection{Depuración}
El siguiente paso en el desarrollo de un programa es la depuración que consiste en verificar que el algoritmo y el programa sean adecuados. No importa que tan bonito esté el programa, si no produce los resultados deseados, simplemente no sirve.
\par
Depurar implica descubrir, localizar y corregir todos los errores que causen que un programa produzca resultados incorrectos o que no produzca ningún resultado.
\subsection{Mantenimiento}
En los programas y trabajos escolares, la tarea termina en el paso anterior, pero en la vida real no es así. 
\par
La etapa de mantenimiento consiste en supervisar la operación de un programa, corregir cualquier error encontrado durante su uso continuo o efectuar modificaciones al mismo, con el propósito de que realice más tareas o de manera diferente a las que tenían contempladas originalmente.
\end{document}