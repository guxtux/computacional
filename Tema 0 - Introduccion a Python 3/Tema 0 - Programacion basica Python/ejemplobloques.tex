\documentclass{beamer}

\mode<presentation> {
  \usetheme{Warsaw}
  \setbeamercovered{transparent}
}

\usepackage[spanish]{babel}
\usepackage[latin1]{inputenc}


\title[Ejemplo del uso de \textbf{Beamer}]{Ejemplo de la generación de presentaciones mediante \textbf{Beamer}}

\subtitle{Algunas diapositivas hechas con \textbf{Beamer}}

\author[Camilo Cubides, Ana María Rojas]
{%
Camilo Cubides\inst{1}\\[-2mm]
\texttt{\tiny eccubidesg@unal.edu.co}\\
    \and%
Ana María Rojas\inst{2}\\[-2mm]
\texttt{\tiny amrojasb@unal.edu.co}\\
}

\institute[Universidad Nacional de Colombia]
{%
  \inst{1}%
    Profesor\vspace{-2mm}
  \and
  \inst{2}%
    Monitora\vspace{-2mm}
}

\date{II semestre de 2006} %

\subject{Generación de presentaciones}

\pgfdeclareimage[height=0.8cm]{university-logo}{logo}
\logo{\pgfuseimage{university-logo}}


\AtBeginSection
{%
\begin{frame}<beamer>
\frametitle{Contenido}
\tableofcontents[currentsection,currentsubsection]
\end{frame}
}

\AtBeginSubsection
{%
\begin{frame}<beamer>
\frametitle{Contenido}
\tableofcontents[currentsection,currentsubsection]
\end{frame}
}

\beamerdefaultoverlayspecification{<+->}

\begin{document}

\begin{frame}
  \titlepage{}
\end{frame}


\begin{frame}
  \frametitle{Contenido}
  \tableofcontents[pausesections]
\end{frame}


\section{Primera sección}

\subsection{Primera subsección}

\begin{frame}
    \frametitle{Título de la primera diapositiva}

    \framesubtitle{Subtítulo de la primera diapositiva}

    \begin{itemize}
      \item  Primer item
      \item  Segundo item
      \item  Tercer item
    \end{itemize}

    \begin{enumerate}
      \item  Primer item
      \item  Segundo item
      \item  Tercer item
    \end{enumerate}

    \begin{description}
      \item[Primero] item 1
      \item[Segundo] item 2
      \item[Tercero] item 3
    \end{description}
\end{frame}


\subsection{Segunda subsección}

\begin{frame}
    \frametitle{Título de la segunda diapositiva}

    \begin{itemize}
      \item<1-> \alert{Primer} item
      \item<2-> \alert{Segundo} item
      \item<2-> \alert{Tercer} item
      \item<1-> Cuarto item
      \item<3-> Quinto item
    \end{itemize}
\end{frame}

\begin{frame}
    \frametitle{Comandos para la aparición de materiales}
    \only<1>{Esta línea se inserta solamente en la diapositiva 1}

    \only<2>{Esta línea se inserta solamente en la diapositiva 2}

    \uncover<3->{Este texto se muestra desde la diapositiva 3 en adelante}

    \visible<2->{Este texto se muestra desde la diapositiva 2 en adelante}

    \invisible<2->{Este texto se oculta desde la diapositiva 2 en adelante}

    \alt<2>{En la diapositiva 2}{En otra diapositiva}

    \temporal<2>{Estamos en 1}{Estamos en 2}{Estamos en 3, \ldots}
\end{frame}

\section{Segunda sección}

\subsection{Primera subsección de la segunda sección}

\begin{frame}
    \frametitle{Entornos I}

    \begin{definition}[definición]
        Entorno utilizado para presentar definiciones.
    \end{definition}

    \begin{example}
        Entorno para presentar ejemplos.
    \end{example}

    \begin{theorem}
        Entorno para presentar teoremas.
    \end{theorem}

    \begin{proof}
        Entorno para presentar pruebas.
    \end{proof}
\end{frame}


\begin{frame}
    \frametitle{Entornos II}

    \begin{block}{Título del bloque}
        Entorno para presentar bloques.
    \end{block}

    \begin{exampleblock}{Título del bloque de ejemplos}
        Entorno para presentar bloques de ejemplos.
    \end{exampleblock}

    \begin{alertblock}{Título del bloque de alerta}
        Entorno para presentar bloques de alerta.
    \end{alertblock}

    \begin{columns}
        \column{.3\textwidth}
            \begin{block}{Titulo del primer bloque}
                Bloque de la primera columna.
            \end{block}
        \column{.4\textwidth}
            \begin{block}{Titulo del segundo bloque}
                Bloque de la segunda columna.
            \end{block}
    \end{columns}
\end{frame}


\section{Tercera sección}

\begin{frame}
    línea 1\\\pause
    línea 2\\\pause
    línea 3\\
    línea 4\\\pause[2]
    línea 5\\
    línea 6\\\pause[4]
    línea 7\\\pause[2]
    línea 8\\\pause
\end{frame}

\end{document}

\usetheme{default}
\usetheme{JuanLesPins}
\usetheme{Goettingen}
\usetheme{Szeged}
\usetheme{Warsaw}

\usecolortheme{crane}

\usefonttheme{serif}
\usefonttheme{structuresmallcapsserif}