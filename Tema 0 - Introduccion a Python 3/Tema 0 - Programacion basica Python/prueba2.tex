\documentclass[12pt]{beamer}
\usepackage[utf8]{inputenc}
%\usepackage[latin1]{inputenc}
\usepackage[spanish]{babel}
%\usetheme{Warsaw}
%\usepackage{euler}
\usepackage{amsmath}
\usepackage{amsthm}
\usepackage{multicol}
\usepackage{graphicx}
\usepackage{tikz}
\usepackage{color}
\usepackage{listings}
\lstset{ %
language=Python,                % choose the language of the code
basicstyle=\footnotesize,       % the size of the fonts that are used for the code
numbers=left,                   % where to put the line-numbers
numberstyle=\footnotesize,      % the size of the fonts that are used for the line-numbers
stepnumber=1,                   % the step between two line-numbers. If it is 1 each line will be numbered
numbersep=5pt,                  % how far the line-numbers are from the code
backgroundcolor=\color{white},  % choose the background color. You must add \usepackage{color}
showspaces=false,               % show spaces adding particular underscores
showstringspaces=false,         % underline spaces within strings
showtabs=true,                 % show tabs within strings adding particular underscores
frame=single,   		% adds a frame around the code
tabsize=4,  		% sets default tabsize to 2 spaces
captionpos=b,   		% sets the caption-position to bottom
breaklines=true,    	% sets automatic line breaking
breakatwhitespace=false,    % sets if automatic breaks should only happen at whitespace
escapeinside={\%}{)}          % if you want to add a comment within your code
}
%\usepackage{epstopdf}
\DeclareGraphicsExtensions{.pdf,.png,.jpg}
\renewcommand {\arraystretch}{1.5}
\mode<presentation>
{
  \usetheme{CambridgeUS}
  \setbeamercovered{transparent}
  % or whatever (possibly just delete it)
}
\title{Tema 0 - Programaci\'{o}n b\'{a}sica con Python II}
\subtitle{Curso de F\'{i}sica Computacional}
\author{M. en C. Gustavo Contreras May\'{e}n}
\date{}
%\email{curso.fisica.comp@gmail.com}
%\ptsize{10}
\begin{document}
\maketitle
\fontsize{14}{14}\selectfont
\spanishdecimal{.}
\begin{frame}{Contenido}
\tableofcontents[pausesections]
\end{frame}
\section{Instrucciones de entrada y salida}
\subsection{Entrada de datos}
\begin{frame}[fragile]
\frametitle{Instrucciones de entrada y salida}
Entrada de datos:
\begin{itemize}
\item \verb|raw_input("entrada")|: lee una l\'{i}nea de entrada que es convertida a \texttt{string}.
\item \verb|eval(string)| : convierte \texttt{string} en un valor num\'{e}rico.
\end{itemize}
\fontsize{10}{10}\selectfont
\begin{minipage}{5.5cm}
\begin{exampleblock}{}<1->
	\verb|>>> a = raw_input("Ingrese a: ")| \\
	\pause
	\textcolor{blue}{\texttt{Ingrese a: 2}}
\end{exampleblock}
\begin{exampleblock}{}<2->
	\verb|>>> print a| \\
	\pause
	\textcolor{blue}{2}
\end{exampleblock}
\begin{exampleblock}{}<3->
	\verb|>>> a| \\
	\pause
	\textcolor{blue}{'a'}
\end{exampleblock}
\begin{exampleblock}{}<4->
	\verb|>>> type(a)| \\
	\pause
	\textcolor{blue}{\texttt{<type 'str'>}}
\end{exampleblock}
\end{minipage}
\hspace{0.5cm}
\begin{minipage}{5.5cm}
\begin{exampleblock}{}<5->
	\verb|>>> b = eval(a)|
\end{exampleblock}
\end{minipage}
\end{frame}
\begin{frame}[fragile]
\begin{minipage}{9cm}
\begin{exampleblock}{}<1->
	\verb|>>> print b, type(b)| \\
	\pause
	\textcolor{blue}{\texttt{2 <type 'int'>}}
\end{exampleblock}
\begin{exampleblock}{}<2->
	\verb|>>> s=eval(raw_input("Ingrese s :"))| \\
	\pause
	\textcolor{blue}{\texttt{Ingrese s: 2*3}}
\end{exampleblock}
\begin{exampleblock}{}<3->
	\verb|>>> print s, type(s)| \\
	\pause
	\textcolor{blue}{\texttt{6 <type 'int'>}}
\end{exampleblock}
\begin{exampleblock}{}<4->
	\verb|>>> m=eval(raw_input("Ingrese m :"))| \\
	\pause
	\textcolor{blue}{\texttt{Ingrese m: hola}} \\
	\pause
	\textcolor{red}{Marca un error, por qué?}
\end{exampleblock}
\end{minipage}
\end{frame}
\subsection{Salida de datos}
\begin{frame}
\frametitle{Salida de datos}
La mayor\'{i}a de programas requiere mostrar un resultado, en ocasiones a la pantalla y en otras, como un conjunto de datos que se enviar\'{a}n a un archivo.
\\
\bigskip
Para facilitar la lectura del resultado conviene elegir el respectivo formato de salida.
\begin{itemize}
\item \texttt{objeto1, objeto2, $\ldots$}
\item $\%$formato1, $\%$formato2, $\ldots$, $\%$tupla
\end{itemize}
\begin{center}
\begin{tabular}{l | c}
\hline Entero & \texttt{d} \\
\hline Punto flotante & \texttt{f} \\
\hline Notaci\'{o}n cient\'{i}fica & e
\end{tabular}
\end{center}
\end{frame}
\begin{frame}[fragile]
\frametitle{Formato de salida de datos}
\fontsize{12}{12}\selectfont
\begin{exampleblock}{}<1->
	\verb|>>> u=6543|
\end{exampleblock}
\begin{exampleblock}{}<2->
	\verb|>>> v=1234.56789|
\end{exampleblock}
\begin{exampleblock}{}<3->
	\verb|>>> print u, v| \\
	\pause
	\textcolor{blue}{\texttt{6543}  \texttt{1234.56789}}
\end{exampleblock}
\begin{exampleblock}{}<4->
	\verb|>>> print "u = %6d" % u| \\
	\pause
	\textcolor{blue}{u=6543}
\end{exampleblock}
\begin{exampleblock}{}<5->
	\verb|>>> print "u = %06d" % u| \\
	\pause
	\textcolor{blue}{\texttt{u=006543}}
\end{exampleblock}
\end{frame}
\begin{frame}[fragile]
\frametitle{Formato de salida de datos}
\fontsize{12}{12}\selectfont
\begin{exampleblock}{}<1->
	\verb|>>> print "v= %7.2f" % v| \\
	\pause
	\textcolor{blue}{\texttt{v=1234.57}}
\end{exampleblock}
\begin{exampleblock}{}<2->
	\verb|>>> print "v= %9.2f" % v| \\
	\pause
	\textcolor{blue}{\texttt{$v=\hspace{0.5cm}1234.57$}}
\end{exampleblock}
\begin{exampleblock}{}<3->
	\verb|>>> print "v= %7.8f" % v| \\
	\pause
	\textcolor{blue}{\texttt{v=1234.56789000}}
\end{exampleblock}
\begin{exampleblock}{}<4->
	\verb|>>> print "v= %.2e" % v| \\
	\pause
	\textcolor{blue}{\texttt{v=1.23e+03}}
\end{exampleblock}
\begin{exampleblock}{}<1->
	\verb|>>> print "u= %6d y v=%8.4e" %(u, v)| \\
	\pause
	\textcolor{blue}{\texttt{u = 6543 y v=1.2346e+03}}
\end{exampleblock}
\end{frame}
\begin{frame}[fragile]
\frametitle{Ejercicio}
A continuaci\'{o}n se presenta el c\'{o}digo que calcula el promedio de dos n\'{u}meros que se proporcionan por el usuario.
\begin{exampleblock}{C\'{o}digo}
\fontsize{10}{10}\selectfont
\verb|A = eval(raw_input('Ingresa A: '))| \\
\verb|Ingresa A: 10| \\
\verb|B = eval(raw_input('Ingresa B: '))| \\
\verb|Ingresa B: 20| \\
\verb|PROM = (A+B)/2.0| \\
\verb|print 'El promedio de %f y %f es %f' % (A,B,PROM)|
\end{exampleblock}
\end{frame}
\section{Estructuras de control}
\begin{frame}
\frametitle{Estructuras de control}
En cualquier lenguaje de programaci\'{o}n se incluye una serie de estructuras de control para ampliar las posibilidades de ejecuci\'{o}n de un programa.
\\
\bigskip
En Python, manejaremos las m\'{a}s comunes, que son relativamente sencillas de usar, cuidado siempre la sintaxis respectiva.
\end{frame}
\subsection{Condicionales}
\begin{frame}
\frametitle{Condicionales}
Una sentencia condicional permite evaluar si se cumple cierta condici\'{o}n, es decir, si su valor es \texttt{True}, se ejecuta una instrucci\'{o}n, en caso de que el valor de la condici\'{o}n no se cumpla, valor \texttt{False}, no se ejecuta la instrucci\'{o}n contenida y se sigue a la siguiente l\'{i}nea de c\'{o}digo.
\end{frame}
\begin{frame}[fragile]
\frametitle{Ejemplo de condicional}
\begin{lstlisting}
a = input('Introduce el valor de a')
if a > 0:
    print "a es positivo"
    a = a + 1
elif a == 0: 
    print "a es 0"
else:
    print "a es negativo"
\end{lstlisting}
\end{frame}
\subsection{Bucles o Loops}
\begin{frame}
\frametitle{Bucles}
Un bucle es una sentencia que eval\'{u}a inicialmente una condici\'{o}n, en caso de que se cumple (valor \texttt{True}) se ejecuta(a) un conjunto de instrucciones, posteriormente, se revisa el valor de la condici\'{o}n, mientras sea verdadero, las instrucciones se ejecutan nuevamente.
\\
\bigskip
Hay que considerar que se puede conocer de antemano, el n\'{u}mero de repeticiones, hay que evitar los bucles infinitos, es decir, sentencias que no modifican el valor de la condici\'{o}n y por tanto, siempre se mantendr\'{a} sin salir del bucle.
\end{frame}
\begin{frame}[fragile]
\begin{lstlisting}
nMax = 5
n = 1
a = [] # Crea una lista vacia
while n < nMax:
    a.append(1.0/n) # agrega un elemento a la lista
    n = n + 1
print a
\end{lstlisting}
\end{frame}
\subsection{Sentancia \texttt{for}}
\begin{frame}[fragile]
\frametitle{Sentencia \texttt{for}}
\begin{lstlisting}
lista = ['Hugo', 'Paco', 'Luis', 'McPato']
nombre = eval(raw_input('Teclea un nombre: '))
for i in range(len(lista)):
    if lista[i] == nombre:
    print nombre, ' es el numero ', i + 1, ' en la lista'
    break
else:
    print nombre, ' no esta en la lista'
\end{lstlisting}
\end{frame}
\section{Control de errores}
\begin{frame}
\frametitle{Control de errores}
Cuando comenzamos a programar, nos podemos encontrar con mensajes de error al momento de ejecutar el programa, siendo las causas m\'{a}s comunes:
\begin{itemize}
\item errores de dedo, escribiendo incorrectamente una instrucci\'{o}n, sentencia, variable o constante.
\item errores al momento de introducir los datos, por ejemplo, si el valor que se debe de ingresar es $123.45$, y si nosotros tecleamos $1234.5$, el resultado ya se considera un error.
\item errores que se muestran en tiempo de ejecuci\'{o}n, es decir, todo est\'{a} bien escrito y los datos est\'{a}n bien introducidos, pero hay un error debido a la l\'{o}gica del programa o del m\'{e}todo utilizado, ejemplo: divisi\'{o}n entre cero.
\end{itemize}
\end{frame}
\begin{frame}[fragile]
\frametitle{Manejo de errores}
\verb|c = 12.0/0.0| \\
\pause
\begin{exampleblock}{}
\verb|Traceback (most recent call last):| \\
\verb|File ''<pyshell#0>'', line 1, in ?| \\
\verb|c = 12.0/0.0| \\
\verb|ZeroDivisionError: float division|
\end{exampleblock}
\pause
\begin{exampleblock}{}
\verb|try:| \\
\verb|     c = 12.0/0.0| \\
\verb|except ZeroDivisionError:| \\
\verb|     print 'Division entre cero'|
\end{exampleblock}
\end{frame}
\section{Funciones}
\begin{frame}
\frametitle{Funciones}
Con lo que hemos revisado sobre Python, tenemos elementos para iniciar la soluci\'{o}n de problemas, una particular manera de agrupar un conjunto de instrucciones, es a trav\'{e}s de funciones.
\\
\bigskip
Las funciones intr\'{i}nsecas de cualquier lenguaje son pocas, pero podemos extenderlas con funciones definidas por el usuario.
\end{frame}
\begin{frame}[fragile]
\frametitle{Estructura de una funci\'{o}n}
La estructura de una funci\'{o}n en Python es la siguiente:
\begin{center}
\begin{exampleblock}{}
\verb| def nombre_funcion(parametro1, parametro2, ...):|
\verb|     conjunto de instrucciones|
\verb|     return valores_devueltos|
\end{exampleblock}
\end{center}
donde parametro1, parametro2 son los par\'{a}metros. Un par\'{a}metro puede ser cualquier objeto de Python, incluyendo una funci\'{o}n.
\\
\medskip
Los par\'{a}metros pueden darse por defecto, por lo que en la funci\'{o}n son opcionales. Si no se utiliza la instrucci\'{o}n \texttt{return}, la funci\'{o}n devuelve un objeto \texttt{null}
\end{frame}
\begin{frame}[fragile]
\frametitle{Ejemplo}
\begin{lstlisting}
>>> def cuadrados(a):
     for i in range(len(a)):
         a[i] = a[i]**2


>>> a = [1, 2, 3, 4]
>>> cuadrados(a)
>>> print a
\end{lstlisting}
\end{frame}
\begin{frame}
\frametitle{C\'{a}lculo de la serie de Fibonacci}
La sucesi\'{o}n fue descrita por Fibonacci como la soluci\'{o}n a un problema de la cr\'{i}a de conejos: ''Cierto hombre ten\'{i}a una pareja de conejos juntos en un lugar cerrado y uno desea saber cu\'{a}nntos son creados a partir de este par en un año cuando es su naturaleza parir otro par en un simple mes, y en el segundo mes los nacidos parir tambi\'{e}n"
\\
\bigskip
\pause
\textcolor{red}{c\'{o}mo le hacemos?}
\end{frame}
\begin{frame}[fragile]
\frametitle{Propuesta de c\'{o}digo}
\begin{lstlisting}
a, b= 0,1
while b < 10:
    print b
    a, b = b, a+b
\end{lstlisting}
\pause
\begin{lstlisting}
a, b= 0,1
while b < 1000:
    print b,
    a, b = b, a+b
\end{lstlisting}
\end{frame}
\section{M\'{o}dulos}
\begin{frame}[fragile]
\frametitle{M\'{o}dulos}
Es una buena pr\'{a}ctica almacenar las funciones en m\'{o}dulos. Un m\'{o}dulo es un archivo en donde se dejan las funciones, el nombre del m\'{o}dulo es el nombre del archivo.
\\
\bigskip
Un m\'{o}dulo se carga al programa con la instrucci\'{o}n
\begin{center}
\verb|from nombre_modulo import *|
\end{center}
Python incluye un n\'{u}mero grande de m\'{o}dulos que contienen funciones y m\'{e}todos para varias tareas. La gran ventaja de los m\'{o}dulos es que est\'{a}n disponibles en internet y se pueden descargar, dependiendo de la tarea que se requiera atender.
\end{frame}
\begin{frame}[fragile]
\frametitle{M\'{o}dulo \texttt{math}}
Muchas funciones matem\'{a}icas no se pueden llamar directo del int\'{e}rprete, pero para ello existe el m\'{o}dulo \texttt{math}.
\\
\bigskip
Hay tres diferentes maneras en las que se puede llamar y utilizar las funciones de un m\'{o}dulo.
\begin{exampleblock}{}
\verb|from math import *|
\end{exampleblock}
De esta manera, se importan todas las funciones definidas en el m\'{o}dulo \texttt{math}, siendo quiz\'{a} un gasto innecesario de recursos, pero tambi\'{e}n generar conflictos con definiciones cargadas de otros m\'{o}dulos.
\end{frame}
\begin{frame}[fragile]
\begin{exampleblock}{}
\verb|from math import func1, func2,...|
\end{exampleblock}
\pause
\begin{exampleblock}{}
\verb|>>> from math import log,sin| \\
\verb|>>> print log(sin(0.5))| \\
\verb|-0.735166686385|
\end{exampleblock}
\end{frame}
\begin{frame}[fragile]
El tercer m\'{e}todo que es el m\'{a}s usado en programaci\'{o}n, es tener disponible el m\'{o}dulo:
\begin{center}
\verb|import math|
\end{center}
Las funciones en el m\'{o}dulo se pueden usar con el nombre del m\'{o}dulo como prefijo:
\begin{exampleblock}{}
\verb|>>> import math| \\
\verb|>>> print math.log(math.sin(0.5))|
\verb|-0.735166686385|
\end{exampleblock}
Podemos ver el contenido de un m\'{o}dulo con la instrucci\'{o}n:
\\
\verb|>>> import math| \\
\verb|>>> dir(math)|
\end{frame}
\end{document}