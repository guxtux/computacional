\documentclass[12pt]{beamer}
\newenvironment{ConCodigo}[1]
  {\begin{frame}[fragile,environment=ConCodigo]{#1}}
  {\end{frame}}
\graphicspath{{Imagenes/}{../Imagenes/}}
\usepackage[utf8]{inputenc}
\usepackage[spanish]{babel}
\usepackage{hyperref}
\usepackage{etex}
\reserveinserts{28}
\usepackage{amsmath}
\usepackage{amsthm}
\usepackage{mathtools}
\usepackage{multicol}
\usepackage{multirow}
\usepackage{tabulary}
\usepackage{booktabs}
\usepackage{nccmath}
\usepackage{biblatex}
\usepackage{epstopdf}
\usepackage{graphicx}
%\usepackage{enumitem,xcolor}
\usepackage{siunitx}
\sisetup{scientific-notation=true}
%\usepackage{fontspec}
\usepackage{lmodern}
\usepackage{float}
\usepackage[format=hang, font=footnotesize, labelformat=parens]{caption}
\usepackage[autostyle,spanish=mexican]{csquotes}
\usepackage{standalone}
\usepackage{blkarray}
\usepackage{algorithm}
\usepackage{algorithmic}
\usepackage{tikz}
\usepackage[siunitx]{circuitikz}
\usetikzlibrary{arrows,patterns,shapes}
\usetikzlibrary{decorations.markings}
\usetikzlibrary{arrows}
\usepackage{color}
\usepackage{xcolor}
%\usepackage{beton}
%\usepackage{euler}
%\usepackage[T1]{fontenc}
\usepackage[sfdefault]{roboto}  %% Option 'sfdefault' only if the base font of the document is to be sans serif
\usepackage[T1]{fontenc}
\renewcommand*\familydefault{\sfdefault}
\DeclareGraphicsExtensions{.pdf,.png,.jpg}
\usepackage{hyperref}
\renewcommand {\arraystretch}{1.5}
\newcommand{\python}{\texttt{python}}
\usefonttheme[onlymath]{serif}
\setbeamertemplate{navigation symbols}{}
\usetikzlibrary{patterns}
\usetikzlibrary{decorations.markings}
\tikzstyle{every picture}+=[remember picture,baseline]
%\tikzstyle{every node}+=[inner sep=0pt,anchor=base,
%minimum width=2.2cm,align=center,text depth=.15ex,outer sep=1.5pt]
%\tikzstyle{every path}+=[thick, rounded corners]
\setbeamertemplate{caption}[numbered]
\newcommand{\ptm}{\fontfamily{ptm}\selectfont}
%Se usa la plantilla Warsaw modificada con spruce
\mode<presentation>
{
  \usetheme{Warsaw}
  \setbeamertemplate{headline}{}
  \useoutertheme{default}
  \usecolortheme{seahorse}
  \setbeamercovered{invisible}
}
% \AtBeginSection[]
% {
% \begin{frame}<beamer>{Contenido}
% \normalfont\mdseries
% \tableofcontents[currentsection]
% \end{frame}
%}

\usepackage{listings}
\lstset{ %
language=Python,                % choose the language of the code
basicstyle=\small,       % the size of the fonts that are used for the code
numbers=left,                   % where to put the line-numbers
numberstyle=\footnotesize,      % the size of the fonts that are used for the line-numbers
stepnumber=1,                   % the step between two line-numbers. If it is 1 each line will be numbered
numbersep=5pt,                  % how far the line-numbers are from the code
backgroundcolor=\color{white},  % choose the background color. You must add \usepackage{color}
showspaces=false,               % show spaces adding particular underscores
showstringspaces=false,         % underline spaces within strings
showtabs=false,                 % show tabs within strings adding particular underscores
frame=single,   		% adds a frame around the code
tabsize=4,  		% sets default tabsize to 2 spaces
captionpos=b,   		% sets the caption-position to bottom
breaklines=true,    	% sets automatic line breaking
breakatwhitespace=false,    % sets if automatic breaks should only happen at whitespace
escapeinside={\#}{)}          % if you want to add a comment within your code
}

\title{Tema 0 - Programación básica con Python IV}
\subtitle{Curso de Física Computacional}
\author[]{M. en C. Gustavo Contreras Mayén}
\date{}
\begin{document}
\maketitle
\fontsize{12}{12}\selectfont
\spanishdecimal{.}
\begin{frame}{Contenido}
\tableofcontents[pausesections]
\end{frame}
\section{Operación entre arreglos}
\subsection{Productos entre arreglos}
\begin{frame}
\frametitle{Productos entre arreglos}
Recordemos que vector es sinónimo de arreglo de una dimensión, y matriz es sinónimo de arreglo de dos dimensiones.
\end{frame}
\begin{frame}[fragile]
\frametitle{Producto interno (vector-vector)}
El producto interno entre dos vectores se obtiene usando la función \texttt{dot} provista por NumPy:
\fontsize{12}{12}\selectfont
\begin{exampleblock}{}
\verb|>>> a = array([-2.8 , -0.88,  2.76,  1.3 ,  4.43])| \\
\verb|>>> b = array([ 0.25, -1.58,  1.32, -0.34, -4.22])| \\
\pause
\verb|>>> dot(a, b)| \\
\pause
\verb|-14.803|
\end{exampleblock}
\end{frame}
\begin{frame}[fragile]
El producto interno es una operación muy común. Por ejemplo, suele usarse para calcular totales:
\fontsize{12}{12}\selectfont
\begin{exampleblock}{}
\verb|>>> precios = array([200, 100, 500, 400, 400, 150])|
\verb|>>> cantidades = array([1, 0, 0, 2, 1, 0])|
\pause
\verb|>>> total_a_pagar = dot(precios, cantidades)|
\pause
\verb|>>> total_a_pagar|
\pause
\verb|1400|
\end{exampleblock}
También se usa para calcular promedios ponderados:
\begin{exampleblock}{}
\verb|>>> notas = array([45, 98, 32])| \\
\pause
\verb|>>> ponderaciones = array([30, 30, 40]) / 100.| \\
\pause
\verb|>>> nota_final = dot(notas, ponderaciones)| \\
\pause
\verb|>>> nota_final| \\
\verb|55.7|
\end{exampleblock}
\end{frame}
\subsection{Producto matriz-arreglo}
\begin{frame}[fragile]
\frametitle{Producto matriz-vector}
El producto matriz-vector es el vector de los productos internos. El producto matriz-vector puede ser visto simplemente como varios productos internos calculados de una sola vez.
\\
\bigskip
Esta operación también es obtenida usando la función \texttt{dot} entre las filas de la matriz y el vector:
\fontsize{12}{12}\selectfont
\begin{exampleblock}{}
\verb|>>> a = array([[-0.6,  4.8, -1.2],| \\
\verb|               [-2. , -3.6, -2.1],| \\
\verb|               [ 1.7,  4.9,  0. ]])| \\
\verb|>>> x = array([-0.6, -2. ,  1.7])| \\
\pause
\verb|>>> dot(a, x)| \\
\pause
\verb|array([-11.28,   4.83, -10.82])|
\end{exampleblock}
\end{frame}
\subsection{Producto matriz-matriz}
\begin{frame}[fragile]
\frametitle{Producto matriz-matriz}
\fontsize{12}{12}\selectfont
El producto matriz-matriz es la matriz de los productos internos entre las filas de la primera matriz y las columnas de la segunda.
\begin{exampleblock}{}
\verb|>>> a = array([[ 2,  8],| \\
\verb|               [-3,  7],| \\
\verb|               [-8, -5]])| \\
\verb|>>> b array([[-3, -5, -6, -3],| \\
\verb|             [-9, -2,  3, -3]])| \\
\pause
\verb|>>> dot(a, b)| \\
\pause
\verb|array([[-78, -26,  12, -30],|\\
\verb|       [-54,   1,  39, -12],| \\
\verb|       [ 69,  50,  33,  39]])| \\
\end{exampleblock}
\end{frame}
\begin{frame}
La multiplicación de matrices puede ser vista como varios productos matriz-vector (usando como vectores todas las filas de la segunda matriz), calculados de una sola vez.
\end{frame}
\begin{frame}
En resumen, al usar la función \texttt{dot}, la estructura del resultado depende de cuáles son los parámetros pasados:
\begin{enumerate}[<+->]
\item \texttt{dot}(vector, vector) $\rightarrow$ número.
\item \texttt{dot}(matriz, vector) $\rightarrow$ vector.
\item \texttt{dot}(matriz, matriz) $\rightarrow$ matriz.
\end{enumerate}
\end{frame}
\section{Solución de sistemas de ecuaciones}
\begin{frame}
\frametitle{Solución de sistemas lineales}
\fontsize{12}{12}\selectfont
Un problema recurrente en Ciencias consiste en obtener cuál es el vector x cuando A y b son dados:
\[Ax=b\]
La ecuación matricial $Ax=b$ es una manera abreviada de expresar un sistema de ecuaciones lineales. Por ejemplo, la ecuación del diagrama es equivalente al siguiente sistema de tres ecuaciones que tiene las tres incógnitas $w$, $y$ y $z$:
\begin{eqnarray*}
36w+51y+13z &=& 3 \\
52w+34y+74z &=& 45 \\
7y+1.1z &=& 33
\end{eqnarray*}
\end{frame}
\begin{frame}
Este sistema se representa matricialmente:
\[ \begin{bmatrix}
36 & 51 & 13 \\
52 & 34 & 74 \\
 & 7 &1.1
\end{bmatrix}
\begin{bmatrix}
w \\
y \\
z
\end{bmatrix} =
\begin{bmatrix}
3 \\
45 \\
33
\end{bmatrix}
\]
\end{frame}
\begin{frame}[fragile]
La teoría detrás de la solución de problemas de este tipo, se puede consultar en cualquier texto de álgebra lineal. Sin embargo, como este tipo de problemas aparece a menudo en la práctica, aprenderemos cómo obtener rápidamente la solución usando Python.
\\
\bigskip
Dentro de los varios módulos incluidos en NumPy, está el módulo \texttt{numpy.linalg}, que provee algunas funciones que implementan algoritmos de álgebra lineal. Dentro de este módulo está la función \texttt{solve}, que entrega la solución $x$ de un sistema a partir de la matriz $A$ y el vector $b$:
\end{frame}
\begin{frame}[fragile]
\fontsize{12}{12}\selectfont
\begin{exampleblock}{}
\verb|>>> a = array([[ 36. ,  51. ,  13. ],| \\
\verb|...            [ 52. ,  34. ,  74. ],| \\
\verb|...            [  0. ,   7. ,   1.1]])| \\
\pause
\verb|>>> b = array([  3.,  45.,  33.])| \\
\pause
\verb|>>> x = solve(a, b)| \\
\pause
\verb|>>> x| \\
\pause
\verb|array([-7.10829222,  4.13213834,  3.70457422])|
\end{exampleblock}
Podemos ver que el vector $x$ en efecto satisface la ecuación $Ax = b$:
\begin{exampleblock}{}
\verb|>>> dot(a, x)| \\
\pause
\verb|array([  3.,  45.,  33.])| \\
\pause
\verb|>>> b| \\
\pause
\verb|array([  3.,  45.,  33.])|
\end{exampleblock}
\end{frame}
\begin{frame}[fragile]
Sin embargo, es importante tener en cuenta que los valores de tipo real casi nunca están representados de manera exacta en la solución numérica, y que el resultado de un algoritmo que involucra muchas operaciones puede sufrir de algunos errores de redondeo. Por esto mismo, puede ocurrir que aunque los resultados se vean iguales en la consola, los datos obtenidos son sólo aproximaciones y no exactamente los mismos valores:
\begin{exampleblock}{}
\verb|>>> (dot(a, x) == b).all()| \\
\pause
\verb|False|
\end{exampleblock}
\end{frame}
\section{Otras funciones dentro de \texttt{numpy.linalg}}
\begin{frame}
\frametitle{Otras funciones dentro de \texttt{numpy.linalg}}
Para extender (y simplificar el trabajo para codificar) el manejo de arreglos en Python, se cuenta con otras funciones que se ocupan de manera continua.
\\
\bigskip
Como se ha mencionado anteriormente, es necesario repasar el álgebra lineal básica para tener en cuenta el proceso con el cual, Pyhon devuelve una respuesta.
\end{frame}
%\subsection{Función \texttt{Eye}}
\begin{frame}[fragile]
\frametitle{Función \texttt{Eye}}
La función \texttt{eye} genera una matriz $N \times	N$ diagonal con \verb|eye(N)|, pero admite otros parámetros que permiten hacer matrices no cuadradas, tener otro tipo de datos y hacer distinto de $0$ otra diagonal diferente.
\begin{exampleblock}{}
\verb|>>>eye(2)| \\
\pause
\verb|array([[ 1.,  0.],| \\
\verb|       [ 0.,  1.]])|
\end{exampleblock}
\end{frame}
\begin{frame}[fragile]
\begin{exampleblock}{}
\verb|>>> eye(2,3)| \\
\pause
\pause{\verb|array([[ 1.,  0.,  0.],| \\
\verb|       [ 0.,  1.,  0.]])|}
\end{exampleblock}
\begin{exampleblock}{}<2->
\verb|>>> eye(2,3,k=1)| \\
\pause{\verb|array([[ 0.,  1.,  0.],| \\
\verb|       [ 0.,  0.,  1.]])|}
\end{exampleblock}
\begin{exampleblock}{}<3->
\verb|>>> eye(2,3,k=1,dtype=complex)| \\
\pause{\verb|array([[ 0.+0.j,  1.+0.j,  0.+0.j],| \\
\verb|       [ 0.+0.j,  0.+0.j,  1.+0.j]])|}
\end{exampleblock}
\end{frame}
%\subsection{Función \texttt{reshape}}
\begin{frame}
\frametitle{Función \texttt{reshape}}
La función \texttt{reshape} permite cambiar las dimensiones de una matriz, siempre respetando el número total de elementos.
\\
\bigskip
No cambia el objeto original, pero devuelve otro objeto que apunta los mismos datos, de forma que si modificamos uno, el otro lo hará también.
\end{frame}
\begin{frame}[fragile]
\verb|>>> a = random.rand(4,4)| \\
\fontsize{10}{10}\selectfont
\pause
\verb|>>> a| \\
\verb|array([[ 0.51878337,  0.93337481,  0.84368137,  0.07324918],| \\
\verb|       [ 0.12929511,  0.92344357,  0.50366378,  0.59754141],| \\
\verb|       [ 0.67841199,  0.73959186,  0.45789404,  0.85003645],| \\
\verb|       [ 0.95552903,  0.81794353,  0.78810869,  0.05192744]])|
\end{frame}
\begin{frame}[fragile]
\verb|>>> b = reshape(a,(2,8))| \\
\fontsize{10}{10}\selectfont
\pause
\verb|>>> b| \\
\pause
\verb|array([[ 0.51878337,  0.93337481,  0.84368137,  0.07324918,  0.12929511,| \\
\verb|         0.92344357,  0.50366378,  0.59754141],| \\
\verb|       [ 0.67841199,  0.73959186,  0.45789404,  0.85003645,  0.95552903,| \\
\verb|         0.81794353,  0.78810869,  0.05192744]])|
\end{frame}
%\subsection{Traza y determinante}
\begin{frame}[fragile]
\frametitle{Traza y determinante}
La traza y el determinante de una matriz, se pueden obtener respectivamente, con la función \texttt{trace} y \texttt{det}:
\begin{exampleblock}{}
\verb|>>> linalg.det(eye(3))| \\
\pause{\verb|1.0|} \\
\pause{\verb|>>> trace(eye(3))|} \\
\pause{\verb|3.0|}
\end{exampleblock}
\end{frame}
%\subsection{Inversa de una matriz}
\begin{frame}[fragile]
\frametitle{Inversa de una matriz}
La inversa de una matriz se calcula con la función \texttt{inv}:
\begin{exampleblock}{}
\fontsize{12}{12}\selectfont
\verb|>>> a = random.rand(2,2)| \\
\pause{\verb|>>> a|} \\
\pause{\verb|array([[ 0.64569289,  0.72496086],| \\
\verb|       [ 0.98555394,  0.02864243]])| }\\
\pause{\verb|>>> linalg.inv(a)|} \\
\pause{\verb|array([[-0.04115328,  1.04161968],| \\
\verb|       [ 1.41603835, -0.92772791]])|} \\
\pause{\verb|>>> dot(a,linalg.inv(a))|} \\
\pause{\verb|array([[ 1.,  0.],| \\
\verb|       [ 0.,  1.]])|}
\end{exampleblock}
\end{frame}
%\subsection{Matriz transpuesta}
\begin{frame}[fragile]
\frametitle{Matriz transpuesta}
La transpuesta de una matriz se obtiene con \texttt{tranpose}, que puede usarse también como método. Otra manera es usar el atributo \texttt{.T}
\begin{exampleblock}{Como función}
\verb|>>> transpose(a)| \\
\pause{\verb|array([[ 0.64569289,  0.98555394],| \\
\verb|       [ 0.72496086,  0.02864243]])|}
\end{exampleblock}
\end{frame}
\begin{frame}[fragile]
\begin{exampleblock}{Como método}
\verb|>>> a.transpose()| \\
\pause{\verb|array([[ 0.64569289,  0.98555394],| \\
\verb|       [ 0.72496086,  0.02864243]])|}
\end{exampleblock}
\begin{exampleblock}{Con el atributo \texttt{.T}}<2->
\verb|>>> a.T| \\
\pause{\verb|array([[ 0.64569289,  0.98555394],| \\
\verb|       [ 0.72496086,  0.02864243]])|}
\end{exampleblock}
\end{frame}
%\subsection{Valores y vectores propios}
\begin{frame}[fragile]
\frametitle{Valores y vectores propios}
La función \texttt{eig} permite obtener los valores y vectores propios:
\begin{exampleblock}{}
\verb|>>> linalg.eig(a)| \\
\verb|(array([ 1.23698756, -0.56265224]),| \\
\verb| array([[ 0.77492825, -0.51447164],| \\
\verb|       [ 0.63204921,  0.85750739]]))|
\end{exampleblock}
\end{frame}
\section{Otro ejemplo de la física}
\begin{frame}
\frametitle{Otro ejemplo de la física}
El método de Euler que usamos para el problema de la bicicleta, se puede generalizr fácilmente para tratar con el movimiento en dos dimensiones espaciales.
\\
\medskip
Para ser más específicos, consideramos un proyectil como una bala que se dispara desde un cañón. Tenemos ya inicialmente un gran cañón en la mente, y el cálculo de la distancia máxima que logra la bala es una tarea común en la física.
\\
\medskip
Si ignoramos la resistencia del aire, las ecuaciones de movimiento, que se obtuvieron de nuevo con la segunda ley de Newton, se puede escribir como:
\end{frame}
\begin{frame}
\frametitle{Ecuaciones de movimiento en dos dimensiones}
\begin{eqnarray*}
\dfrac{d^{2}x}{dt^{2}} &=& 0 \\
\dfrac{d^{2}y}{dt^{2}} &=& -g
\end{eqnarray*}
donde $x$, $y$ son las coordenadas horizontal y vertical respectivamente de la bala, y $g$ es la aceleración debida a la gravedad.
\end{frame}
\begin{frame}
Estas son las ecuaciones diferenciales de segundo orden, a diferencia de las ecuaciones de primer orden que nos hemos encontrado hasta ahora, habrá que generalizar nuestro enfoque.
\\
\medskip
Hemos visto que con una ecuación de primer orden (con la bicicleta), es posible utilizar una aproximación de diferencias finitas para la derivada y obtener una ecuación algebraica simple, que contiene la variable de interés en dos pasos de tiempo adyacentes.
\\
\medskip
Sin embargo, si tuviéramos que adoptar el mismo enfoque con una de las ecuaciones de segundo orden y escribir una diferencia finita aproximación a la segunda derivada, obtendríamos una ecuación algebraica más complicada que implicaría evaluar nuestra variable en tres pasos de tiempo.
\end{frame}
\begin{frame}
Sin embargo, en el presente ejemplo, es posible evitar esta complicación si redefinimos las ecuaciones diferenciales de la siguiente manera.
\\
\medskip
Vamos a escribir cada una de estas ecuaciones de segundo orden en un sistema de dos ecuaciones de primer orden:
\begin{eqnarray*}
\dfrac{dx}{dt} &=& v_{x} \\
\dfrac{dv_{x}}{dt} &=& 0 \\
\dfrac{dy}{dt} &=& v_{y} \\
\dfrac{dv_{y}}{dt} &=& -g
\end{eqnarray*}
\end{frame}
\begin{frame}
Ahora tenemos el doble de ecuaciones que en el ejemplo de la bicicleta, pero podemos usar el método de Euler para resolverlas. La aproximación por diferencias finitas queda como:
\begin{eqnarray*}
x_{i+1} &=& x_{i} + v_{x,i} \Delta t \\
v_{x,i+1} &=& v_{x,i} \\
y_{i+1} &=& y_{i} + v_{y,i} \Delta t \\
v_{y,i+1} &=& v_{y,i} - g \Delta t
\end{eqnarray*}
Dados los valores iniciales de $x$, $y$, $v_{x}$, $v_{y}$, podemos usar las ecuaciones anteriores para estimar valores posteriores de tiempo. Aquí tenemos una \emph{aproximación}, dado que los términos de orden $(\Delta)^{2}$ se cancelan, al elegir un $\Delta t$ pequeño. 
\end{frame}
\begin{frame}
\frametitle{Ejercicio para entregar}
Elabora un código en Python para resolver el problema del lanzamiento de una bala de cañon y calcula la distancia máxima de desplazamiento.
\end{frame}
\end{document}