\documentclass[12pt]{article}
\usepackage[utf8]{inputenc}
\usepackage[spanish]{babel}
\usepackage{amsmath}
\usepackage{amsthm}
\usepackage{multicol,multienum}
\usepackage{graphicx}
\usepackage{float}
\usepackage{tikz}
\usepackage{color}
\usepackage{anysize}
\usepackage{anyfontsize}
%Esta linea obliga a que se marque el punto como indicador de los decimales y no la coma
\spanishdecimal{.}
\usepackage{listings}
\lstset{ %
language=Python,                % choose the language of the code
basicstyle=\small,       % the size of the fonts that are used for the code
numbers=left,                   % where to put the line-numbers
numberstyle=\footnotesize,      % the size of the fonts that are used for the line-numbers
stepnumber=1,                   % the step between two line-numbers. If it is 1 each line will be numbered
numbersep=5pt,                  % how far the line-numbers are from the code
backgroundcolor=\color{white},  % choose the background color. You must add \usepackage{color}
showspaces=false,               % show spaces adding particular underscores
showstringspaces=false,         % underline spaces within strings
showtabs=false,                 % show tabs within strings adding particular underscores
frame=single,   		% adds a frame around the code
tabsize=4,  		% sets default tabsize to 2 spaces
captionpos=b,   		% sets the caption-position to bottom
breaklines=true,    	% sets automatic line breaking
breakatwhitespace=false,    % sets if automatic breaks should only happen at whitespace
escapeinside={\#}{)}          % if you want to add a comment within your code
}

\marginsize{1.5cm}{1.5cm}{2cm}{2cm}
\author{M. en C. Gustavo Contreras May\'{e}n.}
\title{Escribiendo código de Python en un documento \\ \begin{Large} Curso de Física Computacional\end{Large} }
\date{ }
\begin{document}
\maketitle
\fontsize{14}{14}\selectfont
Para escribir código de programación en un documento de \LaTeX como si estuviéramos trabajando con un editor, requiere que usemos el paquete \texttt{listings}, en donde se especifica el lenguaje con el que se trabaja, a las instrucciones y comandos propios del lenguaje, las resalta sin necesidad de hacerlo una por una, es importante tomar en cuenta que los mismos espaciamientos que se indican en el formato del paquete, para evitar que nos marque un error.
\\
\\
Hay que señalar el símbolo que usaremos para los comentarios dentro del código, para que no nos marque un error durante la compilación. Cada bloque de código tendrá una numeración al lado izquierdo de la caja, así como un cuadro que encierra el código, es posible modificar el fondo del cuadro, pensando que mostraremos el código en una diapositiva, pero para los textos en pdf, basta con dejarlo en blanco.
%Nota: aunque el paquete que se usa es listing, el bloque dentro de latex, es lstlisting
\begin{lstlisting}
import matplotlib.pyplot as plt
x=[0.05,0.10,0.15,0.20,0.25,0.30]
y=[0.2000,0.2800,0.3600,0.4200,0.5200,0.5800]

plt.plot(x,y, 'r*', label='resistencia')
plt.xlabel('longitud')
plt.ylabel('resistencia')
plt.legend(loc= "upper left")
plt.show()
\end{lstlisting}
Con lo que tenemos dentro del documento, el código con formato y dentro de una caja.
\end{document}
