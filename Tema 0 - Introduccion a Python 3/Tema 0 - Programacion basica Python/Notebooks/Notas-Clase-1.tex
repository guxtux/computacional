\documentclass[]{article}
\usepackage{lmodern}
\usepackage{amssymb,amsmath}
\usepackage{ifxetex,ifluatex}
\usepackage{fixltx2e} % provides \textsubscript
\ifnum 0\ifxetex 1\fi\ifluatex 1\fi=0 % if pdftex
  \usepackage[T1]{fontenc}
  \usepackage[utf8]{inputenc}
\else % if luatex or xelatex
  \ifxetex
    \usepackage{mathspec}
  \else
    \usepackage{fontspec}
  \fi
  \defaultfontfeatures{Ligatures=TeX,Scale=MatchLowercase}
\fi
% use upquote if available, for straight quotes in verbatim environments
\IfFileExists{upquote.sty}{\usepackage{upquote}}{}
% use microtype if available
\IfFileExists{microtype.sty}{%
\usepackage[]{microtype}
\UseMicrotypeSet[protrusion]{basicmath} % disable protrusion for tt fonts
}{}
\PassOptionsToPackage{hyphens}{url} % url is loaded by hyperref
\usepackage[unicode=true]{hyperref}
\hypersetup{
            pdfborder={0 0 0},
            breaklinks=true}
\urlstyle{same}  % don't use monospace font for urls
\usepackage{color}
\usepackage{fancyvrb}
\newcommand{\VerbBar}{|}
\newcommand{\VERB}{\Verb[commandchars=\\\{\}]}
\DefineVerbatimEnvironment{Highlighting}{Verbatim}{commandchars=\\\{\}}
% Add ',fontsize=\small' for more characters per line
\newenvironment{Shaded}{}{}
\newcommand{\KeywordTok}[1]{\textcolor[rgb]{0.00,0.44,0.13}{\textbf{#1}}}
\newcommand{\DataTypeTok}[1]{\textcolor[rgb]{0.56,0.13,0.00}{#1}}
\newcommand{\DecValTok}[1]{\textcolor[rgb]{0.25,0.63,0.44}{#1}}
\newcommand{\BaseNTok}[1]{\textcolor[rgb]{0.25,0.63,0.44}{#1}}
\newcommand{\FloatTok}[1]{\textcolor[rgb]{0.25,0.63,0.44}{#1}}
\newcommand{\ConstantTok}[1]{\textcolor[rgb]{0.53,0.00,0.00}{#1}}
\newcommand{\CharTok}[1]{\textcolor[rgb]{0.25,0.44,0.63}{#1}}
\newcommand{\SpecialCharTok}[1]{\textcolor[rgb]{0.25,0.44,0.63}{#1}}
\newcommand{\StringTok}[1]{\textcolor[rgb]{0.25,0.44,0.63}{#1}}
\newcommand{\VerbatimStringTok}[1]{\textcolor[rgb]{0.25,0.44,0.63}{#1}}
\newcommand{\SpecialStringTok}[1]{\textcolor[rgb]{0.73,0.40,0.53}{#1}}
\newcommand{\ImportTok}[1]{#1}
\newcommand{\CommentTok}[1]{\textcolor[rgb]{0.38,0.63,0.69}{\textit{#1}}}
\newcommand{\DocumentationTok}[1]{\textcolor[rgb]{0.73,0.13,0.13}{\textit{#1}}}
\newcommand{\AnnotationTok}[1]{\textcolor[rgb]{0.38,0.63,0.69}{\textbf{\textit{#1}}}}
\newcommand{\CommentVarTok}[1]{\textcolor[rgb]{0.38,0.63,0.69}{\textbf{\textit{#1}}}}
\newcommand{\OtherTok}[1]{\textcolor[rgb]{0.00,0.44,0.13}{#1}}
\newcommand{\FunctionTok}[1]{\textcolor[rgb]{0.02,0.16,0.49}{#1}}
\newcommand{\VariableTok}[1]{\textcolor[rgb]{0.10,0.09,0.49}{#1}}
\newcommand{\ControlFlowTok}[1]{\textcolor[rgb]{0.00,0.44,0.13}{\textbf{#1}}}
\newcommand{\OperatorTok}[1]{\textcolor[rgb]{0.40,0.40,0.40}{#1}}
\newcommand{\BuiltInTok}[1]{#1}
\newcommand{\ExtensionTok}[1]{#1}
\newcommand{\PreprocessorTok}[1]{\textcolor[rgb]{0.74,0.48,0.00}{#1}}
\newcommand{\AttributeTok}[1]{\textcolor[rgb]{0.49,0.56,0.16}{#1}}
\newcommand{\RegionMarkerTok}[1]{#1}
\newcommand{\InformationTok}[1]{\textcolor[rgb]{0.38,0.63,0.69}{\textbf{\textit{#1}}}}
\newcommand{\WarningTok}[1]{\textcolor[rgb]{0.38,0.63,0.69}{\textbf{\textit{#1}}}}
\newcommand{\AlertTok}[1]{\textcolor[rgb]{1.00,0.00,0.00}{\textbf{#1}}}
\newcommand{\ErrorTok}[1]{\textcolor[rgb]{1.00,0.00,0.00}{\textbf{#1}}}
\newcommand{\NormalTok}[1]{#1}
\usepackage{longtable,booktabs}
% Fix footnotes in tables (requires footnote package)
\IfFileExists{footnote.sty}{\usepackage{footnote}\makesavenoteenv{long table}}{}
\IfFileExists{parskip.sty}{%
\usepackage{parskip}
}{% else
\setlength{\parindent}{0pt}
\setlength{\parskip}{6pt plus 2pt minus 1pt}
}
\setlength{\emergencystretch}{3em}  % prevent overfull lines
\providecommand{\tightlist}{%
  \setlength{\itemsep}{0pt}\setlength{\parskip}{0pt}}
\setcounter{secnumdepth}{0}
% Redefines (sub)paragraphs to behave more like sections
\ifx\paragraph\undefined\else
\let\oldparagraph\paragraph
\renewcommand{\paragraph}[1]{\oldparagraph{#1}\mbox{}}
\fi
\ifx\subparagraph\undefined\else
\let\oldsubparagraph\subparagraph
\renewcommand{\subparagraph}[1]{\oldsubparagraph{#1}\mbox{}}
\fi

% set default figure placement to htbp
\makeatletter
\def\fps@figure{htbp}
\makeatother


\date{}

\begin{document}

\section{Operadores aritméticos}\label{operadores-aritmuxe9ticos}

\subsection{Python como calculadora}\label{python-como-calculadora}

Una vez abierta la sesión en \(\texttt{python}\), podemos aprovechar al
máximo este lenguaje: contamos con una calculadora a la mano, sólo hay
que ir escribiendo las operaciones en la línea de comandos.

\subsection{Podemos hacer una suma:}\label{podemos-hacer-una-suma}

\begin{Shaded}
\begin{Highlighting}[]
\DecValTok{3}\OperatorTok{+}\DecValTok{200}
\end{Highlighting}
\end{Shaded}

\begin{center}\rule{0.5\linewidth}{\linethickness}\end{center}

\begin{verbatim}
203
\end{verbatim}

\subsection{Una división entre
enteros}\label{una-divisiuxf3n-entre-enteros}

\begin{Shaded}
\begin{Highlighting}[]
\DecValTok{30}\OperatorTok{/}\DecValTok{1234}
\end{Highlighting}
\end{Shaded}

\begin{center}\rule{0.5\linewidth}{\linethickness}\end{center}

\begin{verbatim}
0.024311183144246355
\end{verbatim}

\subsection{Una división entre
reales}\label{una-divisiuxf3n-entre-reales}

\begin{Shaded}
\begin{Highlighting}[]
\FloatTok{3.0}\OperatorTok{/}\FloatTok{4.0}
\end{Highlighting}
\end{Shaded}

\begin{center}\rule{0.5\linewidth}{\linethickness}\end{center}

\begin{verbatim}
0.75
\end{verbatim}

\subsection{Una división entera}\label{una-divisiuxf3n-entera}

\subsubsection{Devuelve el cociente (sin
decimales)}\label{devuelve-el-cociente-sin-decimales}

\begin{Shaded}
\begin{Highlighting}[]
\DecValTok{30}\OperatorTok{//}\DecValTok{4}
\end{Highlighting}
\end{Shaded}

\begin{center}\rule{0.5\linewidth}{\linethickness}\end{center}

\begin{verbatim}
7
\end{verbatim}

\subsection{Otro ejemplo}\label{otro-ejemplo}

\begin{Shaded}
\begin{Highlighting}[]
\DecValTok{4}\OperatorTok{//}\DecValTok{3}
\end{Highlighting}
\end{Shaded}

\begin{center}\rule{0.5\linewidth}{\linethickness}\end{center}

\begin{verbatim}
1
\end{verbatim}

\subsection{La división devuelve un determinado número de dígitos luego
del punto
decimal}\label{la-divisiuxf3n-devuelve-un-determinado-nuxfamero-de-duxedgitos-luego-del-punto-decimal}

\begin{Shaded}
\begin{Highlighting}[]
\DecValTok{4}\OperatorTok{/}\DecValTok{3}
\end{Highlighting}
\end{Shaded}

\begin{center}\rule{0.5\linewidth}{\linethickness}\end{center}

\begin{verbatim}
1.3333333333333333
\end{verbatim}

\subsection{Combinación de
operadores}\label{combinaciuxf3n-de-operadores}

\begin{Shaded}
\begin{Highlighting}[]
\FloatTok{5.0} \OperatorTok{/} \DecValTok{10} \OperatorTok{*} \DecValTok{2} \OperatorTok{+} \DecValTok{5}
\end{Highlighting}
\end{Shaded}

\begin{center}\rule{0.5\linewidth}{\linethickness}\end{center}

\begin{verbatim}
6.0
\end{verbatim}

\textbf{¿por qué obtenemos este resultado??}

\subsection{El resultado cambia cuando agrupamos con
paréntesis}\label{el-resultado-cambia-cuando-agrupamos-con-paruxe9ntesis}

\begin{Shaded}
\begin{Highlighting}[]
\FloatTok{5.0} \OperatorTok{/}\NormalTok{ (}\DecValTok{10} \OperatorTok{*} \DecValTok{2} \OperatorTok{+} \DecValTok{5}\NormalTok{)}
\end{Highlighting}
\end{Shaded}

\begin{center}\rule{0.5\linewidth}{\linethickness}\end{center}

\begin{verbatim}
0.2
\end{verbatim}

Como podemos ver, el uso de paréntesis en las expresiones tiene una
particular importancia sobre la manera en que se evalúan las expresiones

\subsection{Potenciación de un
número}\label{potenciaciuxf3n-de-un-nuxfamero}

Podemos elevar un número a una potencia en particular

\begin{Shaded}
\begin{Highlighting}[]
\DecValTok{2}\OperatorTok{**}\DecValTok{3}\OperatorTok{**}\DecValTok{2}
\end{Highlighting}
\end{Shaded}

\begin{center}\rule{0.5\linewidth}{\linethickness}\end{center}

\begin{verbatim}
512
\end{verbatim}

\subsection{Orden para las potencias}\label{orden-para-las-potencias}

Vemos que elevar a una potencia, la manera en que se ejecuta la
expresión se realiza en un sentido en particular: de derecha a izquierda

\begin{Shaded}
\begin{Highlighting}[]
\NormalTok{(}\DecValTok{2}\OperatorTok{**}\DecValTok{3}\NormalTok{)}\OperatorTok{**}\DecValTok{2}
\end{Highlighting}
\end{Shaded}

\begin{center}\rule{0.5\linewidth}{\linethickness}\end{center}

\begin{verbatim}
64
\end{verbatim}

\subsection{Los paréntesis}\label{los-paruxe9ntesis}

El uso de paréntesis nos indica que la expresión contenida dentro de
ellos, es la que se evalúa primero, posteriormente se sigue la regla de
precedencia de operadores

\subsection{Operador módulo}\label{operador-muxf3dulo}

El operador módulo \((\%)\) nos devuelve el residuo del cociente

\begin{Shaded}
\begin{Highlighting}[]
\DecValTok{17}\OperatorTok{%}\DecValTok{3}
\end{Highlighting}
\end{Shaded}

\begin{center}\rule{0.5\linewidth}{\linethickness}\end{center}

\begin{verbatim}
2
\end{verbatim}

\section{Tabla de operadores}\label{tabla-de-operadores}

\subsection{Precedencia en los operadores
aritméticos}\label{precedencia-en-los-operadores-aritmuxe9ticos}

\subsection{}\label{section}

\begin{longtable}[]{@{}clll@{}}
\toprule
Operador & Operación & Ejemplo & Resultado\tabularnewline
\midrule
\endhead
\(**\) & Potencia & \(2**3\) & \(8\)\tabularnewline
\(*\) & Multiplicación & \(7*3\) & \(21\)\tabularnewline
\(/\) & División & \(10.5/2\) & \(5.25\)\tabularnewline
\(//\) & Div. entera & \(10.5//2\) & \(5.0\)\tabularnewline
\(+\) & Suma & \(3+4\) & \(7\)\tabularnewline
\(-\) & Resta & \(6-8\) & \(-2\)\tabularnewline
\(\%\) & Módulo & \(15\%6\) & \(3\)\tabularnewline
\bottomrule
\end{longtable}

\subsection{Precedencia de los operadores artiméticos
1}\label{precedencia-de-los-operadores-artimuxe9ticos-1}

\begin{enumerate}
\def\labelenumi{\arabic{enumi}.}
\tightlist
\item
  Las expresiones contenidas dentro de pares de paréntesis son evaluadas
  primero. En el caso de expresiones con paréntesis anidados, los
  operadores en el par de paréntesis más interno son aplicados primero.
\end{enumerate}

\subsection{Precedencia de los operadores artiméticos
2}\label{precedencia-de-los-operadores-artimuxe9ticos-2}

\begin{enumerate}
\def\labelenumi{\arabic{enumi}.}
\setcounter{enumi}{1}
\tightlist
\item
  Las operaciones de exponentes son aplicadas después. Si una expresión
  contiene muchas operaciones de exponentes, los operadores son
  aplicados de derecha a izquierda.
\end{enumerate}

\subsection{Precedencia de los operadores artiméticos
3}\label{precedencia-de-los-operadores-artimuxe9ticos-3}

\begin{enumerate}
\def\labelenumi{\arabic{enumi}.}
\setcounter{enumi}{2}
\tightlist
\item
  La multiplicación, división y módulo son las siguientes en ser
  aplicadas. Si una expresión contiene muchas multiplicaciones,
  divisiones u operaciones de módulo, los operadores se aplican de
  izquierda a derecha.
\end{enumerate}

\subsection{Precedencia de los operadores artiméticos
4}\label{precedencia-de-los-operadores-artimuxe9ticos-4}

\begin{enumerate}
\def\labelenumi{\arabic{enumi}.}
\setcounter{enumi}{3}
\tightlist
\item
  Suma y resta son las operaciones que se aplican por último. Si una
  expresión contiene muchas operaciones de suma y resta, los operadores
  son aplicados de izquierda a derecha. La suma y resta tienen el mismo
  nivel de precedencia.
\end{enumerate}

\section{Operadores relacionales}\label{operadores-relacionales}

\subsection{}\label{section-1}

Se comparan dos (o más expresiones) mediante un operador, el tipo de
dato que devuelve es lógico: \(\textbf{True}\) o \(\textbf{False}\), que
también tienen una representación de tipo numérico:

\begin{itemize}
\tightlist
\item
  \(\textbf{True}\) = 1
\item
  \(\textbf{False}\) = 0
\end{itemize}

\subsection{Operaciones aritméticas y
relacionales}\label{operaciones-aritmuxe9ticas-y-relacionales}

\begin{Shaded}
\begin{Highlighting}[]
\DecValTok{1} \OperatorTok{+} \DecValTok{2} \OperatorTok{>} \DecValTok{7}\OperatorTok{-} \DecValTok{3}
\end{Highlighting}
\end{Shaded}

\begin{center}\rule{0.5\linewidth}{\linethickness}\end{center}

\begin{verbatim}
False
\end{verbatim}

\subsection{Se pueden combinar diferentes
expresiones}\label{se-pueden-combinar-diferentes-expresiones}

\begin{Shaded}
\begin{Highlighting}[]
\DecValTok{1} \OperatorTok{<} \DecValTok{2} \OperatorTok{<} \DecValTok{3}
\end{Highlighting}
\end{Shaded}

\begin{center}\rule{0.5\linewidth}{\linethickness}\end{center}

\begin{verbatim}
True
\end{verbatim}

\subsection{El doble signo igual (==) es el operador de
igualdad}\label{el-doble-signo-igual-es-el-operador-de-igualdad}

\begin{Shaded}
\begin{Highlighting}[]
\DecValTok{1} \OperatorTok{>} \DecValTok{2} \OperatorTok{==} \DecValTok{2} \OperatorTok{<} \DecValTok{3}
\end{Highlighting}
\end{Shaded}

\begin{Shaded}
\begin{Highlighting}[]
\DecValTok{1} \OperatorTok{>}\NormalTok{ (}\DecValTok{2} \OperatorTok{==}  \DecValTok{2}\NormalTok{) }\OperatorTok{<}\DecValTok{3}
\end{Highlighting}
\end{Shaded}

\begin{center}\rule{0.5\linewidth}{\linethickness}\end{center}

\begin{verbatim}
False
\end{verbatim}

\subsection{Combinación de
expresiones}\label{combinaciuxf3n-de-expresiones}

\begin{Shaded}
\begin{Highlighting}[]
\DecValTok{3} \OperatorTok{>} \DecValTok{4} \OperatorTok{<} \DecValTok{5}
\end{Highlighting}
\end{Shaded}

\begin{center}\rule{0.5\linewidth}{\linethickness}\end{center}

\begin{verbatim}
False
\end{verbatim}

\subsection{}\label{section-2}

Hay que tener cuidado con el uso de operadores relaciones estrictos

\begin{Shaded}
\begin{Highlighting}[]
\FloatTok{1.0} \OperatorTok{/} \DecValTok{3} \OperatorTok{<} \FloatTok{0.3333}
\end{Highlighting}
\end{Shaded}

\begin{center}\rule{0.5\linewidth}{\linethickness}\end{center}

\begin{verbatim}
False
\end{verbatim}

\subsection{}\label{section-3}

Las expresiones se pueden complicar cada vez más, por lo que hay que
mantener atención al momento de escribirlas

\begin{Shaded}
\begin{Highlighting}[]
\FloatTok{5.0} \OperatorTok{/} \DecValTok{3} \OperatorTok{>=} \DecValTok{11} \OperatorTok{/}\FloatTok{7.0}
\end{Highlighting}
\end{Shaded}

\begin{center}\rule{0.5\linewidth}{\linethickness}\end{center}

\begin{verbatim}
True
\end{verbatim}

\subsection{}\label{section-4}

El utilizar las operaciones aritméticas y relacionales, extiende por
mucho el uso del lenguaje.

\begin{Shaded}
\begin{Highlighting}[]
\DecValTok{2}\OperatorTok{**}\NormalTok{(}\DecValTok{2}\NormalTok{. }\OperatorTok{/}\DecValTok{3}\NormalTok{) }\OperatorTok{<} \DecValTok{3}\OperatorTok{**}\NormalTok{(}\DecValTok{3}\NormalTok{.}\OperatorTok{/}\DecValTok{4}\NormalTok{)}
\end{Highlighting}
\end{Shaded}

\begin{center}\rule{0.5\linewidth}{\linethickness}\end{center}

\begin{verbatim}
True
\end{verbatim}

\subsection{Tabla de operadores
relacionales}\label{tabla-de-operadores-relacionales}

\begin{longtable}[]{@{}clll@{}}
\toprule
Operador & Operación & Ejemplo & Resultado\tabularnewline
\midrule
\endhead
\(==\) & Igual a & \(4==5\) & \(\texttt{False}\)\tabularnewline
\(!=\) & Diferente & \(2!=3\) & \(\texttt{True}\)\tabularnewline
\(<\) & Menor que & \(10<4\) & \(\texttt{False}\)\tabularnewline
\(>\) & Mayor que & \(5>-4\) & \(\texttt{True}\)\tabularnewline
\(<=\) & Menor o igual & \(7<=7\) & \(\texttt{True}\)\tabularnewline
\(>=\) & Mayor o igual & \(3.5 >= 10\) &
\(\texttt{False}\)\tabularnewline
\bottomrule
\end{longtable}

\section{Operadores booleanos}\label{operadores-booleanos}

\subsection{Operadores booleanos}\label{operadores-booleanos-1}

En el caso del operador boleano and y el operador or evalúan una
expresión compuesta por dos (o más términos).

\subsection{}\label{section-5}

En ambas expresiones se espera que cada una tenga el valor de True, en
caso de que esto ocurra, el valor que devuelve la evaluación, es
\(\texttt{True}\).

\subsection{}\label{section-6}

Como se verá en la tabla de verdad, se necesita una condición particular
para que el valor que devuelva la comparación, sea \(\texttt{False}\).

\begin{longtable}[]{@{}llll@{}}
\toprule
Operador & Operación & Ejemplo & Resultado\tabularnewline
\midrule
\endhead
\(\texttt{and}\) & Conjunción & \(\texttt{False and True}\) &
\(\texttt{False}\)\tabularnewline
\(\texttt{or}\) & Disyunción & \(\texttt{False or True}\) &
\(\texttt{True}\)\tabularnewline
\(\texttt{not}\) & Negación & \(\texttt{not True}\) &
\(\texttt{False}\)\tabularnewline
\bottomrule
\end{longtable}

\subsection{Tabla de verdad de los operadores
boleanos}\label{tabla-de-verdad-de-los-operadores-boleanos}

\subsection{}\label{section-7}

\begin{longtable}[]{@{}ccccc@{}}
\toprule
A & B & A and B & A or B & not A\tabularnewline
\midrule
\endhead
True & True & True & True & False\tabularnewline
True & False & False & True & False\tabularnewline
False & True & False & True & True\tabularnewline
False & False & False & False & True\tabularnewline
\bottomrule
\end{longtable}

\section{Tipos de variables}\label{tipos-de-variables}

\subsection{}\label{section-8}

Las variables en python sólo son ubicaciones de memoria reservadas para
almacenar valores.

Esto significa que cuando se crea una variable, se reserva un poco de
espacio disponible en la memoria.

\subsection{}\label{section-9}

Basándose en el tipo de datos de una variable, el intérprete asigna
memoria y decide qué se puede almacenar en la memoria reservada.

Por lo tanto, al asignar diferentes tipos de datos a las variables, se
pueden almacenar \textbf{enteros}, \textbf{decimales} o
\textbf{caracteres (cadenas)} en estas variables.

\subsection{Asignando valores a
variables}\label{asignando-valores-a-variables}

Las variables de python no necesitan una declaración explícita para
reservar espacio de memoria.

La declaración ocurre automáticamente cuando se asigna un valor a una
variable. \emph{El signo igual (=) se utiliza para asignar valores a las
variables}.

\subsection{}\label{section-10}

El término a la izquierda del operador = es el \emph{nombre de la
variable} y el término a la derecha del operador = es el \emph{valor
almacenado} en la variable.

\subsection{Ejemplos}\label{ejemplos}

\begin{Shaded}
\begin{Highlighting}[]
\NormalTok{contador }\OperatorTok{=} \DecValTok{100}          \CommentTok{# Asignacion de tipo entero}
\NormalTok{distancia }\OperatorTok{=} \FloatTok{1000.0}      \CommentTok{# De punto flotante}
\NormalTok{nombre }\OperatorTok{=} \StringTok{"Chucho"}       \CommentTok{# Una cadena de caracteres}

\BuiltInTok{print}\NormalTok{ (contador)}
\BuiltInTok{print}\NormalTok{ (distancia)}
\BuiltInTok{print}\NormalTok{ (nombre)}
\end{Highlighting}
\end{Shaded}

\begin{center}\rule{0.5\linewidth}{\linethickness}\end{center}

\begin{verbatim}
100
1000.0
Chucho
\end{verbatim}

\subsection{Asignación múltiple de
valores}\label{asignaciuxf3n-muxfaltiple-de-valores}

En python podemos asignar un valor único a varias variables
simultáneamente.

\subsection{}\label{section-11}

\begin{Shaded}
\begin{Highlighting}[]
\NormalTok{A }\OperatorTok{=}\NormalTok{ b }\OperatorTok{=}\NormalTok{ c }\OperatorTok{=} \DecValTok{1}
\BuiltInTok{print}\NormalTok{ (A)}
\BuiltInTok{print}\NormalTok{ (b)}
\BuiltInTok{print}\NormalTok{ (c)}
\end{Highlighting}
\end{Shaded}

\begin{center}\rule{0.5\linewidth}{\linethickness}\end{center}

\begin{verbatim}
1
1
1
\end{verbatim}

\subsection{}\label{section-12}

En el ejemplo, se crea un objeto entero con el valor \(1\), y las tres
variables se asignan a la misma ubicación de memoria.

\subsection{}\label{section-13}

También puede asignar varios objetos a varias variables.

\begin{Shaded}
\begin{Highlighting}[]
\NormalTok{A, b, c }\OperatorTok{=} \DecValTok{1}\NormalTok{, }\DecValTok{2}\NormalTok{, }\StringTok{"Alicia"}
\BuiltInTok{print}\NormalTok{ (A)}
\BuiltInTok{print}\NormalTok{ (b)}
\BuiltInTok{print}\NormalTok{ (c)}
\end{Highlighting}
\end{Shaded}

\begin{center}\rule{0.5\linewidth}{\linethickness}\end{center}

\begin{verbatim}
1
2
Ana
\end{verbatim}

\subsection{}\label{section-14}

Aquí, dos objetos enteros con valores \(1\) y \(2\) se asignan a las
variables \(A\) y \(b\) respectivamente, y un objeto de cadena con el
valor \textbf{Alicia} se asigna a la variable \(c\).

\section{Tipos de Datos Estándar}\label{tipos-de-datos-estuxe1ndar}

Los datos almacenados en la memoria pueden ser de varios tipos. Por
ejemplo, la edad de una persona se almacena como un valor numérico y su
dirección se almacena como caracteres alfanuméricos.

\subsection{}\label{section-15}

En python se cuenta con varios tipos de datos estándar que se utilizan
para definir las operaciones posibles entre ellos y el método de
almacenamiento para cada uno de ellos.

\subsection{}\label{section-16}

Los tipos de datos son cinco:

\begin{enumerate}
\def\labelenumi{\arabic{enumi}.}
\tightlist
\item
   Números.
\item
   Cadena.
\item
   Lista.
\item
   Tupla.
\item
   Diccionario.
\end{enumerate}

\subsection{Números}\label{nuxfameros}

Los tipos de datos numéricos almacenan valores numéricos.

Los objetos numéricos se crean cuando se les asigna un valor.

\subsection{}\label{section-17}

\begin{Shaded}
\begin{Highlighting}[]
\NormalTok{Var1 }\OperatorTok{=}\NormalTok{ Var2 }\OperatorTok{=} \DecValTok{10}
\BuiltInTok{print}\NormalTok{ (Var1)}
\BuiltInTok{print}\NormalTok{ (Var2)}
\end{Highlighting}
\end{Shaded}

\begin{center}\rule{0.5\linewidth}{\linethickness}\end{center}

\begin{verbatim}
10
10
\end{verbatim}

\subsection{}\label{section-18}

También se puede eliminar la referencia a un objeto numérico utilizando
la sentencia del

La sintaxis de la sentencia del es:

\(del\) \(var1 [, var2 [, var3 [...., varN]]]]\)

\subsection{}\label{section-19}

Se puede eliminar un solo objeto o varios objetos utilizando la
sentencia del

Por ejemplo:

\begin{Shaded}
\begin{Highlighting}[]
\KeywordTok{del}\NormalTok{ var}

\KeywordTok{del}\NormalTok{ variable1, variable2}
\end{Highlighting}
\end{Shaded}

\subsection{}\label{section-20}

En python se soportan tres tipos numéricos diferentes:

\begin{enumerate}
\def\labelenumi{\arabic{enumi}.}
\tightlist
\item
  Int (enteros con signo)
\item
  Flotante (valores reales de punto flotante)
\item
  Complejos (números complejos)
\end{enumerate}

\subsection{}\label{section-21}

Un número complejo consiste en un par ordenado de números reales de coma
flotante denotados por

\[x + yj\]

donde \(x\) e \(y\) son números reales, \(yj\) es la unidad imaginaria.

\subsection{}\label{section-22}

Todos los números enteros en python 3 se representan como enteros
largos.

\subsection{}\label{section-23}

\begin{longtable}[]{@{}ccc@{}}
\toprule
\(\texttt{int}\) & \(\texttt{float}\) &
\(\texttt{complex}\)\tabularnewline
\midrule
\endhead
\(\texttt{10}\) & \(\texttt{0.0}\) & \(\texttt{3.14j}\)\tabularnewline
\(\texttt{100}\) & \(\texttt{15.20}\) & \(\texttt{45.j}\)\tabularnewline
\(\texttt{100}\) & \(\texttt{15.20}\) & \(\texttt{45.j}\)\tabularnewline
\(\texttt{080}\) & \(\texttt{32.3+e18}\) &
\(\texttt{0.876j}\)\tabularnewline
\(\texttt{-0490}\) & \(\texttt{-90.}\) &
\(\texttt{-.645+0j}\)\tabularnewline
\(\texttt{-0x260}\) & \(\texttt{-32.54e100}\) &
\(\texttt{3e+26j}\)\tabularnewline
\(\texttt{0x69}\) & \(\texttt{70.2-E12}\) &
\(\texttt{4.53e-7j}\)\tabularnewline
\bottomrule
\end{longtable}

\subsection{Cadenas}\label{cadenas}

Las cadenas en \(\texttt{python}\) se identifican como un conjunto
contiguo de caracteres representados en las comillas.

Con python se permite cualquier par de comillas simples o dobles.

\subsection{}\label{section-24}

Los subconjuntos de cadenas pueden ser tomados usando el operador de
corte (\([\;]\) y \([:]\)) con índices comenzando en \(0\) al inicio de
la cadena hasta llegar a \(-1\) al final de la misma.

\subsection{}\label{section-25}

El signo más (\(+\)) es el operador de concatenación de cadenas y el
asterisco (\(*\)) es el operador de repetición.

\subsection{}\label{section-26}

\begin{Shaded}
\begin{Highlighting}[]
\NormalTok{cadena }\OperatorTok{=} \StringTok{'Hola Mundo!'}

\BuiltInTok{print}\NormalTok{ (cadena)          }\CommentTok{# Presenta la cadena completa}
\BuiltInTok{print}\NormalTok{ (cadena[}\DecValTok{0}\NormalTok{])       }\CommentTok{# Presenta el primer caracter de la cadena}
\BuiltInTok{print}\NormalTok{ (cadena[}\DecValTok{2}\NormalTok{:}\DecValTok{5}\NormalTok{])     }\CommentTok{# Presenta los caracteres de la 3a a la 5a posicion}
\BuiltInTok{print}\NormalTok{ (cadena[}\DecValTok{2}\NormalTok{:])      }\CommentTok{# Presenta la cadena que inicia a partir del 3er caracter}
\BuiltInTok{print}\NormalTok{ (cadena }\OperatorTok{*} \DecValTok{2}\NormalTok{)      }\CommentTok{# Presenta dos veces la cadena}
\BuiltInTok{print}\NormalTok{ (cadena }\OperatorTok{+} \StringTok{"PUMAS"}\NormalTok{) }\CommentTok{# Presenta la cadena y concatena la segunda cadena}
\end{Highlighting}
\end{Shaded}

\begin{center}\rule{0.5\linewidth}{\linethickness}\end{center}

\begin{verbatim}
Hola Mundo!
H
la 
la Mundo!
Hola Mundo!Hola Mundo!
Hola Mundo!PUMAS
\end{verbatim}

\subsection{Listas}\label{listas}

Las listas es el tipo de dato más versátil de los tipos de datos
compuestos de python.

Una lista contiene elementos separados por comas y entre corchetes
(\([ \; ]\)).

\subsection{}\label{section-27}

En cierta medida, las listas son similares a los arreglos (arrays) en el
lenguaje C.

Una de las diferencias entre ellos es que todos los elementos
pertenecientes a una lista pueden ser de tipo de datos diferente.

\subsection{}\label{section-28}

Los valores almacenados en una lista se pueden acceder utilizando el
operador de división (\([ \; ]\) y \([:]\)) con índices que empiezan en
\(0\) al principio de la lista y opera hasta el final con \(-1\).

\subsection{}\label{section-29}

El signo más (\(+\)) es el operador de concatenación de lista y el
asterisco (\(*\)) es el operador de repetición.

\subsection{}\label{section-30}

\begin{Shaded}
\begin{Highlighting}[]
\NormalTok{milista }\OperatorTok{=}\NormalTok{ [ }\StringTok{'abcd'}\NormalTok{, }\DecValTok{786}\NormalTok{ , }\FloatTok{2.23}\NormalTok{, }\StringTok{'salmon'}\NormalTok{, }\FloatTok{70.2}\NormalTok{ ]}
\NormalTok{listabreve }\OperatorTok{=}\NormalTok{ [}\DecValTok{123}\NormalTok{, }\StringTok{'pizza'}\NormalTok{]}

\BuiltInTok{print}\NormalTok{ (milista)              }\CommentTok{# Presenta la lista completa}
\BuiltInTok{print}\NormalTok{ (milista[}\DecValTok{0}\NormalTok{])          }\CommentTok{# Presenta el primer elemento de la lista}
\BuiltInTok{print}\NormalTok{ (milista[}\DecValTok{1}\NormalTok{:}\DecValTok{3}\NormalTok{])        }\CommentTok{# Presenta los elementos a partir de la 2a posiciona hasta la 3a}
\BuiltInTok{print}\NormalTok{ (milista[}\DecValTok{2}\NormalTok{:])          }\CommentTok{# Presenta los elementos a partir del 3er elemento}
\BuiltInTok{print}\NormalTok{ (listabreve }\OperatorTok{*} \DecValTok{2}\NormalTok{)       }\CommentTok{# Presenta dos veces la lista}
\BuiltInTok{print}\NormalTok{ (milista }\OperatorTok{+}\NormalTok{ listabreve) }\CommentTok{# Presenta la lista concatenada con la segunda lista}
\end{Highlighting}
\end{Shaded}

\begin{center}\rule{0.5\linewidth}{\linethickness}\end{center}

\begin{verbatim}
['abcd', 786, 2.23, 'salmon', 70.2]
abcd
[786, 2.23]
[2.23, 'salmon', 70.2]
[123, 'pizza', 123, 'pizza']
['abcd', 786, 2.23, 'salmon', 70.2, 123, 'pizza']
\end{verbatim}

\subsection{Tuplas}\label{tuplas}

Una tupla es otro tipo de datos de secuencia que es similar a la lista.

\subsection{}\label{section-31}

Una tupla consiste en un número de valores separados por comas. Sin
embargo, a diferencia de las listas, las tuplas se incluyen entre
paréntesis.

\subsection{}\label{section-32}

La principal diferencia entre las listas y las tuplas son:

\begin{enumerate}
\def\labelenumi{\arabic{enumi}.}
\tightlist
\item
  Las listas están entre corchetes \([\;]\) y sus elementos y tamaño
  pueden cambiarse.
\item
  Las tuplas están entre paréntesis \((\;)\) y \emph{no se pueden
  actualizar}.
\end{enumerate}

Las tuplas pueden ser consideradas como listas de sólo lectura.

\subsection{}\label{section-33}

\begin{Shaded}
\begin{Highlighting}[]
\NormalTok{mitupla }\OperatorTok{=}\NormalTok{ ( }\StringTok{'abcd'}\NormalTok{, }\DecValTok{786}\NormalTok{ , }\FloatTok{2.23}\NormalTok{, }\StringTok{'arena'}\NormalTok{, }\FloatTok{70.2}\NormalTok{  )}
\NormalTok{tuplabreve }\OperatorTok{=}\NormalTok{ (}\DecValTok{123}\NormalTok{, }\StringTok{'playa'}\NormalTok{)}

\BuiltInTok{print}\NormalTok{ (mitupla)                }\CommentTok{# Presenta la tupla completa}
\BuiltInTok{print}\NormalTok{ (mitupla[}\DecValTok{0}\NormalTok{])             }\CommentTok{# Presenta el primer elemento de la tupla}
\BuiltInTok{print}\NormalTok{ (mitupla[}\DecValTok{1}\NormalTok{:}\DecValTok{3}\NormalTok{])           }\CommentTok{# Presenta los elementos a partir del 2o elemento hasta la 3o}
\BuiltInTok{print}\NormalTok{ (mitupla[}\DecValTok{2}\NormalTok{:])            }\CommentTok{# Presenta los elementos a partir del 3er elemento}
\BuiltInTok{print}\NormalTok{ (tuplabreve }\OperatorTok{*} \DecValTok{2}\NormalTok{)         }\CommentTok{# Presenta dos veces la tupla}
\BuiltInTok{print}\NormalTok{ (mitupla }\OperatorTok{+}\NormalTok{ tuplabreve)   }\CommentTok{# Preseta la tupla concatenada con la otra tupla}
\end{Highlighting}
\end{Shaded}

\begin{center}\rule{0.5\linewidth}{\linethickness}\end{center}

\begin{verbatim}
('abcd', 786, 2.23, 'arena', 70.2)
abcd
(786, 2.23)
(2.23, 'arena', 70.2)
(123, 'playa', 123, 'playa')
('abcd', 786, 2.23, 'arena', 70.2, 123, 'playa')
\end{verbatim}

\subsection{}\label{section-34}

El siguiente código es inválido con la tupla, porque intentamos
actualizar una tupla, que la acción no está permitida. El caso es
similar con las listas.

\subsection{}\label{section-35}

\begin{Shaded}
\begin{Highlighting}[]
\NormalTok{mitupla }\OperatorTok{=}\NormalTok{ ( }\StringTok{'abcd'}\NormalTok{, }\DecValTok{786}\NormalTok{ , }\FloatTok{2.23}\NormalTok{, }\StringTok{'edificio'}\NormalTok{, }\FloatTok{70.2}\NormalTok{  )}
\NormalTok{milista }\OperatorTok{=}\NormalTok{ [ }\StringTok{'abcd'}\NormalTok{, }\DecValTok{786}\NormalTok{ , }\FloatTok{2.23}\NormalTok{, }\StringTok{'energia'}\NormalTok{, }\FloatTok{70.2}\NormalTok{  ]}
\NormalTok{mitupla[}\DecValTok{2}\NormalTok{] }\OperatorTok{=} \DecValTok{1000}    \CommentTok{# Sintaxis invalida para la tupla}
\NormalTok{milista[}\DecValTok{2}\NormalTok{] }\OperatorTok{=} \DecValTok{1000}     \CommentTok{# Sintaxis invalida para la lista}
\end{Highlighting}
\end{Shaded}

\begin{center}\rule{0.5\linewidth}{\linethickness}\end{center}

\begin{verbatim}
---------------------------------------------------------------------------

TypeError                                 Traceback (most recent call last)

<ipython-input-19-a99f473d7b8f> in <module>()
      1 mitupla = ( 'abcd', 786 , 2.23, 'edificio', 70.2  )
      2 milista = [ 'abcd', 786 , 2.23, 'energia', 70.2  ]
----> 3 mitupla[2] = 1000    # Sintaxis invalida para la tupla
      4 milista[2] = 1000     # Sintaxis invalida para la lista


TypeError: 'tuple' object does not support item assignment
\end{verbatim}

\subsection{}\label{section-36}

Pero en la lista podemos agregar nuevos elementos que se colocan al
final de la misma:

\subsection{}\label{section-37}

\begin{Shaded}
\begin{Highlighting}[]
\BuiltInTok{print}\NormalTok{(milista)}
\NormalTok{milista.append(}\StringTok{'hola'}\NormalTok{)}
\BuiltInTok{print}\NormalTok{(milista)}
\end{Highlighting}
\end{Shaded}

\begin{center}\rule{0.5\linewidth}{\linethickness}\end{center}

\begin{verbatim}
['abcd', 786, 2.23, 'energia', 70.2]
['abcd', 786, 2.23, 'energia', 70.2, 'hola']
\end{verbatim}

\subsection{ Diccionarios}\label{diccionarios}

Los diccionarios de python son de tipo tabla-hash.

Funcionan como arrays asociativos y consisten en pares
\emph{clave-valor}.

\subsection{}\label{section-38}

Una clave de diccionario puede ser casi cualquier tipo de python, pero
suelen ser números o cadenas.

Los valores, por otra parte, pueden ser cualquier objeto arbitrario de
python.

\subsection{}\label{section-39}

Los diccionarios están encerrados por llaves \(\{ \; \}\) y los valores
se pueden asignar y acceder mediante llaves cuadradas \([ \; ]\).

\subsection{}\label{section-40}

\begin{Shaded}
\begin{Highlighting}[]
\NormalTok{fisicos }\OperatorTok{=} \BuiltInTok{dict}\NormalTok{()}

\NormalTok{fisicos }\OperatorTok{=}\NormalTok{\{}
    \DecValTok{1}\NormalTok{ : }\StringTok{"Eistein"}\NormalTok{,}
    \DecValTok{2}\NormalTok{ : }\StringTok{"Bohr"}\NormalTok{,}
    \DecValTok{3}\NormalTok{ : }\StringTok{"Pauli"}\NormalTok{,}
    \DecValTok{4}\NormalTok{ : }\StringTok{"Schrodinger"}\NormalTok{,}
    \DecValTok{5}\NormalTok{ : }\StringTok{"Hawking"}
\NormalTok{\}}

\BuiltInTok{print}\NormalTok{(fisicos)}
\BuiltInTok{print}\NormalTok{ (fisicos.keys())}
\BuiltInTok{print}\NormalTok{ (fisicos.values())}
\NormalTok{fisicos[}\StringTok{"6"}\NormalTok{] }\OperatorTok{=} \StringTok{"Planck"}    \CommentTok{#agrega un nuevo elemento al diccionario, tanto su clave como valor}
\BuiltInTok{print}\NormalTok{(fisicos)}
\end{Highlighting}
\end{Shaded}

\begin{center}\rule{0.5\linewidth}{\linethickness}\end{center}

\begin{verbatim}
{1: 'Eistein', 2: 'Bohr', 3: 'Pauli', 4: 'Schrodinger', 5: 'Hawking'}
dict_keys([1, 2, 3, 4, 5])
dict_values(['Eistein', 'Bohr', 'Pauli', 'Schrodinger', 'Hawking'])
{1: 'Eistein', 2: 'Bohr', 3: 'Pauli', 4: 'Schrodinger', 5: 'Hawking', '6': 'Planck'}
\end{verbatim}

\subsection{Regla para los
identificadores}\label{regla-para-los-identificadores}

Los identificadores son nombres que hacen referencia a los objetos que
componen un programa: \textbf{constantes}, \textbf{variables},
\textbf{funciones}, etc.

\subsection{}\label{section-41}

Reglas para construir identificadores:

\begin{itemize}
\tightlist
\item
  El primer carácter debe ser una letra o el carácter de subrayado
  (guión bajo)
\item
  El primer carácter puede ir seguido de un número variable de dígitos
  numéricos, letras o carácteres de subrayado.
\item
  No pueden utilizarse espacios en blanco, ni símbolos de puntuación.
\item
  En python se distingue de las mayúsculas y minúsculas.
\end{itemize}

\subsection{Palabras reservadas}\label{palabras-reservadas}

No pueden utilizarse palabras reservadas del lenguaje.

\begin{longtable}[]{@{}cccccc@{}}
\toprule
No usar &\tabularnewline
\midrule
\endhead
del & for & is & raise & assert & elif\tabularnewline
from & lamda & return & break & else & global\tabularnewline
not & try & class & except & if & or\tabularnewline
while & continue & exec & import & pass & yield\tabularnewline
def & finally & in & print & del & system\tabularnewline
\bottomrule
\end{longtable}

\end{document}
