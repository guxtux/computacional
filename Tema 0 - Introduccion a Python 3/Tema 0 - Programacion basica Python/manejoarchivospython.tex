\section{Manejo de archivos}
\begin{frame}
\frametitle{Manejo de archivos}
Hemos visto que cuando realizamos una serie de instrucciones con Python, obtenemos una respuesta en la ventana de la terminal. Conforme vayamos complicando los problemas a resolver, la informaci\'{o}n de respuesta que se nos presente en la pantalla, no tendr\'{i}a mucho sentido, ya que en el an\'{a}lisis de esa informaci\'{o}n se tendr\'{i}a que hacer con todo el conjunto de datos.
\\
\bigskip
En todo lenguaje de programaci\'{o}n se requiere el manejo de archivos, ya sea para \textcolor{blue}{escribir} un conjunto de datos, o para \textcolor{green}{leer} los datos contenidos en un archivo y ocuparlos dentro del programa.
\end{frame}
\begin{frame}[fragile]
\frametitle{Ejemplo}
Es atractiva la salida de la siguiente tabla?
\begin{exampleblock}{}
\begin{lstlisting}
for x in range(1,11):
    print x, x*x, x*x*x
\end{lstlisting}
\end{exampleblock}
\pause
\begin{exampleblock}{Usando un formato}
\begin{lstlisting}
for x in range(1,11):
    print '%2d %3d %4d' %(x, x*x, x*x*x)
\end{lstlisting}
\end{exampleblock}
\end{frame}
\begin{frame}[fragile]
\frametitle{Escribiendo datos en un archivo}
Una vez que ya hemos decidido el formato de los datos, ahora los enviaremos a un archivo.
\begin{lstlisting}
miarchivo = open('midatos.dat','w')
for x in range(1,11):
    print '%2d %3d %4d' %(x,x*x,x*x*x)
    miarchivo.write('%2d %3d %4d \n' %(x,x*x,x*x*x))
miarchivo.close()
\end{lstlisting}
Revisa el archivo de datos en la carpeta en donde se guard\'{o}.
\end{frame}
\begin{frame}
\frametitle{Archivos en Python}
Los archivos en Python son objetos de tipo \texttt{file} creados mediante la funci\'{o}n \texttt{open} (abrir). Esta funci\'{o}n toma como par\'{a}metros una cadena con la ruta del archivo a abrir, que puede ser relativa o absoluta; una cadena opcional indicando el modo de acceso (si no se especifica se accede en modo lectura)
\end{frame}
\begin{frame}[fragile]
\frametitle{Modos de acceso}
El modo de acceso puede ser cualquier combinaci\'{o}n l\'{o}gica de los siguientes modos:
\begin{itemize}[<+->]
\item \textbf{r}: read, lectura. Abre el archivo en modo lectura. El archivo tiene que existir previamente, en caso contrario se lanzar\'{a} una excepci\'{o}n de tipo IOError.
\item \textbf{w}: write, escritura. Abre el archivo en modo escritura. Si el archivo no existe se crea. Si existe, sobreescribe el contenido.
\item \textbf{a}: append, añadir. Abre el archivo en modo escritura. Se diferencia del modo 'w' en que en este caso no se sobreescribe el contenido del archivo, sino que se comienza a escribir al final del archivo.
\item \textbf{b}: binary, binario.
\item \textbf{+}: permite lectura y escritura simult\'{a}neas.
\end{itemize}
\end{frame}
\begin{frame}[fragile]
\frametitle{Lectura de archivos}
Para la lectura de archivos se utilizan los métodos \texttt{read}, \texttt{readline} y \texttt{readlines}.
\\
\bigskip
El m\'{e}todo \texttt{read} devuelve una cadena con el contenido del archivo o bien el contenido de los primeros $n$ bytes, si se especifica el tamaño m\'{a}ximo a leer.
\begin{exampleblock}{M\'{e}todo \texttt{read}}
completo = miarchivo.read()
\end{exampleblock}
\pause
\begin{exampleblock}{M\'{e}todo \texttt{read(n)}}
parte = miarchivo.read(512)
\end{exampleblock}
\end{frame}
\begin{frame}[fragile]
El m\'{e}todo \texttt{readline} sirve para leer las l\'{i}neas del archivo una por una. Es decir, cada vez que se llama a \'{e}ste m\'{e}todo, se devuelve el contenido del archivo desde el puntero hasta que se encuentra un car\'{a}cter de nueva l\'{i}nea, incluyendo este car\'{a}cter.
\begin{exampleblock}{M\'{e}todo \texttt{readline()}}
\texttt{while True: \\
\hspace{0.2cm}linea = f.readline() \\
\hspace{0.2cm}if not linea: break \\
\hspace{0.2cm}print line}
\end{exampleblock}
Por \'{u}ltimo, \texttt{readlines}, funciona leyendo todas las l\'{i}neas del archivo y devolviendo una lista con las l\'{i}neas le\'{i}das. 
\end{frame}
\begin{frame}
Para ocupar el contenido de datos de un archivo dentro de un programa con Python, se requiere manejar los arreglos de una manera din\'{a}mica, que veremos en su momento, ya que tendremos que apoyarnos en la librer\'{i}a \texttt{Numpy}.
\end{frame}