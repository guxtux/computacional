\documentclass[12pt]{article}
\usepackage[utf8]{inputenc}
\usepackage[spanish]{babel}
\usepackage{amsmath}
\usepackage{amsthm}
\usepackage{multicol,multienum}
\usepackage{graphicx}
\usepackage{float}
\usepackage{tikz}
\usepackage{color}
\usepackage{anysize}
\usepackage{anyfontsize}
%Este paquete permite manejar los encabezados del documento
\usepackage{fancyhdr}
%hay que definir el ambiente de la página
\pagestyle{fancy}
%aqui va el texto para todas las paginas l--> izquierda, r--> derecha, hay un C--> para centrar el texto deseado
%\lhead{Curso de Física Computacional}
\fancyhead[R]{\nouppercase{\leftmark}}
%define el ancho de la linea que separa el encabezado del cuerpo del texto
\renewcommand{\headrulewidth}{0.5pt}
\usepackage{hyperref}
%esta parte define el color del marco que aparece en las hiperreferencias.
\definecolor{links}{HTML}{2A1B81}
\hypersetup{colorlinks,linkcolor=,urlcolor=links}
\spanishdecimal{.}
\marginsize{1.5cm}{1.5cm}{0cm}{2cm}
\author{M. en C. Gustavo Contreras Mayén.}
\title{Ejercicios del Tema 0 \\ \begin{Large} Curso de Física Computacional\end{Large} \\
\begin{small}
\texttt{curso.fisica.comp@gmail.com}
\end{small}}
\date{ }
\begin{document}
\maketitle
\fontsize{14}{14}\selectfont
Los ejercicios que se indican a continuación, tienen la finalidad de que practiques con lo que hemos revisado en el Tema 0 del curso. No contabilizan para la calificación final, por tanto, el que los resuelvas, ya es un compromiso moral contigo mismo.
\\
\\
Se recomienda altamente que trabajes en ellos, ya que te permitirá identificar en qué partes debes de reforzar y recuperar en cuanto a programación con \texttt{Python} se refiere. Puedes enviar un correo para consultar alguna duda. La idea es que cada problema lo guardes en un archivo \texttt{*.py} y lo identifiques como \texttt{Tema0-Ejercicio1.py}, \texttt{Tema0-Ejercicio2.py}, etc.
\begin{enumerate}
\item Escribe un programa que convierta un valor dado en metros y devuelva el valor correspondiente en: pulgadas, pies, yardas y millas. Como sabemos, una pulgada equivale a 2.54 cm, un pie tiene 12 pulgadas, una yarda tiene 3 pies y una milla tiene 1760 yardas. El valor lo debe de proporcionar el usuario; para verificar tu programa, revisa que 640 metros equivalen a 25196.85 pulgadas.
\item La densidad de una sustancia está definida como $\rho = \frac{m}{V}$, donde $m$ es la masa y $V$ el volumen. Calcula y muestra el valor de la masa que hay en un litro de las siguientes sustancias (tendrás que consultar el valor, donde las unidades de la densidad estén en $\frac{g}{cm^{3}}$): hierro, aire, gasolina, hielo, cuerpo humano, palta y platino.
\item Sea $p$ la tasa de interés de un banco en porciento anual. Una cantidad inicial $A$ ha crecido entonces a
\[ A \left( 1 + \dfrac{p}{100} \right)^{n} \]
luego de $n$ años- Elabora un programa que calcule cuánto dinero se tiene luego de $1000$ pesos iniciales a los tres años con una tasa de interes del $5\%$.
\item Alguien escribió la siguiente línea de código para calcular el valor de $\sin(1)$:
\begin{verbatim}
x=1; print 'sin(%g)=%g' % (x, sin(x))
\end{verbatim}
Ejecuta esta línea en una terminal, luego responde: ¿dónde está el problema?¿cómo se corrige ese problema?
\item Teclea el siguiente programa y ejecútalo. Si el problema no funciona, revisa que lo tengas tal cual.
\begin{verbatim}
from math import pi

h= 5.0 #altura
b= 2.0 #base
r= 1.5 #radio

area_paralelogramo = h*b
print 'El area del paralelogramo es %.3f' % area_paralelogramo

area_cuadrado = b**2
print 'El area del cuadrado es %g' % area_cuadrado

area_circunferencia = pi*r**2
print 'El area de la circunferencia es %8.3f' % area_circunferencia

volumen_cono = 1.0/3*pi*r**2*h
print 'El volumen del cono es %3.f' % volumen_cono
\end{verbatim}
\item Ejecuta los siguientes programas, si no funcionan: identifica y corrige la(s) instrucción(es)
\begin{enumerate}
\item Identidad trigonométrica.
\begin{verbatim}
from math import, sin, cos
x = pi/4
1_val = sin^2(x) + cos^2(x)
print 1_VAL
\end{verbatim}
\item Movimiento con aceleración constante.
\begin{verbatim}
v0 = 3 m/s
t = 1 s
a = 2 m/s**2
s = v0*t + 1/2 a*t**2
print s
\end{verbatim}
\item Verifica las ecuaciones:
\begin{eqnarray*}
(a+b)^{2} = a^{2} + 2ab + b^{2} \\
(a-b)^{2} = a^{2} - 2ab + b^{2}
\end{eqnarray*}
\begin{verbatim}
a= 3,3    b= 5,3
a2= a**2
b2= b**2

eq1_suma = a2 + 2ab + b2
eq2_suma = a2 - 2ab + b2

eq1_pot = (a + b)**2
eq2_pot = (a - b)**2

print 'Primera ecuacion: %g = %g' , %(eq1_suma, eq1_pot)
print 'Segunda ecuacion: %h = %h' , %(eq2_pot, eq2_pot)
\end{verbatim}
\end{enumerate}
\item La función Gaussina
\[ f(x) = \dfrac{1}{\sqrt{2\pi}s} \exp	\left[ - \dfrac{1}{2} \left( \dfrac{x-m}{s} \right)^{2} \right] \]
es una de las funciones más utilizadas en ciencias y tecnología. Los parámetros $m$ y $s$ son números reales, donde $s$ debe ser mayor que cero. Elabora un prograa que evalúe la función con $m=0$, $s=2$ y $x=1$. Verifica tu resultado comparando con lo que obtengas a mano o con una calculadora.
\item A continuación se presenta una línea de código para convertir grados centígrados a Fahrenheit, algunas líneas no funcionan: identifica cuál(es) es(son) y explica por qué.
\begin{enumerate}
\item \verb|C = 21;   F = 9/5*C + 32;      print F |
\item \verb|C = 21.0; F = (9/5)*C + 32;    print F|
\item \verb|C = 21.0; F = 9*C/5 + 32;      print F|
\item \verb|C = 21.0; F = 9.*(C/5.0) + 32; print F|
\item \verb|C = 21.0; F = 9.0*C/5.0 + 32;  print F|
\item \verb|C = 21;   F = 9*C/5 + 32;      print F|
\item \verb|C = 21.0; F = (1/5)*9*C + 32;  print F|
\item \verb|C = 21;   F = (1./5)*9*C + 32; print F|
\end{enumerate}
\end{enumerate}
\end{document}