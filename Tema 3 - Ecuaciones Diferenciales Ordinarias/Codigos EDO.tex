\documentclass[11pt]{article}
\usepackage[utf8]{inputenc}
%\usepackage[latin1]{inputenc}
\usepackage[spanish]{babel}
\usepackage{anysize}
\usepackage{graphicx} 
\usepackage{amsmath}
\usepackage{color}
\usepackage{listings}
\lstset{ %
language=C++,                % choose the language of the code
basicstyle=\footnotesize,       % the size of the fonts that are used for the code
numbers=left,                   % where to put the line-numbers
numberstyle=\footnotesize,      % the size of the fonts that are used for the line-numbers
stepnumber=1,                   % the step between two line-numbers. If it is 1 each line will be numbered
numbersep=5pt,                  % how far the line-numbers are from the code
backgroundcolor=\color{white},  % choose the background color. You must add \usepackage{color}
showspaces=false,               % show spaces adding particular underscores
showstringspaces=false,         % underline spaces within strings
showtabs=false,                 % show tabs within strings adding particular underscores
frame=single,   		% adds a frame around the code
tabsize=2,  		% sets default tabsize to 2 spaces
captionpos=b,   		% sets the caption-position to bottom
breaklines=true,    	% sets automatic line breaking
breakatwhitespace=false,    % sets if automatic breaks should only happen at whitespace
escapeinside={\%}{)}          % if you want to add a comment within your code
}
%\numberwithin{equation}{list}
\marginsize{1cm}{2cm}{0cm}{2cm}  
\title{Ecuaciones diferenciales ordinarias \\ \begin{large}Curso de Física Computacional\end{large}}
\author{M. en C. Gustavo Contreras Mayén}
\date{ }
\begin{document}
\maketitle
\section*{Método Runge-Kutta de segundo orden.}
\begin{enumerate}
\item \textbf{Descripción.}\\
Este programa calcula la solución de la EDO de segundo orden, mediante el método de RK2. Las constantes se definen en las instrucciones \texttt{DATA}, las condiciones iniciales están dadas en \texttt{YB} y \texttt{ZB}. Los resultados obtenidos se imprimen después de cada 50 pasos.
\item \textbf{Variables.}\\
\texttt{Y}, \texttt{Z}: $y$ y z para un nuevo paso. \\
\texttt{YB}, \texttt{ZB}: $y$ y z del paso anterior. \\
\texttt{BM}, \texttt{KM}: a y b de la ecuación a resolver.
\item \textbf{Código.}
\begin{lstlisting}
		PROGRAM masaresorte
		REAL*8 M, K, K1, K2, L1, L2, KM
		DATA T, K, M, B, Z, Y, H /0.0, 100.0, 0.5, 0.0, 0.0, 1.0, 0.001/
		PRINT *, '	T	Y	Z'
		PRINT 1, T, Y, Z
1		FORMAT (F10.5, 1P2E13.6)

		KM=K/M
		BM=B/M

		OPEN(1,FILE='rk2masares.dat',STATUS='UNKNOWN')
		DO 	N = 1, 20
			DO 	KOUNT = 1, 50
				T = T+H
				K1 = H*Z
				L1 = -H*(BM*Z+KM*Y)
				K2 = H*(Z+L1)
				L2 = - H*(BM*(Z+L1)+KM*(Y+K1))
				Y = Y+(K1+K2)/2
				Z = Z+(L1+L2)/2
			END DO
			WRITE(1,*)T, Y, Z
			PRINT 1, T, Y, Z
		END DO
		CLOSE(1)
		END PROGRAM masaresorte
\end{lstlisting}
\end{enumerate}
\section*{Método de Runge-Kutta de cuarto orden.}
\begin{enumerate}
\item \textbf{Descripción.}
Se presenta un programa de Runge-Kutta de cuarto orden para resolver una ecuación diferencial de primer orden. Antes de ejecutar el programa, el usuario debe de definir la EDO a resolver, en el subprograma \texttt{FUN}. Cuando se ejecuta el programa, se le preguntará al usuario el número de pasos I, en el intervalo de impresión \textit{t}, denotado \texttt{TD}. Entonces, el intervalo de tiempo se hace igual a $h = TD/I$. También se le pregunta al usuario, el máximo \textit{t} en el que debe evaluarse la solución.
\item \textbf{Variables} \\
H : intervalo de tiempo, \textit{h} \\
F: $f(y,t)$ \\
K1, K2, K3, K4: $k_{1}$, $k_{2}$, $k_{3}$, $k_{4}$, respectivamente \\
Y : y \\
YA : \textit{y} en el subprograma que define la ecuación diferencial \\
X : \textit{t} \\
XA : \textit{t} de la ecuación diferencial en el subprograma \\
XL : valor máximo de \textit{t} \\
TD : intervalo de impresión de \textit{t} (la solución se imprime después de cada incremento de \textit{t} por TD).
\item \textbf{Código.}
A continuación se indica el código del programa.\\
\begin{lstlisting}
	PROGRAM rk4ecuacion
      REAL K1, K2, K3, K4
      
      PRINT *
1     PRINT *, 'Esquema de Runge-Kutta de cuarto orden'
      PRINT *
      PRINT *, 'Intervalo de impresion de T?'
      READ *, XPR
      PRINT *, 'Numero de pasos en un intervalo de impresion?'
      READ *, I
      PRINT *, 'Maximo?'
      READ *, XL

! AQUI SE FIJA EL VALOR INICIAL DE LA SOLUCION
      Y=473

! H ES EL INTERVALO DE TIEMPO
      H=XPR/I\end{enumerate}

      PRINT *, 'H= ', H

      OPEN (1,FILE='datostemp.dat', STATUS='REPLACE')
! SE INICIALIZA EL TIEMPO
      XP=0
      HH=H/2
      PRINT *
      PRINT *,'--------------------------------------------'
      PRINT *,'	T			Y'
      PRINT *,'--------------------------------------------'
      PRINT *, XP, Y
      82 FORMAT (1X, F10.6, 7X, 1PE15.6)

!  AVANZA I PASOS EN CADA INTERVALO DE IMPRESION
      
30    DO J= 1,I
	  	XB=XP
	  	XP=XP+H
	  	YN=Y
	  	XM=XB+HH
	  	K1=H*FUN(YB,XB)
	  	K2=H*FUN(YN+K1/2,XM)
	  	K3=H*FUN(YN+K2/2,XM)
	  	K4=H*FUN(YN+K3,XP)
	  	Y=YN+(K1+K2*2+K3*2+K4)/6
      END DO

      PRINT 82, XP, Y
      WRITE(1,*) XP, Y
      IF (XP .LE. XL) GOTO 30
      PRINT *
      PRINT *,'Se ha excedido del limite de X'
      PRINT *
      PRINT *
      
      CLOSE(1)
      PRINT *, 'Oprime 1 para continuar, 0 para terminar'
      READ *, K
      IF (K .EQ. 1) GOTO 1
      PRINT *
	END PROGRAM rk4ecuacion
! +++++++++++++++++++++++++++++++++++++
FUNCTION FUN(Y,X)
      rho = 300
      volumen=0.001
      area = 0.25
      caloresp = 900
      ctransferencia = 30
      emisividad = 0.8
      ctestefanbol = 5.67E-8
     
      constante1 = area/(rho*caloresp*volumen)
      constante2 = emisividad*ctestefanbol

      FUN=X*(constante1*(constante2*(bignumerote-Y**4)+ctransferencia*(297-Y)))
      RETURN
END|
!++++++++++++++++++++++++++
FUNCTION bignumerote()
  integer, parameter:: DD = selected_int_kind(15)
     integer(kind = DD):: bignum
     bignum = 297_DD**4
     RETURN
END
\end{lstlisting}
\end{enumerate}
\textbf{Parámetro KIND.}\\
Especifica el tipo de dato. Si no se especifica, se asume el tipo de dato por defecto. Por ejemplo, Fortran 90 provee tres tipos de datos de parámetro \texttt{KIND} para datos de tipo real:\\
\texttt{
REAL(KIND=4) (o REAL*4)\\
REAL(KIND=8) (o REAL*8)\\
REAL(KIND=16) (o REAL*16)\\
}
\textbf{Función: Selected Int Kind(R)}\\
Devuelve el tipo de dato entero adecuado para representar todos los valores $n$ en el rango\\ $-10^{R} < n <10^R$\\
Si no es posible ese tipo de dato, devuelve el valor de $-1$.
\\
\\
La aplicación del método RK4 a un conjunto de ecuaciones diferenciales es análoga del método de segundo orden. Sean un conjunto de dos ecuaciones:
\begin{eqnarray}
y'=f(y,z,t) \nonumber \\
z'=g(y,z,t)
\end{eqnarray}
El método RK4 para este conjunto es:
\begin{eqnarray}
k_{1} & = & hf(y_{n},z_{n},t_{n}) \nonumber \\
l_{1} & = & hg(y_{n},z_{n},t_{n}) \nonumber \\
k_{2} & = & hf \left( y_{n} + \dfrac {k_{1}}{2},z_{n}+ \dfrac {l_{1}}{2},t_{n}+ \dfrac{h}{2}  \right)  \nonumber \\
l_{2} & = & hg \left( y_{n} + \dfrac {k_{1}}{2},z_{n}+ \dfrac {l_{1}}{2},t_{n}+ \dfrac{h}{2}  \right)  \nonumber \\
k_{3} & = & hf \left( y_{n} + \dfrac {k_{2}}{2},z_{n}+ \dfrac {l_{2}}{2},t_{n}+ \dfrac{h}{2}  \right)  \nonumber \\
l_{3} & = & hg \left( y_{n} + \dfrac {k_{2}}{2},z_{n}+ \dfrac {l_{2}}{2},t_{n}+ \dfrac{h}{2}  \right)  \nonumber \\
k_{4} & = & hf \left( y_{n}+k_{3},z_{n}+l_{3},t_{n}+h \right) \nonumber \\
l_{4} & = & hg \left( y_{n}+k_{3},z_{n}+l_{3},t_{n}+h \right) \\
y_{n+1} & = & y_{n} + \dfrac{1}{6} \left[k_{1}+2k_{2}+2k_{3}+k_{4} \right] \\
z_{n+1} & = & z_{n} + \dfrac{1}{6} \left[l_{1}+2l_{2}+2l_{3}+l_{4} \right]
\end{eqnarray}
Incluso cuando el número de ecuaciones en un conjunto es mayor que dos, el método RK4 es esencialmente el mismo.
\\
\textbf{Ejemplo: } Resuelve el siguiente conjunto de EDO mediante el esquema RK4, utilizando $h=0.2 \pi$ y $h=0.05 \pi$
\begin{eqnarray}
y' & = & z, \hspace{0.5cm} y(0)=1 \nonumber \\
z' & = & -y, \hspace{0.5cm} z(0)=0 \nonumber
\end{eqnarray}
\begin{enumerate}
\item \textbf{Descripción:} Este programa resuelve un conjunto de cualquier número de EDO de primer orden. En la subrutina \texttt{FUNCT} se define el conjunto de ecuaciones diferenciales a resolver. En el programa principal se define el número de ecuaciones \texttt{IM}, así como los valores de las condiciones iniciales. Para correr el programa con un nuevo problema, el usuario debe de cambiar las ecuaciones en \texttt{FUNCT}, el valor de \texttt{IM} y las condiciones iniciales.
\item \textbf{Variables:}\\
Y(1) : y \\
Y(2) : z \\
Y(I) : i-ésima incógnita \\
YN(I) : $y_{n}$ para $I=1$ y $z_{n}$ para $I=2$, etc. \\
YA(I) : $y_{n}+\frac{k_{1}}{2}$ o $y_{n}+\frac{k_{2}}{2}$ o $y_{n}+k_{3}$ para $I=1$; $z_{n}+\frac{l_{1}}{2}$ o $z_{n}+\frac{l_{2}}{2}$ o $z_{n}+l_{3}$ para $I=2$\\
EK(J,1), J=1,2,3,4 : $\hspace{0.5cm} k_{1},k_{2},k_{3},k_{4}$ \\
EK(J,2), J=1,2,3,4 : $\hspace{0.5cm} l_{1},l_{2},l_{3},l_{4}$ \\
EK(J,M), J=1,2,3,4 : similar a la anterior para la m-ésima EDO \\
IM : número de ecuaciones en el conjunto \\
NS : número de intervalos de tiempo en un intervalo de impresión \\
XP : límite máximo de t \\
TD : intervalo de impresión para t
\end{enumerate}
\begin{lstlisting}
	PROGRAM RK4_SISTEMA
	DIMENSION YA(0:10), YN(0:10), EK(0:4,0:10), Y(0:10)
	PRINT *
	PRINT *, "      ESQUEMA DE RK4 PARA UN CONJUNTO DE ECUACIONES"
	PRINT *
!numero de ecuaciones	
	IM = 2

	Y(1) = 1	!condicion inicial para y1 en t=0
	Y(2) = 0	!condicion inicial para y2 en t=0

1	PRINT *
	PRINT *, '¿INTERVALO DE IMPRESION DE T?'
	READ *, TD

	PRINT *, '¿NUMERO DE PASOS EN UN INTERVALO DE IMPRESION DE T?'
	READ *, NS

	PRINT *, '¿T MAXIMO PARA DETENER LOS CALCULOS?'
	READ *, XL

	H =  TD/NS

	PRINT *, '  H =  ', H

	XP = 0
	HH = H/2

	PRINT *

	LI = 0		!inicializacion del numero de linea
	PRINT *, ' LINEA     T       Y(1),       Y(2), .....'
	WRITE(*,98) LI, XP, (Y(I),I=1,IM)

28	LI = LI + 1
	DO N = 1, NS
		XB = XP			!tiempo anterior
		XP = XP + H		!tiempo nuevo
		XM = XB + HH	!tiempo en el punto medio
		
		J =  1			!se calcula k1
		DO I = 1, IM
			YA(I) = Y(I)
		END DO
		XA = XB
		CALL FUNCT(EK,J,YA,H)
		
		J = 2			!se calcula k2
		DO I = 1, IM
			YA(I) = Y(I) + EK(1,I)/2
		END DO
		XA = XM
		CALL FUNCT(EK,J,YA,H)

		J = 3			!se calcula k3
		DO I = 1, IM
			YA(I) = Y(I) + EK(2,I)/2
		END DO
		XA = XM
		CALL FUNCT(EK,J,YA,H)

		J = 4			!se calcula k4
		DO I = 1, IM
			YA(I) = Y(I) + EK(3,I)/2
		END DO
		XA = XP
		CALL FUNCT(EK,J,YA,H)

		DO I = 1, IM	!esquema RK4
			Y(I) = Y(I) + (EK(1,I) + EK(2,I)*2 + EK(3,I)*2 + EK(4,I))/6
		END DO
	END DO
	
	WRITE(*,98) LI,XP,(Y(I),I=1,IM)
98	FORMAT(1X,I2,F10.6,2X,1P4E16.6/(15X,1P4E16.6))
	IF (XP .LT. XL) GOTO 28
200	PRINT *
	PRINT *, 'TECLEA 1 PARA CONTINUAR O 0 PARA TERMINAR'
	READ *, K
	IF (K .EQ. 1) GOTO 1
	PRINT *
	END PROGRAM
!++++++++++++++++++++++++++++++++++++++++++++++
	SUBROUTINE FUNCT(EK,J,YA,H)
	DIMENSION EK(0:4,0:10), YA(0:10)

	EK(J,1) =  YA(2)*H
	EK(J,2) = -YA(1)*H
	RETURN
	END
\end{lstlisting}
\end{document}