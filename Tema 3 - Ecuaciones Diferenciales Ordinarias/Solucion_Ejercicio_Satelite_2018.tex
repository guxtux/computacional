\documentclass[12pt]{beamer}
\newenvironment{ConCodigo}[1]
  {\begin{frame}[fragile,environment=ConCodigo]{#1}}
  {\end{frame}}
\graphicspath{{Imagenes/}{../Imagenes/}}
\usepackage[utf8]{inputenc}
\usepackage[spanish]{babel}
\usepackage{hyperref}
\usepackage{etex}
%\reserveinserts{28}
\usepackage{amsmath}
\usepackage{amsthm}
\usepackage{mathtools}
\usepackage{multicol}
\usepackage{multirow}
\usepackage{tabulary}
\usepackage{booktabs}
\usepackage{nccmath}
\usepackage{physics}
\usepackage{biblatex}
\usepackage[outdir=./]{epstopdf}
%\epstopdfsetup{outdir=./}
\usepackage{graphicx}
%\usepackage{enumitem,xcolor}
\usepackage{siunitx}
%\sisetup{scientific-notation=true}
%\usepackage{fontspec}
\usepackage{lmodern}
\usepackage{float}
\usepackage[format=hang, font=footnotesize, labelformat=parens]{caption}
\usepackage[autostyle,spanish=mexican]{csquotes}
\usepackage{standalone}
\usepackage{blkarray}
\usepackage{algorithm}
\usepackage{algorithmic}
\usepackage{tikz}
\usepackage[siunitx, RPvoltages]{circuitikz}
\usetikzlibrary{arrows,patterns,shapes}
\usetikzlibrary{decorations.markings}
\usetikzlibrary{arrows}
\usepackage{color}
\usepackage{xcolor}
%\usepackage{beton}
%\usepackage{euler}
%\usepackage[T1]{fontenc}
\usepackage[sfdefault]{roboto}  %% Option 'sfdefault' only if the base font of the document is to be sans serif
\usepackage[T1]{fontenc}
\renewcommand*\familydefault{\sfdefault}
\DeclareGraphicsExtensions{.pdf,.png,.jpg}
\usepackage{hyperref}
\renewcommand {\arraystretch}{1.5}
\newcommand{\python}{\texttt{python}}
\usefonttheme[onlymath]{serif}
\setbeamertemplate{navigation symbols}{}
\usetikzlibrary{patterns}
\usetikzlibrary{decorations.markings}
\tikzstyle{every picture}+=[remember picture,baseline]
%\tikzstyle{every node}+=[inner sep=0pt,anchor=base,
%minimum width=2.2cm,align=center,text depth=.15ex,outer sep=1.5pt]
%\tikzstyle{every path}+=[thick, rounded corners]
\setbeamertemplate{caption}[numbered]
\newcommand{\ptm}{\fontfamily{ptm}\selectfont}
%Se usa la plantilla Warsaw modificada con spruce
\mode<presentation>
{
  \usetheme{Warsaw}
  \setbeamertemplate{headline}{}
  \useoutertheme{default}
  \usecolortheme{albatross}
  \setbeamercovered{invisible}
}
% \AtBeginSection[]
% {
% \begin{frame}<beamer>{Contenido}
% \normalfont\mdseries
% \tableofcontents[currentsection]
% \end{frame}
% }

\input{../Preambulos/pre_plantilla_Warsaw_whale}
\input{../Preambulos/pre_codigo}
\makeatletter

% \setbeamercolor{subsection in foot}{bg=blue!30!yellow, fg=red}
%\setbeamercolor{footlinecolor}{bg=black,fg=white}
\setbeamertemplate{footline}
{
  \leavevmode%
  \hbox{%
  \begin{beamercolorbox}[wd=.333333\paperwidth,ht=2.25ex,dp=1ex,center]{section in footline}%
    \usebeamerfont{section in foot} \insertsection
  \end{beamercolorbox}}%
  \begin{beamercolorbox}[wd=.333333\paperwidth,ht=2.25ex,dp=1ex,center]{subsection in foot}%
    \usebeamerfont{subsection in foot}  \insertsubsection
  \end{beamercolorbox}%
  \begin{beamercolorbox}[wd=.333333\paperwidth,ht=2.25ex,dp=1ex,right]{date in head/foot}%
    \usebeamerfont{date in head/foot} \insertshortdate{} \hspace*{2em}
    \insertframenumber{} / \inserttotalframenumber \hspace*{2ex} 
  \end{beamercolorbox}}%
  \vskip0pt%
\makeatother
\normalfont
\usepackage{ccfonts}% http://ctan.org/pkg/{ccfonts}
\usepackage[T1]{fontenc}% http://ctan.or/pkg/fontenc
\renewcommand{\rmdefault}{cmr}% cmr = Computer Modern Roman
\linespread{1.3}
\title{Solución al problema del satélite}
\subtitle{Curso de Física Computacional}
\author{M. en C. Gustavo Contreras Mayén}
\date{\today}
\institute{Facultad de Ciencias - UNAM}
\titlegraphic{\includegraphics[width=1.75cm]{Imagenes/escudo-facultad-ciencias.jpg}\hspace*{4.75cm}~%
   \includegraphics[width=1.75cm]{Imagenes/escudo-unam.jpg}
}
%\newcommand{\tikzmark}[1]{\tikz[overlay,remember picture] \node (#1) {};}
\newcommand{\DrawBox}[3][]{%
    \tikz[overlay,remember picture]{
    \draw[black,#1]
      ($(#2)+(-0.5em,2.0ex)$) rectangle
      ($(#3)+(0.75em,-0.75ex)$);}
}
\begin{document}
\maketitle
\fontsize{14}{14}\selectfont
\spanishdecimal{.}
\begin{frame}
\frametitle{El problema del satélite}
\begin{center}
\begin{tikzpicture}[font=\small]
	\draw (2,2) circle (2cm);
	\draw [dashed] (-0.4,2) -- (5,2);
	\draw [->] (2,2) -- node [midway, fill=white] {$R_{t}$} (1.8,4);
	\draw (5,2) circle (0.05);
	\draw (5.2,1.8) node {H}; 
	\draw [->] (5,2) -- node [near end, right] {$v_{0}$}(5,2.5);
	\draw [->] (2,2) -- node [midway, above] {$r$} (4,3.6);
	\draw (2.7,2) arc (0:45:5.9mm) node [right] {$\theta$};
	\draw [dashed] (5,2) arc (0:95:1.8);
\end{tikzpicture}
\end{center}
Un satélite se lanza desde una altitud $H=772$ km sobre el nivel del mar, con una velocidad inicial $v_{0}=6700$ $m/s$ en la dirección que se muestra.
\end{frame}
\begin{frame}
\frametitle{Sistema de EDO-2}
El conjunto de EDO que describen el movimiento del satélite son:
\[ \ddot{r} = r  \dot{\theta}^{2} - \dfrac{G M_{t}}{r^{2}}  \hspace{2cm} \ddot{\theta} = - \dfrac{2 \dot{r}\dot{\theta}}{r}\]
donde $r$ y $\theta$ son las coordenadas polares del satélite.
\end{frame}
\begin{frame}
\frametitle{Constantes del Problema}
Las constantes involucradas en las expresiones, son:
\begin{eqnarray*}
G &=& 6.672 \times 10^{-11} \mbox{ m}^{3} \mbox{kg}^{-1} \mbox{s}^{2} \\
M_{t} &=& 5.9742 \times 10^{24} \mbox{ kg, Masa de la Tierra} \\
R_{e} &=& 6378.14 \mbox{ km, radio de la Tierra al nivel del mar} 
\end{eqnarray*}
\end{frame}
\begin{frame}
\frametitle{Problemas a resolver}
\setbeamercolor{item projected}{bg=purple!80!white,fg=white}
\setbeamertemplate{enumerate items}[circle]
\begin{enumerate}[<+->]
\item Obtén el conjunto de EDO-1 y las condiciones iniciales del problema, de la forma $\dot{\mathbf{y}} = \mathbf{F}(x,\mathbf{y})$, $\mathbf{y}(0) = \mathbf{b}$.
\item Integra las EDO-1 en el tiempo en que se lanza el satélite y choca en su regreso a la Tierra.
\item Calcula el lugar del impacto, con $\theta$.
\end{enumerate}
\end{frame}
\begin{frame}
\frametitle{Inciso 1: Sistema de EDO-1}
Tenemos que
\[ \begin{split} G M_{t} &= (6.672 \times 10^{-11}) (5.9742 \times 10^{24})  = \\
 &= 3.9860 \times 10^{14} \mbox{m}^{3} \mbox{s}^{2} \end{split} \]
Ahora hacemos
\[ \mathbf{y} = \begin{bmatrix}
y_{0} \\
y_{1} \\
y_{2} \\
y_{3}
\end{bmatrix} =
\begin{bmatrix}
r \\
\dot{r} \\
\theta \\
\dot{\theta}
\end{bmatrix}
\]
\end{frame}
\begin{frame}
\frametitle{Inciso 1: Sistema de EDO-1}
Por lo que el conjunto equivalente de EDO-1, resulta ser
\[ \dot{\mathbf{y}} = \begin{bmatrix}
\dot{y}_{0} \\
\dot{y}_{1} \\
\dot{y}_{2} \\
\dot{y}_{3}
\end{bmatrix} =
\begin{bmatrix}
y_{1} \\
y_{0} y_{3}^{2} - 3.9860 \times 10^{14}/y_{0}^{2} \\
y_{3} \\
-2 y_{1} y_{3} / y_{0}
\end{bmatrix} \]
\end{frame}
\begin{frame}
\frametitle{Condiciones iniciales}
Las condiciones iniciales resultan
\fontsize{12}{12}\selectfont
\begin{eqnarray*}
r(0) &=& R_{t} + H = (6378.14+772) \times 10^{3} = 7.15014 \times 10^{6} \mbox{ m} \\
\dot{r}(0) &=& 0 \\
\theta(0) &=& 0 \\
\dot{\theta}(0) &=& \dfrac{v_{0}}{r(0)} = \dfrac{6700}{7.15014 \times 10^{6}} = 0.937045 \times 10^{-3} \mbox{ rad/s}
\end{eqnarray*}
\fontsize{14}{14}\selectfont
\pause
Por tanto
\[ \mathbf{y}(0) = 
\begin{bmatrix}
7.15014 \times 10^{6} \\
0 \\
0 \\
0.937045 \times 10^{3}
\end{bmatrix}
\]
\end{frame}
\begin{frame}
\frametitle{Inciso 2: Integración de las EDO-1}
Usaremos la función \funcionazul{integrate.odeint} para integrar el sistema de EDO-1.
\\
\bigskip
De lo visto en clase, se necesita la función \funcionazul{y, t} que incluye el sistema de EDO-1.
\end{frame}
\begin{frame}
\frametitle{Inciso 2: Integración de las EDO-1}
El período de integración, es decir, el valor del tiempo que tarda en caer el satélite y choca con la Tierra, hay que estimarlo tentativamente.
\end{frame}
\begin{frame}[plain, allowframebreaks, fragile]
\begin{lstlisting}[caption=Solución con python, style=FormattedNumber, basicstyle=\linespread{1.1}\ttfamily=\small, columns=fullflexible]
def F(y, t):
    F = zeros((4), dtype='float64')
    F[_0_] = y[_1_]
    F[_1_] = y[_0_] * (y[_3_]**2) - 3.9860e14 / (y[_0_]**2)
    F[_2_] = y[_3_]
    F[_3_] = -2.0 * y[_1_] * y[_3_]/y[_0_]
    return F

t = np.linspace(0, 1200, 100)
y_0_ = np.array([7.1514e6, 0., 0., 0.937045e-3])
y_1_ = odeint(F, y_0_, t)

print('tiempo \t altura \t velocidad')
print('-'*30)

for i in range(len(t)):
    print('{0:4.4f} \t {1:1.6e} \t {2:1.6e}'.format(t[i], y_1_[i][_0_], y_1_[i][_1_]))
\end{lstlisting}
\end{frame}
\begin{frame}[fragile]
\frametitle{Resultados}
\fontsize{10}{10}\selectfont
\begin{center}
\hspace*{-1.25cm}
\begin{tabular}{c | c | c | c | c }
tiempo & altura ($r$) & velocidad ($\dot{r}$) & $\theta$ & $\dot{\theta}$ \\ \hline
$0.0000$ & $7.151400e+06$ & $0.000000e+00$ & $0.000000e+00$ & $9.370450e-04$\\ \hline
$12.1212$ & $7.151289e+06$ & $-1.835853e+01$ & $1.135824e-02$ & $9.370742e-04$ \\ \hline
$24.2424$ & $7.150955e+06$ & $-3.671583e+01$ & $2.271718e-02$ & $9.371616e-04$ \\ \hline
\ldots & \ldots & \ldots & \ldots & \ldots \\ \hline
$1030.3030$ & $6.386228e+06$ & $-1.396748e+03$ & $1.043431e+00$ & $1.175043e-03$ \\ \hline
$1042.4242$ & $6.369227e+06$ & $-1.408222e+03$ & $1.057712e+00$ & $1.181324e-03$ \\ \hline
\ldots & \ldots & \ldots & \ldots & \ldots \\ \hline
\end{tabular}
\end{center}
\end{frame}
\begin{frame}[fragile]
\frametitle{Intervalo de tiempo al choque}
El satélite choca con la Tierra cuando $r$ es igual a $R_{t}=6.37814 \times 10^{6}$ m. 
\\
\bigskip
De los resultados, vemos que esto ocurre entre el tiempo $t= 1030$ y $1042$ segundos.
\end{frame}
\begin{frame}[fragile]
\frametitle{Intervalo de tiempo al choque}
\fontsize{10}{10}\selectfont
\begin{center}
\hspace*{-1.25cm}
\begin{tabular}{c | c | c | c | c}
tiempo & altura ($r$) & velocidad ($\dot{r}$) & $\theta$ & $\dot{\theta}$ \\ \hline
$0.0000$ & $7.151400e+06$ & $0.000000e+00$ & $0.000000e+00$ & $9.370450e-04$\\ \hline
$12.1212$ & $7.151289e+06$ & $-1.835853e+01$ & $1.135824e-02$ & $9.370742e-04$ \\ \hline
$24.2424$ & $7.150955e+06$ & $-3.671583e+01$ & $2.271718e-02$ & $9.371616e-04$ \\ \hline
\ldots & \ldots & \ldots & \ldots & \ldots \\ \hline
\tikzmark{top left 1}$1030.3030$ & $6.386228e+06$ & $-1.396748e+03$ & $1.043431e+00$ & $1.175043e-03$ \\ \hline
$1042.4242$ & $6.369227e+06$ & $-1.408222e+03$ & $1.057712e+00$ & $1.181324e-03$ \tikzmark{bottom right 1} \\ \hline
\ldots & \ldots & \ldots & \ldots & \ldots \\ \hline
\end{tabular}
\DrawBox[ultra thick, red]{top left 1}{bottom right 1}
\end{center}
\end{frame}
\begin{frame}
\frametitle{Intervalo de tiempo al choque}
Un valor de $t$ más preciso, lo podemos obtener mediante una interpolación  polinomial, pero si no queremos una precisión alta, con una interpolación lineal, bastará.
\\
\medskip
\pause
Consideremos que $1030.3030 + \Delta t$ el tiempo para el impacto, por lo que escribimos
\[ r(1030.3030 + \Delta t) = R_{t} \]
\end{frame}
\begin{frame}[fragile]
\frametitle{Intervalo de tiempo al choque}
Desarrollando $r$ con dos términos de la serie de Taylor, tenemos que
\[ \begin{split} 
r(1030.3030) + \dot{r}(1030.3030) \Delta t &= R_{t} \\
6.386228 \times 10^{6} + (-1.385053 \times 10^{3}) \Delta t &= R_{t}
\end{split} \]
que al despejar, 
\[ \Delta t =  \dfrac{R_{t} - 6.386228 \times 10^{6}}{-1.385053 \times 10^{3}} =  \SI{5.8394}{\second} \]
\pause
\textcolor{blue}{Por lo que el tiempo de impacto es $1036.1424$ segundos}.
\end{frame}
\begin{frame}
\frametitle{Inciso 3) Coordenada de impacto}
La coordenada $\theta$ del impacto, la podemos calcular de una manera similar, para ello desarrollamos dos términos de la serie de Taylor:
\[ \begin{split}
\theta(1030.3030 + \Delta t) &= \theta(1030.3030) + \dot{\theta}(1030.3030)\Delta t \\
 &= 1.043431 + (1.175043 \times 10^{-3} )(5.8394) \\
 &= 1.5029 \mbox{ rad} = 60.18^{\circ}
\end{split} \]
\end{frame}
\end{document}
