\documentclass[letterpaper]{article}
\usepackage[utf8]{inputenc}
%\usepackage[latin1]{inputenc}
\usepackage[spanish]{babel}
\usepackage{geometry}
\usepackage{anysize}
\usepackage{graphicx} 
\usepackage{amsmath}
\usepackage{lscape}
%\numberwithin{equation}{list}
\marginsize{1cm}{2cm}{0cm}{2cm}  
\title{Autoevaluación Tema 3 (Primera parte) \\ \begin{large}Curso de Física Computacional\end{large}}
\author{M. en C. Gustavo Contreras Mayén}
\date{ }
\begin{document}
\begin{landscape}
\maketitle
\fontsize{12}{12}\selectfont
\spanishdecimal{.}
\begin{flushright}
Fecha: \line(1,0){20} / \line(1,0){20} /\line(1,0){20}
\end{flushright}
Nombre:\line(1,0){400}
\\
\\
Para cada uno de los problemas de la tarea-examen, anota la puntuación que consideras lograste.
\begin{center}
\renewcommand{\arraystretch}{2.5}
\begin{tabular}{| l | c | c | c | c | c | c | c | c |}
\hline
 & Documentación & Modularidad & Eficiencia & Robustez & Graficación & Ejecución & Interpretación & Suma \\ \hline
Problema 1 & & & & & & & & \\ \hline
Problema 2 & & & & & & & & \\ \hline
Problema 3 & & & & & & & & \\ \hline
Problema 4 & & & & & & & & \\ \hline
Problema 5 & & & & & & & & \\ \hline
Problema 6 & & & & & & & & \\ \hline
Problema 7 & & & & & & & & \\ \hline
Problema 8 & & & & & & & & \\ \hline


\end{tabular}
\end{center}
\end{landscape}
\end{document}