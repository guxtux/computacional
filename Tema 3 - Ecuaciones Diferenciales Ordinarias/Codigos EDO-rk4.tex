\documentclass[11pt]{article}
\usepackage[utf8]{inputenc}
%\usepackage[latin1]{inputenc}
\usepackage[spanish]{babel}
\usepackage{anysize}
\usepackage{graphicx} 
\usepackage{amsmath}
\usepackage{color}
\usepackage{listings}
\lstset{ %
language=C++,                % choose the language of the code
basicstyle=\footnotesize,       % the size of the fonts that are used for the code
numbers=left,                   % where to put the line-numbers
numberstyle=\footnotesize,      % the size of the fonts that are used for the line-numbers
stepnumber=1,                   % the step between two line-numbers. If it is 1 each line will be numbered
numbersep=5pt,                  % how far the line-numbers are from the code
backgroundcolor=\color{white},  % choose the background color. You must add \usepackage{color}
showspaces=false,               % show spaces adding particular underscores
showstringspaces=false,         % underline spaces within strings
showtabs=false,                 % show tabs within strings adding particular underscores
frame=single,   		% adds a frame around the code
tabsize=2,  		% sets default tabsize to 2 spaces
captionpos=b,   		% sets the caption-position to bottom
breaklines=true,    	% sets automatic line breaking
breakatwhitespace=false,    % sets if automatic breaks should only happen at whitespace
escapeinside={\%}{)}          % if you want to add a comment within your code
}
%\numberwithin{equation}{list}
\marginsize{1cm}{2cm}{0cm}{2cm}  
\title{Ecuaciones diferenciales ordinarias \\ \begin{large}Curso de Física Computacional\end{large}}
\author{M. en C. Gustavo Contreras Mayén}
\date{ }
\begin{document}
\maketitle
\section*{Método de Runge-Kutta de cuarto orden.}
\begin{enumerate}
\item \textbf{ Descripción.}
Se presenta un programa de Runge-Kutta de cuarto orden para resolver una ecuación diferencial de primer orden. Antes de ejecutar el programa, el usuario debe de definir la EDO a resolver, en el subprograma \texttt{FUN}. Cuando se ejecuta el programa, se le preguntará al usuario el número de pasos I, en el intervalo de impresión \textit{t}, denotado \texttt{TD}. Entonces, el intervalo de tiempo se hace igual a $h = TD/I$. También se le pregunta al usuario, el máximo \textit{t} en el que debe evaluarse la solución.
\item \textbf{Variables} \\
H : intervalo de tiempo, \textit{h} \\
F: $f(y,t)$ \\
K1, K2, K3, K4: $k_{1}$, $k_{2}$, $k_{3}$, $k_{4}$, respectivamente \\
Y : y \\
YA : \textit{y} en el subprograma que define la ecuación diferencial \\
X : \textit{t} \\
XA : \textit{t} de la ecuación diferencial en el subprograma \\
XL : valor máximo de \textit{t} \\
TD : intervalo de impresión de \textit{t} (la solución se imprime después de cada incremento de \textit{t} por TD).
\end{enumerate}
\end{document}