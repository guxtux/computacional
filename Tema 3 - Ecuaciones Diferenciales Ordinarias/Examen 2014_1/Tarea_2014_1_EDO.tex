\documentclass[11pt]{article}
\usepackage[utf8]{inputenc}
%\usepackage[latin1]{inputenc}
\usepackage[spanish,es-noshorthands]{babel}
\decimalpoint
\usepackage{anysize}
\usepackage{graphicx} 
\usepackage{amsmath}
\usepackage{booktabs}
\usepackage{tabulary}
\usepackage{nccmath}
\usepackage{float}
\usepackage{tikz}
\usepackage{siunitx}
\usepackage[american,cuteinductors,smartlabels]{circuitikz}
\usetikzlibrary{calc}
\usetikzlibrary{patterns}
\usetikzlibrary{decorations.markings}
\usepackage{pgfplots}
\usepackage{etex}
\usepackage{color}
\usepackage{listings}
\renewcommand{\arraystretch}{1.5}
\lstset{ %
language=Python,                % choose the language of the code
basicstyle=\normalsize,       % the size of the fonts that are used for the code
numbers=left,                   % where to put the line-numbers
numberstyle=\footnotesize,      % the size of the fonts that are used for the line-numbers
stepnumber=1,                   % the step between two line-numbers. If it is 1 each line will be numbered
numbersep=5pt,                  % how far the line-numbers are from the code
backgroundcolor=\color{white},  % choose the background color. You must add \usepackage{color}
showspaces=false,               % show spaces adding particular underscores
showstringspaces=false,         % underline spaces within strings
showtabs=false,                 % show tabs within strings adding particular underscores
frame=single,   		% adds a frame around the code
tabsize=4,  		% sets default tabsize to 2 spaces
captionpos=b,   		% sets the caption-position to bottom
breaklines=true,    	% sets automatic line breaking
breakatwhitespace=false,    % sets if automatic breaks should only happen at whitespace
escapeinside={\#}{)}          % if you want to add a comment within your code
}
\marginsize{1.5cm}{1.5cm}{1cm}{2cm}  
\title{Tarea Ecuaciones Diferenciales Ordinarias \\ Curso de Física Computacional}
\author{M. en C. Gustavo Contreras May\'{e}n}
\date{ }
\begin{document}
\maketitle
\fontsize{14}{14}\selectfont
\begin{enumerate}
\item La ecuaci\'{o}n diferencial del movimiento de un p\'{e}ndulo simple es
\[ \dfrac{d^{2} \theta}{d t^{2}} = - \dfrac{g}{L} \sin \theta \]
donde
$\theta$ es el desplazamiento angular a partir de la vertical, $g$ es la aceleraci\'{o}n debida a la gravedad y $L$ la longitud del p\'{e}ndulo.
\\
Con el cambio de variable $\tau = t \sqrt{g/L}$, la ecuaci\'{o}n toma la forma:
\[ \dfrac{d^{2} \theta}{d \tau^{2}} = -  \sin \theta\]
Resuelve la ecuaci\'{o}n para determinar el per\'{i}odo del p\'{e}ndulo, si la amplitud es $\theta_{0} = 1$ rad. Considera que para pequeñas amplitudes ($\sin \theta \simeq \theta$) el per\'{i}odo es $2 \pi \sqrt{L/g}$.
\item Un paracaidista de masa $m$ en ca\'{i}da libre vertical experimenta una fuerza de arrastre aerodin\'{a}mica $F_{D} = c_{D} \dot{y}^{2}$, donde $y$ se mide hacia abajo a partir del comienzo de la ca\'{i}da. La EDO que describe la ca\'{i}da es
\[ \ddot{y} = g - \dfrac{c_{D}}{m} \dot{y}^{2}\]
Calcula el tiempo para una ca\'{i}da de 500 m, usa los valores de $g=9.80665 \mbox{ m/s}^{2}$, $c_{D}=0.2028 \mbox{ kg/m}$ y $m=80 \mbox{ kg}$.
\item Un sistema masa-resorte est\'{a} en reposo hasta que se le aplica una fuerza $P(t)$, donde
\[ P(t) = \begin{cases}
10 t \mbox{ N para } t < 2 \mbox{ s} \\
20 \mbox{ N para } t \geq 2 \mbox{ s}
\end{cases} \]
\\
\begin{center}
\begin{tikzpicture}[font=\small]
	\tikzstyle{spring}=[thick,decorate,decoration={zigzag,pre length=0.3cm,post
	  length=0.3cm,segment length=6}]
	\draw (0,0) -- (5,0);
	\draw (0,0) -- (0,2);
	\draw [spring] (0,1) -- node [midway, above]{$k$}(2,1);
	\draw (2.5,0.1) circle (0.1);
	\draw (4,0.1) circle (0.1);
	\draw (2,0.2) rectangle node {$m$}(4.3,1.8);
	\draw [->,thick] (4.3,1) -- node [near end, above] {$P(t)$} (5.3,1);
\end{tikzpicture}
\end{center}
La EDO del movimiento resultante es
\[  \ddot{y} = \dfrac{P(t)}{m} - \dfrac{k}{m} y\]
Calcula el desplazamiento m\'{a}ximo de la masa. Usa $m=0.25$ kg y $k=75$ N/m.
\item Un flotador c\'{o}nico se desliza libremente sobre una varilla vertical. Al tocar el flotador, \'{e}ste pierde su posici\'{o}n de equilibrio, y presenta un movimiento oscilante que se describe por la ecuación diferencial:
\[ \ddot{y} = g (1 - a y^{3}) \]
donde $a=16 \mbox{ m}^{3}$ (que est\'{a}n determinadas por la densidad y dimensiones del flotador) y $g=9.80665 \mbox{ m/s}^{2}$. 
\\
\begin{center}
\begin{tikzpicture}[font=\small]
	\draw [pattern = north east lines, draw =none] (0,0) rectangle (6,0.2);
	\draw [fill = blue!10, postaction={pattern = dots}, draw =none] (0,0.2) rectangle (6,3);
	\draw (0,0.2) -- (6,0.2);
	\draw (0,3) -- (6,3);
	\draw [fill = white](2.8,1.2) -- (3.2,1.2) -- (4.5,3.5) -- (1.5,3.5) -- cycle;
	\draw (2.9,0.2) rectangle (3.1,1.2);
	\draw [dashed](2.9,1.2) rectangle (3.1,3.5);
	\draw (2.9,3.5) rectangle (3.1,4.2);
	\draw (2.8,1.2) -- (1,1.2);
	\draw [<->] (1.2,1.2) -- node [midway, fill=white] {$y$} (1.2,3);
	\draw (0.5,4.2) node {Nivel del agua};
	\draw [->] (0.7,4) -- (0.9,3);
\end{tikzpicture}
\end{center}
Si el flotador se eleva a la posici\'{o}n $y=0.1$ m y se libera, determina el per\'{i}odo y la amplitud de las oscilaciones.
\item Un p\'{e}ndulo est\'{a} suspendido en un collar deslizante. El sistema est\'{a} en reposo, posteriormente se aplica al collar un movimiento oscilante $y(t) = Y \sin \omega t$, en $t=0$. La ecuaci\'{o}n diferencial que describe el movimiento del p\'{e}ndulo es
\[ \ddot{\theta} = - \dfrac{g}{L} \sin \theta + \dfrac{\omega^{2}}{L} Y \cos \theta \sin \omega t\]
\begin{center}
\begin{tikzpicture}[font=\small]
	\draw (-0.1,-0.2) [pattern = north east lines] rectangle (0.1,0.4);
	\draw (4.9,-0.2) [pattern = north east lines] rectangle (5.1,0.4);
	\draw (0.1,0.0) rectangle (4.9,0.2);
	\draw (2.1,-0.1) [fill=white] rectangle (2.9,0.3);
	\draw [dashed] (2.5,-1.5) -- (2.5,0.7);
	\draw [->, thick] (2.5,0.6) -- node [right=0.6cm] {$y(t)$} (3.5,0.6);
	\draw [thick] (2.5,0.1) -- node [near end, above=0.5cm]{$L$}(3.51,-2.);
	\draw (3.6,-2.2) circle (0.2) node [right=0.2cm] {$m$};
	\draw (2.5,-0.45) arc (270:290:7mm) node [below=0.15cm]{$\theta$};
\end{tikzpicture}
\end{center}
Grafica $\theta$ contra $t$ en el intervalo de $t=0$ a $t=10$ segundos, as\'{i} mismo, determina el desplazamiento mayor de $\theta$ durante \'{e}ste per\'{i}odo. Usa $g=9.80665$ m/$s^{2}$, $L=1.0$ m, $Y=0.25$ m y $\omega = 2.5$ rad/s.
\item Tenemos un sistema que consiste en un masa que se desliza sobre una barra gu\'{i}a que est\'{a} en reposo, la masa se ubica en $r=0.75$ m. Al tiempo $t=0$ se enciende un motor que proporiona un movimiento dado por la expresi\'{o}n $\theta(t) = (\pi/12) \cos \pi t$ sobre la barra. La EDO que describe el movimiento resultante de la masa deslizante es:
\[ \ddot{r} = \left( \dfrac{\pi^{2}}{12}\right)^{2}  r \sin^{2} \pi t - g \sin \left( \dfrac{\pi}{12} \cos \pi t \right) \]
\begin{center}
\begin{tikzpicture}[font=\small]
	\draw (0,0) [pattern = north east lines] rectangle (2,0.2);
	\draw (0.2,0.2) -- (0.55,1.2);
	\draw (1.8,0.2) -- (1.45,1.2);
	\draw (0.9,0.9) [rotate around={15:(1,1)}]rectangle (6,1.15);
	\draw (1,1) [fill=white] circle (0.5);
	\draw (1,1) circle (0.1);
	\draw (4,0.7) [fill=white, rotate around={15:(1,1)}]rectangle (4.5,1.4);
	\draw (2,1) -- (4,1);
	\draw (3.4,1) arc (0:25:12mm);
	\draw (3.8,1.3) node {$\theta(t)$};
	\draw (0.8,1.6) [rotate around={15:(0.8,1.6)}] -- (0.8,3);
	\draw (4,2.3) [rotate around={15:(4,2.3)}] -- (4,3);
	\draw (5.75,2.5) [rotate around={15:(5.75,2.5)}] -- (5.75,4.3);
	\draw [<->,rotate around={15:(0.7,2)}] (0.7,2)  -- node [rotate=15,midway, fill=white]{$r$}(4,2);
	\draw [<->,rotate around={15:(0.5,2.8)}] (0.5,2.8)  -- node [rotate=15,midway, fill=white]{$2 \mbox{ m}$}(5.5,2.8);
	\end{tikzpicture}
\end{center}
Calcula el tiempo para el cual, la masa deslizante alcanza el extremo final de la barra gu\'{i}a (la punta de la barra). Usa el valor de $g=9.80665 \mbox{ m/s}^{2}$.
\item Una bala de masa $m=0.25$ kg se lanza con una velocidad $v_{0}= 50$ m/s en la direcci\'{o}n que se indica en la figura. Si la fuerza aerodin\'{a}mica de arrastre sobre la bala es $F_{D}= C_{D} v^{3/2}$, las ecuaciones diferenciales que describen el movimiento son:
\[\ddot{x} = - \dfrac{C_{D}}{m} \dot{x} v^{1/2} \hspace{2cm} \ddot{y}= - \dfrac{C_{D}}{m} \dot{y} v^{1/2} - g\]
donde $v=\sqrt{\dot{x}^{2} + \dot{y}^{2}}$, $C_{D} = 0.03 \mbox{ kg/(ms)}^{1/2}$ y $g=9.80665 \mbox{m/s}^{2}$. Calcular el tiempo de vuelo y el alcance $R$
\begin{center}
\begin{tikzpicture}[font=\small]
	\draw (0,0) [pattern = north east lines, draw = none] rectangle (6,0.5);
	\draw (0,0.5) -- node [midway, above] {$R$} (6.3,0.5);
	\draw (1,0.5) -- node [near end, left] {$y$} (1,3);
	\draw [fill=white] (1,0.5) circle (0.1);
	\draw (0.6,0.8) node {$m$};
	\draw [->, rotate around={30:(1,0.5)}, thick] (1,0.5) -- node [near end, above] {$v_{0}$} (3,0.5);
	\draw (2,0.5) arc (0:30:10mm);
	\draw (2.5,0.8) node {$30^{\circ}$};
	\draw (1,0.5) .. controls (3.8,2.2) and (4.6,2.4) .. (5.5,0.5);
%		  (4,2) .. controls (3.9,1.9).. and (4.1,2.1)..(5.5,0.5);
%		  (3.3,1.4) ..controls (4,1.5) and (4.2,1.3) .. (5.5,0.2);%.. controls (3.8,0.3) and (3.5,0.2) .. (5.5,0);
\end{tikzpicture}
\end{center}
\item La soluci\'{o}n al problema
\[ y'' + \dfrac{1}{x} y' + y \hspace{1.5cm} y(0)=1 \hspace{1cm} y'(0)=0\]
es la funci\'{o}n de Bessel $J_{0}(x)$. Integra num\'{e}ricamente para calcular $J_{0}(5)$ y compara el resultado con $-0.17760$, que es el valor que se obtiene de tablas matem\'{a}ticas. Tip: para evitar la singularidad en $x=0$, inicia la integraci\'{o}n en $x=10^{-12}$.
\end{enumerate} 
\end{document}