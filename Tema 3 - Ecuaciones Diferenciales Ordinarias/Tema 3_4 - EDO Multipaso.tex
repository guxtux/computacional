\documentclass[12pt]{beamer}
\newenvironment{ConCodigo}[1]
  {\begin{frame}[fragile,environment=ConCodigo]{#1}}
  {\end{frame}}
\graphicspath{{Imagenes/}{../Imagenes/}}
\usepackage[utf8]{inputenc}
\usepackage[spanish]{babel}
\usepackage{hyperref}
\usepackage{etex}
%\reserveinserts{28}
\usepackage{amsmath}
\usepackage{amsthm}
\usepackage{mathtools}
\usepackage{multicol}
\usepackage{multirow}
\usepackage{tabulary}
\usepackage{booktabs}
\usepackage{nccmath}
\usepackage{physics}
\usepackage{biblatex}
\usepackage[outdir=./]{epstopdf}
%\epstopdfsetup{outdir=./}
\usepackage{graphicx}
%\usepackage{enumitem,xcolor}
\usepackage{siunitx}
%\sisetup{scientific-notation=true}
%\usepackage{fontspec}
\usepackage{lmodern}
\usepackage{float}
\usepackage[format=hang, font=footnotesize, labelformat=parens]{caption}
\usepackage[autostyle,spanish=mexican]{csquotes}
\usepackage{standalone}
\usepackage{blkarray}
\usepackage{algorithm}
\usepackage{algorithmic}
\usepackage{tikz}
\usepackage[siunitx, RPvoltages]{circuitikz}
\usetikzlibrary{arrows,patterns,shapes}
\usetikzlibrary{decorations.markings}
\usetikzlibrary{arrows}
\usepackage{color}
\usepackage{xcolor}
%\usepackage{beton}
%\usepackage{euler}
%\usepackage[T1]{fontenc}
\usepackage[sfdefault]{roboto}  %% Option 'sfdefault' only if the base font of the document is to be sans serif
\usepackage[T1]{fontenc}
\renewcommand*\familydefault{\sfdefault}
\DeclareGraphicsExtensions{.pdf,.png,.jpg}
\usepackage{hyperref}
\renewcommand {\arraystretch}{1.5}
\newcommand{\python}{\texttt{python}}
\usefonttheme[onlymath]{serif}
\setbeamertemplate{navigation symbols}{}
\usetikzlibrary{patterns}
\usetikzlibrary{decorations.markings}
\tikzstyle{every picture}+=[remember picture,baseline]
%\tikzstyle{every node}+=[inner sep=0pt,anchor=base,
%minimum width=2.2cm,align=center,text depth=.15ex,outer sep=1.5pt]
%\tikzstyle{every path}+=[thick, rounded corners]
\setbeamertemplate{caption}[numbered]
\newcommand{\ptm}{\fontfamily{ptm}\selectfont}
%Se usa la plantilla Warsaw modificada con spruce
\mode<presentation>
{
  \usetheme{Warsaw}
  \setbeamertemplate{headline}{}
  \useoutertheme{default}
  \usecolortheme{albatross}
  \setbeamercovered{invisible}
}
% \AtBeginSection[]
% {
% \begin{frame}<beamer>{Contenido}
% \normalfont\mdseries
% \tableofcontents[currentsection]
% \end{frame}
% }

\include{pre_codigo}
\title{Ecuaciones diferenciales ordinarias 3\\ Métodos multipasos}
\subtitle{Curso de Física Computacional}
\author[]{M. en C. Gustavo Contreras Mayén}
\begin{document}
\maketitle
\fontsize{14}{14}\selectfont
\spanishdecimal{.}
\begin{frame}{Contenido}
\tableofcontents[pausesections]
\end{frame}
\section{Métodos multipaso}
\begin{frame}
\frametitle{Métodos multipaso}
Los métodos que hemos explicado hasta ahora se llaman \emph{métodos de un paso}, por que la aproximación del punto $x_{i+1}$ contiene información proveniente de un solo de los puntos anteriores $x_{i}$.
\\
\medskip
Aunque estas técnicas pueden usar la información relativa a la evaluación de funciones en los puntos entre $x_{i}$ y $x_{i+1}$, no la conservan para utilizarla directamente en aproximaciones futuras.
\\
\medskip
Toda la información que emplean se obtiene dentro del intervalo en el que va a aproximarse la función.
\end{frame}
\begin{frame}
Como la solución aproximada $w_{j}$ está disponible en cada uno de los puntos $x_{0}, x_{1},\ldots,x_{i}$ antes de obtener la aproximación en $x_{i+1}$ y como el error $\vert w_{j} - y(x_{j}) \vert$ tiende a aumentar con $j$, para razonable desarrollar métodos que usen estos datos precedentes más precisos al aproximar la solución en $x_{i+1}$.
\\
\medskip
Se conoce como \emph{métodos multipaso} a aquellos que emplean la aproximación en más de uno de los puntos precedentes para determinar la aproximación en el siguiente punto.
\end{frame}
\subsection{Método multipaso de paso $m$}
\begin{frame}
\frametitle{Método multipaso de paso $m$}
Un \emph{Método multipaso de paso $m$} para resolver el problema con valores iniciales
\[ y' = f(x,y),\hspace{0.5cm} a \leq x \leq b, \hspace{0.5cm} y(a) = \alpha \]
tiene una ecuación en diferencias para obtener la aproximación $w_{i+1}$ en el punto $x_{i+1}$, representada por la siguiente expresión, donde $m>1$:
\[ \begin{split}
w_{i+1} &= a_{m-1} w_{i} + a_{m-2 }w_{i-1} + \ldots + a_{0} w_{i+1-m} + \\ 
&+ h[b_{m} f(x_{i+1},w_{i+1}) + b_{m-1} f(x_{i},w_{i}) + \\
&+ \ldots + b_{0} f(x_{i+1-m}, w_{i+1-m})]
\end{split} \]
\end{frame}
\begin{frame}
\[ \begin{split}
w_{i+1} &= a_{m-1} w_{i} + a_{m-2 }w_{i-1} + \ldots + a_{0} w_{i+1-m} + \\ 
&+ h[b_{m} f(x_{i+1},w_{i+1}) + b_{m-1} f(x_{i},w_{i}) + \\
&+ \ldots + b_{0} f(x_{i+1-m}, w_{i+1-m})]
\end{split} \]
para $i=m-1,m,\ldots,N-1$, donde $h=(b-a)/N$, las $a_{0},a_{1},\ldots,a_{m-1}$ y $b_{0}, b_{1},\ldots,b_{m}$ son constantes y los valores iniciales
\[ w_{0} = \alpha, \hspace{0.2cm} w_{1} = \alpha_{1}, \hspace{0.2cm} w_{2} = \alpha_{2}, \hspace{0.2cm}, \ldots, \hspace{0.2cm} w_{m-1} = \alpha_{m-1} \]
están especificadas.
\end{frame}
\begin{frame}
Cuando $b_{m}=0$, el método es \textbf{explícito} o \textbf{abierto}, ya que la ecuación anterior da entonces el valor de $w_{i+1}$ de manera explícita en términos de los valores previamente determinados.
\\
\medskip
\pause
Cuando $b_{m} \neq 0$, el método es llamado \textbf{implícito} o \textbf{cerrado}, ya que el valor de $w_{i+1}$ se encuentra en ambos lados de la ecuación y se especifica sólo implícitamente.
\end{frame}
\subsection{Método de Adams-Bashforth de 4 orden}
\begin{frame}
\frametitle{Método de Adams-Bashforth de 4 orden}
Las ecuaciones
\[ \begin{split}
w_{i+1} &= w_{i} + \dfrac{h}{24} [ 55 f(x_{i},w_{i}) - 59 f(x_{i-1},w_{i-1}) + \\
&+ 37 f(x_{i-2},w_{i-2}) - 9 f(x_{i-3},w_{i-3}) ]
\end{split} \]
para cada $i=3,4,\ldots, N-1$ definen un método explícito de cuatro pasos llamado \textbf{método de Adams-Bashforth de cuarto orden}.
\end{frame}
\end{document}