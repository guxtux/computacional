\documentclass[12pt]{beamer}
\newenvironment{ConCodigo}[1]
  {\begin{frame}[fragile,environment=ConCodigo]{#1}}
  {\end{frame}}
\graphicspath{{Imagenes/}{../Imagenes/}}
\usepackage[utf8]{inputenc}
\usepackage[spanish]{babel}
\usepackage{hyperref}
\usepackage{etex}
%\reserveinserts{28}
\usepackage{amsmath}
\usepackage{amsthm}
\usepackage{mathtools}
\usepackage{multicol}
\usepackage{multirow}
\usepackage{tabulary}
\usepackage{booktabs}
\usepackage{nccmath}
\usepackage{physics}
\usepackage{biblatex}
\usepackage[outdir=./]{epstopdf}
%\epstopdfsetup{outdir=./}
\usepackage{graphicx}
%\usepackage{enumitem,xcolor}
\usepackage{siunitx}
%\sisetup{scientific-notation=true}
%\usepackage{fontspec}
\usepackage{lmodern}
\usepackage{float}
\usepackage[format=hang, font=footnotesize, labelformat=parens]{caption}
\usepackage[autostyle,spanish=mexican]{csquotes}
\usepackage{standalone}
\usepackage{blkarray}
\usepackage{algorithm}
\usepackage{algorithmic}
\usepackage{tikz}
\usepackage[siunitx, RPvoltages]{circuitikz}
\usetikzlibrary{arrows,patterns,shapes}
\usetikzlibrary{decorations.markings}
\usetikzlibrary{arrows}
\usepackage{color}
\usepackage{xcolor}
%\usepackage{beton}
%\usepackage{euler}
%\usepackage[T1]{fontenc}
\usepackage[sfdefault]{roboto}  %% Option 'sfdefault' only if the base font of the document is to be sans serif
\usepackage[T1]{fontenc}
\renewcommand*\familydefault{\sfdefault}
\DeclareGraphicsExtensions{.pdf,.png,.jpg}
\usepackage{hyperref}
\renewcommand {\arraystretch}{1.5}
\newcommand{\python}{\texttt{python}}
\usefonttheme[onlymath]{serif}
\setbeamertemplate{navigation symbols}{}
\usetikzlibrary{patterns}
\usetikzlibrary{decorations.markings}
\tikzstyle{every picture}+=[remember picture,baseline]
%\tikzstyle{every node}+=[inner sep=0pt,anchor=base,
%minimum width=2.2cm,align=center,text depth=.15ex,outer sep=1.5pt]
%\tikzstyle{every path}+=[thick, rounded corners]
\setbeamertemplate{caption}[numbered]
\newcommand{\ptm}{\fontfamily{ptm}\selectfont}
%Se usa la plantilla Warsaw modificada con spruce
\mode<presentation>
{
  \usetheme{Warsaw}
  \setbeamertemplate{headline}{}
  \useoutertheme{default}
  \usecolortheme{albatross}
  \setbeamercovered{invisible}
}
% \AtBeginSection[]
% {
% \begin{frame}<beamer>{Contenido}
% \normalfont\mdseries
% \tableofcontents[currentsection]
% \end{frame}
% }

\input{../Preambulos/pre_plantilla_Warsaw_whale}
\input{../Preambulos/pre_codigo}
\makeatletter

% \setbeamercolor{subsection in foot}{bg=blue!30!yellow, fg=red}
%\setbeamercolor{footlinecolor}{bg=black,fg=white}
\setbeamertemplate{footline}
{
  \leavevmode%
  \hbox{%
  \begin{beamercolorbox}[wd=.333333\paperwidth,ht=2.25ex,dp=1ex,center]{section in footline}%
    \usebeamerfont{section in foot} \insertsection
  \end{beamercolorbox}}%
  \begin{beamercolorbox}[wd=.333333\paperwidth,ht=2.25ex,dp=1ex,center]{subsection in foot}%
    \usebeamerfont{subsection in foot}  \insertsubsection
  \end{beamercolorbox}%
  \begin{beamercolorbox}[wd=.333333\paperwidth,ht=2.25ex,dp=1ex,right]{date in head/foot}%
    \usebeamerfont{date in head/foot} \insertshortdate{} \hspace*{2em}
    \insertframenumber{} / \inserttotalframenumber \hspace*{2ex} 
  \end{beamercolorbox}}%
  \vskip0pt%
\makeatother
\normalfont
\newcommand{\highlight}[2][yellow]{\mathchoice%
  {\colorbox{#1}{$\displaystyle#2$}}%
  {\colorbox{#1}{$\textstyle#2$}}%
  {\colorbox{#1}{$\scriptstyle#2$}}%
  {\colorbox{#1}{$\scriptscriptstyle#2$}}}%
\usepackage{ccfonts}% http://ctan.org/pkg/{ccfonts}
\usepackage[T1]{fontenc}% http://ctan.or/pkg/fontenc
\renewcommand{\rmdefault}{cmr}% cmr = Computer Modern Roman
\linespread{1.3}
\title{EDO tipo Sturm-Liouville}
\subtitle{Curso de Física Computacional}
\author{M. en C. Gustavo Contreras Mayén}
\date{\today}
\institute{Facultad de Ciencias - UNAM}
\titlegraphic{\includegraphics[width=1.75cm]{Imagenes/escudo-facultad-ciencias.jpg}\hspace*{4.75cm}~%
   \includegraphics[width=1.75cm]{Imagenes/escudo-unam.jpg}
}
\begin{document}
\maketitle
\fontsize{14}{14}\selectfont
\spanishdecimal{.}
\section*{Contenido}
\frame{\tableofcontents[currentsection, hideallsubsections]}
\section{Ecuaciones de Sturm-Liouville}
\frame{\tableofcontents[currentsection, hideothersubsections]}
\subsection{Definición}
\begin{frame}
\frametitle{EDO con valores propios}
Diversos problemas con condiciones en la frontera conducen (mediante el método de separación de variables) a la misma ecuación diferencial ordinaria
\[ X^{\prime \prime}(x) + \lambda \: X(x) = 0, \hspace{1cm} (0 < x < L)\]
con el valor propio $\lambda$
\end{frame}
\begin{frame}
\frametitle{EDO con valores propios}
\[ X^{\prime \prime}(x) + \lambda \: X(x) = 0, \hspace{1cm} (0 < x < L)\]
con el valor propio $\lambda$.
\\
\bigskip
Pero con distintas condiciones en los extremos, según las condiciones de frontera:
\setbeamercolor{item projected}{bg=purple!70!black,fg=white}
\setbeamertemplate{enumerate items}[circle]
\begin{enumerate}[<+->]
\item $X(0) = X(L) = 0$
\item $X^{\prime}(0) = X^{\prime}(L) = 0$
\item $X(0) = X^{\prime}(L) = 0$
\end{enumerate}
\end{frame}
\begin{frame}
\frametitle{Ejemplo de una varilla}
Por ejemplo, en el problema de hallar la temperatura $u(x, t)$ de una varilla $0 \leq x \leq L$ con la temperatura inicial dada $u(x,0)=f(x)$.
\end{frame}
\begin{frame}
\frametitle{Ejemplo de una varilla}
Como problema con valores en la frontera, este problema es igual al problema de determinar la temperatura dentro de una lámina de gran tamaño que ocupe la región $0 \leq x \leq L$ en el espacio $x \ y \ z$.
\end{frame}
\begin{frame}
\frametitle{Ejemplo de una varilla}
Si la temperatura inicial de la varilla depende sólo de $x$ y es independiente de $y$, $z$ (es decir, si $u(x,0)= f(x)$, entonces lo mismo será cierto de su temperatura $u=u(x,y)$ en el instante $t$. 
\end{frame}
\begin{frame}
\frametitle{Ejemplo de una varilla}
Sustituyendo
\[u(x, t) = X(x) \; T(t)\]
en la ecuación de calor
\[ \dfrac{\partial u}{\partial t} = k \dfrac{\partial^{2} u}{\partial x^{2}}\]
\end{frame}
\begin{frame}
\frametitle{Condiciones en los extremos}
Vemos que $X(x)$ satisface la condiciones en los extremos, si las caras $x=0$ y $x=L$ de la lámina se mantienen a temperatura cero.
\end{frame}
\begin{frame}
\frametitle{Condiciones en los extremos}
Las condiciones de $X^{\prime}(0) = X^{\prime}(L) = 0$ si ambas caras están aisladas, y las de $X(0) = X^{\prime}(L) = 0$, si una cara está aislada y la otra se mantiene a temperatura cero.
\end{frame}
\begin{frame}
\frametitle{Condiciones en los extremos}
Pero si cada cara de la varilla pierde calor hacia el medio ambiente (que se encuentra a temperatura cero), de acuerdo con la ley de enfriamiento de Newton, entonces las condiciones en los extremos asumen la forma
\[h \; X(0) - X^{\prime}(0) = 0 = h \; X(L) + X^{\prime}(L)\]
donde $h$ es un coeficiente de transferencia de calor no negativo.
\end{frame}
\begin{frame}
\frametitle{Problema de valores propios}
Al imponer diversas condiciones en los extremos sobre la solución del problema, obtenemos distintos problemas de valores propios, y por ello usamos distintos valores propios ${\lambda_{n}}$ y distintas funciones propias $X_{n}(x)$ en la construcción de una solución formal en términos de una serie de potencias
\end{frame}
\begin{frame}
\frametitle{Problema de valores propios}
\[ u(x, t) = \sum c_{n} \: X_{n}(x) \: T_{n}(x)\]
del problema con valores en la frontera. El paso final en esta construcción es la elección del coeficiente ${c_{n}}$ en la ecuación anterior de modo que
\[u(x,0) = \sum c_{n} \: T_{n}(0) \: X_{n}(x)= f(x) \]
\end{frame}
\begin{frame}
\frametitle{Problema de valores propios}
Por lo que necesitamos un desarrollo en términos de funciones propias de la función dada $f(x)$, en términos de las funciones propias del problema con valores en los extremos correspondientes.
\end{frame}
\subsection{Problemas de Sturm-Liouville}
\begin{frame}
\frametitle{Problemas de Sturm-Liouville}
Para unificar y generalizar el método de separación de variables, es útil formular un tipo general de problema de valores propios que incluya como casos particulares a los ya mencionados.
\end{frame}
\begin{frame}
\frametitle{Problemas de Sturm-Liouville}
La ecuación inicial, con $y$ en vez de $X$ como variable dependiente, se puede escribir como
\[ \dfrac{d}{dx} \left[ p(x) \dfrac{dy}{dx} \right] - q(x) y + \lambda r(x) y = 0\]
donde $p(x)=r(x) \equiv 1$ y $q(x) \equiv 0$
\end{frame}
\begin{frame}
\frametitle{Problemas de Sturm-Liouville}
Podemos asegurar que casi cualquier EDO-2 lineal de la forma
\[ A(x) \: y^{\prime \prime} + B(x) \: y^{\prime} + C(x) \: y + \lambda \: D(x) \: y = 0\]
asume la forma indicada después de multiplicarla por un factor adecuado.
\end{frame}
\begin{frame}
\frametitle{Ejemplo}
Si multiplicamos la ecuación paramétrica de Bessel de orden $n$
\[ x^{2} \: y^{\prime \prime} + x \: y^{\prime} + (\lambda \: x^{2} - n^{2}) \: y = 0, \hspace{1cm} x > 0\]
por $1/x$, podemos escribir el resultado como
\[ \dfrac{d}{dx} \left[ x \: \dfrac{dy}{dx} \right] - \dfrac{n^{2}}{x} \: y + \lambda \: x \: y = 0\]
que tiene la forma de Sturm-Liouville (S-L), con $p(x) = r(x) = x$ y $q(x) = n^{2}/x$
\end{frame}
\begin{frame}
\frametitle{Condiciones de frontera}
Imponiendo sobre las soluciones de la ecuación anterior, en un intervalo abierto acotado $(a,b)$ las siguientes condiciones -lineales- homogéneas en los extremos:
\end{frame}
\begin{frame}
\frametitle{Condiciones de frontera}
\[ \begin{split}
\alpha_{1} \: y(a) - \alpha_{2} \: y^{\prime}(a) =& 0 \\
\beta_{1} \: y(b) - \beta_{2} \: y^{\prime}(b) =& 0
\end{split} \]
donde los coeficientes $\alpha_{1},\alpha_{2},\beta_{1},\beta_{2}$ son constantes. 
\end{frame}
\begin{frame}
\frametitle{Condiciones de frontera}
Además de ser homogéneas, las condiciones están \textit{separadas}, en el sentido de que una de ellas implica los valores de $y(x)$ y $y'(x)$ en un extremo $x=a$, mientras que la otra implica los valores en el otro extremo $x=b$.
\\
\bigskip
Nótese que las condiciones $y(a) = y^{\prime}(b) = 0$ son de la forma dada, con $\alpha_{1} = \beta_{2} = 1$ y $\alpha_{2} = \beta_{1} = 0$
\end{frame}
\subsection{Definición de un problema Sturm-Liouville}
\begin{frame}
\frametitle{Definición de un problema Sturm-Liouville}
Un problema de Sturm-Liouville es un problema con valores en la frontera de la forma
\[\dfrac{d}{dx} \left[ p(x) \: \dfrac{dy}{dx} \right] - q(x) \: y + \lambda \: r(x) \: y = 0, \hspace{1cm} a < x < b \]
\[ \begin{split}
\alpha_{1} \: y(a) - \alpha_{2} \: y^{\prime}(a) =& 0 \\
\beta_{1}  \: y(b) - \beta_{2} \: y^{\prime}(b) =& 0
\end{split} \]
donde $\alpha_{1}$ y $\alpha_{2}$ como $\beta_{1}$ y $\beta_{2}$ son diferentes de cero.
\end{frame}
\begin{frame}
\[\dfrac{d}{dx} \left[ p(x) \: \dfrac{dy}{dx} \right] - q(x) \: y + \lambda \: r(x) \: y = 0, \hspace{1cm} a < x < b \]
\frametitle{Definición de un problema Sturm-Liouville}
El parámetro $\lambda$ es el \textit{eingenvalor} cuyos posibles valores (constantes) se buscan.
\end{frame}
\begin{frame}
\frametitle{Ejemplo}
Se obtienen diferentes problemas de Sturm-Liouville complementando la ecuación diferencial
\[ y^{\prime \prime} + \lambda \: y = 0 \hspace{1cm} 0 < x < L\]
con alguna de las diferentes condiciones de valores en la frontera homogéneas:
\end{frame}
\begin{frame}
\frametitle{Ejemplo}
\setbeamercolor{item projected}{bg=purple!70!black,fg=white}
\setbeamertemplate{enumerate items}[circle]
\begin{enumerate}[<+->]
\item $y(0) = y(L) = 0$, donde $\alpha_{1} = \beta_{1} = 1$ y $\alpha_{2} = \beta_{2} = 0$
\item $y^{\prime}(0) = y^{\prime}(L) = 0$, donde $\alpha_{1} = \beta_{1} = 0$ y $\alpha_{2} = \beta_{2} = 1$
\item $y(0) = y^{\prime}(L) = 0$, donde $\alpha_{1} = \beta_{2} = 1$ y $\alpha_{2} = \beta_{1} = 0$
\end{enumerate}
\end{frame}
\begin{frame}
\frametitle{Solución no trivial}
Nótese que el problema de una EDO tipo S-L siempre tiene la solución trivial $y \equiv 0$.
\\
\bigskip
Por lo que se buscan los valores de $\lambda$ (eingenvalores) para los cuales este problema tiene una solución real \textit{no trivial} (una eigenfunción) y cada eigenvalor cuenta con su eigenfunción asociada (o eigenfunciones).
\end{frame}
\begin{frame}
\frametitle{Solución no trivial}
Puede verse que cualquier constante (diferente de cero) múltiplo de una eigenfunción será también una eigenfunción.
\end{frame}
\subsection{Eigenvalores de Sturm-Liouville}
\begin{frame}
\frametitle{Eigenvalores de Sturm-Liouville}
Supongamos que las funciones $p(x), p^{\prime}(x), q(x)$ y $r(x)$ de la ecuación S-L son continuas en el intervalo $[a,b]$ y que tanto $p(x) > 0$ como $r(x) > 0$ en cada punto de $[a, b]$.
\end{frame}
\begin{frame}
\frametitle{Eigenvalores de Sturm-Liouville}
De este modo los eigenvalores del problema de S-L, constituyen una sucesión creciente
\[ \lambda_{1} < \lambda_{2} < \lambda_{3} < \ldots < \lambda_{n-1} < \lambda_{n} < \ldots\]
de números reales, con
\[ \lim_{n \rightarrow \infty} \lambda_{n} = + \infty\]
Salvo por un factor constante, solo una eigenfunción $y_{n}(x)$ se asocia con cada eigenvalor $\lambda_{n}$.
\end{frame}
\begin{frame}
\frametitle{Propiedad de los eigenvalores}
Además, si $q(x) \geq 0$ en $[a,b]$ y los coeficientes $\alpha_{1},\alpha_{2},\beta_{1},\beta_{2}$ en la definición de S-L, son todos no negativos, entonces, los eigenvalores son todos no negativos.
\end{frame}
\begin{frame}
\frametitle{Propiedad de los eigenvalores}
Algunas veces el problema de S-L se llama \textbf{regular} si se satisface el resultado anterior, en caso contrario, es \textbf{singular}.
\end{frame}
\subsection{Ejemplo: Una varilla delgada}
\begin{frame}
\frametitle{Una varilla delgada de metal}
Sea una varilla delgada de metal con longitud $H$, sus extremos están conectados a distintas fuentes de calor:
\begin{center}
\begin{tikzpicture}[font=\small]
\draw [fill=blue!20](0,0) rectangle node {$T_{L}$} (2,4);
\draw (2,1.8) rectangle (3.5,2.2);
\draw (4,1.8) rectangle (5.5,2.2);
\draw [fill=blue!20] (5.5,0) rectangle node {$T_{R}$} (7.5,4);
\draw [<->](2,1.3) -- node [midway, below] {$x$}(3.5,1.3);
\draw [dashed] (3.5,2.2) -- (3.5,1.25);
\draw [dashed] (4,2.2) -- (4,1.25);
\draw [<-](4,1.3) -- node [midway, below] {$dx$}(4.6,1.3);
\draw [<->] (2,0.1) -- node [midway, above] {$H$} (5.5,0.1);
\draw (3.5,4) node {$T_{\infty}$};
\end{tikzpicture}
\end{center}
\end{frame}
\begin{frame}
\frametitle{Una varilla delgada de metal}
Si el calor sale de la superficie de la varilla únicamente por transferencia de calor, por medio de convección, la ecuación de temperatura es:
\[ -A \: \dfrac{d}{dx} \: k(x) \: \dfrac{d}{dx} \: T(x) + h_{c} \: P \: T(x) = h_{c} \: P \: T_{\infty} + A \: S(x)\]
donde
\end{frame}
\begin{frame}
\frametitle{Una varilla delgada de metal}
\setbeamercolor{item projected}{bg=purple!70!black,fg=white}
\setbeamertemplate{enumerate items}[circle]
\begin{enumerate}[<+->]
\item $T(x)$ es la temperatura del punto que se encuentra a una distancia $x$ del extremo izquierdo.
\item $A$ es el área constante de una sección transversal de la varilla.
\item $k$ es la conductividad térmica.
\item $P$ es el perímetro de la varilla.
\item $h_{c}$ es el coeficiente de transferencia de calor por convección.
\item $T_{\infty}$ es la temperatura neta del aire.
\item $S$ es la fuente de calor.
\end{enumerate}
\end{frame}
\begin{frame}
\frametitle{CDF del problema}
Las condiciones de frontera son:
\[ \begin{split} T(0) =& T_{L} \\
T(H) =& T_{R}
\end{split} \]
Si $T^{0}$ se define como:
\[ T^{0} = T - T_{\infty} \]
\end{frame}
\begin{frame}
\frametitle{Ecuación de temperatura}
La ecuación de temperatura la podemos expresar como:
\begin{equation*}
\highlight[blue!30]{- \dfrac{d}{dx} \: k(x) \dfrac{d}{dx} \: T^{0}(x)} +  \highlight[red!30]{h_{c} \: \dfrac{P}{A} \: T^{0}(x)} =  \boxed{S(x)}
 \end{equation*}
\textcolor{blue}{El primer término representa la difusión del calor} \pause, \textcolor{red}{el segundo es la pérdida de calor en el aire por medio de la convección} y el lado derecho es la fuente de calor.
\end{frame}
\begin{frame}
\frametitle{Otro ejemplo}
Otro ejemplo de una EDO con forma similar es la ecuación de difusión de neutrones dada por:
\[ - \dfrac{d}{dx} \: D(x) \dfrac{d}{dx} \: \Psi (x) + \Sigma_{a} \: \Psi (x) = S(x) \]
Donde $\Psi$ es el flujo de neutrones, $D$ es el coeficiente de difusión y $S$ es la fuente de neutrones.
\end{frame}
\begin{frame}
\frametitle{Otro ejemplo}
\[ - \dfrac{d}{dx} \: D(x) \dfrac{d}{dx} \: \Psi (x) + \Sigma_{a} \: \Psi (x) = S(x) \]
El primer término indica la difusión de neutrones, el segundo la pérdida por absorción y el lado derecho es la fuente de neutrones.
\end{frame}
\begin{frame}
\frametitle{Problemas de difusión}
Considerando en otros casos dentro de la física para problemas con difusión, si se expresa en términos de:
\[- \dfrac{d}{dx} p(x) \dfrac{d}{dx} \phi (x) + q(x) \phi (x) = S(x)  \]
siendo ésta, una ley de conservación de la difusión.
\end{frame}
\begin{frame}
\frametitle{Solución a la ecuación}
Integrando la ecuación anterior en el intervalo $[a,b]$, se obtiene que:
\[ Z(b) - Z(a) + \int_{b}^{a} \: q(x) \: \phi (x) dx = \int_{b}^{a} S(x) dx \]
donde
\[ Z(x) = - p(x) \: \dfrac{d}{dx} \: \phi (x) \]
\end{frame}
\subsection{Problemas con CDF para varillas y láminas}
\begin{frame}
\frametitle{Problemas con CDF para varillas y láminas}
Consideremos una EDO-2 con CDF
\[ - \phi^{\prime \prime} (x) + q \: \phi (x) = S(x), \hspace{1cm} 0 < x < H \]
con condiciones de frontera:
\setbeamercolor{item projected}{bg=purple!70!black,fg=white}
\setbeamertemplate{enumerate items}[circle]
\begin{enumerate}[<+->]
\item $\phi^{\prime}(0) = 0$, condición de frontera izquierda.
\item $\phi^{\prime}(H) = \phi^{\prime}_{R}$, condición de frontera derecha
\end{enumerate}
\end{frame}
\begin{frame}
\frametitle{Malla para la solución}
Si dividimos el dominio en $N$ intervalos de igual longitud, se obtiene una malla donde los intervalos miden $h = H/N$
\begin{center}
\begin{tikzpicture}[font=\small]
\draw (0,1) node {$\phi'=0$};
\draw (7,1) node {$\phi'=\phi_{R}$};
\draw (-1.3,0) -- (8,0);
\foreach \x in {-1,...,7}
	\draw [fill=red!25](\x,0) circle (0.05);
\draw (-1.3,-0.5) node {$x=-h$};
\draw (0,-0.5) node {$0$};
\draw (1,-0.5) node {$h$};
\draw (2,-0.5) node {$2h$};
\draw (3,-0.5) node {$3h$};
\draw (7,-0.5) node {$Nh=H$};
\draw (-1.3,-1) node {$i=0$};
\draw (0,-1) node {$1$};
\draw (1,-1) node {$2$};
\draw (2,-1) node {$3$};
\draw (3,-1) node {$4$};
\draw (6,-1) node {$N$};
\draw (7,-1) node {$N=N+1$};
\end{tikzpicture}
\end{center}
\end{frame}
\begin{frame}
\frametitle{Aproximación por diferencias}
Usando una aproximación por diferencias centrales al primer término de la EDO de segundo orden, obtenemos la ecuación en diferencias para
la $i$-ésima retícula:
\[ \dfrac{(-\phi_{i-1} + 2 \phi_{i} - \phi_{i+1})}{h^{2}} + q \phi_{i} = S_{i} \]
donde $\phi_{i}=\phi(x_{i})$, $S_{i}=S(x_{i})$ y $q$ es constante.
\end{frame}
\begin{frame}
\frametitle{Aproximación por diferencias}
Al multiplicar por $h^{2}$
\[ - \phi_{i-1} + (2-w) \phi_{i} - \phi_{i+1} = h^{2}S_{i} \]
donde $w=qh^{2}$.
\\
\bigskip
Esta ecuación se puede aplicar a todos los puntos de la retícula, excepto cuando $i = 1$ e $i = N+1$.
\end{frame}
\begin{frame}
\frametitle{Condición adiabática}
La condición de la frontera izquierda $\phi'(0) = 0$, es equivalente a una condición simétrica en la frontera llamada condición adiabática en la frontera en el caso de la transferencia de calor.
\end{frame}
\begin{frame}
\frametitle{Condición adiabática}
Si se considera un punto hipotético de la retícula $i = 0$ localizado en $x = -h$, la ecuación anterior en el caso $i = 1$ es:
\[-\phi_{0} + (2+w)\phi_{1} - \phi_{2} = h^{2}S_{1} \]
\end{frame}
\begin{frame}
\frametitle{Uso de la simetría del problema}
En esta ecuación
\[-\phi_{0} + (2+w)\phi_{1} - \phi_{2} = h^{2}S_{1} \]
podemos hacer $\phi_{0} = \phi_{2}$ debido a la simetría. 
\end{frame}
\begin{frame}
\frametitle{Uso de la simetría del problema}
Dividiendo entre dos, obtenemos lo siguiente:
\[ (1 + \dfrac{w}{2}) \phi_{1} - \phi_{2} = \dfrac{1}{2} h^{2}S_{1}\]
como $\phi_{N+1}= \phi(H)=\phi_{R}$ en la frontera derecha, la ecuación con $i=N$ es:
\[ -\phi_{N+1} + (2+w) \phi_{N} = h^{2}S_{N} + \phi_{R}\]
\end{frame}
\begin{frame}[fragile]
\frametitle{Arreglando los términos}
Ordenando los términos anteriores:
\\
\medskip
%\fontsize{10}{10}\selectfont
\scalebox{0.65}{
\begin{equation*}
\begin{matrix}
(1+\frac{w}{2})\phi_{1} & -\phi_{2}      &                &           &            & & = h^{2}\frac{S_{1}}{2} \\
-\phi_{1}               & +(2 + w)\phi_{2} & -\phi_{3}      &           &            & & = h^{2} \: S_{2} \\
                        & \phi_{2}       & +(2 + w) \: \phi_{3} & -\phi_{4} &            & & = h^{2} \: S_{3} \\
                        &                &                &           &            & & \ldots \\
                        &                &                &           &            & & \ldots \\
                        &                &                &           & -\phi_{ N + 1} & +(2 + w) \: \phi_{N} & = h^{2} \: S_{N} + \phi_{R} 
\end{matrix}
\end{equation*}
}
\end{frame}
\begin{frame}[fragile]
\frametitle{Representación matricial}
Que en forma matricial, resulta
\\
\bigskip
\hspace*{-0.5cm}
\scalebox{0.85}{
\begin{equation*}
\begin{bmatrix}
1+w/2 & -1 & 0 & 0 & 0 & 0 \\
-1 & 2+w & -1 & 0 & 0 & 0 \\
0 & -1 & 2+w & -1 & 0 & 0 \\
0 & 0 & 0 + \ddots & -1 & 0 & 0 \\
0 & 0 & 0 & 0 & -1 & 2+w
\end{bmatrix} 
\begin{bmatrix}
\phi_{1} \\
\phi_{2} \\
\phi_{3} \\
\ddots \\
\phi_{N} \\
\end{bmatrix} =
\begin{bmatrix}
h^{2}S_{1}/2 \\
h^{2}S_{2} \\
h^{2}S_{3} \\
\ddots \\
h^{2}S_{N}+\phi_{R} \\
\end{bmatrix}
\end{equation*}}
\end{frame}
\subsection{Solución de sistemas tridiagonales}
\begin{frame}
\frametitle{Solución de sistemas tridiagonales}
Una ecuación tridiagonal la podemos escribir de la siguiente forma:
\[
\begin{bmatrix}
B_{1} & C_{1} & 0 & 0 & 0 & 0 \\
A_{2} & B_{2} & C_{2} & 0 & 0 & 0 \\
0 & A_{3} & B_{3} & C_{3} & 0 & 0 \\
0 & 0 & 0 & \ddots & 0 & 0 \\
0 & 0 & 0 & 0 & A_{n}& B_{n} 
\end{bmatrix}
\begin{bmatrix}
\phi_{1} \\
\phi_{2} \\
\phi_{3} \\
\ddots \\
\phi_{N} \\
\end{bmatrix} =
\begin{bmatrix}
D_{1} \\
D_{2} \\
D_{3} \\
\ddots \\
D_{N} \\
\end{bmatrix}
\]
\end{frame}
\begin{frame}
\frametitle{Solución numérica}
El algoritmo de solución para esta matriz es el siguiente:
\setbeamercolor{item projected}{bg=purple!70!black,fg=white}
\setbeamertemplate{enumerate items}[circle]
\begin{enumerate}[<+->]
\item Se inicializan dos nuevas variables: $B^{\prime}_{1} = B_{1}$ y $D^{\prime}_{1} = D_{1}$.
\item Se calculan de forma recursiva las siguientes ecuaciones, de $i$ hasta $N$:
\[ \begin{split} R &= \dfrac{A_{i}}{B^{\prime}_{i-1}} \\
B^{\prime}_{i} &= B_{i} -R \: C_{i - 1} \\
D^{\prime}_{i} &= D_{i} - R \: D^{\prime}_{i - 1}, \hspace{1cm} i = 2, 3, \ldots, N
\end{split} \]
\seti
\end{enumerate}
\end{frame}
\begin{frame}
\frametitle{Solución numérica}
\setbeamercolor{item projected}{bg=purple!70!black,fg=white}
\setbeamertemplate{enumerate items}[circle]
\begin{enumerate}[<+->]
\conti
\item Se calcula la solución para la última incógnita
\[ \phi_{i} = \dfrac{(D^{\prime}_{i} - C_{i} \: \phi_{i + 1})}{B^{\prime}_{i}}, \hspace{1.5 cm} i = N - 1, N - 2, \ldots, 1 \]
\end{enumerate}
\end{frame}
\begin{frame}
\frametitle{Ejercicio}
Determinar las ecuaciones en diferencias y su solución para el siguiente problema con valores en la frontera:
\[ -2y^{\prime \prime}(x) + y(x) = e^{-0.2 \: x} \]
\end{frame}
\begin{frame}
\frametitle{Ejercicio}
con las siguientes condiciones de frontera
\[ \begin{split} 
y(0) &= 1 \\
y^{\prime}(10) &= -y(10)
\end{split} \]
Supongamos que los intervalos de la malla tienen longitud unitaria.
\end{frame}
\begin{frame}
\frametitle{Solución}
En la siguiente figura se muestra la malla
\begin{center}
\begin{tikzpicture}[font=\small]
\draw (-1.3,0) -- (8,0);
\foreach \x in {-1,...,7}
	\draw [fill=red!25](\x,0) circle (0.05);
\draw (-1.3,-0.5) node {$x=0$};
\draw (0,-0.5) node {$1$};
\draw (1,-0.5) node {$2$};
\draw (6,-0.5) node {$9$};
\draw (7,-0.5) node {$x=10$};
\draw (-1.3,-1) node {$i=0$};
\draw (0,-1) node {$1$};
\draw (1,-1) node {$2$};
\draw (6,-1) node {$9$};
\draw (7,-1) node {$10$};
\end{tikzpicture}
\end{center}
\end{frame}
\begin{frame}
\frametitle{Ecuaciones en diferencias}
Las ecuaciones en diferencias para $i = 1, 2, \ldots, 9$, son:
\[ 2(-y_{i - 1} + 2 \: y_{i} - y_{i + 1}) + y_{i} = e^{-0.2 \: i}, \hspace{1cm} x_{i} = i \]
\end{frame}
\begin{frame}
\frametitle{Ecuaciones en diferencias}
Para $i=1$, sustituimos la condición de frontera $y_{0}=y(0)=1$ en las ecuaciones anteriores, y resulta que:
\[ 5 \: y_{1} - 2 \: y_{2} = e^{-0.2} + 2\]
\end{frame}
\begin{frame}
\frametitle{Ecuaciones en diferencias}
Para $i = 10$, aproximamos la ecuación diferencial por:
\[ - \dfrac{2 \: (y^{\prime}(10)- y^{\prime}(9.5))}{\dfrac{1}{2}} + y(10) = e^{-2}\]
\end{frame}
\begin{frame}
\frametitle{Aproximación a un valor}
Por medio de la aproximación por diferencias centrales, el término $y^{\prime}(9.5)$ es
\[ y^{\prime}(9.5) = \dfrac{y(10) - y(9)}{1}\]
\pause
Sustituimos el resultado anterior y la condición de frontera $y^{\prime}(10) = -y(10)$, para obtener
\[ -2 \: y_{9} + 4.5 \: y_{10} =  0.5 \:  e^{-2} \]
\end{frame}
\begin{frame}
\frametitle{Conjunto de ecuaciones}
Las ecuaciones en diferencias son entonces
\[ \begin{split}
5 \: y_{1} - 2 \: y_{2} &= e^{-0.2} + 2 \\
-2 \: y_{i - 1} + 5 \: y_{i} - 2 \: y_{i + 1} &= e^-0.2 \: x_{i}, \hspace{1cm} i = 2, \ldots, 9 \\
-2 \: y_{9} + 4.5 \: y_{10} &= 0.5 \: e^{-2}
\end{split} \]
Donde $x_{i} = i$
\end{frame}
\begin{frame}[fragile]
\frametitle{La matriz tridiagonal es}
\scalebox{0.8}{
\begin{equation*}
\begin{bmatrix}
5 & -2 & 0 & 0 & 0 & 0 & 0 & 0 & 0 & 0 \\
-2 & 5 & -2 & 0 & 0 & 0 & 0 & 0 & 0 & 0 \\
0 & -2 & 5 & -2 & 0 & 0 & 0 & 0 & 0 & 0 \\
0 & 0 & -2 & 5 & -2 & 0 & 0 & 0 & 0 & 0 \\
0 & 0 & 0 & -2 & 5 & -2 & 0 & 0 & 0 & 0 \\
0 & 0 & 0 & 0& -2 & 5 & -2 & 0 & 0 & 0 \\
0 & 0 & 0 & 0 & 0 & -2 & 5 & -2 & 0 & 0 \\
0 & 0 & 0 & 0 & 0 & 0 & -2 & 5 & -2 & 0 \\
0 & 0 & 0 & 0 & 0 & 0 & 0 & -2 & 5 & -2 \\
0 & 0 & 0 & 0 & 0 & 0 & 0 & 0 & -2 & 4.5 
\end{bmatrix}
\begin{bmatrix}
y_{1} \\
y_{2} \\
y_{3} \\
y_{4} \\
y_{5} \\
y_{6} \\
y_{7} \\
y_{8} \\
y_{9} \\
y_{10} \\
\end{bmatrix} =
\begin{bmatrix}
e^{-0.2} + 2 \\
e^{-0.2 \: x_{2}} \\
e^{-0.2 \: x_{3}} \\
e^{-0.2 \: x_{4}} \\
e^{-0.2 \: x_{5}} \\
e^{-0.2 \: x_{6}} \\
e^{-0.2 \: x_{7}} \\
e^{-0.2 \: x_{8}} \\
e^{-0.2 \: x_{9}} \\
0.5 \:e^{-2}
\end{bmatrix}
\end{equation*}
}
\end{frame}
\begin{frame}
\frametitle{Pensando en el código}
Qué tenemos que hacer para implementar un código con \python{} que nos resuelva el problema?
\\
\medskip
\setbeamercolor{item projected}{bg=purple!70!black,fg=white}
\setbeamertemplate{enumerate items}[circle]
\begin{enumerate}[<+->]
\item Debemos de crear el arreglo tridiagonal.
\item Establecer las condiciones de frontera en el arreglo.
\item Usar una rutina que resuelva el sistema tridiagonal.
\item Presentar los resultados.
\end{enumerate}
\end{frame}
\begin{frame}[fragile]
\frametitle{Paso 1: Creando el arreglo tridiagonal.}
\begin{lstlisting}[caption=Definiendo el arreglo, style=FormattedNumber, basicstyle=\linespread{1.1}\ttfamily=\small, columns=fullflexible]
a = np.zeros(n)
b = np.zeros(n)
c = np.zeros(n)
d = np.zeros(n)

for i in range(n):
    a[i] = -2.0
    b[i] = 5.
    d[i] = np.exp(-0.2 * (i+_1_))

c = a.copy()
\end{lstlisting}
\end{frame}
\begin{frame}[fragile]
\frametitle{Paso 1: Creando el arreglo tridiagonal.}
Toma en cuenta que:
\setbeamercolor{item projected}{bg=purple!70!black,fg=white}
\setbeamertemplate{enumerate items}[circle]
\begin{enumerate}[<+->]
\item No están definidas las condiciones de frontera.
\item Las diagonales $a$ y $c$ son idénticas, que es un caso particular.
\end{enumerate}
\end{frame}
\begin{frame}[fragile]
\frametitle{Paso 2: Definiendo las CDF}
Toma en cuenta que los índices en \python{} comienzan en cero.
\begin{lstlisting}[caption=Estableciendo las CDF, style=FormattedNumber, basicstyle=\linespread{1.1}\ttfamily=\small, columns=fullflexible]
b[_9_] = 4.5
d[_0_] = np.exp(-0.2) + 2
d[_9_] = 0.5 * np.exp(-2)
\end{lstlisting}
\pause
De esta manera se sustituyen los valores que en la diapositiva anterior se indicaron para todos los elementos del arreglo.
\end{frame}
\begin{frame}[plain, allowframebreaks, fragile]
\frametitle{Paso 3: Resolviendo el sistema tridiagonal}
\begin{lstlisting}[caption=La función TRDG, style=FormattedNumber, basicstyle=\linespread{1.1}\ttfamily=\small, columns=fullflexible]
def TRDG(a,b,c,d):
    nf = len(d)
   
    for it in xrange(1, nf):
        mc = a[it]/b[it-_1_]
        b[it] = b[it] - mc * c[it-_1_] 
        d[it] = d[it] - mc * d[it-_1_]

    xc = a
    xc[-_1_] = d[-_1_]/b[-_1_]

    for il in xrange(nf-2, -1, -1):
         xc[il]= (d[il] - c[il] * xc[il+_1_])/b[il]
        
    del b, c, d 
    return xc
\end{lstlisting}
\end{frame}
\begin{frame}[fragile]
\frametitle{Veamos la función por partes}
\begin{lstlisting}[style=FormattedNumber, basicstyle=\linespread{1.1}\ttfamily=\small, columns=fullflexible]
    for it in xrange(1, nf):
        mc = a[it]/b[it-_1_]
        b[it] = b[it] - mc * c[it-_1_] 
        d[it] = d[it] - mc * d[it-_1_]
\end{lstlisting}
En esta parte lo que se resuelve es el cambio de los valores de los coeficientes del arreglo tridiagonal, como se explicó anteriomente.
\end{frame}
\begin{frame}[fragile]
\frametitle{Veamos la función por partes}
\begin{lstlisting}[style=FormattedNumber, basicstyle=\linespread{1.1}\ttfamily=\small, columns=fullflexible]
    xc = a
    xc[-_1_] = d[-_1_]/b[-_1_]

    for il in xrange(nf-_2_, -_1_, -_1_):
         xc[il]= (d[il] - c[il] * xc[il+_1_])/b[il]
        
    del b, c, d 
    return xc
\end{lstlisting}
\end{frame}
\begin{frame}[fragile]
\frametitle{Veamos la función por partes}
Se crea el arreglo \textoazul{xc} a partir de una copia del arreglo \textoazul{a}.
\\
\bigskip
Luego se calcula el último valor de la solución, para ello, se usa el índice \texttt{xc[-1]}.
\end{frame}
\begin{frame}[fragile]
\frametitle{Veamos la función por partes}
El otro ciclo \funcionazul{for} realiza la sustitución hacia atrás, por ello, usamos un paso negativo ya que el valor inicial es mayor que el valor final.
\\
\bigskip
Se eliminan los arreglos \texttt{b},\texttt{c} y \texttt{d}. Se devuelve el arreglo \texttt{xc} con la solución.
\end{frame}
\begin{frame}[plain]
\frametitle{Solución}
Implementando el código en \python, tenemos los siguientes valores
\\
\medskip
\begin{minipage}{4cm}
\begin{tabular}{c | c}
punto & solución \\ \hline
$0$ & $1.00000000$ \\
$1$ & $0.84643489$ \\
$2$ & $0.70672186$ \\
$3$ & $0.58520973$ \\
$4$ & $0.48189665$ \\
$5$ & $0.39486742$ \\
\end{tabular}
\end{minipage}
\hspace{0.5cm}
\begin{minipage}{4cm}
\begin{tabular}{c | c}
punto & solución \\ \hline
$6$ & $0.32133217$ \\
$7$ & $0.25786591$ \\
$8$ & $0.20003411$ \\
$9$ & $0.14127111$ \\
$10$ & $0.07049423$ \\
 & 
\end{tabular}
\end{minipage}
\end{frame}
\end{document}