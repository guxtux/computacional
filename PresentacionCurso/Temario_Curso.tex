\documentclass[12pt]{article}
\usepackage[utf8]{inputenc}
\usepackage[spanish]{babel}
\usepackage{amsmath}
\usepackage{amsthm}
\usepackage{graphicx}
\usepackage{color}
\usepackage{float}
\usepackage{multicol}
\usepackage{enumerate}
\usepackage{anyfontsize}
\usepackage{anysize}
\marginsize{1.5cm}{1.5cm}{-1cm}{2cm}
%\setlength{\parskip}{21pt}
%\linespread{1.3}
\author{M. en C. Gustavo Contreras May\'{e}n.\\
F\'{i}s. Ludmila Palomino Mendoza.}
\title{Curso de F\'{i}sica Computacional\\{\large Grupo 8142}}
\date{ }
\begin{document}
%\renewcommand\theenumii{\arabic{theenumii.enumii}}
\renewcommand\labelenumii{\theenumi.{\arabic{enumii}}}
\maketitle
\fontsize{12}{12}\selectfont
\textbf{Objetivos:}
El prop\'{o}sito del curso es enseñar al estudiante las ideas de computabilidad usadas en distintas \'{a}reas de la  f\'{i}sica para resolver un conjunto de problemas modelo. A partir de planteamientos anal\'{i}ticos se pretende obtener resultados num\'{e}ricos reproducibles consistentes, y que predigan situaciones f\'{i}sicas asociadas al problema bajo estudio.
\\
\\
El alumno debe asimilar las ideas b\'{a}sicas del análisis num\'{e}rico, como son las de estabilidad en el c\'{a}lculo y la sensibilidad de las respuestas a las perturbaciones en la estructura del problema.El curso tambi\'{e}n le dar\'{a} al estudiante capacidad de juicio sobre la calidad de los resultados num\'{e}ricos obtenidos. En particular se har\'{a} \'{e}nfasis en la confiabilidad de los resultados respecto a los errores tanto del algoritmo de soluci\'{o}n como de las limitaciones num\'{e}ricas de la computadora. Esta capacidad se adquirir\'{a} a lo largo del curso comparando resultados num\'{e}ricos con otros tipos de an\'{a}lisis, en las regiones en las cuales se pueden llevar ambos a cabo. Por otra parte permitir\'{a} al estudiante explorar regiones de comportamiento f\'{i}sico sólo accesibles al c\'{a}lculo num\'{e}rico.
\\
\\
\textbf{Lugar: }Laboratorio de Enseñanza en Cómputo de Física, Edificio Tlahuizcalpan.
\\
\textbf{Horario: } Martes y Jueves de 18 a 21 horas.
\\
\\
\section{Metodolog\'{i}a de Enseñanza}
\textbf{Antes de la clase.}
\\
\\
Para facilitar la discusi\'{o}n en el aula, el alumno revisar\'{a} el material de trabajo que se le proporcionar\'{a} oportunamente, de tal manera que ya llegar\'{a} conociendo el tema a desarrollar a la clase. Damos por entendido de que el alumno realizar\'{a} la lectura.
\\
\\
\textbf{Durante la clase}
\\
\\
Se dar\'{a} un tiempo para la exposici\'{o}n con di\'{a}logo y discusi\'{o}n del material de trabajo con los temas a cubrir durante el semestre. Se busca que sea un curso totalmente pr\'{a}ctico por lo que se va a trabajar con los equipos de c\'{o}mputo del laboratorio.
\\
\\
\textbf{Después de la clase}
\\
\\
El curso requiere que le dediquen al menos el mismo n\'{u}mero de horas de trabajo en casa, es decir, les va a demandar seis horas como m\'{i}nimo; si cuentan con una experiencia en programaci\'{o}n, tienen un paso adelantado, pero si no han programado, se ver\'{a}n en la necesidad de dedicarle m\'{a}s tiempo.
\\
\\
Las t\'{e}cnicas de programaci\'{o}n que vayan adquiriendo ser\'{a}n el reflejo de su trabajo fuera de clase. En caso de no trabajar o dedicarle el tiempo al curso, se complicar\'{a} bastante, situaci\'{o}n que esperamos no se presente.

\section{¿Programaci\'{o}n?}
El planteamiento de los algoritmos y la soluci\'{o}n de un problema, se expondr\'{a} de manera general, propiamente \textcolor{blue}{NO es un curso de programación bajo alg\'{u}n lenguaje en particular}, es importante que cuenten con conocimientos de programaci\'{o}n b\'{a}sicos en alg\'{u}n lenguaje com\'{u}n.
\\
\\
En el curso utilizaremos Python para programar, se dar\'{a} un breviario de programaci\'{o}n con Python b\'{a}sico como Tema 0, que no ser\'{a} evaluado ni formar\'{a} parte de la calificaci\'{o}n final.
\\
\\
Cuentan con la completa libertad de elegir el lenguaje o software para trabajar durante el curso:
\begin{multicols}{2}
\begin{itemize}
\item Fortran
\item Java
\item C++
\item C
\item Delphi
\item Mathematica
\item Maple
\item Matlab
\item Scilab
\item Octave
\end{itemize}
\end{multicols}
\subsection{Software adicional}
Para la ejecuci\'{o}n de los algoritmos con c\'{o}digos de Python, ser\'{a} necesario usar un editor de texto, pueden elegir el de su gusto y preferencia, para aprovechar al m\'{a}ximo la integraci\'{o}n de herramientas, usaremos en el curso: Spyder2.
\\
\\
Adem\'{a}s ser\'{a} necesario usar software para graficaci\'{o}n de datos, para ello usaremos gnuPlot y la librer\'{i}a matplotlib integrada en Python, pero igual pueden usar el que ya conozcan y manejen bien.
\subsection{Opcionales}
Pueden traer una laptop para el trabajo en el curso, no es requisito, ya que tenemos equipos suficientes en el laboratorio. Deber\'{a}n de configurar en su equipo, ya sea en Linux o en Windows, la paqueter\'{i}a necesaria afortunadamente est\'{a} disponible de manera libre, es decir, es software GNU.
%\newpage
\section{Temario del curso}
Llevaremos el temario oficial del curso, que est\'{a} disponible en la p\'{a}gina de la Facultad, haciendo un ajuste en el orden de los temas, siendo entonces:
\\
\\  
\textbf{Tema 1: Escalas, condici\'{o}n y estabilidad}
\begin{enumerate}
\item Introducci\'{o}n.
\item Sistemas num\'{e}ricos de punto flotante y lenguajes.
\item Dimensiones y escalas.
\item Errores num\'{e}ricos y su amplificaci\'{o}n.
\item Condici\'{o}n de un problema y estabilidad de un m\'{e}todo.
\end{enumerate}
\textbf{Tema 2: Operaciones matem\'{a}ticas b\'{a}sicas}
\begin{enumerate}
\item Interpolaci\'{o}n y extrapolaci\'{o}n.
\item Diferenciaci\'{o}n num\'{e}rica.
\item Integraci\'{o}n num\'{e}rica.
\item Evaluaci\'{o}n num\'{e}rica de soluciones.
\end{enumerate}
\textbf{Tema 3: Ecuaciones diferenciales ordinarias}
\begin{enumerate}
\item M\'{e}todos simples.
\item M\'{e}todos impl\'{i}citos y de multipasos.
\item M\'{e}todos de Runge-Kutta.
\item Estabilidad de las soluciones.
\item Orden y caos en el movimiento de dos dimensiones.
\end{enumerate}
\textbf{Tema 4: An\'{a}lisis num\'{e}rico de problemas matriciales}
\begin{enumerate}
\item Inversi\'{o}n de matrices y n\'{u}mero de condici\'{o}n.
\item Valores propios de matrices tridiagonales.
\item Discretizaci\'{o}n de la ecuaci\'{o}n de Laplace y m\'{e}todos iterativos de soluci\'{o}n.
\item Solución num\'{e}rica de ecuaciones diferenciales el\'{i}pticas en una y dos dimensiones.
\end{enumerate}
\textbf{Tema 5: Problemas cl\'{a}sicos y cu\'{a}nticos de valores propios}
\begin{enumerate}
\item Algoritmo de Numerov.
\item Integraci\'{o}n de problemas con valores en la frontera.
\item Formulaci\'{o}n matricial para problemas de valores propios.
\item Formulaciones variacionales.
\end{enumerate}
\textbf{Tema 6: Simulaci\'{o}n computacional}
\begin{enumerate}
\item M\'{e}todo de Monte Carlo.
\item Din\'{a}mica molecular.
\item Otros algoritmos de simulaci\'{o}n.
\item Aplicaci\'{o}n a problemas de f\'{i}sica de inter\'{e}s actual.
\end{enumerate}
\textbf{Tema 7: Ecuaciones de evoluci\'{o}n}
\begin{enumerate}
\item La ecuaci\'{o}n de ondas y su discretizaci\'{o}n en diferencias finitas. Criterio de Courant.
\item La ecuaci\'{o}n de Fourier para el calor y su discretizaci\'{o}n en diferencias finitas. Estabilidad del esquema.
\end{enumerate}
\section{Bibliograf\'{i}a}
\begin{itemize}
\item Kahaner, D., Moler, C., Nash, S., 1989, Numerical methods and software, Prentice Hall, USA.
\item Klein, A., Godunov, A. Introductory Computational Physics. Cambridge University Press. 2006.
\item Gould, H., Tobochnik, J., 1988, An introduction to computer simulation methods: Applications to physical systems, Addison Wesley Publishing Company, USA.
\item Vesely, F., 1994, Computational physics: An introduction, Plenum Press, USA.
\item Rojas, J.F., Morales, M.A., Rangel, A., Torres, I. F\'{i}sica computacional: una propuesta educativa. Revista Mexicana de F\'{i}sica E 55 (1) 97–111, Junio 2009.
\item Janert, P. K. Gnuplot in action. Understanding data with graphs. Manning Publications Co. 2010.
\item Mej\'{i}a, C.E., Restrepo, T., Trefftz, C. LAPACK, una colecci\'{o}n de rutinas para resolver problemas de \'{a}lgebra lineal num\'{e}rica. Universidad Eafit, julio-septiembre, número 123, Universidad Eafit, Medill\'{i}n, Colombia, pp. 73-80. 2001.
\end{itemize}
\section{Evaluaci\'{o}n}
Se distribuye de la siguiente manera:
\begin{itemize}
\item \textbf{Ejercicios en clase $\mathbf{20\%}$:} para tener derecho a este porcentaje se requiere estar presente en la clase, es decir, el ejercicio se entregar\'{a} en la clase o se dejar\'{a} para la siguiente, en caso de que no asistan y se enteren del ejercicio, se les revisar\'{a} el trabajo que entreguen, pero no se les tomar\'{a} en cuenta para el porcentaje, (moraleja: hay que asistir a clase) 
\item \textbf{Tareas $\mathbf{40\%}$} : Se entregar\'{a} una tarea por tema, se les proporcionar\'{a} de manera adelantada y con fecha de entrega definida, no se reciben tareas extempor\'{a}neas, ni por correo.
\item \textbf{Ex\'{a}menes $\mathbf{40\%}$} : Uno por tema, de tipo teóricos-pr\'{a}cticos. 
\end{itemize}
\underline{No habr\'{a} reposiciones de ex\'{a}menes parciales.}
\section{Examen final}
\textbf{Para tener derecho al examen final:} se deber\'{a}n de haber presentado todos los ex\'{a}menes parciales y haber entregado todas las tareas del curso. Habr\'{a} dos rondas de examen final, si en la primera de ellas no se acredita el examen, ser\'{a} posible presentarlo en una segunda y \'{u}ltima ronda, se aclara que para tener derecho al segundo examen, se debe de presentar el primero.
\\
\\
En caso de haber presentado al menos un examen parcial o haber entregado al menos una tarea, y el promedio final sea menor a 6, la calificación final que se asentar\'{a} en el acta, ser\'{a} 5. No hay renuncias a calificaciones.
\section{Fechas importantes}
\begin{itemize}
\item 5 de agosto. Inicio del semestre.
\item 22 de noviembre. Fin de Semestre.
\item 25 al 29 de noviembre, primera semana de finales.
\item 2 al 6 de diciembre, segunda semana de finales.
\end{itemize}
\end{document}