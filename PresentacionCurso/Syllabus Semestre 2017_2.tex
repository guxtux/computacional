\documentclass[12pt]{article}
\usepackage[utf8]{inputenc}
\usepackage[spanish]{babel}
\usepackage{amsmath}
\usepackage{amsthm}
\usepackage{hyperref}
\usepackage{graphicx}
\usepackage{color}
\usepackage{float}
\usepackage{multicol}
\usepackage{enumerate}
\usepackage{anyfontsize}
\usepackage{anysize}
\renewcommand{\baselinestretch}{1.5}
\marginsize{1.5cm}{1.5cm}{-1cm}{2cm}
\author{M. en C. Gustavo Contreras Mayén. \texttt{curso.fisica.comp@gmail.com}\\
M. en C. Abraham Lima Buendía. \texttt{abraham3081@ciencias.unam.mx}}
\title{Curso de Física Computacional\\{\large Semestre 2017-2 Grupo 8198}}
\date{ }
\begin{document}
%\renewcommand\theenumii{\arabic{theenumii.enumii}}
\renewcommand\labelenumii{\theenumi.{\arabic{enumii}}}
\maketitle
\fontsize{12}{12}\selectfont
\textbf{Lugar: }Laboratorio de Enseñanza en Cómputo de Física, Edificio Tlahuizcalpan.
\\
\textbf{Horario: } Martes y Jueves de 18 a 21 horas.
\\
\\

\textbf{Objetivos y Temario:} Se trabajará el temario oficial de la asignatura, que está disponible en: \href{http://www.fciencias.unam.mx/asignaturas/715.pdf}{http://www.fciencias.unam.mx/asignaturas/715.pdf}
\section{Metodología de Enseñanza}
\textbf{Antes de la clase.}
\\
Para facilitar la discusión en el aula, el alumno revisará el material de trabajo que se le proporcionará oportunamente, de tal manera que llegará conociendo el tema a desarrollar a la clase. Daremos por entendido de que el alumno realizará la lectura y actividades establecidas.
\\
\textbf{Durante la clase.}
\\
Se dará un tiempo para la exposición con diálogo y discusión del material de trabajo con los temas a cubrir durante el semestre. Se busca que sea un curso totalmente práctico por lo que se va a trabajar con los equipos de cómputo del laboratorio.
\\
\textbf{Después de la clase.}
\\
El curso requiere que le dediquen al menos el mismo número de horas de trabajo en casa, es decir, les va a demandar seis horas como mínimo; si cuentan con una experiencia en programación, tienen un paso adelantado, pero si no han programado, se verán en la necesidad de dedicarle más tiempo.
\section{Evaluación}
Para tener derecho a calificación, se requiere la asistencia mínima del 80\%.
\\
\bigskip
Los elementos y el peso de la calificación del curso, se distribuyen de la siguiente manera:
\begin{itemize}
\item \textbf{Ejercicios en clase $\mathbf{30\%}$:} para tener derecho a este porcentaje se requiere estar presente en la clase, es decir, el ejercicio se entregará en la clase o se dejará para la siguiente, en caso de que no asistan y se enteren del ejercicio, se les revisará el trabajo que entreguen, pero no se les tomará en cuenta para el porcentaje, (moraleja: hay que asistir a clase) 
\item \textbf{Tareas $\mathbf{30\%}$} : Son cinco tareas en todo el curso, se les proporcionará de manera adelantada y con fecha de entrega definida, no se reciben tareas extemporáneas, ni por correo.
\item \textbf{Exámenes $\mathbf{40\%}$} : Habrá cinco exámenes, de tipo teóricos-prácticos. 
\end{itemize}
\underline{No habrá reposiciones de exámenes parciales.}
\\
\\
La calificación final se obtendrá de los porcentajes indicados para los ejercicios en clase, de las tareas y de los exámenes. En el caso de obtener un promedio mayor o igual a 6, es el que se asentará en el acta, que corresponde a la calificación definitiva del curso.
\section{Examen final}
\textbf{Para tener derecho al examen final:} se deberán de haber presentado todos los exámenes parciales y haber entregado todas las tareas del curso. Habrá dos rondas de examen final, si en la primera de ellas no se acredita el examen, será posible presentarlo en una segunda y última ronda, se aclara que para tener derecho al segundo examen, se debe de presentar el primero.
\\
\\
En caso de haber presentado al menos un examen parcial o haber entregado al menos una tarea, y si se presenta una combinación de los siguientes casos:
\begin{itemize}
\item Ya no se presenten al curso (no cubran el 80 \% de asistencia).
\item No presenten un examen y/o tarea.
\end{itemize}
La calificación final que se asentará en el acta del curso será 5 (cinco). De acuerdo al Reglamento de Estudios Profesionales: No hay renuncias a calificaciones finales.
\section{Fechas importantes}
\begin{itemize}
\item Lunes 30 de enero. Inicio del semestre.
\item Lunes 10 a viernes 14 de abril. Semana Santa.
\item 25 de mayo. Fin de Semestre.
\item 29 de mayo al 2 de junio, primera semana de finales.
\item 5 al 9 de junio, segunda semana de finales.
\end{itemize}
\end{document}