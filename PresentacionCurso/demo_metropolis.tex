\documentclass[12pt]{beamer}
\newenvironment{ConCodigo}[1]
  {\begin{frame}[fragile,environment=ConCodigo]{#1}}
  {\end{frame}}
\graphicspath{{Imagenes/}{../Imagenes/}}
\usepackage[utf8]{inputenc}
\usepackage[spanish]{babel}
\usepackage{hyperref}
\usepackage{etex}
%\reserveinserts{28}
\usepackage{amsmath}
\usepackage{amsthm}
\usepackage{mathtools}
\usepackage{multicol}
\usepackage{multirow}
\usepackage{tabulary}
\usepackage{booktabs}
\usepackage{nccmath}
\usepackage{physics}
\usepackage{biblatex}
\usepackage[outdir=./]{epstopdf}
%\epstopdfsetup{outdir=./}
\usepackage{graphicx}
%\usepackage{enumitem,xcolor}
\usepackage{siunitx}
%\sisetup{scientific-notation=true}
%\usepackage{fontspec}
\usepackage{lmodern}
\usepackage{float}
\usepackage[format=hang, font=footnotesize, labelformat=parens]{caption}
\usepackage[autostyle,spanish=mexican]{csquotes}
\usepackage{standalone}
\usepackage{blkarray}
\usepackage{algorithm}
\usepackage{algorithmic}
\usepackage{tikz}
\usepackage[siunitx, RPvoltages]{circuitikz}
\usetikzlibrary{arrows,patterns,shapes}
\usetikzlibrary{decorations.markings}
\usetikzlibrary{arrows}
\usepackage{color}
\usepackage{xcolor}
%\usepackage{beton}
%\usepackage{euler}
%\usepackage[T1]{fontenc}
\usepackage[sfdefault]{roboto}  %% Option 'sfdefault' only if the base font of the document is to be sans serif
\usepackage[T1]{fontenc}
\renewcommand*\familydefault{\sfdefault}
\DeclareGraphicsExtensions{.pdf,.png,.jpg}
\usepackage{hyperref}
\renewcommand {\arraystretch}{1.5}
\newcommand{\python}{\texttt{python}}
\usefonttheme[onlymath]{serif}
\setbeamertemplate{navigation symbols}{}
\usetikzlibrary{patterns}
\usetikzlibrary{decorations.markings}
\tikzstyle{every picture}+=[remember picture,baseline]
%\tikzstyle{every node}+=[inner sep=0pt,anchor=base,
%minimum width=2.2cm,align=center,text depth=.15ex,outer sep=1.5pt]
%\tikzstyle{every path}+=[thick, rounded corners]
\setbeamertemplate{caption}[numbered]
\newcommand{\ptm}{\fontfamily{ptm}\selectfont}
%Se usa la plantilla Warsaw modificada con spruce
\mode<presentation>
{
  \usetheme{Warsaw}
  \setbeamertemplate{headline}{}
  \useoutertheme{default}
  \usecolortheme{albatross}
  \setbeamercovered{invisible}
}
% \AtBeginSection[]
% {
% \begin{frame}<beamer>{Contenido}
% \normalfont\mdseries
% \tableofcontents[currentsection]
% \end{frame}
% }

\usetheme[progressbar=frametitle]{metropolis}
\usepackage{appendixnumberbeamer}
\usepackage{booktabs}
\usepackage[scale=2]{ccicons}
\usepackage{pgfplots}
\usepgfplotslibrary{dateplot}
\usepackage{xspace}
\newcommand{\themename}{\textbf{\textsc{metropolis}}\xspace}
\title{Curso de Física Computacional}
\subtitle{Semestre 2018-2}
\author[]{M. en C. Gustavo Contreras Mayén \\ M. en C. Abraham Lima Buendía}
\institute{Facultad de Ciencias - UNAM}
%\titlegraphic{\includegraphics[width=2cm]{escudo-facultad-ciencias.jpg}\hspace*{5.75cm}~%
%   \includegraphics[width=2cm]{escudo-unam.jpg}}
\date{\today}
\begin{document}
\maketitle
\begin{frame}{Table of contents}
  \setbeamertemplate{section in toc}[sections numbered]
  \tableofcontents[hideallsubsections]
\end{frame}
\section{Presentación del curso}
\frame{\tableofcontents[currentsection, hideothersubsections]}
\subsection{Objetivos}
\begin{frame}
\frametitle{Objetivos 1}
El propósito del curso es enseñar al estudiante las ideas de computabilidad usadas en distintas áreas de la  física para resolver un conjunto de problemas modelo. 
\end{frame}
\begin{frame}
A partir de planteamientos analíticos se pretende obtener resultados numéricos reproducibles consistentes, y que predigan situaciones físicas asociadas al problema bajo estudio.
\end{frame}
\begin{frame}
\frametitle{Objetivos 2}
El alumno debe asimilar las ideas básicas del análisis numérico, como son las de estabilidad en el cálculo y la sensibilidad de las respuestas a las perturbaciones en la estructura del problema.
\end{frame}
\begin{frame}
\frametitle{Objetivos 3}
El curso también le dará al estudiante capacidad de juicio sobre la calidad de los resultados numéricos obtenidos.
\end{frame}
\begin{frame}
\frametitle{Objetivos 4}
En particular se hará énfasis en la confiabilidad de los resultados respecto a los errores tanto del algoritmo de solución como de las limitaciones numéricas de la computadora. 
\\
\bigskip
Esta capacidad se adquirirá a lo largo del curso comparando resultados numéricos con otros tipos de análisis, en las regiones en las cuales se pueden llevar ambos a cabo.
\end{frame}
\begin{frame}
\frametitle{Objetivos 5}
 Por otra parte permitirá al estudiante explorar regiones de comportamiento físico sólo accesibles al cálculo numérico.
\end{frame} 
\section{Sobre el curso}
\frame{\tableofcontents[currentsection, hideothersubsections]}
\subsection{Lugar y horario}
\begin{frame}
\frametitle{Lugar y horario} 
\textbf{Lugar: }Laboratorio de Enseñanza en Cómputo de Física, Edificio Tlahuizcalpan.
\\
\bigskip
\textbf{Horario: } Martes y Jueves de 18 a 21 horas.
\end{frame}
\subsection{Metodología de Enseñanza}
\begin{frame}
\frametitle{Metodología de Enseñanza - 1}
\textbf{Antes de la clase.}
\\
\medskip
Para facilitar la discusión en el aula, el alumno revisará antes de la clase el material de trabajo que se le proporcionará oportunamente, de tal manera que ya llegará a la misma conociendo el tema a desarrollar durante la clase.
\\
\bigskip
Daremos por entendido de que el alumno realizará la lectura y/o actividades.
\end{frame}
\begin{frame} 
\frametitle{Metodología de Enseñanza - 2}
\textbf{Durante la clase.}
\\
\medskip
Se dará un tiempo para la exposición con diálogo por parte de los profesores y discusión del material de trabajo con los temas a cubrir durante el semestre.
\\
\bigskip
Se busca que sea un curso totalmente práctico por lo que se va a trabajar con los equipos de cómputo del laboratorio.
\end{frame}
\begin{frame}
\frametitle{Herramienta de programación}
Será necesario utilizar una herramienta computacional para resolver ejercicios y problemas que se revisen en clase.
\\
\bigskip
Usaremos el lenguaje de programación \textoazul{\python} dada su versatilidad y facilidad de manejo.
\end{frame}
\begin{frame}
Las técnicas de programación que vayan adquiriendo serán el reflejo de su trabajo fuera de clase. En caso de no trabajar o dedicarle el tiempo al curso, se complicará bastante, situación que esperamos no se presente.
\end{frame}
\begin{frame}
\frametitle{Guías adicionales de apoyo.}
Se han elaborado guías de apoyo complementarias para la consulta tanto de los conceptos principales de la física involucrada en el problema, así como de programación con \textoazul{\python}.
\\
\bigskip
De esta manera tendrán una referencia inicial y ya por su cuenta, consultar otros materiales y con ello, lograr un entendimiento completo del problema y su solución.
\end{frame}
\subsection{¿Programación?}
\begin{frame}
\frametitle{¿Programación?}
La solución de un problema, requiere de realizar una abstracción del mismo, es decir, debemos de plantear el problema físico, a un problema que permita ser resuelto mediante un algoritmo.
\end{frame}
\begin{frame}
El algoritmo que se proponga como solución deberá de \enquote{probarse} por lo que debemos de revisar la solución, así como la congruencia de la misma con la física y sobre todo, el margen de error que devuelve la solución numérica.
\end{frame}
\begin{frame}
El curso de Física Computacional \textoazul{NO es un curso de programación bajo algún lenguaje en particular}.
\\
\bigskip
Es altamente recomendable que cuenten con conocimientos de programación básicos en algún lenguaje o software.
\end{frame}
\begin{frame}
\frametitle{Tema 0 del curso}
En el curso utilizaremos \textoazul{\python} para programar, se dará un breviario de programación básica como Tema 0, que no será evaluado ni formará parte de la calificación final.
\end{frame}
\begin{frame}
Tendremos un panorama general del uso del lenguaje, pero NO debemos de confiarnos y pensar que con esto, ya podremos programar con facilidad, mientras más práctica tengan, poco a poco mejorarán sus técnicas de programación.
\end{frame}
\begin{frame}
\frametitle{Ya se programar!!}
Cuentan con la completa libertad de elegir el lenguaje o software para trabajar durante el curso:
\begin{multicols}{2}
\begin{itemize}
\item Fortran
\item Java
\item C++
\item C
\item Delphi
\item Wolfram
\item Mathematica
\item Maple
\item Matlab
\item Scilab
\item Octave
\end{itemize}
\end{multicols}
\end{frame}
\begin{frame}
Si es el caso que pueden trabajar con algún otro lenguaje o software, deberán de entregar su código fuente y el archivo ejecutable.
\end{frame}
\begin{frame}
\frametitle{Software}
Usaremos dentro del curso la suite \textoazul{Anaconda}, que es de libre distribución y contiene una serie de herramientas y programas con lo que programar con \textoazul{\python}, será una tarea más sencilla.
\end{frame}
\begin{frame}
\frametitle{Anaconda}
La suite incluye un \emph{entorno de desarrollo}, terminales, sistema de debug y de consulta.
\\
\bigskip
Como es multiplataforma, se puede utilizar en entornos linux, iOS y Windows. En los equipos del laboratorio tienen instalado linux y Fedora como distribución.
\end{frame}
\begin{frame}
\frametitle{Opcionales}
Pueden traer una laptop para el trabajo en el curso, no es requisito, ya que tenemos equipos suficientes en el laboratorio.
\\
\medskip
Se recomienda que cuenten en sus equipos con el mismo software, las guías que hemos comentado, les brindarán la información para instalar los programas.
\end{frame}
\begin{frame}
\frametitle{Metodología de Enseñanza - 3}
\textbf{Después de la clase.}
\\
\medskip
El curso \emph{requiere que le dediquen al menos el mismo número de horas de trabajo en casa}, es decir, \textoazul{les va a demandar seis horas como mínimo}; si cuentan con una experiencia en programación, tienen un paso adelantado, pero si no han programado, se verán en la necesidad de dedicarle más tiempo.
\end{frame}
\section{Temario}
\frame{\tableofcontents[currentsection, hideothersubsections]}
\subsection{Temario}
\begin{frame}
\frametitle{Temario del curso}
Llevaremos el temario oficial del curso, que está disponible en la página de la Facultad \href{http://www.fciencias.unam.mx/asignaturas/715.pdf}{- Temario -}, haciendo un ajuste en el orden de los temas, siendo entonces:
\end{frame}
\begin{frame}
\frametitle{\textbf{Tema 1: Escalas, condición y estabilidad}}
\setbeamercolor{item projected}{bg=blue!70!black,fg=yellow}
\setbeamertemplate{enumerate items}[circle]
\begin{enumerate}[<+->]
\item Introducción.
\item Sistemas numéricos de punto flotante y lenguajes.
\item Dimensiones y escalas.
\item Errores numéricos y su amplificación.
\item Condición de un problema y estabilidad de un método.
\end{enumerate}
\end{frame}
\begin{frame}
\frametitle{\textbf{Tema 2: Operaciones matemáticas básicas}}
\setbeamercolor{item projected}{bg=blue!70!black,fg=yellow}
\setbeamertemplate{enumerate items}[circle]
\begin{enumerate}[<+->]
\item Interpolación y extrapolación.
\item Diferenciación numérica.
\item Integración numérica.
\item Evaluación numérica de soluciones.
\end{enumerate}
\end{frame}
\begin{frame}
\frametitle{\textbf{Tema 3: Ecuaciones diferenciales ordinarias}}
\setbeamercolor{item projected}{bg=blue!70!black,fg=yellow}
\setbeamertemplate{enumerate items}[circle]
\begin{enumerate}[<+->]
\item Métodos simples.
\item Métodos implícitos y de multipasos.
\item Métodos de Runge-Kutta.
\item Estabilidad de las soluciones.
\item Orden y caos en el movimiento de dos dimensiones.
\end{enumerate}
\end{frame}
\begin{frame}
\frametitle{\textbf{Tema 4: Análisis numérico de problemas matriciales}}
\setbeamercolor{item projected}{bg=blue!70!black,fg=yellow}
\setbeamertemplate{enumerate items}[circle]
\begin{enumerate}[<+->]
\item Inversión de matrices y número de condición.
\item Valores propios de matrices tridiagonales.
\item Discretización de la ecuación de Laplace y métodos iterativos de solución.
\item Solución numérica de ecuaciones diferenciales elípticas en una y dos dimensiones.
\end{enumerate}
\end{frame}
\begin{frame}
\frametitle{\textbf{Tema 5: Problemas clásicos y cuánticos de valores propios}}
\setbeamercolor{item projected}{bg=blue!70!black,fg=yellow}
\setbeamertemplate{enumerate items}[circle]
\begin{enumerate}[<+->]
\item Algoritmo de Numerov.
\item Integración de problemas con valores en la frontera.
\item Formulación matricial para problemas de valores propios.
\item Formulaciones variacionales.
\end{enumerate}
\end{frame}
\begin{frame}
\frametitle{\textbf{Tema 6: Simulación computacional}}
\setbeamercolor{item projected}{bg=blue!70!black,fg=yellow}
\setbeamertemplate{enumerate items}[circle]
\begin{enumerate}[<+->]
\item Método de Monte Carlo.
\item Dinámica molecular.
\item Otros algoritmos de simulación.
\item Aplicación a problemas de física de interés actual.
\end{enumerate}
\end{frame}
\begin{frame}
\frametitle{\textbf{Tema 7: Ecuaciones de evolución}}
\setbeamercolor{item projected}{bg=blue!70!black,fg=yellow}
\setbeamertemplate{enumerate items}[circle]
\begin{enumerate}[<+->]
\item La ecuación de ondas y su discretización en diferencias finitas. Criterio de Courant.
\item La ecuación de Fourier para el calor y su discretización en diferencias finitas. Estabilidad del esquema.
\end{enumerate}
\end{frame}
\section{Evaluación del curso}
\frame{\tableofcontents[currentsection, hideothersubsections]}
\subsection{Evaluación}
\begin{frame}
\frametitle{Evaluación}
Se distribuye de la siguiente manera:
\setbeamercolor{item projected}{bg=blue!70!black,fg=yellow}
\setbeamertemplate{enumerate items}[circle]
\begin{enumerate}[<+->]
\item Ejercicios en clase $\mathbf{10\%}$
\item Tareas $\mathbf{50\%}$
\item Exámenes en salón $\mathbf{40\%}$
\end{enumerate}
\end{frame}
\begin{frame}
\frametitle{Ejercicios en clase $\mathbf{10\%}$}
Este porcentaje considera necesariamente la asistencia del alumno en clase, ya que habrá ejercicios que requieran completarse y se deberá de entregar la solución en la siguiente sesión.
\\
\medskip
En el caso de que no asistan a la clase y se enteren del ejercicio, se les revisará el trabajo que entreguen, pero no se les tomará en cuenta para el porcentaje, (moraleja: hay que asistir a clase) 
\end{frame}
\begin{frame}
\frametitle{Tareas $\mathbf{50\%}$}
Serán 3 tareas en total durante el semestre, se les proporcionarán los ejericicios de manera adelantada y con fecha de entrega definida, no se recibirán tareas extemporáneas, ni se enviarán por correo.
\\
\bigskip
Se calificarán sólo aquellas tareas que entreguen el $50\%$ de los ejercicios. 
\end{frame}
\begin{frame}
\frametitle{Trabajo en equipo}
Podrán reunirse y colaborar para discutir, debatir, proponer y bosquejar la solución a los ejercicios de las tareas.
\\
\bigskip
En el dado caso de encontrar códigos idénticos, se cancelarán no sólo los ejercicios tipo copy-paste, sino la tarea completa del(los) alumnos involucrados.
\end{frame}
\begin{frame}
\frametitle{Exámenes $\mathbf{40\%}$}
Se realizarán tres exámenes durante el semestre, siendo del tipo teóricos-prácticos.
\\
\bigskip
Se aplicarán en el aula de cómputo y el trabajo será individual.
\end{frame}
\begin{frame}
\frametitle{Una sola reposición de examen}
Considerando que sólo habrá tres exámenes en el semestre, se considera la posibilidad de presentar una única reposición si y sólo si se cumplen los siguientes puntos:
\begin{itemize}[<+->]
  \item Sólo un examen parcial tenga una calificación no aprobatoria, es decir, que la calificación del examen parcial sea menor a 6 (seis)
  \item Se debieron de haber presentado los otros dos exámenes parciales.
  \item Se debieron de haber entregado las tres tareas completas.
\end{itemize}
\end{frame}
\begin{frame}
En caso de contar con un promedio final aprobatorio del curso (los tres exámenes parciales aprobados), no se aplicará una reposición de algún examen para subir el promedio final del curso.
\end{frame}
\begin{frame}
\frametitle{Examen final}
El examen final del curso se presentará si y sólo si:
\begin{itemize}[<+->]
\item Se presentaron los tres los exámenes parciales.
\item Entregaron las tres tareas del curso.
\item Hay dos exámenes parciales con calificación menor a seis.
\end{itemize}
\end{frame}
\begin{frame}
La calificación obtenida en el examen final, es la que se asentará en el acta de calificaciones del curso de Física Computacional.
\\
\bigskip
Ya no se promediará con las tareas ni con los ejercicios de clase.
\end{frame}
\begin{frame}
\frametitle{Muy importante}
Habrá dos rondas de examen final, si en la primera de ellas no se acredita el examen, será posible presentarlo en una segunda y última ronda, se aclara que para tener derecho al segundo examen, se debe de presentar el primero.
\end{frame}
\begin{frame}
\emph{En caso de haber presentado al menos un examen parcial y/o haber entregado al menos una tarea} se promediarán respectivamente las tareas y exámenes.
\\
\bigskip
Sólo se asentará en el acta de calificaciones \textoazul{NP} si el(la) alumn{@} no entrega tarea alguna y no presenta algún examen. (¿?)
\end{frame}
\begin{frame}
\frametitle{Más importante}
De acuedo al Reglamento General de Exámenes de la UNAM, se considera una calificación aprobatoria aquella que sea mayor o igual a $6$ seis.
\\
\medskip
\begin{itemize}
\item No \enquote{se guardan calificaciones}.
\item No se renuncia a una calificación.
\end{itemize}
\end{frame}
\section{Notas importantes}
\frame{\tableofcontents[currentsection, hideothersubsections]}
\subsection{Consideraciones importantes}
\begin{frame}
\frametitle{Consideraciones importantes 1}
\begin{itemize}[<+->]
\item El cupo para el curso es de 25 alumnos.
\item Se le dará prioridad en la inscripción a los alumnos que están cursando regularmente la carrera, es decir, alumnos que están inscritos en el séptimo semestre.
\item Si consideran quedarse en el curso y se les firma la tira de materias, entendemos que completarán en el curso, si quieren revisar otras opciones de horarios o profesores, se les pide amablemente no requieran la firma, para darle oportunidad a quienes ya están seguros de llevar el curso.
\end{itemize}
\end{frame}
\begin{frame}
\frametitle{Consideraciones importantes 2}
\begin{itemize}[<+->]
\item Si alguien desea participar como oyente sin inscripción, podrá hacerlo siempre y cuando haya espacio de trabajo o traiga laptop, pero NO se guardarán calificaciones.
\item Les pedimos gentilmente que revisen detalladamente la organización de sus horarios, para evitar empalmes con otras asignaturas, el curso de Física Computacional les exigirá la atención y trabajo necesarios.
\end{itemize}
\end{frame}
\subsection{Fechas importantes}
\begin{frame}
\frametitle{Fechas importantes}
\begin{itemize}[<+->]
\item Lunes 29 de enero. Inicio del semestre 2018-2.
\item Lunes 5 de febrero, día feriado.
\item Lunes 19 de febrero, día feriado.
\item \textcolor{red}{Del lunes 26 de marzo al viernes 30 de marzo, Semana Santa.}
\item \textcolor{red}{Martes 1 de mayo, día feriado.}
\item \textcolor{red}{Jueves 10 de mayo, día feriado.}
\item \textcolor{red}{Martes 15 de mayo, día feriado.}
\item Viernes 29 de mayo. Fin de Semestre.
\item Del 28 de mayo al 1 de junio, primera semana de finales.
\item Del 4 al 8 de junio, segunda semana de finales.
\end{itemize}
\end{frame}
\section{Introduction}

\begin{frame}[fragile]{Metropolis}

  The \themename theme is a Beamer theme with minimal visual noise
  inspired by the \href{https://github.com/hsrmbeamertheme/hsrmbeamertheme}{\textsc{hsrm} Beamer
  Theme} by Benjamin Weiss.

  Enable the theme by loading

  \begin{verbatim}    \documentclass{beamer}
    \usetheme{metropolis}\end{verbatim}

  Note, that you have to have Mozilla's \emph{Fira Sans} font and XeTeX
  installed to enjoy this wonderful typography.
\end{frame}
\begin{frame}[fragile]{Sections}
  Sections group slides of the same topic

  \begin{verbatim}    \section{Elements}\end{verbatim}

  for which \themename provides a nice progress indicator \ldots
\end{frame}

\section{Titleformats}

\begin{frame}{Metropolis titleformats}
	\themename supports 4 different titleformats:
	\begin{itemize}
		\item Regular
		\item \textsc{Smallcaps}
		\item \textsc{allsmallcaps}
		\item ALLCAPS
	\end{itemize}
	They can either be set at once for every title type or individually.
\end{frame}

{
    \metroset{titleformat frame=smallcaps}
\begin{frame}{Small caps}
	This frame uses the \texttt{smallcaps} titleformat.

	\begin{alertblock}{Potential Problems}
		Be aware, that not every font supports small caps. If for example you typeset your presentation with pdfTeX and the Computer Modern Sans Serif font, every text in smallcaps will be typeset with the Computer Modern Serif font instead.
	\end{alertblock}
\end{frame}
}

{
\metroset{titleformat frame=allsmallcaps}
\begin{frame}{All small caps}
	This frame uses the \texttt{allsmallcaps} titleformat.

	\begin{alertblock}{Potential problems}
		As this titleformat also uses smallcaps you face the same problems as with the \texttt{smallcaps} titleformat. Additionally this format can cause some other problems. Please refer to the documentation if you consider using it.

		As a rule of thumb: Just use it for plaintext-only titles.
	\end{alertblock}
\end{frame}
}

{
\metroset{titleformat frame=allcaps}
\begin{frame}{All caps}
	This frame uses the \texttt{allcaps} titleformat.

	\begin{alertblock}{Potential Problems}
		This titleformat is not as problematic as the \texttt{allsmallcaps} format, but basically suffers from the same deficiencies. So please have a look at the documentation if you want to use it.
	\end{alertblock}
\end{frame}
}

\section{Elements}

\begin{frame}[fragile]{Typography}
      \begin{verbatim}The theme provides sensible defaults to
\emph{emphasize} text, \alert{accent} parts
or show \textbf{bold} results.\end{verbatim}

  \begin{center}becomes\end{center}

  The theme provides sensible defaults to \emph{emphasize} text,
  \alert{accent} parts or show \textbf{bold} results.
\end{frame}

\begin{frame}{Font feature test}
  \begin{itemize}
    \item Regular
    \item \textit{Italic}
    \item \textsc{SmallCaps}
    \item \textbf{Bold}
    \item \textbf{\textit{Bold Italic}}
    \item \textbf{\textsc{Bold SmallCaps}}
    \item \texttt{Monospace}
    \item \texttt{\textit{Monospace Italic}}
    \item \texttt{\textbf{Monospace Bold}}
    \item \texttt{\textbf{\textit{Monospace Bold Italic}}}
  \end{itemize}
\end{frame}

\begin{frame}{Lists}
  \begin{columns}[T,onlytextwidth]
    \column{0.33\textwidth}
      Items
      \begin{itemize}
        \item Milk \item Eggs \item Potatos
      \end{itemize}

    \column{0.33\textwidth}
      Enumerations
      \begin{enumerate}
        \item First, \item Second and \item Last.
      \end{enumerate}

    \column{0.33\textwidth}
      Descriptions
      \begin{description}
        \item[PowerPoint] Meeh. \item[Beamer] Yeeeha.
      \end{description}
  \end{columns}
\end{frame}
\begin{frame}{Animation}
  \begin{itemize}[<+- | alert@+>]
    \item \alert<4>{This is\only<4>{ really} important}
    \item Now this
    \item And now this
  \end{itemize}
\end{frame}
\begin{frame}{Figures}
  \begin{figure}
    \newcounter{density}
    \setcounter{density}{20}
    \begin{tikzpicture}
      \def\couleur{alerted text.fg}
      \path[coordinate] (0,0)  coordinate(A)
                  ++( 90:5cm) coordinate(B)
                  ++(0:5cm) coordinate(C)
                  ++(-90:5cm) coordinate(D);
      \draw[fill=\couleur!\thedensity] (A) -- (B) -- (C) --(D) -- cycle;
      \foreach \x in {1,...,40}{%
          \pgfmathsetcounter{density}{\thedensity+20}
          \setcounter{density}{\thedensity}
          \path[coordinate] coordinate(X) at (A){};
          \path[coordinate] (A) -- (B) coordinate[pos=.10](A)
                              -- (C) coordinate[pos=.10](B)
                              -- (D) coordinate[pos=.10](C)
                              -- (X) coordinate[pos=.10](D);
          \draw[fill=\couleur!\thedensity] (A)--(B)--(C)-- (D) -- cycle;
      }
    \end{tikzpicture}
    \caption{Rotated square from
    \href{http://www.texample.net/tikz/examples/rotated-polygons/}{texample.net}.}
  \end{figure}
\end{frame}
\begin{frame}{Tables}
  \begin{table}
    \caption{Largest cities in the world (source: Wikipedia)}
    \begin{tabular}{lr}
      \toprule
      City & Population\\
      \midrule
      Mexico City & 20,116,842\\
      Shanghai & 19,210,000\\
      Peking & 15,796,450\\
      Istanbul & 14,160,467\\
      \bottomrule
    \end{tabular}
  \end{table}
\end{frame}
\begin{frame}{Blocks}
  Three different block environments are pre-defined and may be styled with an
  optional background color.

  \begin{columns}[T,onlytextwidth]
    \column{0.5\textwidth}
      \begin{block}{Default}
        Block content.
      \end{block}

      \begin{alertblock}{Alert}
        Block content.
      \end{alertblock}

      \begin{exampleblock}{Example}
        Block content.
      \end{exampleblock}

    \column{0.5\textwidth}

      \metroset{block=fill}

      \begin{block}{Default}
        Block content.
      \end{block}

      \begin{alertblock}{Alert}
        Block content.
      \end{alertblock}

      \begin{exampleblock}{Example}
        Block content.
      \end{exampleblock}

  \end{columns}
\end{frame}
\begin{frame}{Math}
  \begin{equation*}
    e = \lim_{n\to \infty} \left(1 + \frac{1}{n}\right)^n
  \end{equation*}
\end{frame}
\begin{frame}{Line plots}
  \begin{figure}
    \begin{tikzpicture}
      \begin{axis}[
        mlineplot,
        width=0.9\textwidth,
        height=6cm,
      ]

        \addplot {sin(deg(x))};
        \addplot+[samples=100] {sin(deg(2*x))};

      \end{axis}
    \end{tikzpicture}
  \end{figure}
\end{frame}
\begin{frame}{Bar charts}
  \begin{figure}
    \begin{tikzpicture}
      \begin{axis}[
        mbarplot,
        xlabel={Foo},
        ylabel={Bar},
        width=0.9\textwidth,
        height=6cm,
      ]

      \addplot plot coordinates {(1, 20) (2, 25) (3, 22.4) (4, 12.4)};
      \addplot plot coordinates {(1, 18) (2, 24) (3, 23.5) (4, 13.2)};
      \addplot plot coordinates {(1, 10) (2, 19) (3, 25) (4, 15.2)};

      \legend{lorem, ipsum, dolor}

      \end{axis}
    \end{tikzpicture}
  \end{figure}
\end{frame}
\begin{frame}{Quotes}
  \begin{quote}
    Veni, Vidi, Vici
  \end{quote}
\end{frame}

{%
\setbeamertemplate{frame footer}{My custom footer}
\begin{frame}[fragile]{Frame footer}
    \themename defines a custom beamer template to add a text to the footer. It can be set via
    \begin{verbatim}\setbeamertemplate{frame footer}{My custom footer}\end{verbatim}
\end{frame}
}

\begin{frame}{References}
  Some references to showcase [allowframebreaks] \cite{knuth92,ConcreteMath,Simpson,Er01,greenwade93}
\end{frame}

\section{Conclusion}

\begin{frame}{Summary}

  Get the source of this theme and the demo presentation from

  \begin{center}\url{github.com/matze/mtheme}\end{center}

  The theme \emph{itself} is licensed under a
  \href{http://creativecommons.org/licenses/by-sa/4.0/}{Creative Commons
  Attribution-ShareAlike 4.0 International License}.

  \begin{center}\ccbysa\end{center}

\end{frame}

{\setbeamercolor{palette primary}{fg=black, bg=yellow}
\begin{frame}[standout]
  Questions?
\end{frame}
}

\appendix

\begin{frame}[fragile]{Backup slides}
  Sometimes, it is useful to add slides at the end of your presentation to
  refer to during audience questions.

  The best way to do this is to include the \verb|appendixnumberbeamer|
  package in your preamble and call \verb|\appendix| before your backup slides.

  \themename will automatically turn off slide numbering and progress bars for
  slides in the appendix.
\end{frame}

\begin{frame}[allowframebreaks]{References}

  \bibliography{demo}
  \bibliographystyle{abbrv}

\end{frame}

\end{document}